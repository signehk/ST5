\section{Fremtidigt arbejde}

Fremtidigt arbejde med MTV-elementerne vil indeholde følgende under de fire områder:

Teknologi:
Her analyseres krav til aktivitetstracker, for derefter at vælge et aktivitetsarmbånd til den videre analyse, på baggrund af studier vedr. blandt andet præcision og egenskaber. Efterfølgende beskrives hvilke teknologier det valgte aktivitetsarmbånd består af, samt hvordan disse kan anvendes i forbindelse med aktivitetstracking. Sidst undersøges hvorvidt der er mulighed for at anvende et allerede eksisterende interface til dataoverførsel fra patient til læge.

Patient: 
Under dette område beskrives hvilke kriterier patienten skal opfylde, for at få udleveret et armbånd. Her ses der også på studier vedrørende brugervenligheden af det valgte armbånd, samt krav til patienten ift. videresendelse af data. Det vil også blive undersøgt om brugen af en aktivitetstracker kan motivere patienten til at dyrke mere motion, samt etiske aspekter når det kommer til overvågning af aktivitet og eventuel GPS-position fra lægens side. 

Organisation: 
Undersøgelse af hvor mange patienter, der bliver henvist til sekundær sektor som følge af hypertension, samt hvordan mere fysisk aktivitet kan påvirke dette antal. Det undersøges desuden hvordan udredning af hypertension foregår, samt hvordan brugen af aktivitetsarmbånd vil passe ind i den allerede eksisterende behandlingsmetode. Sidst undersøges også krav til efteruddannelse, indkøb og vejledning ift. brug af aktivitetsarmbåndet.

Økonomi:
For økonomien vil udgifter i forbindelse med sygdommen og videresendelse af patienter blive undersøgt, og herunder vil antallet af nuværende indlæggelser, fraværsdage, førtidspensionister og tidlige dødsfald grundet sygdommen også blive set på i et økonomisk aspekt. Udgifter ved indkøb og vedligehold af aktivitetsarmbåndene vil også blive kigget på, og sidst vil der blive set på de økonomiske fordele ved nedbringelse af fraværsdage, førtidspensionister og for tidligt døde, som følge af forbedret monitorering af patienterne. 