\section{Fremtidigt arbejde}

Jævnfør vores nuværende disposition for MTV'en vil det fremtidige arbejde med de fire MTV-element indeholde følgende: 

\subsubsection{Teknologi}
Her analyseres krav til aktivitetstracker ift. optagelse af de ønskede data, for derefter at vælge et aktivitetsarmbånd til den videre analyse på baggrund af studier vedr. blandt andet præcision og egenskaber. Efterfølgende beskrives, hvilke teknologier det valgte aktivitetsarmbånd består af, samt hvordan disse kan anvendes i forbindelse med aktivitetstracking. Sidst undersøges, hvorvidt der er mulighed for at anvende et allerede eksisterende interface til dataoverførsel fra patient til læge.

\subsubsection{Patient}
Under dette område beskrives, hvilke kriterier patienten skal opfylde for at få udleveret et aktivitetsarmbånd. Her ses også på studier vedrørende brugervenligheden af det valgte armbånd, samt krav til patienten ift. brugen af armbåndet og en evt. videresendelse af data til egen læge. Det vil også blive undersøgt, om brugen af en aktivitetstracker kan motivere patienten til at dyrke mere motion. Til sidst ses der også på etiske aspekter, når det kommer til lægens overvågning af patientens aktivitet og eventuel GPS-position.

\subsubsection{Organisation}
Dette afsnit vil indeholde en undersøgelse af, hvor mange patienter, der bliver henvist til den sekundære sektor som følge af hypertension, samt hvordan forøget fysisk aktivitet kan påvirke dette antal. Det undersøges desuden, hvordan udredning af hypertension foregår, samt hvad en implementering af aktivitetsarmbånd til behandling af hypertension vil have af betydning for både patient og læger. Sidst undersøges også krav til efteruddannelse, indkøb og vejledning ift. brug af aktivitetsarmbåndet.

\subsubsection{Økonomi}
For økonomiafsnittet vil udgifter i forbindelse med sygdommen og videresendelse af patienter blive undersøgt. Herunder vil antallet af nuværende indlæggelser, fraværsdage, førtidspensionister og tidlige dødsfald grundet sygdommen også blive set på i et økonomisk aspekt. Udgifter ved indkøb og vedligehold af aktivitetsarmbåndene vil også blive undersøgt, og sidst findes om der er økonomiske fordele ved nedbringelse af fraværsdage, førtidspensionister og for tidligt døde, som følge af forbedret monitorering af patienterne. 