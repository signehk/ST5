\section{Analyse af problemet}
Fysisk aktivitet defineres i det danske sundhedsvæsen som værende en aktivitet, der forhøjer energiomsætningen - både ved målrettet fysisk træning og ved hverdagsaktiviteter såsom gang eller indkøb \citep{motionsraad2007, terkelsen2015}

Da der findes flere definitioner af fysisk inaktivitet, er der i dette projekt valgt at tage udgangspunkt i Sundhedsstyrelsens og WHO's definition af fysisk inaktivitet som værende mindre end 2,5 times fysisk aktivitet per uge \citep{motionsraad2007}. 

Det er påvist, at mange sygdomsramte personer har gavn af fysisk aktivitet som en behandling eller en metode til at forebygge sygdomsprogression, samt at fysisk inaktivitet kan være en faktor i forbindelse med udviklingen af flere sygdomme \citep{motionsraad2007,pedersen2011}
Af denne grund vælges der at tage udgangspunkt i én sygdom, og fysisk aktivitets påvirkning på netop denne lidelse, som fokusområde i dette projekt.

Hypertension udgør en risikofaktor for følger som apopleksi, myokardieinfarkt, hjerteinsufcciens samt pludselig død, og ifølge nuværende definitioner af hypertension har omkring $20~\%$ af befolkningen denne sygdom \citep{pedersen2011}. Fysisk inaktivitet øger risikoen for hypertension, og motion har en synlig blodtrykssænkende effekt \citep{olsen2015}. Af den grund vælges hypertension som udgangspunktet for projektet og problemanalysen. 

Der er en række sundhedsmæssige risici forbundet med hypertension, idet sygdommen medfører et øget pres på kroppens blodkar, hvilket forøger risikoen for udvikling af arteriesklerose, aneurismer, hjerteanfald og apopleksi. Længerevarende hypertension er af denne grund ofte årsag til kronisk nyresvigt og hjerte-kar-sygdomme \citep{martini2015}. Ifølge Statens Institut for Folkesundhed er omkring 4 \% af alle dødsfald i Danmark relateret til hypertension \citep{juel2006}.

På trods af de sundhedsmæssige risici ved hypertension får 2/3 af de diagnosticerede patienter ikke tilstrækkelig behandling, således at de kan opnå det anbefalede blodtryk; under 140/90 mmHg \citep{martini2015, paulsen2012}. Behandlingen kan foregå enten farmakologisk eller non-farmakologisk, hvor den non-farmakologiske behandling blandt andet består af anbefalinger i forhold til kost- og motionsvaner. Hvis hverken den non-farmakologiske eller 3-stofs antihypertensiv farmakologisk behandling ikke forbedrer patientens blodtryk, vil patienten blive videresendt til en hypertensionsklinik for at udrede sygdommen yderligere \cite{lodberg2016, bech2015}.

Med intentionen om at spare sundhedsvæsenet for penge samt forbedre hypertensive patienters livskvalitet, anses det af ovenstående grunde at være relevant at undersøge, hvorvidt videresendelsen af hypertensive patienter fra almen praksis kan begrænses. Så vidt muligt bør begrænsningen ske gennem forbedringer i behandlingsmetoderne hos den praktiserende læge, ved at skabe større mulighed for monitorering af hverdagsvaner såsom fysisk aktivitet, så behandlingen sker non-farmakologisk.

Monitorering af fysisk aktivitet sker almindeligvis ved en selvudfyldt dokumentation, der giver et indblik i typen af aktivitet, intensitet, hyppighed samt tidsperiode for hver enkelt aktivitet. Dette kan monitoreres ved en aktivitetslog, aktivitetsdagbog eller spørgeskema \citep{adamo2009}. 

Da disse monitoreringsmetoder er subjektive, har patienter en tendens til at overvurdere mængde eller intensitet af deres ydede fysiske aktivitet \citep{adamo2009}. De subjektive metoder er således ikke altid i stand til at repræsentere den reelle fysiske aktivitet, selvom metoderne anses som værende valide \citep{motionsraad2007, pedersen2011}. 

Alternativt findes objektive målemetoder, som kan give et indblik i den ydede fysiske aktivitet over en længere tidsperiode uden at være tidskrævende og dermed til gene for patienten. Disse objektive målemetoder kan være i form af skridttællere, dobbeltmærket vand, pulsmålere og aktivitetsarmbånd \citep{motionsraad2007}.

Til monitorering af aktivitetsniveau hos kronikere i almen praksis vil det være relevant at anvende en eller flere af disse målemetoder, med henblik på at opnå et mere konkret og objektivt indblik i patienternes aktivitetsmønstre. Fordelen ved metoderne er, at der ikke opstår bias som følge af vurdering af egen fysisk aktivitet, mens ulemper involverer blandt andet pris og tilvænning til ny elektronik. 

Da skridttællere og dobbeltmærket vand anvendes til at måle gennemsnitsaktivitet over en tidsperiode fremfor aktivitetsmønstre fra dag til dag fravælges disse teknologier. 
Pulsmåling fravælges ligeledes, da metoden ikke fungerer godt til at skelne mellem inaktivitet og let aktivitet, samt at følelser som eksempelvis forskrækkelse også vil vise udsving i pulsen uden at øge energiomsætningen betragteligt \citep{motionsraad2007}. 

Af disse grunde vælges det at fokusere på armbånd til aktivitetstracking. Disse giver  mulighed for at opnå indsigt i patientens daglige aktivitetsmønster, da de kan bæres døgnet rundt og giver mulighed for synkronisering med blandt andet computere, hvorved dataoverførsel og -analyse kan gøres i hjemmet og ved lægebesøg. Dette leder frem til den valgte problemformulering: 

\begin{center}
\textit{Hvilke effekter vil implementeringen af aktivitetsarmbånd til registrering og objektivisering af fysisk aktivitet hos hypertensive patienter have i den almene praksis?}
\end{center}