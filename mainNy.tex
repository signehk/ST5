

\documentclass[a4paper,11pt,fleqn,dvipsnames,oneside,openright,oldfontcommands]{memoir} 	% Openright aabner kapitler paa hoejresider (openany begge)


%%%%%%%%% Indsat random
%makes it possible to refer to the name of a chapter rather than just the number.
\usepackage{nameref}
\usepackage{pdfpages}
\usepackage{marvosym}
\usepackage{setspace}
\usepackage{graphicx} % For at sætte 2 billeder ved siden af hinanden

%package for writing program code in latex
\usepackage{listings}
%%%%%%%%%%%%%%%%%%%%%%

% ¤¤ Oversaettelse og tegnsaetning ¤¤ %
\usepackage[T1]{fontenc}					% Output-indkodning af tegnsaet (T1)
\usepackage[danish]{babel}					% Dokumentets sprog
\usepackage[utf8]{inputenc}					% Input-indkodning af tegnsaet (UTF8)
\usepackage{ragged2e,anyfontsize}			% Justering af elementer
\usepackage{fixltx2e}						% Retter forskellige fejl i LaTeX-kernen							
				
																							
% ¤¤ Figurer og tabeller (floats) ¤¤ %
\usepackage{graphicx} 						% Haandtering af eksterne billeder (JPG, PNG, EPS, PDF)
%\usepackage{eso-pic}						% Tilfoej billedekommandoer paa hver side
%\usepackage{wrapfig}						% Indsaettelse af figurer omsvoebt af tekst. \begin{wrapfigure}{Placering}{Stoerrelse}
\usepackage{multirow}                		% Fletning af raekker og kolonner (\multicolumn og \multirow)
\usepackage{multicol}         	        	% Muliggoer output i spalter
\usepackage{rotating}						% Rotation af tekst med \begin{sideways}...\end{sideways}
\usepackage{colortbl} 						% Farver i tabeller (fx \columncolor og \rowcolor)
\usepackage{xcolor}							% Definer farver med \definecolor. Se mere: http://en.wikibooks.org/wiki/LaTeX/Colors
\usepackage{flafter}						% Soerger for at floats ikke optraeder i teksten foer deres reference
\let\newfloat\relax 						% Justering mellem float-pakken og memoir
\usepackage{float}							% Muliggoer eksakt placering af floats, f.eks. \begin{figure}[H]
\usepackage{array,booktabs,xcolor,longtable} % kan lave \hdashline i tabellertabe
\usepackage{arydshln}
\usepackage{tabu}

	
	
% ¤¤ Matematik mm. ¤¤
\usepackage{amsmath , amsthm , amsfonts , amssymb, float, stmaryrd} 		% Avancerede matematik-udvidelser
%\usepackage{mathtools}						% Andre matematik- og tegnudvidelser
\usepackage{textcomp}                 		% Symbol-udvidelser (f.eks. promille-tegn med \textperthousand )
\usepackage{rsphrase}						% Kemi-pakke til RS-saetninger, f.eks. \rsphrase{R1}
\usepackage[version=3]{mhchem} 				% Kemi-pakke til flot og let notation af formler, f.eks. \ce{Fe2O3}
\usepackage{siunitx}						% Flot og konsistent praesentation af tal og enheder med \si{enhed} og \SI{tal}{enhed}
\sisetup{output-decimal-marker = {,}}		% Opsaetning af \SI (DE for komma som decimalseparator) 

% ¤¤ Referencer og kilder ¤¤ %
\usepackage[danish]{varioref}				% Muliggoer bl.a. krydshenvisninger med sidetal (\vref)
\usepackage[numbers]{natbib}				% Udvidelse med naturvidenskabelige citationsmodeller
%\usepackage{xr}							% Referencer til eksternt dokument med \externaldocument{<NAVN>}
%\usepackage{glossaries}					% Terminologi- eller symbolliste (se mere i Daleifs Latex-bog)
\usepackage{lastpage}					% Gør det mulig at refere til sidste side 
%\bibliographystyle{bibtex/vancouver} 

% ¤¤ Misc. ¤¤ %
\usepackage{listings}						% Placer kildekode i dokumentet med \begin{lstlisting}...\end{lstlisting}
\usepackage{lipsum}							% Dummy text \lipsum[..]
\usepackage[shortlabels]{enumitem}			% Muliggoer enkelt konfiguration af lister
\usepackage{pdfpages}						% Goer det muligt at inkludere pdf-dokumenter med kommandoen \includepdf[pages={x-y}]{fil.pdf}	
\pdfoptionpdfminorversion=6					% Muliggoer inkludering af pdf dokumenter, af version 1.6 og hoejere
\pretolerance=2500 							% Justering af afstand mellem ord (hoejt tal, mindre orddeling og mere luft mellem ord)


% Kommentarer og rettelser med \fxnote. Med 'final' i stedet for 'draft' udloeser hver note en error i den faerdige rapport.
\usepackage[footnote,draft,danish,silent,nomargin]{fixme}		


%%%% CUSTOM SETTINGS %%%%

% ¤¤ Marginer ¤¤ %
\setlrmarginsandblock{3.0cm}{2.5cm}{*}		% \setlrmarginsandblock{Indbinding}{Kant}{Ratio}
\setulmarginsandblock{2.5cm}{3.0cm}{*}		% \setulmarginsandblock{Top}{Bund}{Ratio}
\checkandfixthelayout 						% Oversaetter vaerdier til brug for andre pakker

%	¤¤ Afsnitsformatering ¤¤ %
\setlength{\parindent}{6mm}           		% Stoerrelse af indryk
\setlength{\parskip}{0mm}          			% Afstand mellem afsnit ved brug af double Enter
\linespread{1,1}							% Linie afstand



% ¤¤ Indholdsfortegnelse ¤¤ %
\setsecnumdepth{subsection}		 			% Dybden af nummerede overkrifter (part/chapter/section/subsection)
\maxsecnumdepth{subsection}					% Dokumentklassens graense for nummereringsdybde
\settocdepth{section} 					% Dybden af indholdsfortegnelsen

% ¤¤ Lister ¤¤ %
\setlist{
  topsep=0pt,								% Vertikal afstand mellem tekst og listen
  itemsep=-1ex,								% Vertikal afstand mellem items
} 

%hyperlinks in the tabel of contents - comment this out before the report is printed.
\usepackage{hyperref}
\hypersetup{
	bookmarks = true,  % Show 'bookmark'-frame in pdf.
	colorlinks = true, % True = colored links, False = framed links.
	citecolor = black,  % Link color for references.
	linkcolor = black,  % Link color in table of contents.
	urlcolor = black,   % Link color for extern URLs.
}

% ¤¤ Opsaetning af figur- og tabeltekst ¤¤ %
\usepackage{caption}
%\usepackage{subcaption}
\captionnamefont{\small\bfseries\itshape}	% Opsaetning af tekstdelen ('Figur' eller 'Tabel')
\captiontitlefont{\small}					% Opsaetning af nummerering
\captiondelim{. }							% Seperator mellem nummerering og figurtekst
\hangcaption								% Venstrejusterer flere-liniers figurtekst under hinanden
%\captionwidth{0.9\textwidth}					% Bredden af figurteksten
\setlength{\belowcaptionskip}{0pt}			% Afstand under figurteksten
\captionsetup[figure]{labelfont={bf,it},font={it}} % sætter nummer til fed og kursis. Resten til fed + skriften er mindre end resten
\captionsetup[table]{labelfont={bf,it},font={it}} 


% ¤¤ Opsaetning af listings ¤¤ %

\definecolor{commentGreen}{RGB}{34,139,24}
\definecolor{stringPurple}{RGB}{208,76,239}

\lstset{language=Matlab,					% Sprog
	basicstyle=\ttfamily\scriptsize,		% Opsaetning af teksten
	keywords={for,if,while,else,elseif,		% Noegleord at fremhaeve
			  end,break,return,case,
			  switch,function},
	keywordstyle=\color{blue},				% Opsaetning af noegleord
	commentstyle=\color{commentGreen},		% Opsaetning af kommentarer
	stringstyle=\color{stringPurple},		% Opsaetning af strenge
	showstringspaces=false,					% Mellemrum i strenge enten vist eller blanke
	numbers=left, numberstyle=\tiny,		% Linjenumre
	extendedchars=true, 					% Tillader specielle karakterer
	columns=flexible,						% Kolonnejustering
	breaklines, breakatwhitespace=true,		% Bryd lange linjer
}

% ¤¤ Navngivning ¤¤ %
\addto\captionsdanish{
	\renewcommand\appendixname{Bilag}
	\renewcommand\contentsname{Indholdsfortegnelse}	
	\renewcommand\appendixpagename{Bilag}
	\renewcommand\appendixtocname{Bilag}
	\renewcommand\cftchaptername{\chaptername~}				% Skriver "Kapitel" foran kapitlerne i indholdsfortegnelsen
	\renewcommand\cftappendixname{\appendixname~}			% Skriver "Appendiks" foran appendiks i indholdsfortegnelsen
}

% ¤¤ Kapiteludssende ¤¤ %
%\definecolor{numbercolor}{gray}{0.7}		% Definerer en farve til brug til kapiteludseende
%\newif\ifchapternonum

\makechapterstyle{AAU}
{
	% Afstand mellem sidehovedet og kapitel+tal+kapitelnavnet defineres til:
	\setlength{\beforechapskip}{0cm}

	% Afstanden mellem kapitelnavnet og body-teksten defineres til:
	\setlength{\afterchapskip}{2cm}

	% Typografiopsætningen til kapitel+tal defineres til:
	\renewcommand\chapnamefont{\sffamily\bfseries\LARGE\raggedright}
	
	% Typografiopsætningen til kapitel+tal defineres til:
	\renewcommand\chaptitlefont{\sffamily\bfseries\huge\color[cmyk]{1.00,0.38,0.00,0.64}}

	% Forårsager, at der til kapitlet også tilføjes dets respektive tal:
	\renewcommand\chapternamenum{}
	\renewcommand\printchapternum
	{
		\makebox[0pt][l]
		{
			\color[cmyk]{1.00,0.38,0.00,0.64}
			\hspace{0.1cm}
			\resizebox{!}{1cm}{\chapnamefont\bfseries\sffamily\thechapter}
		}
	}
	
	% Definitionen af linjenstykket mellem ``Kapitel #'' samt ``kapitelnavnet'':
			\renewcommand\afterchaptertitle{\par\hspace{1.5cm}\hrule height 1pt\vskip\midchapskip}
}

% Aktivering af selve kapitellayoutet med dét navn, som definerer kapitellayoutet (ses fra tidligere):
\chapterstyle{AAU}

%\makechapterstyle{jenor}{					% Definerer kapiteludseende frem til ...
%  \renewcommand\beforechapskip{0pt}
%  \renewcommand\printchaptername{}
%  \renewcommand\printchapternum{}
% % \renewcommand\printchapternonum{\chapternonumtrue}
%  \renewcommand\chaptitlefont{\fontfamily{pbk}\fontseries{db}\fontshape{n}\fontsize{20}{25}\selectfont\raggedright}
%  \renewcommand\chapnumfont{\fontfamily{pbk}\fontseries{m}\fontshape{n}\fontsize{1in}{0in}\selectfont\color{numbercolor}}
% \renewcommand\printchaptertitle[1]{
%    \noindent
%    \ifchapternum
%     \begin{tabularx}{\textwidth}{XI}
%	{\let\\\newline\chaptitlefont ##1\par}     
%    \end{tabularx}
%    \par\vskip-2.5mm\hrule
%    \else
%    \begin{tabularx}{\textwidth}{X}
%      {\parbox[b]{\linewidth}{\chaptitlefont ##1}} & \raisebox{-15pt}{\chapnumfont \thechapter}
%    \end{tabularx}
%    \par\vskip2mm\hrule
%    \fi
%  }
%}											% ... her
%
%\chapterstyle{jenor}						% Valg af kapiteludseende - Google 'memoir chapter styles' for alternativer

% ¤¤ Sidehoved ¤¤ %

\makepagestyle{AAU}							% Definerer sidehoved og sidefod udseende frem til ...
\makepsmarks{AAU}{%
	\createmark{chapter}{left}{shownumber}{}{. \ }
	\createmark{section}{right}{shownumber}{}{. \ }
	\createplainmark{toc}{both}{\contentsname}
	\createplainmark{lof}{both}{\listfigurename}
	\createplainmark{lot}{both}{\listtablename}
	\createplainmark{bib}{both}{\bibname}
	\createplainmark{index}{both}{\indexname}
	\createplainmark{glossary}{both}{\glossaryname}
}
\nouppercaseheads											% Ingen Caps oenskes

\makeoddhead{AAU}{Gruppe 16gr5404}{}{\leftmark}				% Definerer lige siders sidehoved (\makeevenhead{Navn}{Venstre}{Center}{Hoejre})
\makeevenhead{AAU}{\rightmark}{}{Aalborg Universitet}		% Definerer ulige siders sidehoved (\makeoddhead{Navn}{Venstre}{Center}{Hoejre})
\makeevenfoot{AAU}{Side \thepage\ af \pageref{LastPage}}{}{}							% Definerer lige siders sidefod (\makeevenfoot{Navn}{Venstre}{Center}{Hoejre})
\makeoddfoot{AAU}{}{}{Side \thepage\ af \pageref{LastPage}}								% Definerer ulige siders sidefod (\makeoddfoot{Navn}{Venstre}{Center}{Hoejre})
\makeheadrule{AAU}{\textwidth}{0.5pt}						% Tilfoejer en streg under sidehovedets indhold
\makefootrule{AAU}{\textwidth}{0.5pt}{1mm}					% Tilfoejer en streg under sidefodens indhold

\copypagestyle{AAUchap}{AAU}								% Sidehoved for kapitelsider defineres som standardsider, men med blank sidehoved
\makeoddhead{AAUchap}{}{}{}
\makeevenhead{AAUchap}{}{}{}
\makeheadrule{AAUchap}{\textwidth}{0pt}
\aliaspagestyle{chapter}{AAUchap}							% Den ny style vaelges til at gaelde for chapters
															% ... her
															
\pagestyle{AAU}												% Valg af sidehoved og sidefod


%%%% CUSTOM COMMANDS %%%%

% ¤¤ Billede hack ¤¤ %
\newcommand{\figur}[4]{
		\begin{figure}[H] \centering
			\includegraphics[width=#1\textwidth]{billeder/#2}
			\caption{#3}\label{#4}
		\end{figure} 
}

% ¤¤ Specielle tegn ¤¤ %
\newcommand{\decC}{^{\circ}\text{C}}
\newcommand{\dec}{^{\circ}}
\newcommand{\m}{\cdot}


%%%% ORDDELING %%%%

\hyphenation{}

%%%%Fra engelsk til dansk i \autoref{•} %%%%
\renewcommand{\figureautorefname}{figur}
\renewcommand{\sectionautorefname}{afsnit}
\renewcommand{\subsectionautorefname}{afsnit}
\renewcommand{\subsubsectionautorefname}{afsnit}
\renewcommand{\tableautorefname}{tabel}
\renewcommand{\appendixautorefname}{bilag}
\renewcommand{\equationautorefname}{ligning}
\renewcommand{\itemautorefname}{punkt}
\renewcommand{\chapterautorefname}{kapitel}
\input{settings/macros.tex}
\raggedbottom % Soerger for at LaTeX ikke "straekker" teksten

%-------------DEF. ALIAS--------------------
\defcitealias{hasse2012}{'Teknologiforståelse - på skoler og hospitaler'}
\defcitealias{kaewkannate2016}{'A comparison of wearable fitness devices'} 
\defcitealias{mercer2016}{'Acceptance of Commercially Available Wearable Activity Trackers Among Adults Over 50 and With Chronic Illness: A Mixed-Methods Evaluation'}
\defcitealias{rapp2016}{'Personal informatics for everyday life: How users without prior self-tracking experience engage with personal data'}
\defcitealias{pedersen2011}{'Fysisk Aktivitet - Håndbog om forebyggelse og behandling'}
\defcitealias{evenson2015}{'Systematic review of the validity and reliability of consumer-wearable activity trackers'}
\defcitealias{galper2006}{'Inverse Association between Physical Inactivity and Mental Health in Men and Women'}
\defcitealias{munck2007}{'Projektgruppen for hypertensionsaudit ved Anders Munck'}



%-------------Formaliteter-------------------
\begin{document}

\frontmatter	 
\clearpage
\thispagestyle{empty}

\begin{figure}[H]
	\raggedleft
		\includegraphics[width=0.2\textwidth]{figures/aaulogo-da.png}
\end{figure} 
\vspace*{\fill} 
\begin{center}
\begin{Huge}
\textbf{Registrering og objektivisering af fysisk aktivitetsniveau hos kronikere i almen praksis via aktivitetsarmbånd}\\
\vspace{5 mm}
5. semesterprojekt - Efterår $2016$\\
\vspace{3 mm}
\end{Huge}
{\Large Gruppe $16$gr$5404$}
\begin{figure}[H]
	\centering
	\includegraphics[width=0.8\textwidth]{figures/forside}
\end{figure}	
\end{center}
\vspace*{\fill}

\begin{center}
\line(1,0){400}
\end{center}
%\begin{document} 
\thispagestyle{empty}
%\begin{titlepage}
\begin{nopagebreak}
{\samepage 

\begin{tabular}{r}
\parbox{\textwidth}{ 
 {\includegraphics[height=2.5cm]{figures/aaulogo-da.png}}
\hfill \hspace{2cm} \parbox{8cm}
{\begin{tabular}{l} %4.90
{\small \textbf{5. semester}}\\
{\small \textbf{School of Medicine and Health}}\\
%{\small \textbf{\textcolor{MidnightBlue}{}}}\\ 
{\small \textbf{Sundhedsteknologi}}\\
{\small Fredrik Bajers Vej $7$A} \\
{\small $9220$ Aalborg} \\
%{\small \textcolor{NavyBlue}{\emph{http://www.smh.aau.dk/}}}
\end{tabular}}}
\end{tabular}

\hspace{-1.5cm}\begin{tabular}{cc}
\parbox{7cm}{
\begin{description}

\item {Titel:}

Kan man registrere og objektivisere fysisk aktivitetsniveau hos kronikere i almen praksis?\\

\item {Tema:} 

\small{
Klinisk teknologi
}

\end{description}

\parbox{8cm}{

\begin{description}
\item {Projektperiode:}\\
   Efteråret $2016$\\
   $02/09/2016$ - $19/12/2016$\\
   
\item {Projektgruppe:}\\
  $16$gr$5404$\\
  
\item {Medvirkende:}\\
Birgithe Kleemann Rasmussen\\
Mads Kristensen\\
Signe Hejgaard Kristoffersen\\
Simon Bruun\\
Suado Ali Haji Diriyi\\
Toby Steven Waterstone

\hspace{2cm}
\item {Vejledere:}\\
Ole Hejlesen, Morten Sig Ager Jensen \\
og Mads Nibe Stausholm\\
\end{description}

}\\
\begin{description}
\item {Sider: ??} \\
\item {Appendikser: ??}\\
\hspace{1.5cm}
%\item {Afsluttet: $27/05/2016$}
\end{description}
\vfill } &
\parbox{7cm}{
  \vspace{.15cm}
  \hfill 
  \begin{tabular}{l}
  {Synopsis:}\bigskip \\
  \fbox{
    \parbox{8cm}{\bigskip
     {\vfill{\small Denne rapport belyser ideen af anvende et aktivitetsarmbånd til monitorering af hypertensive patienter's aktivitetsniveau. 
Hypertension er en folkesygdom og i Danmark er det vurderet til at 20 \% har sygdommen. Disse patienter anbefales fysisk aktivitet, da dette resulterer i reduktion i blodtryk, hvorpå risikoen for følgesygdomme og medicinering ligeledes reduceres eller udskydes. 
Formålet har således været at undersøge et udvalgt aktivitetsarmbånd, Fitbit Flex, og vurdere om dette kan anvendes i behandlings- og/eller udredningsforløb af hypertension. 
Den registrerede aktivitet vil give læger og andet relevant sundhedsfagligt personale indblik i patienters aktivitetsniveau, hvoraf en korrekt rådgivning kan gives til patienten. 

Vurderingen af Fitbit Flex som registrering- og objektiviseringsmetode er fortaget med udgangspunkt i håndbogen for medicinsk teknologi vurdering.  

\vspace{5cm}

\textbf{Hvad konkluderes der?}
     \bigskip}}
     }}
   \end{tabular}}
\end{tabular}} \hspace{-1.5cm}%\vspace{1.0cm}

\vfill
{\footnotesize\itshape \noindent Offentliggørelse af rapportens indhold, med kildeangivelse, må kun ske efter aftale med forfatterne.}
\\
\end{nopagebreak}
%\end{titlepage}
%\end{document}
\chapter*{Forord}

Denne rapport er skrevet af gruppe 5404, sundhedsteknologistuderende på 5. semester, Aalborg Universitet i projektperioden fra 2 september til 19 december 2016.

Rapporten er skrevet i relation til en medicinsk teknologivurdering med det overordnede tema 'klinisk teknologi'. 

Projektet omhandler teknologien aktivitetsarmbånd som non farmakologisk behandling af hypertensive patienter i den almene praksis, med det formål at monitorere og motivere patienterne til en mere fysisk aktiv hverdag, for at give lægen i den almene praksis et mere objektivt billede af patienternes daglige aktivitetsniveau end mere subjektive metoder vil give, samt at give patienterne mere bevidsthed om deres daglige aktivitetsniveau. 

Teknologien aktivitetsarmbånd bliver analyseret i forhold til de fire aspekter, teknologi, patient, organisation og økonomi, for at kunne vurdere om teknologien vil være et egnet værktøj til behandling af hypertensive patienter.

Der rettes tak til vejler Ole Hejlesen, Morten Sig Ager Jensen, samt Mads Nibe Stausholm for god vejledning og konstruktiv kritik igennem projektperioden. \\\\\\\\\\


\begin{tabular}{lcl}
   \hspace{-1cm} \rule{7cm}{0.5pt} & \hspace{1cm} & \rule{7cm}{0.5pt} \\
   \hspace{-1cm}  Birgithe Klemann Rasmussen&  & Mads Kristensen
\end{tabular}
\vspace{1.5cm}


\begin{tabular}{lcl}
   \hspace{-1cm} \rule{7cm}{0.5pt} & \hspace{1cm} & \rule{7cm}{0.5pt} \\
    \hspace{-1cm} Signe Hejgaard Kristoffersen & & Simon Brunn
\end{tabular}
\vspace{1.5cm}
      

\begin{tabular}{lcl}
    \hspace{-1cm} \rule{7cm}{0.5pt} & \hspace{1cm} & \rule{7cm}{0.5pt} \\
  \hspace{-1cm} Suado Ali Haji Diriyi & & Toby Steven Waterstone
\end{tabular}
\vspace{1.5cm}
    
\newpage
Skriv forord her.... 


\section*{Læsevejledning}
Denne rapport består af et initierende problem, en problemanalyse, en problemformulering, MTV-spørgsmål og -analyser samt en syntese af disse analyser, der gerne skal besvare problemformuleringen. 

Det initiernede problem og problemanalysen belyser og analyserer projektets problemstillinger og leder frem til en problemformulering igennem en problemafgrænsning. MTV-spørgsmålene og -analyserne beskæftiger sig med de fire MTV-elementer; patient, teknologi, organisation og økonomi. Syntesen dækker over en diskussion af MTV-analyserne, en konklusion på problemformuleringen samt en perspektivering til valgte teknologi i projektet. 


\subsection*{Kildeangivelse}
I denne rapport bliver kilder angivet ved Vancouver-metoden, hvor kilden henvises til som et nummer i kantede parenteser. Information omkring kilden findes i litteraturlisten.


%Teknologiafsnittet vil beskrive den valgte teknologi, og hvilke variationer af teknologien der eksistere i dag. En sammenligning af variationerne vil blive fortaget, med henblik på at fremhæve fordele og ulemper. Yderligere vil teknologien også blive sammenlignet med de nuværende løsningsmuligheder der anvendes i dag, for at se hvordan de adskiller sig fra hinanden.  

%Patientafsnittet i MTV’en undersøges den afgrænsede patientgruppe nærmere i forhold til teknologien. Der undersøges blandt andet om teknologien vil have en betydelig påvirkning på patienternes hverdag, og om den kan forbedre deres livskvalitet. Yderligere undersøges om eventuelle etiske problemstillinger forekommer ved anvendelse af teknologien.

%Den organisatoriske analyse vil hovedsageligt behandle ændringer i interaktionen mellem patienter og sundhedspersonale, samt det organisatoriske aspekt i forhold til samarbejdet mellem forskellige sundhedsinstitutioner: primær og sekundær sundhedssektor.

%Det økonomiske aspekt blive undersøgt, med udgangspunkt i at finde frem til omkostningerne relateret til de teknologiske løsninger, som er undersøgt i teknologianalysen. Dette omhandler eventuelle besparelser eller ekstraudgifter, der kan forekomme ved implementering af den nye teknologi.







%\section{Metode} \label{metode}

%Denne medicinske teknologivurdering (MTV) vil afvige fra opbygningen beskrevet i MTV-håndbogen, som følge af projektet samtidig indeholder elementer fra problembaseret læring (PBL). Som et resultat af blandingen, tages der udgangspunkt i en medicinsk problemstilling, som analyseres for at udarbejde en problemformulering. Analysen i forbindelse med PBL-tilgangen vil desuden indeholde MTV-elementer såsom etik, målgruppe og interessentanalyse.

%Efterfølgende vil problemformuleringen skabe grundlag for at arbejde videre med MTV-håndbogens elementer. Her vil de fire områder, teknologi, patient, organisation og økonomi, blive anvendt til at stille mere konkrete spørgsmål. Fremgangsmåden betyder at der vil være tale om en problem- og teknologiorienteret MTV, da der søges at finde en løsning på et medicinsk problem gennem en vurdering af, hvorvidt en ny teknologi vil afhjælpe de problemer, der er ved den nuværende løsningsmetode.

%Teknologiafsnittet vil indeholde en beskrivelse af egenskaberne for den nuværende teknologi, samt en undersøgelse af den alternative behandlingsmetode, der ligger til grunde for MTV'en. Efter den nuværende og den alternative teknologi er beskrevet, vil disse blive sammenholdt, med henblik på at finde fordele og ulemper ved de to løsningsforslag.

%I forbindelse med patientafsnittet i MTV-modellen afgrænses patientgruppen, med henblik på at gøre problemet konkret, hvorved målgruppen for teknologien kan undersøges nærmere. Der undersøges blandt andet hvorvidt teknologien vil have en betydelig påvirkning på patienternes hverdag og om der skal tages højde for etiske problemstillinger.

%Den organisatoriske analyse vil hovedsageligt behandle ændringer i interaktionen mellem patienter og sundhedspersonale, samt det organisatoriske aspekt i forhold til samarbejdet mellem forskellige sundhedsinstitutioner.

%Som et led i MTV'en vil det økonomiske aspekt blive undersøgt med udgangspunkt i, at finde frem til omkostningerne relateret til de teknologiske løsninger, som er undersøgt i teknologianalysen. Her undersøges desuden hvilke besparelser eller ekstraudgifter, der kan forekomme ved implementering af den nye teknologi.

%Analysen af de fire MTV-elementer vil dernæst blive anvendt i syntesen, der indeholder en diskussion med udgangspunkt i fordele og ulemper ved både den nuværende og den undersøgte teknologi. Herigennem vil PBL-metoden også komme til udtryk, i og med syntesen leder frem til en konklusion, som vil besvare den indledende problemformulering. 


\section*{Ordliste}
\begin{description}[leftmargin=!,labelwidth=\widthof{\bfseries The longeeeeeeest label}]
\item [PLO] Praktiserende Lægers Organisation
\item [RLTN] Regionernes Lønnings- og TakstNævn
\item [Applikation] Et program, der anvendes under et operativsystem, som bruges til at løse specifikke opgaver
\item [MEMS] Micro Electro-Mechanical System
\item [Parkinsons sygdom] Kronisk nervesygdom karakteriseret ved langsomme bevægelser, stivhed i musklerne og rysten.
\item [Demens] Symptomer på svigtende hjernefunktion.
\item [Likert skale] Skala til måling af holdninger.
\item [Aktivitetstracker] Apparat der måler aktivitet, men ikke nødvendigvis i form af et armbånd.
\item [Osteoporose] Knogleskørhed
\item [GWB] The General Well-Being Schedule
\item [CES-D] Center for Epidemiologic Studies Depression Scale 
\item [GPS] Global Positioning System
\item [MTV] Medicinsk Teknologi Vurdering
\end{description}

\clearpage

%-----------Indholdsfortegnelse-------------
\phantomsection									
\pdfbookmark[0]{Indholdsfortegnelse}{indhold}	
\tableofcontents*								

\mainmatter

\part{ Problemstilling}

%-------------Indledning-----------------
\chapter{Indledning} \label{sec:indledning}
I Danmark dør 4.500 mennesker årligt i forbindelse med fysisk inaktivitet, svarende til $7-8~\%$ af alle dødsfald \citep{aagaard2014}. Fysisk inaktivitet har konsekvenser for kroppens fysiologiske tilstand og helbred, da det er en risikofaktor for psykiske sygdomme, livsstilssygdomme, såsom type-2 diabetes eller visse hjertekarsygdomme, samt en for tidlig død for blandt andet patienter med type-2 diabetes og hypertension \citep{motionsraad2007}. 

Statens Institut for Folkesundhed har desuden fundet, at fysisk inaktive personer dør 5-6 år tidligere end aktive personer, og manglende aktivitet anses som værende en af de mest betydende faktorer i relation til for tidlig død på verdensplan. Ud over dette, resulterer fysisk inaktivitet nationalt årligt i $100.000$ hospitalsindlæggelser, $3,1$ millioner fraværsdage, $2,6$ millioner kontakter til praktiserende læge og $1.200$ førtidspensioner \citep{christensen2012}.

Fysisk inaktivitet påvirker blandt andet kroppens kredsløb, muskler, knogler og metabolisme, hvilket vil resultere i en reduceret arbejdskapacitet for kroppen og et eventuelt funktionstab \citep{motionsraad2007}.

Aktivitet i dagligdagen er nødvendigt i alle aldersgrupper, og anbefalingerne er specificeret til de enkelte aldersgrupper.
Sundhedsstyrelsen anbefaler, at voksne bør være aktive minimum 30 minutter dagligt med moderat intensitet \citep{pedersen2011}.

Fysisk aktivitet kan anvendes til at forebygge flere sygdomme, og en struktureret fysisk træning kan yderligere benyttes som en del af en behandling eller til at forebygge en eventuel videreudvikling af flere sygdomme \citep{motionsraad2007}. Dette kræver, at der fokuseres på fysisk aktivitet under behandling af patienter.


Afhængigt af sygdommens stadie vil patienten befinde sig i enten den primære eller sekundær sundhedssektor. I den primære sundhedssektor, som er borgernes lokale adgang til sundhedsvæsenet, kan patienterne komme i kontant med praktiserende læger, som kan opstille en diagnose og behandle patienterne for adskillige sygdomme. Hvis der kræves yderligere komplekse undersøgelser eller behandlinger af patient videre sendes vedkommende til den sekundær sundhedssektor, bestående af syghuset. 

I dette projekt fokuseres på almen praksis, som også er en del af det danske sundhedsvæsenet. I almen praksis er samarbejdet mellem læge og patient tæt, hvor hertil lægen kan følge den enkelte patients sociale liv til blandt andet formål at kunne løse og forstå kroniske sygdomme som patienterne døjer med. 
Foruden at give borgerne en lige adgang til lægebesøg er almen praksis også med til at lemme presset på sygehusene ift. antal henvendelser. \citep{vedsted2014}





\section{Initierende problem}
\textit{Hvordan monitoreres/dokumenteres patienters aktivitetsniveau i dagligdagen som led i en sygdomsbehandling?}  


%-------------Problemanalyse----------------
\chapter{Problemanalyse}
\section{Fysisk aktivitet}

Inden for det danske sundhedsvæsen, defineres fysisk aktivitet som værende en aktivitet, der forhøjer energiomsætningen. Dette betyder at alt mellem indkøb og gåture, til målrettet fysisk træning, kan defineres som værende fysisk aktivitet\citep{motionsraad2007, terkelsen2015}.

Som nævnt i \ref{sec:indledning} anbefaler Sundhedsstyrelsen et aktivitetsniveau på mindst 30 minutters motion af moderat intensitet hver dag. I forbindelse med dette, er moderat intensitet defineret som $40-59~\%$ af maksimal iltoptagelse, $64-74~\%$ af makspuls eller aktivitet, hvilket gør patienten lettere forpustet, uden at forhindre muligheden for samtale. 

Definitionen af både fysisk aktivitet og inaktivitet varierer, afhængigt af hvilken sundhedsinstans, der har opstillet definitionen. Center for Disease Control and Prevention (CDC) i USA. har valgt at anbefale mindst $30$ minutters moderat arbejdsintensitet, såsom rask gang eller havearbejde, $5$ dage om ugen, eller 20 minutters aktivitet af høj intensitet 3 dage om ugen. Samtidig definerer CDC forskellige niveauer af fysisk inaktivitet, som går fra utilstrækkelig fysisk aktivitet med mere end $10$ minutters aktivitet om ugen. Under dette niveau er inaktivitet, der defineres som mindre end $10$ minutters aktivitet om ugen \citep{motionsraad2007}.

Sundheds- og sygelighedsundersøgelsen definere fysisk inaktivitet ud fra et enkelt spørgsmål, vedrørende den bedste beskrivelse af patientens fritidsaktiviteter. Her indeholder $3$ af $4$ svar forskellige niveauer af aktivitet, og besvarer patienten spørgsmålet med svaret "Læser, ser på fjernsyn eller har anden stillesiddende beskæftigelse.", kategoriseres patienten som fysisk inakiv\citep{motionsraad2007}.

Både Sundhedsstyrelsen og World Health Organization (WHO) definere fysisk inaktivitet som værende mindre end 2,5 timers fysisk aktivitet om ugen. Af den grund vælges det i projektet, at tage udgangspunkt i Sundhedsstyrelsen og WHO's definition af fysisk inaktivitet, når begrebet omtales senere i projektet\citep{motionsraad2007}.
\section{Effekter af fysisk aktivitet}
Fysisk aktivitet kan påvirke kroppen fysiologisk på mange måder, herunder kan det i forskellige grader forbedre blandt andet immunforsvar, lungefunktion, blodtryk, muskelstyrke- og udholdenhed samt kroppens bevægelighed og vægt. Desuden sker en forbedring af glukosetransportering til muskelcellerne, hvilket medfører, at insulinniveauet er lavere hos folk, der udfører fysisk aktivitet. cite{andersen2001, martini2015}. Dette medfører, at forskellige sygdomme, der relateres til nogen af de nævnte fysiologiske funktioner, kan påvirkes ved fysisk aktivitet.

Flere studier indikerer, at fysisk aktivitet kan have en forebyggende effekt på forskellige folke- og livsstilssygdomme cite{warburton2010}. Nogle af disse folke- og livsstilssygdomme er muskel-og skeletlidelser, stress, samt en række kredsløbssygdomme såsom hjertekarsygdomme, hypertension, overvægt og type-2 diabetes. Foruden disse forebygger fysisk aktivitet nogle kræfttyper, herunder tyktarmskræft og brystkræft. De nævnte lidelser er gældende for alle aldersgrupper, og foruden disse er særlige effekter af fysisk aktivitet gældende for enkelte aldersgrupper. Eksempelvis udskyder eller reducerer den ældre befolkning, der udfører fysisk aktivitet, det aldersrelaterede reducering i funktionsevne, som forventes med alderen. Risikoen for apopleksi og islæmisk hjertesygdom nedsættes som følge af fysisk aktivitet hos ældre \cite{pedersen2011,
warburton2010}. 

Fysisk aktivitet kan ligeledes have en psykisk effekt. Det kan blandt andet forebygge psykiske lidelser som depression, angst og demens cite{pedersen2011}. De psykologiske påvirkninger kan skyldes, at endorfinkoncentrationen i blodet ændres ved fysisk aktivitet, og endorfin virker som kroppens egen produktion af morfinlignende stoffer \cite{kessing2016}. Større overskud, mere selvtillid og overskud samt bedre social trivsel kan ligeledes være en effekt af fysisk aktivitet \cite{sundhedsstyrelsen2006}. 

Mange af de nævnte sygdomme, både de fysiske og psykiske, kan desuden behandles med fysisk aktivitet. Den fysiske aktivitet kan være den primære behandlingsmetode eller det kan være en del af behandlingen, eksempelvis i samspil med medicinsk behandling. Type-2 diabetikere og hypertensive er eksempler på patientgrupper, hvor fysisk aktivitet ofte er en del af behandlingsforløbet, hvor graden af lidelsen har betydning for om fysisk aktivitet og andre livsstilsændringer er den primære behandling eller om det kombineres med medicinsk behandling. Ved behandling af sygdomme skal tages hensyn til, hvilken form for fysisk aktivitet, som udføres af forskellige patientgrupper, da det ellers kan have en skadende effekt. Nogle af disse patientgrupper er eksempelvis artrosepatienter, der specielt skal undgå overbelastning af led, samt gravide, som skal undgå fysisk aktivitet, hvor uventede stød kan forekomme \cite{andersen2001, pedersen2011}.

\section{Fysisk inaktivitet} \label{sec:fys_inaktivitet}

Definitionen af både fysisk aktivitet og inaktivitet afhængiger af, hvilken sundhedsinstans, der har opstillet definitionen. Center for Disease Control and Prevention (CDC) i USA anbefaler mindst $30$ minutters moderat arbejdsintensitet, såsom rask gang eller havearbejde, $5$ dage om ugen, eller $20$ minutters aktivitet af høj intensitet $3$ dage om ugen \citep{motionsraad2007,christensen2012}.
Samtidig definerer CDC forskellige niveauer af fysisk inaktivitet, hvoraf disse er henholdsvis anbefalet fysisk aktivitet, utilstrækkelig fysisk aktivitet, inaktivitet samt inaktivitet i fritiden. Utilstrækkelig fysisk aktivitet svarer til et aktivitetsniveau, der ligger under det anbefalede aktivitetsniveau, hvor der dog udføres mere end $10$ minutters fysisk aktivitet ugentligt. Ved niveuaet inaktivitet udføres der mindre end $10$ minutters ugentligt fysisk aktivitet ved moderat eller høj intensitet \citep{motionsraad2007,christensen2012}.

Sundheds- og sygelighedsundersøgelsen af \citeauthor{christensen2012} definerer fysisk inaktivitet ud fra ét enkelt spørgsmål vedrørende den mest passende beskrivelse af patientens fritidsaktiviteter igennem det sidste år. Svarmulighederne til dette spørgsmål er 'hård træning flere gange om ugen', 'motionsidræt eller tungt arbejde mindst fire timer om ugen', 'lettere motion mindst fire timer om ugen' samt 'stillesiddende aktivitet'. Besvarer patienten spørgsmålet med 'stillesiddende aktivitet', kategoriseres patienten som værende fysisk inaktiv \citep{motionsraad2007,christensen2012}.

Både Sundhedsstyrelsen og World Health Organization (WHO) definerer fysisk inaktivitet, som værende mindre end $2,5$ timers fysisk aktivitet om ugen. Det er valgt at tage udgangspunkt i Sundhedsstyrelsen og WHOs definition af fysisk inaktivitet, når begrebet omtales senere i projektet. Ud fra denne definition anslår Sundhedsstyrelsen, at $30-40~\%$ af den voksne danske befolkning er fysisk inaktive \citep{motionsraad2007}. 
\subsection{Årsager til fysisk inaktivitet}
Fysisk inaktivitet er forårsaget af forskellige faktorer, eksempelvis livsstil og den teknologiske udvikling gennem tiden. Manglende tid, motivation og interesse er dog nogle af de overordnede årsager til fysisk inaktivitet \citep{ottesen2005}.  

\subsubsection{Teknologiske faktorer}  
Siden den industrielle revolution er teknologi et område, der er i konstant udvikling, og anvendes blandt andet som skåneredskab for at aflaste den almene arbejder for fysisk hårdt arbejde, samt invaliditet heraf \citep{hallal2012}. 
Ligeledes har udviklingen ledt til en reduktion i mængden af fysisk aktivitet, der er krævet for at komme igennem hverdagen. Dette betyder blandt andet let adgang til mad og drikkevarer, som ikke kræver stor energiomsætning for at skaffe \citep{motionsraad2007,hallal2012}. Transport foregår ofte med bil eller bus, og teknologier som tv, trådløs kommunikation, internet og lignende bidrager til fysisk inaktivitet \citep{hallal2012}.  

\subsubsection{Kropslige faktorer}
På verdensplan ses det, at fysisk inaktivitet stiger i takt med alderen \citep{guthold2008}. 
Årsagen til dette hos danske ældre er, at de ikke føler det nødvendige overskud til fysisk aktivitet efter hverdagens daglige gøremål. 
Nogle overvægtige oplever frygt og manglende motivation ved fysisk aktivitet, da de forbinder det med ubehag og usikkerhed i, hvad deres krop reelt kan holde til. 
Psykiske forhindringer for fysisk aktivitet fremtræder som flovhed for at vise sig frem i et træningscenter, samt at individer ikke føler, de passer ind med omgivelserne under fysisk aktivitet. 
Dertil forekommer ligeledes manglende motivation og/eller interesse \citep{ottesen2005}.

\subsubsection{Økonomiske faktorer}
Fysisk inaktivitet kan også være forårsaget af økonomiske årsager, hvor eksempelvis betaling for medlemskab af et træningscenter vil sætte en begrænsning for nogle personer. Yderligere kan fysisk aktivitet opfattes som værende for tidskrævende eller besværligt at få plads til i hverdagen \citep{ottesen2005}.
\subsection{Fysiologiske følger af fysisk inaktivitet}\label{sec:FoeglerAfFiA}
Hvis kroppen ikke udsættes for belastning ved fysisk aktivitet, tilpasses den ved at nedgradere nogle biologiske mekanismer og processer. Omvendt forstærkes de ved stimulation. 
Blandt disse biologiske mekanismer og processer kan nævnes kredsløbet, stofskiftet, muskel- og knoglevækst \citep{motionsraad2007}.

De fysiske påvirkninger, forårsaget af inaktivitet, der beskrives i dette afsnit, kan være årsag til flere alvorlige kroniske lidelser, hvis der fortsættes en inaktiv livsstil. 
Fysisk inaktivitet kan eksempelvis føre til osteoporose, hjertekarsygdomme, overvægt samt insulinresistens, der øger risikoen for type-2 diabetes \citep{motionsraad2007}.

\subsubsection{Kredsløb}
Kredsløbet er en af de mekanismer som påvirkes relativt hurtigt ved fysisk inaktivitet. 
Et studie af \citeauthor{Convertino1995}, som foregik over fire uger, har påvist et fald i aerob kapacitet, som angiver den maksimale iltoptagelse i kroppen under fysisk arbejde over tid, med 5-6 \% pr. uge. 
Testpersonerne var både kvinder og mænd i aldersgruppen 18 til 45 år. 
Et fald i aerob kapacitet kan skyldes en reducering af hjertets slagvolumen både i hvile og under arbejde, grundet reducering i kroppens samlede blodvolumen. 
For at kompensere for dette øges pulsen for at opretholde minutvolumen af blod, der pumpes ud i kroppen. 
Et fald i blodvolumen udgør en kortsigtet reducering af aerob kapacitet \citep{Convertino1995}. 
Under tidsperioder med inaktivitet varende længere end ca. 12 uger, kan der yderligere ses en reduceret iltekstraktion i det perifere kredsløb \citep{Coyle1985}.

\subsubsection{Muskelvæv}
Ved fysisk inaktivitet stimuleres musklerne i mindre grad, hvilket fører til tab af muskelmasse grundet, at hastigheden for proteinnedbrydning i musklerne forløber hurtigere end proteinnydannelse, også kaldet proteinsyntese. 
Musklerne bliver derfor mindre, hvilket betegnes muskelatrofi. 
Flere studier påpeger, at der efter én til to ugers inaktivitet, kan ses en reduktion i muskelmasse, og at reduktionen af muskelmasse udelukkende skyldes en reduceret proteinsyntese \citep{Douglas2006, Bloomfield1995}. 
Desuden vil der også opleves et betydeligt tab af muskelkraft hos personer, der er inaktive over længere tid \citep{Bloomfield1995}. 

\subsubsection{Knoglevæv}
Ligesom musklerne, skal knogler og sener stimuleres, for at kunne opretholde deres styrke. 
Hvis ikke vævet stimuleres eksempelvis gennem fysisk aktivitet, som inkluderer en form for stress, vil der begynde at ske en nedbrydning af knoglevævet. 
Allerede efter én uge har et studie af \citeauthor{Bloomfield1995} kunnet observere øget calciumudskillelse i urin og afføring. 
Dog varer det ofte op mod én til to måneder før, der kan detekteres forandringer i knoglernes mineralindhold, da knoglevæv omsættes langsomt \citep{Bloomfield1995}.

\subsubsection{Stofskifte}
Stofskiftet indgår i reguleringen af de kemiske processer i kroppen. 
Hormoner spiller en vigtig rolle inden for stofskiftet, heriblandt hormonet insulin, som er vigtigt for glukoseoptagelse i musklerne og for regulering af glukosekoncentrationen i blodet. 
En inaktiv livsstil vil føre til nedsat insulinfølsomhed og derfor en nedsat evne til at regulere glukosekoncentrationen i blodet. 
Efter en uges inaktivitet kan der ses en reducering i musklernes insulinfølsomhed ifølge  \citeauthor{Mikines1991}. 
Grunden til dette kan skyldes, at der bliver mindre af det glukosetransporterende protein GLUT4 i muskelcellerne. 
Endvidere vil muskelatrofi føre til, at der er mindre muskelvæv, hvori glukosen kan optages \citep{Tabata1999}.
\subsection{Psykologiske følger af fysisk inaktivitet}
Som nævnt i \autoref{sec:effekterafaktivitet} er fysisk inaktivitet en risikofaktor for visse psykiske lidelser. Eksempelvis er det påvist, at forekomsten af depression er lavere blandt fysisk aktive end blandt fysisk inaktive \citep{motionsraad2007}. Ud over depression er der nogen evidens for, at andre psykiske sygdomme såsom angst, misbrug, skizofreni og spiseforstyrrelser kan have gavn af større eller mindre mængde fysisk aktivitet i relation til sygdomsbehandlingen \citep{kessing2016}. Fysisk inaktivitet kan både have en rolle for sygdomsudviklingen samt den videre progredieren af sygdommen, hvor fysisk inaktivitet kan forværre symptomer og patientens generelle tilstand \citep{motionsraad2007,kessing2016}.

\subsubsection{Depression samt følelsesmæssig trivsel}
I studiet \citetalias{galper2006} af \citeauthor{galper2006}, undersøges sammenhængen mellem fysisk inaktivitet og depression samt følelsesmæssig trivsel. Forsøgspersonerne hertil blev delt op i grupper af inaktive, utilstrækkeligt aktive, tilstrækkeligt aktive og meget aktive, og disse grupper blev så vurderet, om de havde depressive symptomer, og om de trivedes følelsesmæssigt. 
Til dette benyttedes en skala, The General Well-Being Schedule (GWB), som dermed forsøger at kvantificere forsøgspersonernes følelsesmæssige trivsel samt en skala, Center for Epidemiologic Studies Depression Scale (CES-D), til at kvantificere depressive symptomer \citep{galper2006}.

\begin{figure}[H]
\centering
\includegraphics[width=0.6\textwidth]{figures/inaktivitet_gwb}
\caption{Fysisk inaktivitet sammenholdt med følelsesmæssig trivsel. På x-aksen fremgår grupperingen i fysisk aktivitetsniveau for henholdsvis mænd og kvinder. På y-aksen fremgår den gennemsnitlige GWB-score, hvilket indikerer følelsesmæssig trivsel på en skala fra $0-110$ \citep{galper2006}.}
\label{fig:inaktivitet_gwb}
\end{figure}

\begin{figure}[H]
\centering
\includegraphics[width=0.6\textwidth]{figures/inaktivitet_dep}
\caption{Fysisk inaktivitet sammenholdt med depressionssymptomer. På x-aksen fremgår grupperingen i fysisk aktivitetsniveau for henholdsvis mænd og kvinder. På y-aksen fremgår den gennemsnitlige CES-D-score, hvilket indikerer depressive symptomer på en skala fra $0-60$ \citep{galper2006}.}
\label{fig:inaktivitet_dep}
\end{figure}

\noindent
Resultater herfra, som fremgår af \autoref{fig:inaktivitet_gwb} og \autoref{fig:inaktivitet_dep}, viser, at fysisk inaktive, især kvinder, har en højere tendens til depressive symptomer, end andre, der er mere fysisk aktive. På samme måde fremgik det af studiet, at fysisk inaktive forsøgspersoner  ikke trives følelsesmæssigt, sammenlignet med dem, der er mere aktive \citep{galper2006}. 

Studiet konkluderer derved, at der er en sammenhæng mellem fysisk inaktivitet og psykiske følger, som eksempelvis depression og forværret følelsesmæssig trivsel \citep{galper2006}. Yderligere er der evidens for, at fysisk inaktivitet forværrer allerede eksisterende depressionstilstande samt dårlig følelsesmæssig trivsel \citep{motionsraad2007}.

\section{Sygdomsafgrænsning}
Som nævnt i \autoref{sec:effekterafaktivitet}, er fysisk inaktivitet forbundet med fysiologiske og psykiske konsekvenser, hvortil det er påvist, at mange sygdomsramte personer har gavn af fysisk aktivitet som en behandling eller en metode til at forebygge sygdomsprogression \cite{motionsraad2007,pedersen2011}. Fysisk aktivitet har effekt ved mange typer sygdomme, som påvirker forskellige aldersgrupper, hvorfor fysisk aktivitet generelt kan siges at være gavnligt \cite{pedersen2011}. Der vælges at tage udgangspunkt i én sygdom og fysisk aktivitets påvirkning på denne lidelse som fokusområde i dette projekt, da dette giver mulighed for at fokusere på en konkret patientgruppe og behandlingen af denne i analysen af de fire MTV-aspekter.

Hypertension udgør en risikofaktor for følger som apopleksi, myokardieinfarkt, hjerteinsufficiens samt pludselig død, og ifølge nuværende definitioner af hypertension har omkring $20~\%$ af befolkningen denne sygdom, hvorfor det kan betegnes som værende en folkesygdom \cite{pedersen2011}. Fysisk inaktivitet øger risikoen for hypertension, og motion har en synlig blodtrykssænkende effekt \cite{olsen2015}. Af den grund vælges hypertension som udgangspunktet for projektet og problemanalysen. 
\section{Hypertension}

Ud af de 20 \% voksne danskere med hypertension, er omkring 30 \% af disse ikke diagnosticeret, idet der ofte ikke er tydelige symptomer på lidelsen. \cite{kronborg2008} 
Der er en række sundhedsmæssige risici forbundet med hypertension. Blandt andet er længerevarende hypertension ofte årsag til kronisk nyresvigt og hjerte-kar-sygdomme. Hypertension medfører et øget pres på kroppens blodkar, hvilket forøger risikoen for udvikling af arteriesklerose, aneurismer, hjerteanfald og apopleksi [2]. Det kan være svært at estimere de nøjagtige tal for dødeligheden som følge af hypertension, idet folk med hypertension ofte dør af følgevirkninger heraf og årsagen til dødsfaldet kan være uklar. Ifølge Statens Institut for Folkesundhed er omkring 4 \% af alle dødsfald i Danmark relateret til hypertension. \cite{juel2006} \textbf{(ikke heeeelt sikker)}
 
På trods af de sundhedsmæssige risici ved hypertension får 2/3 af de diagnosticerede patienter ikke tilstrækkelig behandling, således at de kan komme ned under det anbefalede blodtryk. \cite{paulsen2012}
Blodtryk er karakteriseret ved et systolisk og et diastolisk blodtryk, som henholdsvis er trykket i arterierne, når hjertet trækker sig sammen og mellem to hjerteslag. Blodtryk skrives som ”systole/diastole” og måles i enheden millimeter kviksølv (mmHg). Det anbefales, at blodtrykket ligger under 140/90 mmHg, hvor blodtryk over denne grænse betegnes hypertension. Ligger blodtrykket mellem 120/80 og 139/89 mmHg kaldes dette prehypertension og der bør gøres opmærksom på dette for at undgå hypertension. [2]
I de fleste tilfælde er årsagen til hypertension ukendt, men der er patientgrupper, der har særlig høj risiko for at udvikle hypertension. En lidelse, der ofte forbindes med hypertension, er diabetes. De to lidelser er begge resultatet af metabolisk syndrom, som er forstyrrelser i kroppens metabolisme og er ofte grundet overvægt. \cite{cheung2012} \textbf{(Skal der skrives mere om det?)}
Behandling af hypertension kan ske farmakologisk eller non-farmakologisk. Alle patienter med hypertension bør i non-farmakologisk behandlingen, som består af en række anbefalinger, der bør følges, herunder er eksempelvis motion og kostændringer. Ved farmakologisk behandling tages der højde for graden af hypertension samt hvorvidt der er udviklet følgesygdomme. \cite{lodberg2016}

\section{Hvordan er fysisk aktivitet defineret i medicinsk sammenhæng, og hvad gør lægerne for at monitorere patientens aktivitetsniveau?}

Inden for det danske sygehusvæsen, defineres fysisk aktivitet som værende en aktivitet, der forhøjer energiomsætningen. 
Dette betyder at alt mellem indkøb og gåture, til målrettet fysisk træning, kan defineres som værende fysisk aktivitet.\citep{gupta2013, terkelsen2015}

Sundhedsstyrelsen anbefaler desuden et aktivitetsniveau på mindst 30 minutters motion af moderat intensitet hver dag hele ugen. 
I forbindelse med dette, er moderat densitet blevet defineret som $40-59\%$ af maksimal iltoptagelse, $64-74\%$ af makspuls eller aktivitet, der gør patienten lettere forpustet, uden at forhindre muligheden for samtale. 
For at patienten bliver defineret som værende fysisk inaktiv, kræver det af den grund mindre end $2.5$ timers fysisk aktivitet om ugen.\citep{gupta2013}

I forbindelse med monitorering af aktivitetsniveauet for patienter ved klinikbesøg, kan den fysiske aktivitet bestemmes med udgangspunkt i flere forskellige undersøgelsesmetoder \citep{gupta2013}. 
Måden hvorpå aktiviteten monitoreres, kan opdeles i to kategorier: objektiv og subjektiv \citep{gupta2013, adamo2009}. 

En almindelig subjektiv metode, der anvendes er selvudfyldt dokumentation, der typisk giver et indblik i type af aktivitet, intensitet, hyppighed, samt tidsperiode for ydet aktivitet \citep{adamo2009}. Dertil er der forskellige måder at dokumentere den fysiske aktivitet, som f.eks. en aktivitetslog, aktivitetsdagbog, spørgeskemaer og lignende \citep{adamo2009}. 

Spørgeskemaer tager udgangspunkt i faste spørgsmål omhandlende patientens fysiske aktivitet i løbet af dagligdagen \citep{muller2009}. 
Disse omhandler blandt andet transport til og fra arbejde, motionsvaner, tid brugt foran eksempelvis computer eller TV og ønsker om eventuelle ændringer af patientens aktivitetsvaner \citep{gupta2013, vestergaard2012}. 

Alternativt anvendes aktivitetsdagbøger \citep{muller2009} for at opnå en mere fyldestgørende indsigt i patientens aktivitetsmønster \citep{gupta2013}. 
Dagbogen fungerer som en logbog, hvori den primære aktivitet siden sidste notation, nedskrives med bestemte intervaller. 
Denne monitoreringsmetode giver et bedre indblik i patientens fysiske aktivitet gennem dagen, men er også mere tidskrævende at anvende for især patient men også læge.\citep{gupta2013}



%%%%%%%%%%%%%%%%%%%%%%%%%%%%%%%%%%%%%%%%%%%%%%%%%%%%%%%%%%%%%%%%%%%

Den subjektive metode anvendes på grund af dens lave omkostning, lave patientbyrde, og generelle accept, samtidig med at den er velegnet til dokumentation af diversiteten i forhold til hvilken fysisk aktivitet der er ydet \citep{adamo2009}.


Denne type aktivitetsførelse forbindes dog med en fejlrepræsentation i forhold til den reelle fysiske aktivitet, da de noteret værdier er præget af subjektive holdninger \citep{adamo2009}. Dette påvirker den noteret værdi, i den forstand at patienter har en tendens til enten at over- eller undervurderer den fysiske aktivitet \citep{adamo2009}.
Et studie oplyser at 72 \% af patienter, af alderen 19 eller derunder, overestimerer deres fysiske aktivitet ved selvudfyldelse, i forhold til aktiviteten målt med objektiv/direkte målemetoder (accelerometer, pedometer, og lignende.) \citep{adamo2009}


Problematikken ved denne type undersøgelse, er at patientens svar ikke altid vil være i overensstemmelse med sandheden grundet over- eller underrapportering af aktivitetsniveauet \citep{gupta2013}.

%%%%%%%%%%%%%%%%%%%%%%%%%%%%%%%%%%%%%%%%%%%%%%%%%%%%%%%%%%%%%%%%%%%

Som et led i behandling af kronikere, såsom overvægtige eller diabetespatienter,, udleveres der også skridttællere (accelerometre)\citep{muller2009, jensen2012, snorgaard2010}. 
Accelerometret vil give et mere detaljeret overblik over patientens aktivitetsmønster end spørgeskemaer og dagbøger, grundet muligheden for at monitorere kontinuert gennem længere tid. 
Der opstår dog komplikationer i forbindelse med anvendelsen, som følge af accelerometrets manglende evne til at opfange forskellige aktiviteter. 
Af den grund anvendes det kun til at danne et billede af, hvor meget tid patienten bruger på generel bevægelse. \citep{gupta2013}
\section{Alternative metoder til aktivitetsmåling}
%% Objektiv metode - Overordnet

\subsection{Pedometer}
Skridttællerens primær funktion er at vise antal gået skridt indenfor en bestemt afstand.
Skridttælleren kan hertil også måle antal forbrændte kalorier, totale træningstid og afstanden
som brugeren har gået afhængigt af designet. Skridttæller kaldes også pedometer, og findes i
både mekanisk og elektronisk form. [1]

\subsection{Dobbeltmærket van}
Dobbeltmærket vand anvendes til måling af energiforbruget i en periode på 1 - 2 uger. Metoden er baseret på at der først indtages en mængde vand, som indeholder isotoperne brint (2^H) og ilt (18^O), som afviger fra normalt vand ved at have flere neutroner. Disse isotoper bliver optaget i vævsvæsken og bliver fordelt rundt omkring kroppen. Brintisotopen (2^H) udskilles som vand mens iltisotopen (18^O) både kan udskilles som kuldioxid (CO^2) og vand. CO^2 produktionen kan beregnes ved at trække eliminationen af 2^H fra eliminationen af 18^O. CO^2 produktionen bruges her som et udtryk for energiforbruget. [2] [3] [4] 
En begrænsning ved brugen af dobbeltmærket vand metoden er at den kun viser det gennemsnitlige aktivitetsmønster over en periode på 1-2 uger, i stedet for aktiviteten fra dag til dag [3]. Denne metode kan dog kombineres med den subjektive metode, spørgeskema, som belæg. Derudover kræver metoden tilgængelighed af isotoperne og indsprøjtning af disse for at kunne udfører målingen [3].

\subsection{Pulsmåler}
Pulsmålere bruges til at måle hjertefrekvensen. Der findes forskellige metoder til at detektere
puls, for eksempel måling af den elektriske spændingsforskel under hjertets cyklus. Denne
metode anvender typisk et bælte som patienten har rundt om thorax. [2] Nogle pulsmåler indeholder elektroder, og ved kontakt med hudens overflade vil den elektriske spændingsforskel blive målt. Fordelen ved pulsmålere er blandt andet tilstrækkelig hukommelse med høj tidsopløsning, god sammenhæng mellem pulsfrekvens og arbejdsintensitet ved en moderat intensitet eller højere. [2] En anden målemetode kaldet pulsoximeter måler iltmætningen i blodet for heraf at kunne registrere pulsen. Oximeteret består af lysemitterende dioder som udsender lysstråler gennem vævet og opfanges af en fotodetektor, hvorved iltindholdet i blodet kan bestemmes med udgangspunkt i (noget med lysmængde og iltmætning)[5] Selvom alle aldersgrupper kan anvende pulsmåling skal man være opmærksom på medicinering, da dette kan have virkning på hjertets rytme. Eksempelvis kan betablokkere sænke pulsen. [2] [6]
Selvom pulsmålere giver et godt overblik over pulsfrekvensen ved et moderat eller højere intensitet, indebærer den også en begrænsning, ved registrering af pulsen i forbindelse med inaktivitet ved let aktivitet (Vel ved inaktivitet og let aktivitet?). For at pulsen ikke bliver påvirket af følelsesmæssige ændringer på kroppen, såsom forskrækkelse, hvor energiforbruget vil afvige lidt, bruges flex-puls metoden. Denne metode har først en kalibreringsligning som bruges til at bestemme sammenhængen mellem arbejdsintensitet og puls hos den enkelte person. Ud fra kalibreringsligningen findes hvilepulsen, som kan bruges til at finde en flex-puls dvs. gennemsnittet mellem hvilepulsen og pulsen under letteste arbejde. [3]

\subsection{Aktivitetsarmbånd}
Aktivitetsarmbånd, som også er et elektronisk måleinstrument, bruges af forskellige
målgrupper for eksempel elite atleter/udøvere eller til almindelig dagligdags brug til holde styr på den fysiske aktivitet. Aktivitetsarmbånd er primært en kombination af pulsmåling og skridttæller. [7] Afhængigt af designet af aktivitetsarmbåndet kan yderligere funktioner/egenskaber såsom søvn monitorering, antal forbrændte/indtaget kalorier, sende notifikationer til mobilen, bærebare under vand og alarmering måles. [8] [9]  
Aktivitetsarmbånd kan udover skridttælling også måle fysiske parametre som puls, den tilbagelagte distance, søvn og kalorieindtag samt kalorier forbrændt [2] [7]. Aktivitetsarmbåndet kan blive synkroniseret til andre enheder såsom computer og mobil, på denne måde kan data blive overført, analyseret og sammenlignet over en længere periode. [2]
I forhold til aktivitetsmåler har et studie sammenlignet 8 forskellige aktivitetsmålere med et Oxygon Mobile, som er en bærbar udstyr der måler den metaboliske respons ved udførelse af arbejde. Studiet viste en gennemsnitlig procent fejl i et spændet 9% - 24%. [11] [10]

\subsection{Sammenligning/sammenfatning}
Til monitorering af aktivitetsniveautet hos kronikere i praksis kan nævnte objektive målemetoder anvendes. Det afgørende for valget af målemetode er hensigten med monitoreringen. Hvis patientens energiforbrug ønskes at blive målt dagligt er dobbeltmærket vand ikke optimalt. Derimod kan skridttæller anvendes, da denne som tidligere nævnt kan vise antal forbrændte kalorier. Både skridttæller og aktivitetsarmbånd kan patienten monitorere på håndleddet og kan anvendes under træning eller i hverdagen.Teknologierne kan bruges i forbindelse med selvkontrol af aktivitetsniveau, hvilket vil betyde at patienten har mere ansvar for monitorering af aktivitetsniveauet. Fordelen ved et aktivitetsarmbånd er målingen af pulsen og andre eventuelle ønsket fysiologiske parametre og således udvides spændet for hvilke sygdomme der kan arbejdes på.  Da målgruppen er bred skal aktivitetsarmbåndet tilpasses den enkelte person. En udfordring i aktivitetsarmbånd er hvordan man sørger for patienten beholder armbåndet under hele målingsperioden. Da aktivitetsarmbåndet indeholder nødvendige målingsparametre vurderes dette mest hensigtsmæssigt at arbejde videre på i projektet. 


Indledning
En anden måde at dokumentere fysisk aktivitet på er ved anvendelse af objektive målemetoder. Disse metoder er ikke præget af patienternes egen vurdering af den fysiske aktivitet, men måler derimod mængden af aktivitet direkte. Dette kan give mere præcist måling end subjektive målemetoder som er baseret på patients egne erindringer/oplevelse af den fysiske aktivitetsniveau, hvilket ikke altid er i overensstemmelse med definitionen af fysisk aktivitet.   Disse metoder metoderne kombineres med hinanden eller bruges individuelt afhængigt af analysens formål. Derudover kan metoderne også kombineres med de subjektive metoder til at styrke dokumentationen af den udførte fysiske aktivitet. En stigning af brugen af objektive metoder til måling af fysisk aktivitet ses hos nyere undersøgelser. Denne metode er blevet mere udbredt gennem de seneste år, hvor den fysiske aktivitet måles ved anvendelse af for eksempel dobbeltmærket vand, accelerometre/pedometre, pulsmåler og aktivitetsarmbånd. [2] [3] 



%Et studie af Eduardo Ferriolli et al (Physical Activity Monitoring: A Responsive and Meaningful Patient-Centered Outcome for Surgery, Chemotherapy, or Radiotherapy?) målte 

%[1] http://www.explainthatstuff.com/how-pedometers-work.html 

%[2] 

%[3]http://sundhedsstyrelsen.dk/publ/MER/2007/FYSISK_INAKTIVITET-KONSEKVENSER_OG_SAMMENHAENGE2007.PDF (objektive målemetoder)

%[4] Pulse Oximetry Training Manual (se under mappen litteratur)

%Alt muligt andet pis:

%“Background: With the ever-increasing availability of health information technology (HIT) enabling health consumers to measure, store, and manage their health data (e.g., self-tracking devices) ”
%FYSISK_INAKTIVITET-KONSEKVENSER_OG_SAMMENHAENGE2007.PDF
\section{Teknologiafgrænsning} \label{afgraensning_tek}

Igennem en undersøgelse af hvilke funktioner, der vil være relevante i forbindelse med aktivitetstracking, opstilles krav som udgangspunkt for valget af den endelige teknologi. Kravene til funktion vil blive stillet ud fra den primære aktivitetsform hos patientgruppen, således aktivitetsarmbåndet er optimeret til netop denne aktivitetstype.

I Danmark stiger prævalensen af hypertension med alderen. Det ses blandt andet, at der kun er $1~\%$ af de $20-29$ årige, som lider af hypertension, mens omkring $69~\%$ af de $80-89$ årige har sygdommen \citep{olsen2015}. Som følge af den forøgede risiko for hypertension i sammenhæng med alderen, ses der på den primære anbefalede fysiske aktivitet for ældre over $65$ år, hvilket er $30$ minutters aktivitet med moderat intensitet om dagen og mindst $2$ gange $20$ minutters muskelstyrkende eller konditionsforøgende aktivitet om ugen \citep{pedersen2011}.

Hos ældre anses gang over $6$ km/t som konditionsforøgende aktivitet, og gang med $4-5$ km/t som moderat aktivitet. Med udgangspunkt i foregående, samt Sundhedsstyrelsens anbefalinger i \citetalias{pedersen2011} tages udgangspunkt i gangregistrering med mulighed for udvidelse til svømning og cykling \citep{pedersen2011}.

\subsection{Krav til funktionalitet}

Såfremt aktivitetsarmbåndet skal anvendes i hverdagen, er det vigtigt, at det er kompakt og bærbar, samt at det ikke har behov for opladning på daglig basis. Som følge af at den primære aktivitet for patientgruppen er gang, er det vigtigt, at enheden kan måle dette præcist, således målingerne kan anvendes som valide data. 

Da Sundhedsstyrelsen også anbefaler svømning og cykling, hvis patienten har mulighed for dette, vil det være relevant, men ikke påkrævet, at aktivitetsarmbåndet har mulighed for at måle denne type aktivitet. Registrering af disse aktiviteter kræver både vandtæthed og GPS eller mulighed for kommunikation med en ekstern cykelcomputer på patientens cykel.

\subsection{Valg af aktivitetsarmbånd}

For at finde den mest optimale aktivitetsarmbånd til formålet, tages der udgangspunkt i studier, som har undersøgt præcisionen af forskellige aktivitetsarmbånd ved blandt andet antal skridt, energiforbrug og afstand. Ud over dette er brugerfladen også bedømt, hvorfor dette også er relevant at tage med i overvejelserne. Det er valgt at fokusere på Fitbit Flex, da dette indeholder den nødvendige teknologi til tracking, samt at muligheden for reproducering af målinger er høj \citep{kaewkannate2016}. Oven i dette udkom Fitbit Flex $2$ i $2016$, og den nye version giver mulighed for tracking af svømning, hvilket er væsentligt for patiengruppen, men som følge af mangel på studier vedrørende repeterbarheden og præcisionen af den nye model, fokuseres på den gamle model \citep{fitbitflex}.

Yderligere fordele ved Fitbit Flex inkluderer muligheden for at gemme data i op til $30$ dage, muligheden for at sammenligne med andres aktivitet, vandtæthed og kompatibilitet med fitness-apps på smartphones og computer \citep{kaewkannate2016, fitbitflex}. Især de sociale egenskaber ved Fitbit's armbånd, samt muligheden for at tracke aktiviteten med apps, kan give anledning til øget aktivitetsniveau hos patienterne \citep{karapanos2016, rooksby2014}. Fitbit Flex er dog ikke udstyret med GPS, men hvis funktionen er nødvendig for tracking af bestemte aktiviteter, har patienten mulighed for at kombinere GPS-data fra eksempelvis smartphones med Fitbit's data \citep{fitbitflex}.
\section{Problemformulering}

Her vil være en opsummering af problemanalysen, for at tydeliggøre hvorfor problemformuleringen lyder som følgende:

\textit{Hvilke effekter vil implementeringen af aktivitetsarmbånd til registrering og objektivisering af fysisk aktivitet hos hypertensive patienter have i den almene praksis?}

%-------------METODE-----------------
\part{ Metode}
\chapter{Rapportens struktur} \label{metode}
Denne rapport tager udgangspunkt i metoden for en MTV, hvor en medicinsk problemstilling analyseres \citep{mtvhaandbog}. Yderligere er rapporten udarbejdet som et semesterprojekt på Aalborg Universitet, hvorfor den også tager udgangspunkt i problembaseret læring, hvor der opstilles et initierende problem, laves en problemanalyse og en problemformulering, der forsøges at besvare. 

\begin{figure}[H]
	\centering
	\includegraphics[width=0.5\textwidth]{figures/metodemodel}
	\caption{Model for den brugte metode i projektet.}
	\label{fig:metodemodel}
\end{figure}

\noindent
Som illustreret på \autoref{fig:metodemodel}, starter projektet bredt med et initierende problem, som analyseres og afgrænses i en problemformulering. Denne problemformulering forsøges besvaret gennem en MTV. 

En MTV er en vurdering baseret på forskning og kan herved anvendes som et evidensbaseret grundlag, når der skal tages beslutninger om, hvorvidt nye teknologier bør anvendes i sundhedsvæsenet. Målgruppen for en MTV er beslutningstagere, såsom politikere, ledelser på sygehuse og organisationer. De fire hovedområder indenfor MTVen; teknologi, patient, organisation og økonomi, kræver forskellige metoder, videnskabelige teorier, forskingstilgange med mere, og på baggrund af dette er der ofte fagfolk fra relevante områder involveret i udarbejdelsen af en MTV \citep{mtvhaandbog}. 

MTV-analyserne belyser forskellige aspekter af teknologien ved at inddele MTV'en i fire områder; teknologi, patient, organisation og økonomi. 

Hvert aspekt har et tilhørende metodeafsnit til at beskrive, hvilke analysemetoder og MTV-spørgsmål, der anvendes, og som er relevante for at kunne besvare den opstillede problemformulering. Informationen er primært fundet gennem systematiske informationssøgninger og er tilpasset, med henblik på besvarelse af MTV-spørgsmålene i de fire aspekter. Metoden for informationssøgning beskrives yderligere i \autoref{sec:metode_soeg}. Ud over en systematisk informationssøgning indsamles information ud fra egne erfaringer om teknologien, hvilke kan være relevante for besvarelsen af visse MTV-aspekter. Disse erfaringer ses især relevante for de områder, hvor den eksisterende viden ikke har været tilstrækkelig for at kunne besvare de fokuserede spørgsmål under MTV-analyserne.

I syntesen vil de fire MTV-områder blive diskuteret, og der vil være en samlet konklusion på problemformuleringen ud fra delkonklusionerne i de fire MTV-analyser. 


Idet denne MTV-inspirerede rapport er udarbejdet af en sundhedsteknologi projektgruppe på 5. semester, anses det, at rapporten kan anvendes som vidensgrundlag til en beslutningstagen eller til videre undersøgelser. 


\chapter{MTV-analyse}
Følgende kapitel beskriver de anvendte metoder i henholdsvis teknologi-, patient-, organisations- og økonomianalysen. Endvidere vil MTV-spørgsmålene til hver analyse fremgå. 

\section{Teknologi}\label{sec:metode_tek}
Teknologiafsnittet er skrevet ud fra MTV-spørgsmål, som vil beskrive teknologien og redegøre for og vurdere, hvilke teknologiske krav, Fitbit Flex skal opfylde for at kunne benyttes til at måle aktivitetsniveau hos hypertensive patienter. 
Foruden en tilpasset litteratursøgning, foretages beskrivelse af teknologien ud fra erfaringer ved anvendelse af software, der er relateret til teknologien \citep{mtvhaandbog}. Dette giver anledning til følgende MTV-spørgsmål: 
\subsection{MTV-spørgsmål}
\begin{itemize}
\item Hvordan fungerer Fitbit Flex, og hvordan kan dette anvendes, således at en alment praktiserende læge får dokumenteret patientens aktivitetsniveau?
\item Repræsenterer Fitbit Flex den fysiske aktivitet tilstrækkeligt, til at data kan anvendes af praktiserende læger til vurdering af patientens fysiske aktivitetsniveau?
\end{itemize}

\section{Patient}\label{sec:metode_pat}
Til analyse af patientaspektet, og hvordan teknologien påvirker denne, analyseres sociale, kommunikative, økonomiske, individuelle og etiske forhold, samt samspillet mellem disse. Dette gøres ud fra metoden for en MTV \citep{mtvhaandbog}.

\begin{figure}[H]
\centering
\includegraphics[width=0.8\textwidth]{figures/patientaspekter}
\caption{Fokusområder for patientanalysen. Cirklerne repræsenterer henholdsvis de sociale, økonomiske, etiske, individuelle og kommunikative forhold, og overlappene illustrerer samspillet derimellem \citep{mtvhaandbog}.}
\label{fig:patientaspekter}
\end{figure}

\noindent
Figur \ref{fig:patientaspekter} viser de forskellige forhold for patientaspektet, der tages højde for i analysen \citep{mtvhaandbog}. I forhold til Fitbit Flex fokuseres der i denne analyse på sociale forhold, herunder hvordan denne teknologi påvirker patientens arbejds- og uddannelsesliv, familie og livskvalitet, individuelle forhold, herunder hvordan patienten oplever teknologien, kommunikative forhold, samt etiske forhold, herunder risiko for misbrug af personlige data. Dette giver anledning til følgende MTV-spørgsmål: 

\subsection{MTV-spørgsmål}
\begin{itemize}
\item Hvilke kriterier skal være opfyldt for, at patienten kan få udleveret Fitbit Flex?
\item Er teknologien brugervenlig og motiverer den patienten til at få en mere aktiv hverdag?
\item Hvordan påvirker teknologien patienternes individuelle og sociale forhold i dagligdagen?
\item Hvor stor en andel af patienter oplever en positiv virkning ved anvendelse af teknologien, hvad er tidshorisonten for denne virkning, og hvad spiller en rolle for, at teknologien giver et succesfuldt forløb?
\item Hvilke etiske dilemmaer opstår ved at monitorere patientens fysiske aktivitet?
\end{itemize} 

\section{Organisation}\label{sec:metode_org}
Det ønskes at undersøge de organisatoriske forudsætninger samt mulige konsekvenser ved implementering af Fitbit Flex til monitorering af fysisk aktivitet i almen praksis. Undersøgelsen tager udgangspunkt i den modificerede Leavitt organisationsmodel der ses af \autoref{fig:leavittmodel}, for at analysere konsekvenserne af en eventuel ændring i organisationen \citep{mtvhaandbog}. 

\begin{figure}[H]
\centering
\includegraphics[width=0.9\textwidth]{figures/leavitt}
\caption{Leavitts modificerede organisationsmodel. Pilen mellem de forskellige områder betegner sammenspillet mellem dem. Dertil står omgivelser uden for de andre fire områder, da dette betegner, hvem der har interesse for de organisatoriske ændringer, der vil forekomme \citep{mtvhaandbog}.}
\label{fig:leavittmodel}
\end{figure}

\noindent
Leavitts modificerede organisationsmodel benyttes, da denne tager højde for omgivelsernes påvirkning på teknologi, aktører, opgaver, struktur, disses indbyrdes påvirkning og påvirkning på omgivelserne. 
Teknologi omhandler arbejdsprocesser, procedurer og rutiner, i relation til teknologien.  
Aktører er de ansatte i organisationen, og deres holdninger og ekspertise i relation til organisationens opgaveløsninger. 
Opgaver dækker over karakteren af de opgaver, som organisationen forsøger at løse. 
Struktur omhandler formelle mønstre i organisationen, som arbejdsdeling og formalisering.  
Omgivelser er udvalgte interessenter, der er relevante i forhold til de organisatoriske ændringer \citep{mtvhaandbog}. Dette giver anledning til følgende MTV-spørgsmål:

\subsection{MTV-spørgsmål}
\begin{itemize}
\item Hvordan passer Fitbit Flex i almen praksis? 
\item Hvilke krav vil implementeringen stille til alment praktiserende læger, og hvem skal stå for en eventuel efteruddannelse? 
\item  Hvordan vil patientfordelingen mellem den primære og sekundære sundhedssektor blive påvirket, og hvad vil en ændring i arbejdsfordelingen medføre?
\end{itemize}

\section{Økonomi}\label{sec:metode_oeko}
I økonomianalysen undersøges mulige omkostninger ved implementering og anvendelse af Fitbit Flex som monitoreringsenhed for fysisk aktivitet til brug i almen praksis.
Ligeledes undersøges omkostninger for den nuværende monitoreringsmetode, samt hvilke økonomiske konsekvenser, der forekommer, når patienten ikke dyrker den anbefalede mængde motion.
Det ønskes at foretage cost-effectiveness og cost-utility analyser for at afgøre, om Fitbit Flex skal implementeres. I en cost-effectiveness analyse opgøres omkostninger og konsekvenser ved den nuværende monitoreringsmetode og Fitbit Flex for at afgøre, hvilken teknologi der er mest omkostningseffektiv i forhold til både valuta og antal vundne leveår. En cost-utility analyse benyttes til at tage højde for kvalitetsjusterede leveår (QALY), hvor de vundne leveår kvalitetsjusteres med den helbredsrelaterede livskvalitet \citep{mtvhaandbog}.
I denne økonomianalyse vil blive fremhævet mulige sundhedsmæssige resultater som følge af implementeringen af Fitbit Flex, i forhold til udgifterne dertil, uden at foretage egentlige cost-effectiveness eller cost-utility analyser, eftersom de benyttede værdier i analysen er estimerede.  
Disse estimerede værdier er baseret på udregninger ud fra funden litteratur omkring sundhedsøkonomi. Dette giver anledning til følgende MTV-spørgsmål: 

\subsection{MTV-spørgsmål}
 
\begin{itemize}
\item Hvad er omkostningerne ved nuværende monitoreringsmetode, samt konsekvenserne ved utilstrækkelig aktivitetsydelse? 
\item Hvilke omkostninger er forbundet med brug af Fitbit Flex til patienter med hypertension, og hvad er den økonomiske konsekvens af dette, hvis brug af aktivitetsarmbånd resulterer i et øget antal kvalitetsjusterede leveår?
\end{itemize}


%  Søgeprotokollen findes i \autoref{app:soegeprotokol}.
\chapter{Søgestrategi}

MTV’en vil primært blive dokumenteret ved brug af videnskabelig litteratur fundet fra forskellige videnskabelige databaser. For at overskueliggøre dette vil der sideløbende med MTV’ens udformning blive udarbejdet en søgeprotokol. Heri lægges fokus på blandt andet være inklusions- og eksklusionskriterier for at kunne fokusere søgningen til det mest relevante litteratur i forhold til de fire områder i MTV’en. Formålet med søgeprotokollen er dels at få et overblik over de kilder, der anvendes, og for at kunne indskrænke søgningerne tilstrækkeligt, så den anvendte litteratur er relevant i forhold til den opstillede problemformulering. 

%-------------MTV--------------------
\part{ MTV-analyse}
%-------------Teknoglogi-------------
\chapter{Teknologi}
Dette kapitel har fokus på det teknologiske element, hvor teknologien vil blive karakteriseret, analyseret og vurderet.










% Indhold i afsnittet\\
%\section{Besvarelse}
Indhold: Dette afsnit vil indeholde forskellige underemner, der vil beskæftige sig med forskellige aspekter, så MTV-spørgsmålene kan besvares. 

\section{Teknologibeskrivelse}
Aktivitetsarmbånd bliver i stigende grad mere udbredt. Ifølge IDC er der sket en stigning i salget af aktivitetsarmbånd fra 11,8 millioner enheder i første kvartal af 2015 til 19,7 millioner i første kvartal af 2016 \citep{IDC2016}.

Det har ikke været muligt at finde statistisk data vedrørende udbredelsen af Fitbit Flex armbåndet, dog ses at Fitbit udgør stor andel af markedet for aktivitetsarmbånd, og at der fra første kvartal i 2015 til første kvartal i 2016 er sket en stigning i salget på 1 million enheder \citep{IDC2016}.  
Fitit Felx armbåndet, som det vil båret af brugeren, ses af \autoref{fig:fitbitflexarmbånd}. 

\begin{figure}[H]
	\centering
	\includegraphics[width=0.35\textwidth]{figures/fitbitflex}
	\caption{Fitbit flex armbånd \citep{fitbitflex}.}
	\label{fig:fitbitflexarmbånd}
\end{figure}

Overordnet består et Fitbit Flex aktivitetsarmbånd af en flex tracker, oplader kabel, trådløs synkroniserings dongle og armbånd til flex tracker \citep{fitbitflex}. Disse kan ligeledes ses af \autoref{fig:fitbitflexindhold}. 

\begin{figure}[H]
	\centering
	\includegraphics[width=0.6\textwidth]{figures/fitbitflexindhold}
	\caption{Fra venstre mod højre ses flex tracker, oplader kabel, trådløs synkroniserings dongle, armbånd \citep{fitbitflex}.}
	\label{fig:fitbitflexindhold}
\end{figure}

Fitbit Flex er i stand til at måle antal skridt, forbrændte kalorier, afstand dækket, minutter brugeren er aktiv og længden samt kvalitet af søvn. 
For at brugeren kan se den registrerede aktivitet, som er blevet opsamlet af armbåndet, skal dette synkroniseres med en kompatibel enhed, da armbåndet kun besidder et display bestående af fem LED'er. 
Synkronisering foregår trådløst, ved brug af bluetooth low energy og kan foregå mellem forskellige enheder som for eksempel smartphone og computer. 
Synkronisering mellem flex tracker og computer kræver dog anvendelse af den trådløs synkroniserings dongle, der ses af \autoref{fig:fitbitflexindhold}.
Forudsætninger for, at data kan synkroniseres er, at en kompatibel enhed har den korrekte applikation installeret, hvor synkroniseringen ellers sker automatisk idet applikationen åbnes.  
Yderligere skal der oprettes en brugerkonto på www.fitbit.com, hvor brugeren oplyser personlig info: køn, alder, højde og vægt. Dette er nødvendigt i forhold til optimering af dataopsamling og estimering af forbrændte kalorier.  
Gennem applikationen visualiseres den registrerede aktivitet, hvor brugeren har mulighed for at se data fra starttidspunktet for anvendelsen af armbåndet. Data kan også observeres via  Fitbits hjemmeside, hvor det er muligt at logge ind via brugerkontoen. 
Således ville alle i besiddelse af brugerkontoen have adgang til den synkroniserede data, uden fysisk at havde hverken bruger eller armbånd til rådighed. 
Til den daglige aktivitet har brugeren mulighed for at sætte bestemte mål til den fysiske aktivitet. Alt efter hvilke mål brugeren sætter for sig selv, kan progressionen ses ud fra de fem LED'er på armbåndet, ved at brugeren trykker to gange på armbåndet.   
Når et af brugerens mål gennemføres, visualiseres dette ved at de 5 LED'er blinker og at armbåndet vibrerer. 
Fitbit Flex armbåndet er ikke i stand til at visualisere batteriniveauet for armbåndet, dette kan dog ses ved brug af applikationen. 
Hukommelsen i flex trackeren tillader detaljeret data at blive lagret i perioder op til 7 dage og består af minut til minut målinger.  
Yderligere lagres summeringer af daglig aktivitet i op til 30 dage. 
Ved jævnlig synkronisering er det muligt for brugeren at bevare detaljeret data, da informationen tilknyttes brugerkontoen. 
Fitbit anbefaler én daglig synkronisering, dog er det ikke en nødvendighed \citep{fitbitflex}. 

\subsection{Hardware}
Fitbit flex trackeren har forskellige hardware elementer, hvorfra trackeren signalere, og detektere fysisk aktivitet. Hardwaren i trackeren udgøres af et display, sensor, motorer og batteri.
 
\textbf{Display:} 
Flex trackeren er udstyret med fem LED'er, der ved forskellige operationstilstande signalerer til brugeren. 
LED'erne fungerer for eksempel, som indikator for progressionen i forhold til det brugerdefinerede fysiske mål for dagen. Hertil vil hver LED repræsentere en procentvis progressionen i intervaller af $20 \%$. Eksempelvis hvis brugeren har opfyldt $73 \%$ af det fysisk mål, vil de første tre LED'er lyse og den fjerde vil blinke. Dette indikerer, at brugeren har nået $60 \%$ af målet, og at brugeren nu befinder sig mellem $60 \%$ og $80 \%$. 
Det samme gør sig gældende når flex trackeren sættes til opladning. Her indikerer LED'erne, hvor langt armbåndet er fra fuld opladning, som signaleres ved at alle fem LED'er lyser. 
I tilfælde af synkroniseringsfejl vil dette også fremgå af LED'erne. Her vil armbåndet lyse med et mønster, skiftevis mellem at have ingen eller alle LED'er tændt. 
Ved manuel aktivering og de-aktivering af sleep mode, vil LED'erne indikere dette gennem forskellige indikationsmønstre.

\textbf{Sensor:} 
Flex trackeren registrer den fysiske aktivitet ved anvendelse af et MEMS 3-akses accelerometer, hvilket er den eneste sensor, som er implementeret i armbåndet. Ud fra algoritmer analyseres bevægelsesmønstre, hvorved der kan oplyses hvor mange skridt der er foretaget under løb eller gang, den tilbagelagte afstand, med mere. 

\textbf{Motorer:}
Flex trackeren er yderligere udstyret med en vibrationsmotor, der aktiveres under forskellige funktioner når armbåndet anvendes. Disse fungerer i sammenspil med displayet, som et kommunikationsredskab for brugeren. Vibration aktiveres ved anvendelse af alarm funktion og ved aktivering eller de-aktivering af sleep mode, samt når det daglige fysiske mål nåes. 
 
\textbf{Batteri:} 
Fitbit Flex indeholder et genopladeligt batteri, der lades ved brug af det medfølgende kabel. Dette ses af \autoref{fig:fitbitflexindhold}. Kablet tilsluttes en computer og opladningen begynder, hvis computeren er tændt. 
Levetiden på batteriet er op til 5 dage, dog kan mindre forventes ved omstændigt brug.


\subsection{Software}
Applikationen er brugerfladen hvorfor den synkroniserede data formidles til brugeren. Her oplyses skridt, forbrændte kalorier med mere. 
Alt efter brugerens engagement, kan der også udfyldes informationer omkring indtaget kost ved brug af applikationen. Brugeren kan ud fra dette få et estimat af hvor mange kalorier der indtages, hvortil dette kan sammenlignes med antal kalorier forbrændt. Anvendelsen af denne kost-log er dog ikke en nødvendighed for anvendelsen af armbåndet eller applikationen, dog kunne dette give en praktiserende læge indblik i om patienten overholder anbefalingerne for hypertensive patienter, både i forhold til kostvaner, samt fysisk aktivitet.  

\begin{figure}[H]
	\centering
	\includegraphics[width=0.45\textwidth]{figures/burgerfladeoversigt}
	\caption{Ooverordnet oversigt af fysisk aktivitet, der vises idet applikationen åbnes. Her vises antal skridt taget, afstand rejst, kalorier forbrændt med mere.}
	\label{fig:brugerfladeoversigt}
\end{figure}

Af \autoref{fig:brugerfladeoversigt} ses oversigt over den registrerede aktivitet som er målt gennem armbåndet. Her ses skridt taget, afstand rejst og kalorier forbrændt. Af oversigten ses også hvor langt brugeren er fra at opfylde de forskellige aktivitetsmål, og er repræsenteret af den blå cirkel omkring de forskellige angivelser. 
Af bunden ses fire forskellige oversigter, hvor der fra venstre mod højre ses dashboard, challenges, friends og account. Plus tegnet i midten fungerer som en genvej til forskellige funktioner under de fire oversigte. 
\textbf{Dashboardet} er den overordnede oversigt, som oplyser det førnævnte (ydet aktivitet). 
\textbf{Challenges} viser en oversigt over tilvalgte aktivitetsudfordringer, hvor brugeren har mulighed for at opstille udfordringer med venner samt andre brugere af applikationen. 
\textbf{Friends} giver brugeren et overblik og venner der er tilføjet til applikationen. 
\textbf{Account} viser overblik over hvilken bruger der er logget ind, og hvilken fitibt enhed der er tilsluttet applikationen. Yderligere kan der fortages ændringer af profil og mål for daglig fysisk aktivitet. 

En detaljeret oversigt over ydet aktivitet kan ses under den overordnede oversigt, ved at tykke på de specifikke målinger. Ved at trykke på steps ses eksemplet der fremgår af \autoref{fig:brugerfladesteps}.  

\begin{figure}[H]
	\centering
	\includegraphics[width=0.45\textwidth]{figures/brugerfladesteps}
	\caption{Oversigt over skridt taget for forhenværende dage, som graf (øverst) og tabel (nederst).}
	\label{fig:brugerfladesteps}
\end{figure}

Af \autoref{fig:brugerfladesteps} ses der foroven en graf over skridt taget inden for den sidste uge. Af grafen ses en hvid tværgående linje, der repræsenterer målet for antal skridt for dagen. Hertil ses at dage hvor målet er blevet opfyldt markeres med en stjerne.  

Under grafen ses en oversigt over antal skridt taget for de forhenværende dage, rækkende tilbage til den første anvendelsesdato. Heraf ses ligeledes at dagene hvor målet nåes, er indikeret med en stjerne. 

Ved at trykke på en den givne dag eller en af de forhenværende dage, kan der ses en mere detaljeret oversigt over skridt taget i løbet af den pågældende dag. Dette ses af \autoref{fig:specifiksteps}, hvor det er muligt at se på hvilke tider af dagen brugeren er mest aktiv.  

\begin{figure}[H]
	\centering
	\includegraphics[width=0.45\textwidth]{figures/specifiksteps}
	\caption{Til venstre ses den grafiske oversigt over antal skridt taget, og højre figur viser skridt taget i en tabel.}
	\label{fig:specifiksteps}
\end{figure}


\subsection{Brugertilpasning}
Der er forskellige muligheder for at tilpasse armbåndet optimalt til den givne bruger. Heriblandt er der mulighed og at udskifte armbåndet til andre længder, og at tilpasse skridtlængden til den enkelte bruger. Foruden dette tilegner armbåndet sig og at bliver brugt under forskellige vejrforhold, da Fitbit Flex er vandafvisende. 

\subsubsection{Forskellige størrelser armbånd}
For brugeren er der mulighed for at vælge mellem to forskellige længder af armbånd. Dette tillader muligheden for bedre tilpasning omkring håndleddet. Armbåndene kan ligeledes fås i forskellige farver. % Swag 

\subsubsection{Kalibrering}
Som standard vurderer applikationen brugerens skridtlængde, ud fra de angivne oplysninger ved oprettelsen af brugerkontoen. Brugeren har dog mulighed for at kalibrerer denne værdi, i tilfælde af at brugeren opdager uoverensstemmelse mellem registrerede værdier og reelle værdier. Brugeren kan under indstillinger i applikationen ændre den pre-defineret skridtlængde, til en mere passende. Fitbit oplyser på deres support-hjemmeside guidelines for hvordan brugeren selv udregner værdier til en mere passende skridtlængde.   

\begin{comment}
Hvad består teknologien af?
Hvor udbredt er teknologien?
Hvordan tilpasses teknologien den enkelte person?
Levetid for teknologien?
Hvilke muligheder er der for lagring og videregivelse af information til en læge?
\end{comment}

\subsection{Fordele og begrænsninger ved teknologien}
Sammenlignes anvendelse af aktivitetsarmbåndet Fitbit Flex med nuværende anvendte metoder til objektivisering af aktivitetsniveau i almen praksis, er der forskellige fordele og begrænsninger ved denne alternative metode. Som tidligere nævnt i \autoref{alternativemetoder} giver anvendelsen af aktivitetsarmbånd den almen praktiserende læge en mere nøjagtig vurdering af en patients aktivitetsniveau sammenlignet med subjektive besvarelser såsom spørgeskemaer.

Informationer om aktivitetsniveauet opsamles automatisk, og det er dermed ikke nødvendigt for patienten selv at holde styr på, hvor meget aktivitet, der udføres. Lægens opfattelse af, hvor meget fysisk aktivitet patienten udfører, afhænger derved heller ikke af patientens hukommelse eller evne til at formidle. Det giver lægen en mere præcis oversigt over aktiviteten udført over tid, og dette kan hjælpe med at se en eventuel udvikling eller tilbagegang i aktivitetsniveauet hos patienten. 

Det er desuden ikke nødvendigvis ekstra tidskrævende at anvende aktivitetsarmbåndet sammenlignet med de nuværende metoder i \autoref{NuMetode}, såfremt patienten har modtaget tilstrækkelig information om, hvordan udstyret anvendes korrekt. Problemer kan opstå, hvis patienten oplever besvær ved anvendelsen af aktivitetsarmbåndet, og ikke bruger det korrekt eller vælger helt at undgå at bruge det.

Fitbit Flex er generelt ikke velegnet til tracking af andet end gang og løb. Anden aktivitet end dette kan medføre, at lægen får forkert indblik i patientens aktivitetsniveau, hvis patienten ikke har meddelt lægen, at patienten foruden den registrerede aktivitet også dyrker anden sport. Sammenlignet med de subjektive metoder, der anvendes i klinikkerne, kan alle former for fysisk aktivitet medtages på én gang ved besvarelse af eksempelvis spørgeskema eller under samtale om patientens aktivitetsniveau.
En ulempe ved anvendelse af Fitbit Flex sammenlignet med andre aktivitetsarmbånd er, at der ikke er indbygget GPS i denne. Kræver den målte aktivitet GPS-input for højere præcision, skal en ekstern GPS, eksempelvis på en smartphone, anvendes. Det er dog ikke en nødvendighed for lægen at vide placeringen, da lægen som udgangspunkt kun er interesseret i at kende aktivitetsniveauet.

En begrænsning i forhold til aktivitetsarmbånd sammenlignet med nuværende metoder kan være, at det er en mere teknologisk metode, der kan kræve adgang til smartphone eller PC og muligvis internet. Størstedelen af de hypertensive patienter er en del af den ældre befolkningsgruppe. I 2014 var der ifølge Danmarks Statistik 41 \% af de 75-89 årige, der aldrig har brugt internettet \citep{dst2014}. Hvis lægen skal have mulighed får at tilgå patientens data vedrørende aktivitetsniveauet, når patienten ikke er fysisk til stede, eksempelvis ved telefonkonsultationer, er det nødvendigt, at patienten har adgang til internettet for at synkronisere data fra aktivitetsarmbåndet med dennes brugerkonto.

Som nævnt i \autoref{sec:teknologibeskrivelse} skal Fitbit Flex synkroniseres med en smartphone eller PC for at kunne se den registrerede aktivitet. Dette kræver, at patienten er i besiddelse af en af disse, hvis patienten selv ønsker at følge med i aktiviteten. Har patienten ikke mulighed for at komme i besiddelse af smartphone eller PC, kan detaljeret data lagres i aktivitetsarmbåndet i op til syv dage. Dette kan dog imødegås ved, at lægen tjekker data fra aktivitetsarmbåndet indenfor de syv dage, hvis det ønskes at se detaljeret data. Ellers kan Fitbit Flex gemme mindre detaljeret data i op til 30 dage. Fitbit Flex har desuden et batteri, der genoplades ved at koble den til en PC. 

\subsection{Nøjagtighed af aktivitetsmåling}

Præcisionen af Fitbit Flex armbåndet vil have indvirkning på brugbarheden af de målte data, og problematikken i forhold til armbåndets begrænsninger er beskrevet i \ref{afgraensning_tek}. Det blev fundet at armbåndets præcision ved almindelig gang er $99,6~\%$, mens det jævnfør "A comparison of warable fitness devices" af \citep{kaewkannate2016}, underestimerer aktiviteten ved gang på trapper eller løbebånd \citep{kaewkannate2016}.

For at armbåndet kan anvendes i praksis, er det vigtigt at armbåndet ikke har tendens til overestimering af patientens aktivitetsniveau. Dette krav er stillet, som følge af et forhøjet aktivitetsniveau gavner patienten, hvorfor det ikke vil være hensigtsmæssigt at implementere et armbånd, der giver patienten et indtryk af at have opnået den ønskede aktivitetstid, før patienten reelt set har opnået de daglige mål. Her vil det være mere gavnligt for patientens tilstand, hvis armbåndet underestimere aktiviteten, således patienten kommer til at bevæge sig mere end det ønskede mål.

Betydningen af over- og underestimering af aktivitetsniveauet, gør at Fitbit Flex anses som værende anvendeligt til aktivitetsestimering hos patienterne i almen praksis, eftersom præcisionen ved almindelig gang er tæt på $100~\%$, mens den underestimerer andre gangtyper. Derved opnås ekstra aktivitet hos patienten, før han/hun bliver gjort opmærksom på at de daglige mål er nået, hvilket anses som et positivt resultat af fejlestimering ved anvendelse af aktivitetsarmbåndet.
\section{Delkonklusion}

Hvordan fungerer fitbit flex, og hvordan kan dette anvendes, således at en almen praktiserende læge får dokumenteret patientens aktivitetsniveau?!?
 
- Fungerer ved at måle tid aktiv (aktive timer) og skridt fra et accelerometer?
	- Måler i forhold til den enkelte brugers tilpasninger i brugerkontoen som tilhører det enkelte armbånd for at estimere aktiviteten..

- Ved upload muligt at se data flere steder?
	- Både på smartphone og PC/mac er der adgang til data fra Fitbit Flex, så det er nemt at komme til data hvis man har "adgang", men det har de jo fordi lægen skal jo have data fra patienten.
	
- Virker den brugervenlig og er det optimalt til dataopsamling flere steder?
	- I princippet skal patienten bare gå med armbåndet (i en periode?) og så opsamler det data, patienten kan se på LED'erne om han/hun har bevæget sig "nok". Upload og opladning er krævet for at kunne bruge aktivitetsarmbåndet... Kan målgruppen finde ud af det? ... Hvis patienterne er meget engagerede kan de selv se deres data på smartphone og PC/mac, hvilket kan være en motiverende faktor. Lægen kan se dataene fra patienten lige så snart patienten har uploaded data fra armbåndet ved at synkronisere armbåndet med smartphone eller PC/mac...

Repræsentere fitbit flex den fysiske aktivitet tilstrækkeligt, til at data kan anvendes af praktiserende læger til vurdering af patientens fysiske aktivitetsniveau?

- Hvor godt virker den?
	- Indrage oplysninger fra afsnittet om præcision, at den måler gang med næsten 100 \% præsicion... Dog skal patienten oplyse om andre aktiviter ud over hvad Fitbit Flex kan måle, for eksempel cykling eller svømning, for at give lægen det fulde billede, fordi Fitbit Flex kan ikke måle disse aktiviteter lige så præcist som gang... Måske tilføje noget af det med at gang er nemt at forholde sig til og hellere underestimere og sikre at de når målet med gang / løb , frem for ikke at nå det... Derfor vurderes det at Fitbit Flex kan være en god metode til at .. ja ..
	
- Det vurderes at Fitbit Flex, (eller lignende teknologier)... vil være et godt redskab... til objektiv monitorering af patienternes fysiske aktivitet, så længe patienterne kan finde ud af at anvende teknologien og at det tages i betragtning at der er visse former for aktivitet som armbåndet ikke kan registrere tilstrækkeligt... 

%-------------Patient----------------
\chapter{Patienten}
Dette kapitel har fokus på patientaspektet, hvor teknologiens påvirkning på patienten vil blive karakteriseret, analyseret og vurderet. 




% Indhold i afsnittet
\section{Patientkriterier for tildeling af aktivitetsarmbånd}

Det vil være fordelagtigt at definere nogle kriterier, som patienten vil skulle overholde for at få tildelt et aktivitetsarmbånd til monitorering af fysisk aktivitet til hypertensive patienter. Dette gøres for at indskrænke gruppen af patienter for at sikre, at aktivitetsarmbåndene gives til patienter, der vil få mest gavn af denne form for ekstra monitorering, så omkostningseffektiviteten holdes så lav som muligt.

Dette kan eksempelvis defineres ud fra graden af fysisk inaktivitet, tendens til overvurdering/estimering af egen fysisk aktivitet og patienter med høj risiko for udvikling af hypertension eller følger til hypertension. 

Disse kriterier kunne være, at patienten er fysisk inaktiv ud fra definitionen i \autoref{sec:fys_inaktivitet}, at egen læge vurderer, at patienten overestimerer mængden af fysisk aktivitet, som de dyrker til dagligt, eller at egen læge vurderer, at patienten har høj sandsynlighed for at udvikle symptomer på hypertension eller følger til tilstanden, der vil forringe patientens livskvalitet. Dette vil indskrænke gruppen af patienter, der vil få udleveret et aktivitetsarmbånd til de, der vil få mest gavn af brugen af armbåndet. 
\subsection{Brugertilfredshed}

Et vigtigt element ved indførelse af ny teknologi er brugertilfredsheden, og denne har stor betydning for virkningen af den nye teknologi. En lav brugertilfredshed i forbindelse med anvendelse af aktivitetsarmbånd, vil resultere i lavere anvendelsesprocent, hvilket betyder teknologien ikke vil give lægen et fyldestgørende indblik i patientens aktivitetsmønster.

\subsubsection{Brugerbedømelse af Fitbit Flex}

For det valgte aktivitetsarmbånd, Fitbit Flex, er det fundet at armbåndet ikke scorer højest hvad angår tilfredsheden vedrørende egenskaber for armbåndet. I undersøgelsen 'A comparison of wearable fitness devices' har forsøgspersonerne anvendt hvert af armbåndene i en uge, hvorefter de er bedømt på en skala fra $1$ til $5$. Blandt fordele ved armbåndet, kan der blandt andet nævnes at forsøgspersonerne er tilfredse med det strømlinede design, at applikationens brugerflade er farverig, sjov og nem at bruge, samt at det er vandafvisende. Ulemperne inkludere blandt andet langsom synkronisering og problemer med tracking af gang på trapper \citep{kaewkannate2016}.

\begin{figure}[H]
	\centering
	\includegraphics[width=0.6\textwidth]{figures/FeatureSatisfaction}
	\caption{Sammenligning af tilfredsheden vedrørende aktivitetsarmbåndenes egenskaber \citep{kaewkannate2016}.}
	\label{fig:FeatureSatisfaction}
\end{figure}

På \figref{fig:FeatureSatisfaction} ses det at Fitbit Flex's bedømmelse ligger omkring midten af tilfredsheds-skalaen anvendt i studiet 'A comparison of wearable fitness devices'. Dette betyder at brugerne finder armbåndet lettere anvendeligt og tilfredsstillende hvad angår synkronisering, brugerflade og batteritid, mens det bedømmes som moderat anvendeligt og tilfredsstillende vedrørende hardware design og brugervenlighed \citep{kaewkannate2016}.

\begin{figure}[H]
	\centering
	\includegraphics[width=0.6\textwidth]{figures/FunctionSatisfaction}
	\caption{Sammenligning af tilfredsheden vedrørende aktivitetsarmbåndenes funktioner \citep{kaewkannate2016}.}
	\label{fig:FunctionSatisfaction}
\end{figure}

Funktionaliteten er bedømt på \figref{fig:FunctionSatisfaction}, hvor det ses der ikke er en stor variation i brugernes bedømmelse af armbåndenes funktioner. Her er Fitbit Flex bedømt mellem lettere og moderat anvendeligt og tilfredsstillende ved optælling af skridt, søvnmåling og afstandsmåling. Alle armbånd er bedømt som meget lidt til lettere anvendeligt og tilfredsstillende i forbindelse med kostanalyse- og opmåling \citep{kaewkannate2016}.

\subsubsection{Anvendelse af aktivitetstracker i hverdagen}

For at opnå forståelse for patientens oplevelse ved brug af aktivitetstrackere i hverdagen, tages udgangspunkt i studierne 'Acceptance of Commercially Available Wearable Activity Trackers Among Adults Aged Over 50 and With Chronic Illness: A Mixed-Methods Evaluation' og 'Personal informatics for everyday life: How users without prior self-tracking experience engage with personal data'. Det førstnævnte studie undersøger implementeringen af aktivitetstrackere til motionsmonitorering af kronisk syge over 50 år, hvor forsøgspersonerne tester en simpel skridttæller og fire aktivitetstrackere, for til sidst at bedømme forskellige aspekter ved anvendelse af disse. I det andet studie undersøges hvordan forsøgspersoner uden tidligere erfaring med aktivitetstrackere oplever at måle deres aktivitetsniveau.


https://www.ncbi.nlm.nih.gov/pmc/articles/PMC3961803/ (Måske ikke så relevant)
https://www.ncbi.nlm.nih.gov/pmc/articles/PMC4749845/
http://www.sciencedirect.com/science/article/pii/S107158191630060X


\subsubsection{Motivation ved aktivitetstracking}

https://www.ncbi.nlm.nih.gov/pubmed/19902981
http://link.springer.com/article/10.1186/s13612-016-0042-6


Er teknologien brugervenlig og motiverer den patienten til en mere aktiv hverdag?

Er aktivitetsarmbåndet brugervenligt/let at anvende/lære at anvende?

Hvad kræver det af patienten at anvende aktivitetsarmbåndet?
\section{Patientens sociale og individuelle forhold i dagligdagen}
Idet en patient får tildelt et Fitbit Flex-aktivitetsarmbånd, er der forskellige tilhørende faktorer, der kan have betydning for patientens brug af armbåndet. 
Dette relaterer til, hvor avanceret teknologien er, og hvilke muligheder der er for at formidle den registrerede aktivitet for brugeren selv og omgangskreds. Dertil er det muligt at opdele disse faktorer i sociale og individuelle forhold og kan være hæmmende eller motiverende for patienten.   

\subsection{Sociale forhold}
En implementering af Fitbit Flex kan påvirke patienten og dennes sociale forhold. Fitbit Flex muliggør sammenligning med andre brugere af aktivitetsarmbånd over internettet, hvis patienten ønsker dette. Dette skaber en form for onlinefællesskab, hvor patienterne kan interagere med andre, der kan have lignende mål vedrørende daglig fysisk aktivitet \citep{karapanos2016}. 
Dette giver mulighed for, at patienten kan sammenligne sig med og konkurrere mod venner, kollegaer, familie, fremmede eller blot egne tidligere rekorder. På denne måde kan der skabes incitament til motion, hvis der konkurreres mod andre, da det kan virke som en motiverende faktor \citep{rooksby2014}.

I forhold til den valgte patientgruppe, kan alderen af patienten være afgørende, da prævalensen af hypertension stiger med alderen. I aldersgruppen $>50$ år har næsten $50~\%$ af befolkningen hypertension \citep{kronborg2008}. Denne aldersgruppe, især den ældste del af patienterne, vil ikke nødvendigvis kunne benytte sig af de sociale aspekter af aktivitetsarmbåndene, hvis de ikke er bekendte med sociale medier til dagligt. Disse vil udelukkende få gavn af de simple funktioner af et aktivitetsarmbånd, hvorfor det skal tages højde for, at alle patienter ikke vil få det samme udbytte af brugen af teknologien \citep{mercer2016}. Hvis de ældre patienter er i stand til at synkronisere sit armbånd, så familiemedlemmer vil kunne tilgå dennes data via internettet, vil dette muligvis kunne fungere som en motiverende faktor, hvis de er klar over og har givet samtykke til, at familien følger med i deres aktivitetsniveau.

\subsection{Individuelle forhold}
I forhold til den valgte patientgruppe, kan der være nogle individuelle forhold, der afgør, om aktivitetsarmbåndet vil blive brugt af patienterne. Dette kan især være aldersgruppen af patienterne. 

Den ældre del af patientgruppen kan være tilbageholdende over for ny sundhedsteknologi, som de selv skal betjene, da dette kræver en indsigt i, hvordan denne slags teknologi fungerer. Ikke alle i aldersgruppen, $>50$ år, har erfaring med brug af denne type teknologi, hvilket kan gøre nogle patienter tilbageholdende fra at tage teknologien til sig, selvom den er relativt brugervenlig \citep{mercer2016}. 

I et studie omhandlende en gruppe af kronisk syge i alderen $>50$ år, som havde gået med aktivitetsarmbånd over en periode, ville $73~\%$ af studiets deltagere købe en aktivitetstracker, da de generelt var tilfredse med én eller flere af de afprøvede aktivitetstrackere. I dette studie lagde patienterne blandt andet vægt på, om den var behageligt at gå med, og om den var pæn, så den fungerede som en form for smykke \citep{mercer2016}.
\section{Effekter af monitorering af aktivitetsniveau}
%Indhold: I dette afsnit vil vi analysere effekter af en implementering af Fitbit Flex med hensigt på monitorering af aktivitetsniveau. Vi vil herunder komme ind på, hvilke effekter der er mulige at opnå, samt tidshorisonten på effekterne, om teknologien vil være en motiverende faktor for patienten til en mere aktiv hverdag, og om den kan være demotiverende, fx hvis der ikke sker en ændring. Vi vil se, om vi kan estimere, hvor stor en andel af patienter, der oplever en positiv virkning/effekt. 

Hypertensive patienters anvendelse af Fitbit Flex kan have forskellige virkninger på patienter og deres holdning til fysisk aktivitet, og det kan desuden påvirke forholdet mellem patient og læge.

Der er ikke fundet tal, der direkte viser sammenhængen mellem anvendelse af aktivitetsarmbånd og effekten af dette, så det er usikkert, hvor mange patienter, der reelt vil have en positiv effekt af anvendelse af Fitbit Flex. Ud fra forskellige studier kan det estimeres, hvordan patienter med hypertension vil påvirkes ved implementering af aktivitetsarmbånd som en del af behandlingen.

Der er en række studier, blandt andet studiet af \citeauthor{mercer2016}, der indikerer, at brugen af aktivitetsarmbånd kan give motivation til en mere aktiv hverdag. Det nævnes i studiet, at anvendelsen af aktivitetsarmbåndet giver brugeren bevidsthed om egen sundhed og aktivitetsniveau, hvilket i nogle tilfælde kan medføre et øget aktivitetsniveau. 

Testpersonerne, der deltog i studiet, blev, ved anvendelse af en likert skala, spurgt om, hvorvidt de følte, at aktivitetsarmbånd hjalp dem med at blive mere aktive. Til dette bliver der i gennemsnit givet et neutralt svar, således at der hverken var uenighed eller enighed om, at aktivitetsarmbånd medførte øget aktivitetsniveau hos den enkelte \citep{mercer2016}. Det kan derfor antages, at halvdelen af testpersonerne har oplevet, at aktivitetsarmbånd havde en positiv effekt på deres aktivitetsniveau.

Studiet konkluderer, at der er potentiale i at anvende aktivitetsarmbånd til at forbedre kronikeres motionsvaner. Desuden nævnes det, at implementering af aktivitetsarmbånd i sundhedssektoren ville kunne forbedre relationen mellem patient og læge, da det kan hjælpe lægen med at give patienten et bedre indblik i og bedre vejledning om patientens fysiske sundhed og vigtigheden af det \citep{mercer2016}. 

Et studie af \citeauthor{nelson2016} har undersøgt sammenhængen mellem anvendelse af aktivitetsarmbånd og testpersonens følelse af empowerment, hvilket er en følelse af handleevne og kontrol over beslutninger, der påvirker deres helbred \citep{toennesen2005}.

I studiet konkluderes, at forskellige egenskaber, såsom muligheden for at være en del af et fællesskab via en app og den feedback aktivitetsarmbåndet giver, har en positiv indflydelse på brugerens følelse af empowerment. Det faktum, at testpersonerne blev monitoreret havde dog ingen virkning på følelsen af empowerment, hvilket ifølge studiet kan skyldes, at monitorering associeres med negative konsekvenser, såsom invasion af privatlivet.
Studiet finder ligeledes, at jo større en brugers følelse af empowerment er, des mere tilskyndes denne til at opnå sine opstillede mål. Det påpeges dog i studiet, at testpersonernes generelle engagement til at opnå opstillede mål, var lavere end det er set i andre studier. Dette kan skyldes, at testpersonerne i dette studie ikke fik opstillet generelle mål for deres fysiske aktivitet, hvorimod testpersoner i andre studier skulle forsøge at nå mål opstillet af eller i samarbejde med andre. 

%- Nogle folk er motiveret. Nogle føler selv at det hjælper på deres sundhed. Kan få folk til at engagere sig mere i at opnå opstillede mål (måske mål opstillet sammen med læge, så det ikke er helt egne mål).  antage, at det vil have positiv på nogle patienter. Kilder siger desuden at kun patienter der er ”motiveret” til det kommer i non-farmakologisk behandling. 

%\citep{nelson2016}: http://www.sciencedirect.com.zorac.aub.aau.dk/science/article/pii/S0747563216302369 

\section{Etiske problemstillinger}

Ved implementering af ny teknologi eller nye ideer i sundhedssektoren vil der ofte opstå etiske problemstillinger, som skal adresseres. Der vil derfor i dette afsnit blive forsøgt belyst, hvilke etiske problemstillinger Fitbit Flex, som aktivitetsmonitoreringsenhed, vil kunne skabe i sundhedssektoren.

Rapporten af \citeauthor{patienthome2015} beskriver, at hvis monitoreringen er hyppig, kan det føles som overvågning for nogle patienter. Det kan derfor blive en problemstilling, som skal tages op med patienten, hvorvidt vedkommende ønsker monitoreringen \citep{patienthome2015, SundhedsstyrelsenPatientersRetsstilling2016}.

Studierne af \citeauthor{Mittelstand2011} og \citeauthor{Nordgren2013} har undersøgt en række etiske afspekter ved brugen af enheder, der monitorerer patienten i privatlivet. Studierne understreger visse etiske aspekter vedrørende monitorering af patienten i privatlivet.

Af disse etiske aspekter vil Fitbit Flex primært kunne komme til at ramme områder, som omhandler privatliv og persondata, synlighed, autonomi, pålidelighed, helbred samt uafhængighed \citep{Nordgren2013,Mittelstand2011}.


\subsection{Privatliv og persondata}

Som tidligere nævnt vil nogle patienter muligvis komme til at føle den kontinuerlige monitorering som overvågning. Fitbit Flex kan følge patienters aktivitetsniveau i form af skridt og aktive timer, herved vil eventuelle private detaljer ved aktiviteter ikke kunne spores, da Fitbit Flex ikke inkluderer GPS.
Data omhandlende fysisk aktivitet er ikke direkte personfølsomt, da det ikke indeholder CPR-nummer eller kan bruges til at identificere eller skade patienten, hvis brugernavnet til Fitbit-kontoen jævnfør \autoref{sec:teknologibeskrivelse} anonymiseres. På denne måde kan data ikke misbruges  \citep{Mittelstand2011}.


\subsection{Synlighed}

Synlighed vedrører enhedens fysiske fremtræden og om, hvorvidt nogle patienter vil komme til at føle sig stigmatiseret som syg, hvorved teknologien vil være et symbol for sygdommen, som er synlig for andre \citep{Mittelstand2011}. Fitbit Flex bæres som og ligner et armbåndsur og anses derfor ikke som værende generende for patienten, da patienten på denne måde ikke skiller sig ud. Aktivitetsarmbånd bruges derudover også af dele af befolkningen, der ikke er syge, men som ønsker at monitorere deres aktivitetsniveau. 

\subsection{Autonomi}

Behandlingen, som patienten modtager, skal være forståelig, og patienten skal være indbefattet med, hvad behandlingen kommer til at betyde for vedkommendes liv \citep{Mittelstand2011}. Lægens introduktion af teknologien er derfor vigtig, så patienten kan give et informeret samtykke. Patienten vil skulle tilpasse sig en ny livsstil ved implementeringen af Fitbit Flex, da monitoreringen har til formål at sætte konkrete tal på patientens daglige aktivitet.

\subsection{Pålidelighed}

Pålidelighed spiller en rolle for, at den dokumenterede aktivitet, som Fitbit Flex registrerer, kan gengives tilstrækkeligt præcist \citep{Nordgren2013}.
Hvis Fitbit Flex ikke viser patienternes fysiske aktivitetsniveau, eksempelvis da visse former for aktivitet ikke kan registreres, som nævnt i \autoref{noejagtighed}, kan monitoreringen betyde, at lægen underestimerer mængden af fysisk aktivitet. Dette kan betyde, at patienten får en forkert vejledning eller behandling. 
 
\subsection{Helbred}

Monitoreringen har til hensigt at øge patientens aktivitetsniveau og herved gøre patientens helbredstilstand bedre - eller undgå at forværre den. Fitbit Flex har som aktivitetstracker til formål at dokumentere den fysiske aktivitet, patienterne udfører, samt motivere patienterne til at udføre mere aktivitet i hverdagen ved for eksempel at sætte mål for patienten. Den kontinuerlige måling kan medføre, at patienten kan føle, at vedkommendes liv vil centrere sig omkring sundhed og sygdom. Dette vil kunne betyde, at patienten vil opfatte sig selv som en syg person, i stedet for en person med en sygdom \citep{Nordgren2013}.

\subsection{Uafhængighed}

Ved udlån af et Fitbit Flex-armbånd fra egen læge er det i forventningen om, at patienten skal få en mere aktiv hverdag. I den forbindelse skal det overvejes, om patienten vil føle sig forpligtet eller tvunget til at dyrke mere motion, fremfor at føle sig motiveret og opfordret. Patienten skal ikke føle sig tvunget til at fuldføre denne forventning, da teknologien ikke har til formål at bestemme over patienternes aktiviteter, og hvad de foretager sig \citep{Nordgren2013}.

\subsection{Yderligere etiske overvejelser}
Et andet etisk aspekt ved patientmonitorering er, hvem og hvor mange der skal have ejerskab og ansvar over det data, som bliver indsamlet. Data skal prioriteres, så det kun er det data, der kan bruges til den reelle behandling, som bliver gemt. Der skal derfor tages stilling til, hvad der er relevant data \citep{patienthome2015}. Eksempelvis vil det relevante data fra Fitbit Flex være de data, som fortæller noget om, hvor aktiv patienten er i hverdagen, altså; aktive timer, skridt gået og forbrændte kalorier. Data som søvnovervågning eller andet, som aktivitetarmbåndet kan registrere, vil være overflødigt for aktivitetsarmbåndets brug i forhold til behandling af hypertension, da dette ikke giver nogen direkte indikation vedrørende patientens aktivitetsniveau. Data, som sendes til lægen, skal derfor begrænses, så det tilpasses til det, som er nødvendigt for patientens forløb. 
\section{Delkonklusion}

Med henblik på at opnå et succesfuldt forløb ved brug af aktivitetsarmbåndene, er det nødvendigt at opstille kriterier for patienten, da eksempelvis behandlingsresistent hypertension eller andre sygdomme kan have indvirkning på forløbet. Her vil det være op til den enkelte praktiserende læge at vurdere, hvorvidt patienten er i stand til at anvende aktivitetsarmbånd i forbindelse med behandlingen af hypertension. Det er i et studie fundet at $73~\%$ af forsøgspersoner over $50$ år efter endt studie, havde interesse i anskaffelse af aktivitetsarmbånd, hvorfor aldersgruppen ikke er et argument for afvisning af aktivitetsarmbånd.

Ved anvendelse af aktivitetsarmbåndet, er det vigtigt at patienterne finder det let anvendeligt i hverdagen, hvorfor både brugertilfredshed og -venlighed har en væsentlig betydning for, hvilke resultater anvendelsen vil give. I de undersøgte studier, er det fundet at brugere syntes Fitbit Flex var let at anvende og havde en underholdende brugerflade, hvilket er fundet som en motiverende faktor i forhold til kontinuert anvendelse af armbåndet. Af andre motiverende faktorer til kontinuert anvendelse, kan nævnes muligheden for at undersøge ændringer i egne aktivitetsvaner, nem synkronisering, samt komfort og design af aktivitetsarmbåndet, for at undgå at skille sig ud ved aktivitetsmonitorering.

I et studie, som introducerede kronisk syge over $50$ år til forskellige aktivitetstrackere, er det desuden fundet at monitoreringen af og målsætning for egen aktivitet, motiverer patienter til højere aktivitetsniveau. Dette vil være gavnligt for sygdomsforløbet, såfremt effekten på patienternes aktivitetsniveau, vil være det samme som vist i studiet. Her er det yderligere fundet, at de sociale egenskaber i Fitbit's brugerflade, kan give grundlag for yderligere motivation til højere aktivitetsniveau, hos patienter med kendskab til andre sociale medier, mens monitoreringen potentielt kan påvirke aktivitetsniveauet i en negativ retning.

Samtidig blev det fundet, at patienter uden tidligere erfaring med smartphones eller tablets, ikke havde yderligere problemer med tilvænning til aktivitetstrackerne, sammenlignet med patienter, som anvendte tidligere nævnte elektronikudstyr i hverdagen. Hvis samme mønster opstår hos hypertensive patienter i almen praksis, antages det at patienter uden stor erfaring med IT også vil være i stand til at anvende Fitbit Flex. 

Ved undersøgelse af de etiske problemstillinger ved anvendelse af Fitbit Flex, er den største problematik, at lægen har mulighed for kontinuert overvågning af patienten, hvilket kræver lægen i samråd med patienten, fastsætter hvilke data lægen har adgang til. Patienten kan potentielt anse den kontinuerte overvågning, som en krænkelse af privatlivet. Som udgangspunkt vil aktivitetsmønstret uden tilhørende GPS-tracking, anses som dataindsamling på niveau med døgnblodtryksmåling eller monitorering af hjerterytme, hvorfor det antages at patienterne er villige til at acceptere overvågning af aktivitetsvaner efter samtale med lægen. 

%-------------Organisation-----------
\chapter{Organisation}
Dette kapitel omhandler de organisatoriske ændringer, der kan forekomme ved implementering af Fitbit Flex til monitorering af hypertensive patienters aktivitetsniveau i den almene praksis. Leavitts modificerede organisationsmodel, beskrevet i \autoref{sec:metode_org}, anvendes til at beskrive disse ændringer. I kapitlet undersøges tilrettelæggelse og opgavefordeling i den primære og sekundære sundhedssektor, som påvirkes ved implementeringen af Fitbit Flex. Det undersøges også, hvilke behov læger i den primære sektor og hypertensive patienter har for introduktion til teknologien i forbindelse med implementeringen.   


% Indhold i afsnittet
\section{Patientforløb med hypertension}
Med udgangspunkt i Leavitts modificerede organisationsmodel, ses hvilke påvirkninger der forkommer i modellens fem områder ved implementering af Fitbit Flex. Dertil ses hvilken virkning teknologien har på behandlingsforløbet.

\subsection{Diagnose og udredning af hypertension} \label{sec:dia_hypertension}
Som beskrevet i \autoref{sec:hypertension} ses der sjældent symptomer ved hypertension og opdages derfor ofte ved en tilfældighed ved eksempelvis sundhedstjek hos patientens alment praktiserende læge. 
Diagnosen hypertension kan ikke stilles på baggrund af én blodtryksmåling foretaget hos lægen, da patienten kan være nervøs og derved have et højere blodtryk og påvirke resultatet. Patienten bør få foretaget enten døgnblodtryksmåling eller hjemmeblodtryksmåling, hvis målinger i klinikken viser forhøjet blodtryk. Patienten kan desuden sidde i et rum uden tilstedeværelse af sundhedspersonale og få foretaget blodtryksmålinger med en automatisk blodtryksmåler \citep{lodberg2016, bech2015}.

En hypertensionskontrol består af forskellige  undersøgelser. I rapporten \citetalias{munck2007} blev personalet spurgt om hvilke områder hypertensionskontrollen bestod af. I over 50 \% af de deltagende praksis omhandlede kontrollen, en blodtrykskontrol, en vægtkontrol, information omkring rygeafvænning, kostvejledning og vejleding omkring motion og fysisk aktivitet. Data fra studiet oplyste yderligere at 79,9 \% af personalet inkluderede instruktion i hjemmeblodtryksmåling ved hypertensionskontrollerne. Dette omhandlede blandt andet instruktion i udfyldelse af et registreringsskema for blodtryksmålingerne \citep{munck2007}. 

Såfremt flere blodtryksmålinger viser et forhøjet blodtryk skal patienten igennem en videre udredning. Patientens tidligere sygehistorie vil tages i betragtning, herunder blandt andet forskellige risikofaktorer for hypertension såsom lavt aktivitetsniveau, diabetes og familiær disposition til blandt andet hypertension, diabetes og nyresygdomme. Foruden dette foretages en objektiv undersøgelse af patienten, hvor blandt andet højde, vægt og abdominalomfang måles. Der tages desuden EKG-målinger, blodprøver og urinprøver, og ved kliniske tegn på hjertesvigt, henvises patienten til sekundærsektor for at få foretaget røntgen af thorax og ekkokardiografi \citep{lodberg2016, bech2015}.

Er den hypertensive patient under $40$ år, har et meget højt blodtryk eller har behandlingsresistent hypertension, bør patienten undersøges for sekundær hypertension for at sikre, at det ikke er en eller flere bagvedliggende sygdomme, der har hypertension som en følge \citep{lodberg2016}. Sekundær hypertension forekommer hos mindre end $5~\%$ af tilfældene, og den hyppigste årsag er nyresygdomme \citep{lodberg2008}. Patienten kan eksempelvis henvises til nefrologisk afdeling for yderligere undersøgelser, hvis der findes eller er mistanke om en bagvedliggende nyresygdom \citep{lodberg2016, sundhedsstyrelsen2010}. 

\subsubsection{Samspil mellem primær og sekundær sektor}
Under udredning og behandling kan hypertensive patienter, hvis nødvendigt, blive henvist til forskellige afdelinger af den alment praktiserende læge. Regionen, hvori patienten er bosat, har betydning for, hvor patienten henvises til. Henvisning kan eksempelvis være til Blodtrykscenteret i Region Midtjylland, hvor Nyremedicinsk, Hjertemedicinsk og Endokronologisk Afdeling i samarbejde behandler hypertension og eventuelle følgesygdomme, samt patienter med sekundære hypertension. Patienter kan henvises til Blodtrykscenteret ved behandlingsresistent hypertension, hypertension i forbindelse med nogle former for hjertekarsygdomme, mistanke om sekundær hypertension eller hvis nyopdaget hypertension skal verificeres ved hjælp af døgnblodtryksmåling. Ved en henvisning til Blodtrykscenteret vil patienten forinden være forsøgt udredt af egen læge, hvor informationer om udredningen vedlægges henvisningen \citep{aarhusuniversitetshospital}. 

I regioner uden et center såsom Blodtrykscenteret vil hypertensive patienter typisk blive henvist til nefrologisk, kardiologisk eller endokronologisk afdeling \citep{buur2011}. Når det vurderes af lægerne på den pågældende afdeling, at patienten har fået den tilstrækkelige behandling og/eller udredning på afdelingen, afsluttes behandlingen, og patienten kan efterfølgende gå til kontrol ved alment praktiserende læge \citep{sundhedsstyrelsen2010, lodberg2016}.

Hvis Fitbit Flex bliver implementeret som en del af behandlingen af hypertension kan dette påvirke strukturen i organisationen ved at ændre antallet af patienter, der bliver henvist til forskellige afdelinger i den sekundære sektor. Hvis aktivitetsarmbåndet har en positiv effekt, og flere patienter opnår reduktion i blodtryk som følge af højere aktivitetsniveau, kan der være færre følgevirkninger såsom nyresygdomme og hjerteproblemer. Hvis ikke dette var tilfældet kunne man i stedet opnå en udskydelse af diverse følgesygdomme. Dette kan resulterer i at færre patienter henvises til den sekundære sundhedssektor. 

\subsection{Behandling af hypertension}

Når den alment praktiserende læge har diagnosticeret og vurderet patienten, kan behandlingen påbegyndes. Behandling af hypertension afhænger af, hvilken grad af hypertension patienten har, samt hvorvidt det er sekundær hypertension. Det er herved forskelligt, hvor og hvem der varetager behandlingen. Opstår følgevirkninger, der kræver yderligere behandling, henvises patienten til en specialiseret afdeling, hvor behandlingen varetages, indtil patienten er stabil og kan fortsætte hypertensionskontrol hos egen læge \citep{sundhedsstyrelsen2010}.

Som skrevet i \autoref{sec:hypertension} har lægen i den almene klinik mulighed for at behandle hypertensive patienter farmakologisk og non-farmakologisk. Behandlingen vurderes ud fra, om patienten har risiko for kardiovaskulær sygdom, hvor lægen blandt andet undersøger om patienten har risikofaktorer såsom hypertensive organskader, diabetes og nyresygdomme \citep{promedicin2016}.

Farmakologisk antihypertensiva-behandling startes, hvis patienten har et blodtryk på over $180$/$110$ mmHg, da det på dette tidspunkt ikke er tilstrækkeligt med omlægning af livsstil, herunder rygestop, motion, kostændringer og saltindtag \citep{pedersen2016, bech2015}. 

\begin{figure}[H]
\centering
\includegraphics[width=0.9\textwidth]{figures/behandlingsvejl}
\caption{Behandlingstilgang i relation til risikofaktorer og målt blodtryk. $10$ års risiko for apopleksi eller myokardieinfarkt - Rød: meget høj risiko ($>30~\%$), orange: høj risiko ($20-30~\%$), gul: middel risiko ($15-20~\%$) og grøn: lav risiko ($<15~\%$) samt hvilken konsekvens, som bør drages af inddelingen. (HT: hypertension; SBT: systolisk blodtryk; DBT: diastolisk blodtryk). *: CVD (kardiovaskulær sygdom), **: CKD (kronisk nyresygdom), ***: Evt. strammere blodtryksmål hos visse patienter med diabetes og patienter med proteinuri \citep{bech2015}.}
\label{fig:behandlingsvejl}
\end{figure}

\noindent
I \autoref{fig:behandlingsvejl} ses hvilke tiltag der er nødvendige i relation til patientens blodtryk, samt hvilke risikofaktorer patienten er vurderet til at have. Har patienten en mild grad af hypertension, hvilket illustreres med grøn på \autoref{fig:behandlingsvejl}, kan patienten muligvis nøjes med livsstilsændringer, hvis der er få risikofaktorer. Fra middel til meget høj risiko for følgesygdomme af hypertension, illustreret med gul, orange og rød på \autoref{fig:behandlingsvejl}, bør der tillægges blodtryksmedicin \citep{bech2015}.
\citetalias{munck2007} er et projekt ved Forskningsenheden for Almen Praksis i Odense, hvor en rapport er udgivet omhandlende registreringer af hypertension i 184 almene praksisser. Ifølge denne rapport var $35,3~\%$ af de hypertensive patienter for lidt fysisk aktive, og dermed var lavt aktivitetsniveau den tredje hyppigste risikofaktor \citep{munck2007}.

Yderligere var $11~\%$ af de registrerede hypertensive patienter i non-farmakologisk behandling, samtidig var det kun $1,7~\%$ af de registrerede patienter, der ikke fik nogle former for farmakologisk behandling \citep{munck2007}.
 %(Jeg ved ikke hvor jeg er på vej hen… Hjælp. Er alt det her overhovedet relevant at have med? DET ER COOLT MED TAL, HVOR MANGE DER EGENTLIG BLIVER BEHANDLER ALTSÅ HVOR MANGE DER ER REGISTRERET. JA JEG SYNTES OGSÅ AT DET VIRKER MEGET FORNUFTIGT)
Anvendelse af aktivitetsarmbånd til behandling af hypertension vil høre under kategorien non-farmakologisk behandling i form af livsstilsændringer. Hertil vil lægen eller sygeplejersken få til opgave at oplære patienten i brug af armbåndet og siden følge op på aktivitetsniveauet. Lægen kan ud fra data om aktivitetsniveauet vejlede patienten om motion. 
Skulle en stigning af blodtrykket forekomme under behandlingen, kan lægen vurdere, om det kan skyldes et fald i aktivitetsniveau. Ligeledes ville et lavere blodtryk kunne skyldes, at aktivitetsniveauet er steget. Det vil på denne måde være muligt at sammenligne objektive målinger af aktivitetsniveauet med patientens prognose. 

\subsection{Aktivitetsarmbånd i den nuværende organisation}
I relation til Leavitts modificerede organisationsmodel er mulige interessenter for anvendelse af aktivitetsarmbånd, som en del af behandling mod hypertension, de praktisserende læger.

I \citetalias{munck2007} blev personale i almene praksisser adspurgt hvorvidt det ønskes, at personalet involveres mere i patientbehandlingen af hypertension, hvortil $67,3~\%$ svarede ja. Ud af disse svarede omkring $49~\%$, at de ønsker øget involvering indenfor vejledning om motion og fysisk aktivitet \citep{munck2007}. Disse tal kan tyde på, at en stor andel af de almene praksisser kan være åbne for nye muligheder eller forbedringer inden for vejledning om fysisk aktivitet i forbindelse med hypertension. Ved indførelse af Fitbit Flex som en del af behandlingen af hypertension har personalet i den almene praksis mulighed for at monitorere patientens fysiske aktivtetsmønster og kan dermed få bedre mulighed for at tilpasse vejledningen til patienten. 

Andre mulige interessenter kan også være læger og andet personale i den sekundære sektor. Blodtrykscenteret i Region Midtjylland reklamerer blandt andet med, at forskning og implementering af nye behandlingsmetoder foregår med udgangspunkt i Blodtrykscenteret \citep{aarhusuniversitetshospital}. Foruden dette kan afdelinger på sygehuse, der behandler mange hypertensive patienter, have interesse i at få implementeret behandlingen i de almene klinikker. Hvis antallet af svære tilfælde af hypertension reduceres, kan det dermed påvirke samspillet mellem primær og sekundær sektor. Ligeledes ville patienterne være interesseret i teknologien, da en eventuel stigning i aktivitet ville resultere i færre eller en udskydelse af følgevirkninger der kan forekomme ved hypertension. 

Yderligere fungerer læger og sygeplejersker i almen praksis også som aktører, da de vil få til opgave at lære at anvende aktivitetsarmbånd som en del af et behandlings- og monitoreringsforløb for hypertensive patienter. 
Indførelse af aktivitetsarmbånd kræver derfor, at de alment praktiserende læger ønsker at innovere behandlingen og anvende den alternative behandlingsmetode i klinikkerne. Nogle vil muligvis være skeptiske over for indførelse af en ny teknologi, såsom Fitbit Flex. De læger, der har interesse i at benytte aktivitetsamrbånd i form af Fitbit Flex, kan indføre det i praksis. Har indførelsen en positiv effekt, kan teknologien udvides til flere alment praktiserende klinikker, hvis disse ændrer mening.

For at gøre det mere attraktivt for de praktiserende læger at anvende en ny teknologi, der kræver omstilling i forhold til den almindelige arbejdsgang, kan der indføres et honorar for anvendelse af aktivitetsarmbånd som en del af behandlingen for hypertension. Ved at honorere anvendelse af nye teknologier i almen praksis, kan anvendelsen af udstyret øges, hvilket eksempelvis har gjort sig gældende ved hjemmeblodtryksmåling \citep{bang2006}.
\section{Organisatoriske ændringer}\label{sec:org_aendringer}

For at anvende FitBit Flex i en medicinsk sammenhæng, skal både patient og læge have viden om teknologien. Det essentielle er at både patient og læge skal være indforstået med, hvordan armbåndet bruges som et middel til dokumentation af patientens aktivitetsmønster.

Hertil skal lægen vide hvordan motion kan have en virkning på blodtrykket, og hvad det vil betyde for patienten hvis vedkommende ikke er tilstrækkeligt aktiv, i forhold til Sundhedsstyrelsens anbefalinger. Sammenhæng mellem aktivitet og blodtryk er beskrevet i \autoref{sec:effekterafaktivitet}.

Lægen skal yderligere kunne vurdere hvilke hypertensive patienter, der er egnet til at få et armbånd, hvorfor der i \autoref{sec:kriterier} opstilles forslag til patient-kriterier for tildeling af et Fitbit Flex. I forlængelse med dette skal teknologien også tilpasses den enkelte patient, hvilket vil betyde at lægen yderligere skal have færdigheder i brugertilpasning, som er beskrevet i \autoref{sec:brugertilpasning}. 

Derudover skal lægen ved udlevering af Fitbit Flex, kunne instruere patienten i brugen af armbåndet, samt de tilhørende dele og programmer, således patienten er i stand til at anvende armbåndets forskellige funktioner på egen hånd. 

Instruktionen af armbåndet ville som minimum indeholde viden omkring: 
\begin{itemize}
\item Forståelse vedrørende anskaffelse og installation af applikation på mobil, computer eller lignende enhed. 
\item Forudsætningerne for at armbåndet virker, i forhold til hvordan registrer fysisk aktivitet. 
\item Brugerfladen af applikationen, samt hvordan patienten selv kan aflæse den registrerede fysiske aktivitet.  
\item Vigtigheden af jævnlig synkronisering af data. 
\item Hvordan batteriniveau aflæses og hvordan det genoplades. 
\end{itemize} 

Yderligere udleveres en brugermanual med armbåndet, som patienten kan anvende i tilfælde af tvivl eller spørgsmål.   

Det registrerede data skal efterfølgende også kunne tolkes af lægen, og det er af den grund igen relevant at lægen har indblik i brugerinterfacets funktioner. For at lægen kan se patients udvikling er det nødvendigt at lægen kan oplære patienten i synkronisering af data, således patienten kan gemme, opbevare og uploade data over en længere periode.


 
\subsection{Efteruddannelse af personale som følge af implementering af aktivitetsarmbånd}
Når en ny teknologi indføres i den almene praksis, skal lægerne eller andet sundhedspersonale i praksissen lære at anvende teknologien, hvis den skal anvendes i et behandlingsforløb. 
Ifølge forskere indenfor læring og teknologiforståelse, som arbejder med forskningsprojektet Technucation, er forståelsen for teknologi vigtig i lægernes erfaring, da det er vigtigt at kunne se og forstå, hvordan teknologier anvendes og udnyttes på bedste vis \citep{aarhusuniversitet2013}. Herfor vil eventuel efteruddannelse kunne ses som relevant for lægerne i almen praksis, hvis et nyt redskab, som aktivitetsarmbånd til objektiv monitorering af patienters fysiske aktivitet, skal implementeres for at sikre, at de har en effekt i deres arbejde og, at de ydelser de medfører for patienterne lever op til forventningerne om at kunne monitorere patienternes fysiske aktivitet objektivt. Bogen Teknologiforståelse - på Skoler og Hosptitaler, der er skrevet i samarbejde med Technucation, beskriver teknologien på forskellige områder. Blandt andet beskrives teknologien som vigtig for at mindske fejl i systemet, da der for eksempel kan undgåes ulæselig håndskrift eller lignende ved anvendelse af teknologi. Ved at implementere aktivitetsarmbånd til monitorering af patienters daglige aktivitetsniveau, vil patienternes daglige fysiske aktivitetsniveau blive registreret på en mere objektiv måde end hvis patienterne selv skal fortælle om deres fysiske aktivitet, hvilket vil være mere subjektivt og derfor mindre troværdigt. Dokumentationen for patienternes fysiske aktivitet vil derfor være mere gyldig for at få et bedre overblik over patienten, hvis der anvendes objektive målemetoder \citep{hasse2012}. 
Efteruddannelsesfonden er etableret, så læger i den almene praksis, som deltager i efteruddannelse, har en konto, hvor lægen kan trække beløb fra til fravær, transport, kursusafgift og undervisningsmateriale. Denne konto er på cirka 13000 kr. Dog skal bestemte kriterier være opfyldt for, at en læge kan få dækket udgifterne til efteruddannelse \citep{vedsted2005}. Et af disse kriterier er eksempelvis, at lægen skal arbejde efter overenskomst om almen praksis mellem Praktiserende Lægers Organisation og Regionernes lønnings- og takstnævn \citep{fondenforalmenpraksis2016}. Efteruddannelse af læger kan passende foregå i samarbejde med de uddannelsesgrupper (tolvmandsforeninger), som ofte indgår i en læges netværk. Der vil i denne forbindelse kunne afholdes foredrag af en foredragsholder relateret til aktivitetsarmbånd og brugen i disse og hvordan de fungerer, for at lægerne får indsigt og uddannelse nødvendig for at kunne anvende dem til diagnosticering eller behandling af patienter \citep{vedsted2005}. 
\\
\subsubsection{Analysering af data fra aktivitetsarmbånd}  
Data opsamlet fra aktivitetsarmbåndet skal analyseres med henblik på at se, inden for den periode patienten har gået med armbåndet, hvor meget patienten har været fysisk aktiv, og om denne fysiske aktivitet opfylder målene for at være aktiv nok.  
Med Fitbit Flex, som er yderligere beskrevet i \autoref{XXXX}, vil den data, der er opsamlet blive overført ved at synkronisere enheden via en trådløs forbindelse, enten til smartphone eller PC, med bluetooth eller ved brug af den trådløse sync dongle, som følger med produktet. Denne synkronisering kan patienten foretage i eget hjem, hvorefter lægen kan se patientens data i klinikken ved at logge på patientens Fitbit bruger konto. Data, der opsamles, inkluderer blandt andet antal skridt gået, hvor lang en distance dette svarer til, antallet af kalorier forbrændt samt diverse grafer, som kan give patienten et overblik over, hvor meget den pågældende patient er aktiv. Dataene vil da kunne ses som tal eller grafisk ved hjælp af software programmet som hører til Fitbit \citep{fitbitflex}. Disse data vil kunne tastes ind i de databaser, som lægerne bruger, hvis dette er relevant for behandlingen af patienten eller som diagnosticerende middel. De data som lægen vil få fra aktivitetsarmbåndet vil være, antal skridt gået eller løbet, distancen, antal aktive timer, samt et estimat på kalorier forbrændt. Disse data vil skulle have en plads i lægens database, hvilket enten vil betyde at allerede anvendte programmer i den almene praksis skal tilpasses de nye data, eller der skal anvendes den allerede eksisterende platform fra fitbit, hvorved der vil skulle implementeres et nyt program i den almene praksis hvor aktivitetsarmbånd anvendes. 

\subsection{Indkøb af udstyr}	
Hvis aktivitetsarmbånd implementeres som en del af behandlingen af hypertension, skal det besluttes hvem, der har ansvar for anskaffelse af udstyret. Sundhedsvæsenet kan eksempelvis stå for indkøb af udstyr for at undgå, at patienten selv skal belastes økonomisk. 
Fitbit Flex kan købes over producentens egen hjemmeside, her er det muligt at forespørge om en større ordre via deres hjemmeside. Detailhandel kan derfor foregå for eventuelt at kunne spare penge ved at handle ind i et større parti af deres produkt.
Patienten kan eventuelt købe aktivitetsarmbånd billigere med en anbefaling fra lægen, eller få udleveret et aktivitetsarmbånd af lægen, som patienten beholder i et bestemt tidsinterval, eventuelt inden et kontrolbesøg hos lægen. Dette kan foregå ligesom ved udlevering af apparat til hjemmeblodtryksmåling, hvor patienten modtager klare instruktioner fra læge eller sygeplejerske inden brug og kan låne udstyret med hjem.

\subsection{Kontakt og information}
Problemer eller spørgsmål kan opstå enten fra læge til virksomheden, der producerer aktivitetsarmbåndet eller fra patient til læge. Herfor er det nødvendigt at kunne imødegå dette for at undgå, at der sker fejl eller misforståelser under forløbet, hvor patienten får målt deres fysiske aktivitetsniveau igennem hverdagen. 

\subsubsection{Mellem praktiserende afdeling/læge og producent}
Den praktiserende afdeling kan ved hjælp af producenten, Fitbits hjemmeside kontakte deres kundeservice på e-mail eller telefon, hvis der opstår problemer med produktet, eller hvis der er brug for at få en viden, som ikke kan findes "på egen hånd". 

\subsubsection{Mellem patient/borger og læge}
Hvis en patient efter at have fået udleveret et aktivitetsarmbånd bliver i tvivl om brugen af denne, vil det være nødvendigt for patienten at kunne kontakte lægen for yderligere information og vejledning. Ved at det er muligt for patienter at kontakte deres læge telefonisk kan mange misforståelser undgås og eventuel fejlbrug af udstyr. Lægen i den almene praksis skal som minimum have én time sat af til telefontid om dagen. Typisk er det i de sene morgentimer lægerne kan kontaktes telefonisk, dog kan der forekmme ventetider alt efter, hvor mange der prøver at komme i kontakt med den pågældende læge \citep{vedsted2005}. 

% kilde1 - Teknologiforståelse i sundhedsvæsnet (aarhusuniversitet2013)\\ 
% kilde2 - Almen praksis i Danmark (vedsted2005)
% En bog vi kan bruge "Teknologiforståelse på skoler og hospitaler", måske :) \\
% kilde3 - Teknologiforståelse på skoler og hospitaler (bog) (hasse2012)\\
% kilde4 - Fibit flex manual (fitbitflex)
% fondenforalmenpraksis2016: http://www.laeger.dk/portal/pls/portal/!PORTAL.wwpob_page.show?_docname=11221408.PDF

%Andre kilder som er værd at kigge på \\\\

%http://www.sst.dk/~/media/408300A6052D4D92BA02D486BD617613.ashx \\
%http://technucation.dk/laeringsaktivteter/laeringsaktivitet-ny-teknologi/ \\
%http://www.sst.dk/~/media/408300A6052D4D92BA02D486BD617613.ashx \\
%http://lvvl.dk/file/241981/vv12.pdf \\

%http://videnskab.dk/kultur-samfund/ny-teknologi-kan-skade-patienter-og-skoleelever-mere-end-den-gavner


\section{Delkonklusion}
Indhold: Dette afsnit vil indeholde en delkonklusion af denne del af MTV'en og dette kapitel, som vil lede frem til en endelig konklusion i syntesen. 

%-------------Økonomi----------------
%%%%%%%%%%%%%%%%%%%%%%%%%%%%%%%%%%%%%%%%%%%%%%%%%%%%%%%%
%\chapter{Økonomi}
%Dette afsnit omhandler det økonomiske aspekt ift. MTV analysen.
\chapter{Økonomi}

\section{Metode}
% Hvad koster den patientgruppe for samfundet? Hvad koster det, hvis de ikke passer deres sygdom ordentlig (i forhold til motion), og dermed får følgevirkninger (fx hospitalsophold, medicin)? (Cost/Benefit analyse??)
% Cost-effectiveness analysen (konsekvenserne måles i naturlige enheder)
%Cost-utility analysen (konsekvenserne måles i kvalitetsjusterede leveår (QALYs))
% Cost-benefit analysen (konsekvenserne opgøres i kroner og øre)
% Hvordan er omkostningerne sammenlignet med alternativerne?
% I forlængelse af, hvad aktivitetstrackeren skal kunne: Er de billige tilstrækkelige, eller er det nødvendigt at købe de dyre?
% Brugerbetaling eller egenbetaling??

I økonomianalysen undersøges hvilke omkostninger der er forbundet med anvendelse af aktivitetsmåler som dokumenteringsenhed for aktivitet i den almene praksis/medicin.
Ligeledes undersøges omkostninger for nuværende anvendelsesmetoder, samt hvilke økonomiske konsekvenser der forekommer når patienten ikke opretholder anbefalet aktivitetskvote.
Dette er med henblik på at fremhæve sundhedsgevinsterne i forhold til udgifterne.   
Omkostningerne og konsekvenser er opgjort af sundhedsøkonomiske analyser, som cost-effectiveness analyse (CEA), cost-utility analyse (CUA) og cost-benefit analyse (CBA), og oplyses i henholdsvis narturlige enheder (f.eks. vunde leveår), kvalitetsjusterede leveår og korner øre. 
De estimeret værdier fra de forskellige analyser er baseret på eksisterne litteratur fundet ved anvendelse af søgeprotokol, samt basale økonomiske udregninger.    

\subsection{Spørgsmål}
%Eksempler
%\begin{itemize}
%\item Hvad vil forskellige modeller for vaccinationsprogram have af nytteeffekten i forhold til omkostninger?

%\item Hvordan ville en eventuel screening påvirke organisering og økonomi? 

%\item Hvad er de ressourcemæssige konsekvenser?
%\end{itemize}

\noindent
Økonomisk relateret MTV-spørgsmål:  
\begin{itemize}
\item Hvilke økonomiske ændringer er der forbundet ved udlevering af aktivitetsmålere til patienter med ???.

\item Hvordan ville et aktivitetsarmbånd påvirke organisering og økonomi?

\item Hvilke omkostninger er forbundet med kvantificering af patientaktivitet, i forhold til nuværende anvendelsesmetoder (Aktivitetslog)?.  

\item Hvad er omkostningerne ved nuværende anvendelsesmetoder, samt konsekvenserne ved utilstrækkelig aktivitetsydelse? 
\end{itemize}
 




% Indhold i afsnittet
%\input{contents/hMTV/fOekonomi/temp_oek_indhold}
\section{Omkostninger i sundhedssektoren}
Det er relevant at se på omkostningerne i sundhedssektorens primære og sekundære sektor ved brug af den nuværende målemetode. 

\subsection{Primær sektor}
Den subjektive målemetode, der på nuværende tidspunkt benyttes af $27,7~\%$ af praktiserende læger, foregår ved spørgeskema under en konsultation, medfører udgifter i den primære sundhedssektor \citep{munck2007}. Afhængigt af antal konsultationer, som den enkelte kronikere har behov for, kan et spørgeskema følge med hver konsultation, og omkostningerne til denne målemetode vil derved stige. Udarbejdelse og udskrifter af et spørgeskema vil have relativt lave omkostninger.
\citeauthor{munck2007} udarbejdede i 2007 en rapport om hypertension i almen praksis. Her blev 159 kontaktpersoner i almen praksis spurgt: "Sætter I jeres hypertensionspatienter til kontrol med fast tidsinterval? Hvis ja, angiv det typiske interval". Her svarede 92,5 \%, at de sætter deres patienter til kontrol med et fast tidsinterval, og i gennemsnit er dette interval udregnet til 3,9 måneder \citep{munck2007}. 

Hvis det, jævnfør \citeauthor{kronborg2008}, antages at $1/5$ af den voksne danske befolkning har hypertension, vil dette svare til omkring 900.000 danskere \citep{folketal2016}. Hvis 900.000 danskere skal til lægekonsultation á 137,93 kr hver 3,9. måned, vil dette svare til en årsomkostning for sundhedssektoren på omkring 380 millioner kr. 

Den samlede medicinudgift i den primære sundhedssektor i Danmark i 2014 lå på 11,6 mia kroner, og Danmarks Statistik påpeger i denne sammenhæng, at blodtrykssænkende medicin og hjertemedicin er nogle af de mest anvendte former for medicin i Danmark \citep{dst2016}. 

\subsection{Sekundær sektor}
Tal fra landspatientregistret viser, at der i 2012 var 310 ambulante besøg i forbindelse med blodtryksforhøjelse af ukendt årsag i den private sektor. Dette tal steg i 2013 til 639 ambulante besøg. I 2014 steg tallet yderligere til 1186 i 2014, hvilket vil betyde en stigning på cirka 85,6 \% \citep{sundhedsdatastyrelsen2016} . Med denne stigning vil udgifterne for antal henvendelser i almen praksis dermed også stige proportionelt. 

\section{Omkostninger ved implementering af Fitbit Flex} \label{sec:armbaand_omkost}

Implementering af Fitbit Flex til patienter, som lider af hypertension, vil kunne medføre og antageligvis ændre de økonomiske udgifter i både den primære og sekundære sundhedssektor. I den primære sundhedssektor, som består af de praktiserende læger som behandlere, er det relevant at belyse de direkte omkostninger, der vil kunne forekomme i forbindelse med indkøb og implementering af teknologien, samt de indirekte omkostninger i form af, hvordan både den primære og sekundære sektor påvirkes i tilfælde af, at teknologien har en gavnlig effekt for patienterne. Hertil skal det overvejes, om teknologien skal være betalt af det offentlige, eller om den skal være brugerbetalt. 

\subsection{Direkte omkostninger} \label{sec:dir_omkost}
De direkte omkostninger er alle  udgifter, som relaterer til teknologien for eksempel ved indkøb af Fitbit Flex, implementering og brug i den almene praksis.  
Indkøbsprisen for et Fitbit Flex-armbånd er $749$ kr. hvis det købes gennem producentens hjemmeside. Dette beløb dækker over de omkostninger, der vil være tilknyttet til privat køb. 
Ved køb gennem sundhedssektoren betragtes dette som et virksomhedskøb, hvortil der ikke pålægges momsafgifter, hvilket udgør $20~\%$ af den opgivne pris. 
Der kan eventuelt indgås købsforhandlinger hos producenten, hvortil aftale om yderligere besparelse kan besluttes ved køb af større parti af produktet. Hvis produktet købes gennem sundhedssektoren, vil betaling af produktet med offentlig økonomisk støtte være mest hensigtsmæssig, da udstyret antageligvis kun vil blive udlånt til patienterne i over en periode og derefter leveret tilbage, hvorefter det kan genbruges af andre patienter. Det kan være fordelagtigt at indføre en testperiode, hvor en praksis indkøber et parti af produktet og efterfølgende følger op på patienterne før, der købes et endeligt antal af produktet for at sikre effekten. 

Udgifter, der kan være forbundet med en implementering, er efteruddannelse af personale som teknologien kommer til at berøre. Efteruddannelsesfonden stiller årligt $13.000$ kr. til rådighed \citep{vedsted2005}. Efteruddannelse af personale beskrives ligeledes i \autoref{sec:efteruddannelse}.   
Ved anvendelse af Fitbit Flex vil der opstå udgifter, der relaterer sig til udlån og introduktion af aktivitetsarmbåndet til patienterne. Til at bedømme omkostningerne forbundet ved brug af Fitbit Flex, tages der udgangspunkt i honorartabellen fra PLO \citep{honorartabel2016}. Tabellen er et opslag over, hvilke honorarer der kan gives praktiserende læger ved forskellige ydelser.
Da der ikke forekommer nogle direkte omkostninger i forhold til udlån og introduktion af aktivitets armbånd, sammenlignes der i stedet med udgifterne ved hjemmeblodtryksmåling. Udlevering af og introduktion til udstyr til hjemmeblodtryksmåling vil typisk forekomme ved en konsultation, hvorfor der også vil blive lagt honorar til dette. Honoraret for en konsultation ligger på $137,83$ kr. som tidligere nævnt i \autoref{sec:nuv_primaer}. Udlån og instruktion af hjemmeblodtryksmåler har et honorar på $141,68$ kr. hvilket vurderes at være sammenligneligt med et aktivitetsarmbånd, da det cirka omfatter de samme procedurer, dette kan eventuelt forekomme flere gange efter behov hos patienterne. Dette er et samlet honorar på $279,51$ kr. Hvis patienten bliver i tvivl om brugen af aktivitetsarmbåndet eller har flere spørgsmål, vil en telefonkonsultation være relevant at medtage; honoraret per telefonkonsultation er på $26,99$ kr.

\subsection{Indirekte omkostninger} \label{sec:indir_omkost}
De indirekte omkostninger vil typisk være de omkostninger, som er relateret til medicinforbrug, adminstration, samt forbedringer ved patienternes helbred som resultat af teknologien. 
Som det fremgår af \autoref{sec:primaer_sektor_omkostninger} udleveres $563$ milioner DDD af det mest populære blodtrykssænkende medicin fra apoteket. Hvis implementeringen af aktivitetsarmbånd har en positiv virkning på patienternes helbred, og dermed blodtryk, i og med, at de dyrker mere motion og får en sundere livsstil, vil der kunne skæres ned på behovet for medicin nødvendig for at behandle patienterne. Omkostningerne for disse DDD, vil dermed også kunne reduceres. Yderligere vil bivirkninger ved medicinen reduceres, hvilket vil øge patienternes livskvalitet, lette presset for behandlerne og forebygge bivirkningsrelaterede indlæggelser. 
Videresendelser af hypertensive patienter til den sekundære sektor vil ligeledes kunne blive reduceret, hvorfor patienterne kan forblive i et forløb som kun vedrører den primære sundhedssektor.   %\textbf{Det bliver lidt "punktet", det skal måske lige omformuleres lidt så der kommer et bedre flow i teksten :) }

\begin{comment}
Hvad koster et Fitbit Flex? 
Hvilke besparelser tilbydes der så sundhedsektoren? 

Hvad koster det så at introducere patienterne til teknologien? 
	Hvad dækker den her introduktion minimum over, for at kunne anvende armbåndet? (Anvendelse af app og hvordan den skal oplades.)
	
Hvad koster det hvis de har spørgsmål vedr. teknologien? 


%Diskussion orienteret 
Langsigtet omkostninger - hvis behandlingen hjælper/ikke hjælper
- Besparelser vedr. medicinering 
- Besparelser vedr. ambulant forløb 
- Forebyggelse af behandlingsresistent hypertension = $$$$



EVT: Dags-takster i sekundær sektor (Ambulant).



En model i almen praktsis for implementeringen af aktivitetsarmbånd?

Honorartabel = \citep{honorartabel2016}
\end{comment}
\subsection{Omkostninger i sundhedssektoren}
% har kombineret primær og sekundær sundhedssektor - gav ikke mening at adskille.. 

%Indhold: I dette afsnit vil vi gerne analysere og belyse mulige udgifter/besparelser ift. et muligt antal (reducerede) henvendelser til almen praksis ved benyttelse af ny teknologi, hvis dette er muligt at estimere. Herunder vil vi se på, om man vil kunne "nøjes" med telefonsamtaler eller emails for nogle af kontrollerne i stedet for egentlige konsultationer. 

%Indhold: I dette afsnit vil vi gerne analysere og belyse udgifter/besparelser ift. et muligt antal (reducerede) hospitalsindlæggelser ved benyttelse af ny teknologi, hvis dette er muligt at estimere.

Ved en implementering af Fitbit Flex, vil dette medføre ændringer i strukturen af den primære og sekundære sundhedssektor, som nævnt i \autoref{sec:org_aendringer}, hvor nogle mulige organisatoriske ændringer er beskrevet. Med de organisatoriske ændringer, vil der ligeledes følge ændringer i udgifter til sundhedssektoren. 
Håbet ved en implementering af Fitbit Flex er, at den vil medføre færre, økonomisk tunge, henvendelser til det offentlige. 

I den primære sektor, vil dette kunne ske, hvis antallet af blodtrykskontroller faldt. Dette kunne eksempelvis ske, hvis nogle fysiske kontroller erstattedes med telefoniske konsultationer, da honoraret for en telefonisk konsultation er 110 kroner lavere end en fysisk konsultation. Ud over denne økonomiske gevinst, skal der tages højde for, at der vil kunne være et øget antal henvendelser til egen læge i forbindelse med vurdering af patientens egnethed til brug af Fitbit Flex, udlevering og instruktioner i brugen af aktivitetsarmbåndet. 
 
Ligeledes vil der muligvis kunne være økonomiske besparelser, hvis en øget mængde fysisk aktivitet vil kunne reducere mængden af henvendelser til den sekundære sundhedssektor. Dette er muligt, da et ambulant besøg er 1.283 kroner dyrere end en konsultation, eller 1.394 kroner dyrere end en telefonisk konsultation ved egen læge. Hvis antallet af hospitalsindlæggelser endvidere kan mindskes, vil udgifterne dertil falde med 12.597 kroner per dag, det ikke er nødvendigt ikke at indlægge en patient. 

Regionerne

På denne måde, vil der kunne følge besparelser med en implementering af Fitbit Flex, der muligvis vil kunne udligne udgifterne til implementeringen af aktivitetsarmbåndet, som nævnt i \autoref{sec:armbaand_omkost}. 

Økonomi-afsnittet bliver nok primært spekulativt, da der er mange faktorer, som kan være voldsomt svære at estimere, men jeg synes, der er nogle fine betragtninger med. Tænk specielt over, at det godt kan være, at man har øgede omkostninger forbundet med armbåndet i primærsektoren, men at det man sparer i sekundærsektoren og i medicinudgifter opvejer dette på samfundsniveau. Det kan jo også være, at konklusionen simpelthen er, at det er for dyrt i forhold til gevinsten. En hård økonomisk konklusion kan nok være svær at nå til.
\section{Delkonklusion}

Indledningsvist vil implementeringen af Fitbit Flex i den almene praksis give en udgift i forhold til de anvendte metoder relateret til aktivitetsregistrering. Udsigterne for eventuel besparelse på hypertensionsområdet vil være udtrykt ved en reduktion i brugen af medicin, samt udskydelse af følgesygdomme, såfremt der opnås en positiv effekt ved implementeringen af aktivitetsarmbåndet. Omkostningerne ved en implementering vil give sig til udtryk i form af direkte omkostninger til indkøb, efteruddannelse, samt vedligeholdelse af Fitbit Flex. Anvendelse af Fitbit Flex antages at være tilsvarende udgifter som ved anvendelse af hjemmeblodtryksmonitorering. 

Såfremt Fitbit Flex har en positiv virkning på aktivitetsniveauet hos hypertensive patienter, vil der potentielt forekomme besparelser i den sekundære sundhedssektor i form af færre ambulante besøg og indlæggelser på hospitalet.

Ligeledes vil kunne forekomme besparelser i den sekundære sundhedssektor, samt i forbindelse med medicinudgifterne ved en reduktion i DDD. Hertil kan en testperiode med Fitibit Flex i den almene praksis være værd at overveje for at sikre, om teknologien har den tiltænkte effekt på hypertensive patienter.  

%-------------Syntese--------------------
\part{ Syntese}
%\chapter{Syntese}
\chapter{Diskussion}
Indhold: Dette afsnit skal indeholde en diskussion af rapportens indhold med udgangspunkt i problemformuleringen. 

\section{Metode}
Litteraturen har været valgt ud fra inklusions- og eksklusionskriterier. Disse kriterier har varieret efter formål og vidensbehov for de forskellige analyser. Det har ved manglende litteratur, været nødvendigt at opstille flere inklusionskriterier, med henblik på at finde anvendelig og relevant litteratur. Dette har dog haft betydning for antagelser og konklusioner der har kunne tages under de forskellige analyser, hvortil bestemte områder kun forbliver spekulative. 

Den inkluderede litteratur er blevet sammenholdt med det opstillede evidensniveau, for at vurdere kildens troværdighed. Ranglisten er ikke blevet anvendt med henblik på at give hver enkelt kilde en bestemt rang, men blot på at give et indtryk af kildens evidensniveau. Såfremt det ikke har været muligt at finde kilder af høj evidens, er der i stedet blevet fundet flere lavt rangeret kilder med samme udtalelser og konklusioner, for at styrke validiteten af kilderne. 

Nogle af analyserne kunne gøres mere omfattende ved udarbejdelse af interviews eller spørgeskemaer, dog har dette ikke været anset nødvendigt grundet tilgængeligt litteratur har været tilstrækkeligt.

Observationer foretaget af Fitbit Flex har kun været relateret til Fitbit applikationen. Hertil har det været muligt at få indblik i, hvordan data synkroniseres og formidles til brugeren, dog kunne en mere omstændig observation have været foretaget ved indkludering af selve Fitbit Flex armbåndet. Dette ville kunne have bidraget til en beskrivelse af brugervenligheden, samt hvor hurtigt armbåndet reelt drænes for strøm.  

\section{Teknologi} \label{sec:dis_teknologi}
Valget af Fitbit Flex er et simpelt alternativt i forhold til andre aktivitetsarmbånd på markedet. 
Grundet armbåndet kun anvender et accelerometer til registrering af aktivitet, er det kun velegnet til gang eller løb, hvorfor mange andre aktivitetstyper ikke kan registreres. 
Nødvendigheden for at kunne tracke andre aktivitetstyper kan dog vurderes som værende af mindre betydning, da målgruppen er hypertensive patienter, hvortil tilfælde af hypertension stiger med alderen. 
Dermed er det primært ældre, der vil være målgruppe, hvor gang og løb er anbefalet for at mindske risioen for skader, samt at dette i forvejen er den mest udøvede aktivitetstype blandt ældre. 

Til trods for at Fitbit Flex armbåndet er i stand til at registrere gang og løb, er den ikke velegnet til formidling af information, da den kun kan kommunikerer via 5 LED’er og vibration. 
Dette gør at patienten skal have adgang til enten smartphone eller PC, for at se konkret information omkring aktivitet eller batteritilstand. 

Der findes andre armbånd fra Fitbit, samt andre producenter, der har et mere omfattende display, således det er muligt for patienten at få de førnævnte informationer via armbåndet. 
Dertil ville armbåndet blive en mere selvstændig enhed og kun være afhængig af andet elektronik ved synkronisering af data. Anvendelsen af én enhed til sammenhold af informationer kan dog påvirke formålet med simpliciteten af armbåndet til gavn for målgruppen. 

På smartphone-enheden bliver data formidlet på engelsk, hvilket kan ses som en begrænsning for den kun dansk- eller andet sprogtalende population blandt målgruppen. Hertil kan en kontrolgruppe være relevant til undersøgelse af behovet for tilpasning af sproget, eller om den nuværende brugerflade er intuitiv. 

Ved jævnlig synkronisering kan der bevares et detaljeret overblik over registreret aktivitet. 
Hertil vides dog ikke om dette er et behov, for at lægen at kan få korrekt indblik i patientens aktivitetsniveau. Detaljeret data kan dog være en fordel ved vejledning omkring aktivitet, såfremt patienten opfylder det anbefalede aktivitetsniveau. 

Pålideligheden af Fitbit Flex er også relevant, da der ideelt ønskes et armbånd, der registrerer aktivitet $100~\%$ korrekt, hvilket dog ikke er muligt. Hertil tillades individuel tilpasning af armbåndet, således en bedre repræsentation af aktivitetsniveau opnås. 

Et af argumenterne for valget af Fitbit Flex var præcisionen ved måling af gang, og at den havde en tendens til at underestimere frem for at overestimere antal skridt taget. Dette er anset som en fordel, da dette er med til at sikre at patienten opnår det ønskede aktivitetsniveau.
\section{Patient} \label{sec:dis_patient}
På trods af, at \citeauthor{mercer2016} beskriver brug af aktivitetstrackers i en gruppe af kronisk syge, der er over 50 år, kommer denne artikel ikke med en endelig konklusion på anvendeligheden af disse. Dette skyldes, at patienterne i artiklen og denne rapports målgruppe, har blandede forhold til modtagelse og brug af teknologien. Størstedelen har imidlertid haft et godt udbytte og følt sig motiveret af teknologien, mens andre har følt sig meget udfordrede i brugen af en ny teknologi, da de ikke er vant til dette i deres dagligdag. Det er herved vigtigt at overveje, hvordan ikke-teknologivante patienter bedst muligt kan benytte teknologien; eksempelvis ved hjælp fra familie, venner eller plejepersonale. Dette skal undersøges yderligere før en eventuel implementering af Fitbit Flex. 

Det er værd at reflektere over, om Fitbit Flex har en høj motivationsfaktor, eftersom den visuelle feedback i form af armbåndets LED'er er begrænset i forhold til andre aktivitetsarmbånd. 
Teknologivante patienter vil muligvis have et større udbytte af aktivitetsmonitoreringen, hvis armbåndet gjorde brug af GPS, da dette giver mere præcise målinger af tilbagelagt afstand samt medfører, at aktivitetsarmbåndet kan benyttes til monitorering af andre aktivitetsformer end gang og løb. Ved overvejelse af en implementering af aktivitetsarmbånd med GPS, skal de tilhørende etiske aspekter i form af en følelse af overvågning undersøges.
Nogle patienter vil kunne have gavn af flere informationer direkte fra armbåndet, heriblandt antal aktive timer og tilbagelagt afstand, som en yderligere motivation - dette vil kunne lade sig gøre igennem de enheder, som Fitbit Flex synkroniserer med, men informationen kan ikke tilgås fra armbåndet i sig selv. Det skal hertil vurderes, om enkelheden af Fitbit Flex er en fordel for nogle patienter, da dette gør aktivitetsarmbåndet simpelt at interagere med for de, der muligvis ikke er vant til brug af ny teknologi. 


\section{Organisation} \label{sec:dis_organisation}

Fitbit Flex ikke være tilbøjeligt til at skabe ændringer i patientens hverdag, eftersom teknologien ikke er nær så krævende og forstyrrende at bære, som andre teknologier, der kan lånes med hjem fra lægen; eksempelvis EKG-apparat med elektroder eller døgnblodtryksmåler, som aktiverer blodtryksmanchetten med et fast interval. Udlevering af Fitbit Flex vil derfor kræve, at patienten introduceres til teknologien, hvorefter forløbet vil være tilsvarende udleveringen af andet monitoreringsudstyr i en periode, eksempelvis over en måned.

En anden mulighed for implementering vil være, at patienterne selv anskaffer armbåndet, hvorved aktivitetsmonitoreringen ikke begrænses til perioden, hvor armbåndet er udlånt af sundhedssektoren. Dette vil envidere sikre, at motiveringsfaktoren beskrevet i \autoref{sec:dis_patient} opretholdes længere end perioden, hvor armbåndet er udleveret. Af denne grund vil det være relevant at undersøge muligheden for at skabe en tilskudsordning, således patienterne har mulighed for at købe et armbånd til nedsat pris efter endt monitorering med det udlånte armbånd.

%Andre muligheder for implementering vil være, at patienterne selv anskaffer armbåndet efter endt monitorering med det udlånte armbånd, hvorved aktivitetsmonitoreringen ikke begrænses til perioden, hvor armbåndet er udlånt af sundhedssektoren. Dette vil envidere sikre, at motiveringsfaktoren beskrevet i \autoref{sec:dis_patient} opretholdes længere end perioden, hvor armbåndet er udleveret. En rutinegørelse af bevidstheden om fysisk aktivitet hos patienterne vil herved kunne opnås ved guidance fra armbåndet. Af den grund vil det være relevant at skabe en tilbudsordning, således patienterne har mulighed for at købe et armbånd med rabat.

I tilfælde hvor patienten oplever problemer med teknologien, eller har spørgsmål vedrørende funktionen, bør hverken læger eller sygeplejersker agere support-personale, som følge af blandt andet høje udgifter og et til tider presset arbejdsskema. Her er det muligt at kontakte Fitbit direkte på deres engelske hjemmeside, eller det sted som udbyder udstyret, og som alternativ kan lægesekretærer introduceres til armbåndet, således de kan hjælpe med eventuelle problemer.

Fitbit Flex kommer ikke til at fungere som et alternativ til den nuværende behandling, men blot som et supplement og potentiel erstatning for blodtryksmedicin ved let hypertension. Som det ses på \autoref{fig:behandlingsvejl} vedrørende behandlingsforløb i relation til forskellige grader af hypertension, ses det at alle hypertensive patienter, uanset graden, har gavn af livsstilsændringer. Derfor vil armbåndet være anvendeligt som både supplement og potentiel erstatning for andre behandlingsmetoder afhængigt af graden af hypertension. Såfremt det skal erstatte blodtryksmedicin ved mild grad af hypertension, kræver det en undersøgelse af effekten på aktivitetsniveauet og blodtrykket hos patienter, som får udleveret armbåndet.

Armbåndet kan, grundet motivationsfaktoren beskrevet i \autoref{sec:dis_patient} øge aktivitetsniveauet, og derfor potentielt have en positiv indvirkning på blodtrykket hos patienterne, hvorfor det kan reducerende antallet af videresendte patienter fra primær til sekundær sektor. Dette vil have en positiv effekt på den sekundære sektor ved at lette arbejdsbyrden og spare penge i forbindelse med blandt andet indlæggelser og undersøgelser.

For at sikre effekten af aktivitetsarmbånd som monitorerings og motivatonsfaktor til hypertensive patienter, vil en testperiode kunne være relevant, at indføre på få klinikker til at starte med og derefter udvide til flere, hvis disse viser positive resultater.
\section{Økonomi} \label{sec:dis_oekonomi}
Økonomianalysen har til formål at klarlægge mulige omkostninger og besparelser ved implementering af Fitbit Flex. I analysen estimeres disse omkostninger og besparelser, hvorfor der ud fra analysen ikke er nogen endelig konklusion på, om teknologien er omkostningseffektiv. Det vil være gavnligt med en større viden omkring sundhedsøkonomi, når det skal undersøges, hvad omkostningerne ved for lavt aktivitetsniveau for patienter med hypertension er, hvad omkostningerne ved implementering af Fitbit Flex er, samt de sundhedsmæssige gevinster ved brug af et aktivitetsarmbånd til monitorering i almen praksis. 

Da der i analysen er benyttet estimater, er der ikke blevet lavet egentlige cost-effectiveness eller cost-utility analyser. Disse ville være relevante at udarbejde, hvis værdierne for omkostninger og gevinster, i form af både vundne leveår, QALY og valuta, var mere valide. En cost-effectiveness analyse vil være værdifuld for en beslutningstagen omkring implementeringen af Fitbit Flex, da denne værdisætter omkostninger og konsekvenser af henholdsvis den subjektive monitoreringsmetode og aktivitetsarmbåndet. Yderligere vil en cost-utility analyse kunne benyttes for at fokusere på kvaliteten af vundne leveår, hvis fysisk aktivitet øger patientens livskvalitet.

Ved udarbejdelse af de førnævnte analyser vil beslutningsmatricen på \autoref{fig:cost_effect} kunne anvendes. 

\begin{table}[H]
	\centering
	\includegraphics[width=0.9\textwidth]{figures/cost-effectiveness}
	\caption{Beslutningsmatrice ved brug af cost-effectiveness og cost-utility analyser. $E$ står for effect, men kan også benyttes til utility, og $C$ står for omkostninger. $n$ er den nye teknologi, og $g$ er den gamle \citep{mtvhaandbog}.}
	\label{fig:cost_effect}
\end{table}

\noindent
Ud fra estimaterne i økonomianalysen, vil Fitbit Flex have potentiale til at have en højere effekt end den nuværende monitoreringsmetode. De direkte omkostninger beskrevet i \autoref{sec:dir_omkost} vil være højere ved den nye teknologi, mens det er uklart, om de indirekte besparelser beskrevet i \autoref{sec:indir_omkost} er høje nok til at dække de direkte omkostninger. Hvis dette er tilfældet, vil situationen svare til beslutningsudfald 8, hvor den nye teknologi skal implementeres. Hvis omkostningerne er højere ved den nye teknologi, vil dette svare til beslutningsudfald 9, hvor der ingen klar beslutning fremgår. 
\section{Samlet diskussion}

For at kunne implementere Fitbit Flex til udredning og behandling af hypertensive patienter skal der gøres en række overvejelser. Disse omhandler blandt andet tidshorisonten for motivation ved brug af armbåndet, implementeringsmetoden og patientgruppens accept af den nye teknologi. 

Ved implementeringen skal det fremhæves, at armbåndet ikke fungerer som et fuldkomment alternativ til den nuværende behandling, men som en motivationsfaktor til livsstilsændringer, der kan udgøre den non-farmakologiske del af behandlingen. Jævnfør \autoref{sec:beh_hypertension}, vil livsstilsændringer gavne patienterne uanset, hvilken grad af hypertension patienten lider af, hvorfor det er relevant at motivere dem til en mere aktiv hverdag, samtidig med at opnå større mulighed for monitorering af livsstilsændringer.

I forbindelse med livsstilsændringerne er den mulige motiverende faktor, som beskrevet i \autoref{sec:dis_patient}, vigtig at tage med i overvejelserne, når implementeringsmetoden skal vælges. Ved hjemlån af armbåndet, vil lægen grundet denne motiverende faktor ikke nødvendigvis kunne få et præcist billede af patientens almindelige fysiske aktivitet, hvis patienten dyrker mere motion i monitoreringsperioden. Af den grund vil en undersøgelse af motivationsfaktoren, samt hvorvidt denne aftager efter endt monitoreringsperiode, kunne danne grundlag for en evidensbaseret konklusion vedrørende langtidseffekten af implementeringen.

Ved en undersøgelse af videre aktivitetsmønster efter endt monitoreringsforløb, kan det i samme omgang overvejes at undersøge brugervenligheden og præcisionen af det nye Fitbit Flex 2. Dette kan med fordel gøres således, at den nyeste tilgængelige teknologi implementeres, hvis undersøgelser viser positive resultater ved brug af Fitbit Flex 2. Vedrørende valget af Fitbit Flex, vil det være en fordel, at armbåndet ikke har et display, som viser brugerens progression. Dette kan gøre at brugeren er mere tilbøjelig til at synkronisere med smartphone eller PC, som følge af nysgerrighed vedrørende egen aktivitet, hvorved brugeren ikke glemmer at uploade data. Havde uret haft indbygget display, kunne brugeren nøjes med at kigge på uret, for at holde sig opdateret vedrørende aktivitetsniveauet, og dermed glemme at synkronisere med Fitbit-kontoen.

Med henblik på at opretholde den motiverende faktor i form af den visuelle repræsentation af egen aktivitet, kan det samtidig overvejes en anden implementeringsmetode; at give patienterne muligheden for at købe et Fitbit Flex-aktivitetsarmbånd med tilskud gennem sundhedsvæsenet. Dette vil formentlig kunne vedligeholde motivationen til at fortsætte livsstilsændringerne i form af øget aktivitetsniveau, men potentielt vil patienter også kunne anvende søvn- og fødevaretracking, for at opnå større indsigt i egen sundhed. Her skal det samtidigt nævnes, at armbåndet, hvis det vurderes omkostningseffektivt, bør implementeres på lige fod med hjemmeblodtryksmålere og EKG-målere til brug i hjemmet, hvor sundhedsvæsenet har ansvaret for indkøb og vedligehold. Således oplever patienten ikke udgifter i forbindelse med monitorering af aktiviteten, førend de overvejer at købe et armbånd efter endt monitoreringsperiode.
\section{Konklusion}
Indhold: Dette afsnit skal indeholde en konklusion af rapporten med udgangspunkt i delkonklusionerne fra teknologi, patient, organisation og økonomi og problemformuleringen. 
\subsection{Anbefalinger}
Indhold: Dette afsnit kan indeholde nogle anbefalinger, hvis rapportens konklusion ender med at lede frem til nogle.

%-------------Bibliografi--------------------

\begingroup \label{litteratur}
\raggedright
\bibliographystyle{bibliography/unsrtnat}
\bibliography{bibliography/bibliography}
\endgroup


%-------------Bilag-------------------------
\begin{appendices}
\chapter{Patientforløb}

\begin{figure}[H]
\centering
\includegraphics[width=0.9\textwidth]{figures/patientforloeb2}
\caption{Flowchart af patientforløbet fra mistanke om hypertension til behandling og efterfølgende hypertensionskontroller. Illustrationen viser den beslutningstagen, der følger med diagnosticeringen af hypertension, samt sammenspillet mellem sundhedssektorens forskellige dele; almen praksis og den sekundære sektor.}
\label{fig:patientforloeb}
\end{figure}
\end{appendices}



\end{document}