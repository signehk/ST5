\section{Metode} \label{metode}
I denne rapport anvendes kombinationen af AAU model og Medicinsk teknologi vurdering (MTV). Følgende afsnit beskriver disse og kombinationen af hvordan dette anvendes i projektet. 

Da denne rapport er sammensat med udgangspunkt i en medicinsk problemstilling, er det med fordel at kombinere disse modeller..

På \ref{fig:metodemodel} ses en sammensætningen af AAU- modellen og MTV-modellen, som illustrerer opbygningen af dette projekt.

\begin{figure}[H]
	\centering
	\includegraphics[width=0.5\textwidth]{figures/metodemodel}
	\caption{Model for den brugte metode i projektet.}
	\label{fig:metodemodel}
\end{figure}

AAU modellen er problembaseret og starter meget bredt med en initierende problemformulering, hvoraf der fortages en videregående problemanalyse, for at fremhæve omfang, konsekvenser og nuværende løsningsmidler. 
Af analysen foretages der en yderligere indsnævring af problemstillingen, for at opstille en endelig problemformulering. Denne formulering vil omhandle hvorvidt en ny teknologi vil afhjælpe problemstillingen, som problemanalysen belyser, og forsøges besvaret gennem en teknologivurdering. 



Til teknologivurdering benyttes medicinsk teknologivurdering (MTV), med udgangspunkt i den relaterende håndbogen \citep{mtvhaandbog}. 
MTV'en belyser forskellige afspekter af teknologien ved at inddele vurderingen i fire områder: \textbf{teknologi}, \textbf{patient}, \textbf{organisation}, og \textbf{økonomi}. Områderne uddybes nødvendigvis ikke ligeligt, da teknologivurderingen kun er MTV-inspireret. 
Hvert område vil have et indledende metodeafsnit, for beskrive hvilken tilgang der tages under de forskellige områder, såsom analysemetoder og fokuserede spørgsmål. 

Teknologiafsnittet vil beskrive den valgte teknologi, og hvilke variationer af teknologien der eksistere i dag. En sammenligning af variationerne vil blive fortaget, med henblik på at fremhæve fordele og ulemper. Yderligere vil teknologien også blive sammenlignet med de nuværende løsningsmuligheder der anvendes i dag, for at se hvordan de adskiller sig fra hinanden.  

Patientafsnittet i MTV’en undersøges den afgrænsede patientgruppe nærmere i forhold til teknologien. Der undersøges blandt andet om teknologien vil have en betydelig påvirkning på patienternes hverdag, og om den kan forbedre deres livskvalitet. Yderligere undersøges om eventuelle etiske problemstillinger forekommer ved anvendelse af teknologien.

Den organisatoriske analyse vil hovedsageligt behandle ændringer i interaktionen mellem patienter og sundhedspersonale, samt det organisatoriske aspekt i forhold til samarbejdet mellem forskellige sundhedsinstitutioner: primær og sekundær sundhedssektor.

Det økonomiske aspekt blive undersøgt, med udgangspunkt i at finde frem til omkostningerne relateret til de teknologiske løsninger, som er undersøgt i teknologianalysen. 
Dette omhandler eventuelle besparelser eller ekstraudgifter, der kan forekomme ved implementering af den nye teknologi.



Analysen af de fire MTV-områder vil dernæst blive anvendt i syntesen, der indeholder en diskussion med udgangspunkt i fordele og ulemper ved både den nuværende og den undersøgte teknologi. 
Afsluttende vil konklusionen fremhæve om teknologien kan anvendes i relation til problemstilling, og dermed besvare den endelige problemformulering.




MTV’en vil primært blive dokumenteret ved brug af videnskabelig litteratur fundet fra forskellige videnskabelige databaser. For at overskueliggøre dette vil der sideløbende med MTV’ens udformning blive udarbejdet en søgeprotokol. I søgeprotokollen vil der blandt andet være inklusions og eksklusionskriterier for at kunne fokusere søgningen til det mest relevante litteratur i forhold til de fire områder i MTV’en. Formålet med søgeprotokollen er dels at få et overblik over de kilder, der anvendes og for at kunne dokumentere MTV’ens indhold, da det er muligt ved hjælp af søgeprotokollen at se hvor, hvad og hvordan der er søgt litteratur, hvorved det er muligt at genskabe MTV’ens indhold. Søgeprotokollen findes i \autoref{app:soegeprotokol}.



%\section{Metode} \label{metode}

%Denne medicinske teknologivurdering (MTV) vil afvige fra opbygningen beskrevet i MTV-håndbogen, som følge af projektet samtidig indeholder elementer fra problembaseret læring (PBL). Som et resultat af blandingen, tages der udgangspunkt i en medicinsk problemstilling, som analyseres for at udarbejde en problemformulering. Analysen i forbindelse med PBL-tilgangen vil desuden indeholde MTV-elementer såsom etik, målgruppe og interessentanalyse.

%Efterfølgende vil problemformuleringen skabe grundlag for at arbejde videre med MTV-håndbogens elementer. Her vil de fire områder, teknologi, patient, organisation og økonomi, blive anvendt til at stille mere konkrete spørgsmål. Fremgangsmåden betyder at der vil være tale om en problem- og teknologiorienteret MTV, da der søges at finde en løsning på et medicinsk problem gennem en vurdering af, hvorvidt en ny teknologi vil afhjælpe de problemer, der er ved den nuværende løsningsmetode.

%Teknologiafsnittet vil indeholde en beskrivelse af egenskaberne for den nuværende teknologi, samt en undersøgelse af den alternative behandlingsmetode, der ligger til grunde for MTV'en. Efter den nuværende og den alternative teknologi er beskrevet, vil disse blive sammenholdt, med henblik på at finde fordele og ulemper ved de to løsningsforslag.

%I forbindelse med patientafsnittet i MTV-modellen afgrænses patientgruppen, med henblik på at gøre problemet konkret, hvorved målgruppen for teknologien kan undersøges nærmere. Der undersøges blandt andet hvorvidt teknologien vil have en betydelig påvirkning på patienternes hverdag og om der skal tages højde for etiske problemstillinger.

%Den organisatoriske analyse vil hovedsageligt behandle ændringer i interaktionen mellem patienter og sundhedspersonale, samt det organisatoriske aspekt i forhold til samarbejdet mellem forskellige sundhedsinstitutioner.

%Som et led i MTV'en vil det økonomiske aspekt blive undersøgt med udgangspunkt i, at finde frem til omkostningerne relateret til de teknologiske løsninger, som er undersøgt i teknologianalysen. Her undersøges desuden hvilke besparelser eller ekstraudgifter, der kan forekomme ved implementering af den nye teknologi.

%Analysen af de fire MTV-elementer vil dernæst blive anvendt i syntesen, der indeholder en diskussion med udgangspunkt i fordele og ulemper ved både den nuværende og den undersøgte teknologi. Herigennem vil PBL-metoden også komme til udtryk, i og med syntesen leder frem til en konklusion, som vil besvare den indledende problemformulering.