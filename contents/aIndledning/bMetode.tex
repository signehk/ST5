\section{Metode} \label{metode}

Denne medicinske teknologivurdering (MTV) vil afvige fra opbygningen beskrevet i MTV-håndbogen, som følge af projektet samtidig indeholder elementer fra problembaseret læring (PBL). Som et resultat af blandingen, tages der udgangspunkt i en medicinsk problemstilling, som analyseres for at udarbejde en problemformulering. Analysen i forbindelse med PBL-tilgangen vil desuden indeholde MTV-elementer såsom etik, målgruppe og interessentanalyse.

Efterfølgende vil problemformuleringen skabe grundlag for at arbejde videre med MTV-håndbogens elementer. Her vil de fire områder, teknologi, patient, organisation og økonomi, blive anvendt til at stille mere konkrete spørgsmål. Fremgangsmåden betyder at der vil være tale om en problem- og teknologiorienteret MTV, da der søges at finde en løsning på et medicinsk problem gennem en vurdering af, hvorvidt en ny teknologi vil afhjælpe de problemer, der er ved den nuværende løsningsmetode.

Teknologiafsnittet vil indeholde en beskrivelse af egenskaberne for den nuværende teknologi, samt en undersøgelse af den alternative behandlingsmetode, der ligger til grunde for MTV'en. Efter den nuværende og den alternative teknologi er beskrevet, vil disse blive sammenholdt, med henblik på at finde fordele og ulemper ved de to løsningsforslag.

I forbindelse med patientafsnittet i MTV-modellen afgrænses patientgruppen, med henblik på at gøre problemet konkret, hvorved målgruppen for teknologien kan undersøges nærmere. Der undersøges blandt andet hvorvidt teknologien vil have en betydelig påvirkning på patienternes hverdag og om der skal tages højde for etiske problemstillinger.

Den organisatoriske analyse vil hovedsageligt behandle ændringer i interaktionen mellem patienter og sundhedspersonale, samt det organisatoriske aspekt i forhold til samarbejdet mellem forskellige sundhedsinstitutioner.

Som et led i MTV'en vil det økonomiske aspekt blive undersøgt med udgangspunkt i, at finde frem til omkostningerne relateret til de teknologiske løsninger, som er undersøgt i teknologianalysen. Her undersøges desuden hvilke besparelser eller ekstraudgifter, der kan forekomme ved implementering af den nye teknologi.

Analysen af de fire MTV-elementer vil dernæst blive anvendt i syntesen, der indeholder en diskussion med udgangspunkt i fordele og ulemper ved både den nuværende og den undersøgte teknologi. Herigennem vil PBL-metoden også komme til udtryk, i og med syntesen leder frem til en konklusion, som vil besvare den indledende problemformulering.