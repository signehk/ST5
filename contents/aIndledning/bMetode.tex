\section{Metode} \label{metode}

Denne medicinske teknologivurdering (MTV) vil tage udgangspunkt i opbygningen beskrevet i MTV-håndbogen, samtidig vil den indeholde elementer fra problembaseret læring (PBL), da denne MTV skrives som del af 5. semesterprojektet på sundhedsteknologi uddannelsen. Som et resultat af denne kombination, tages der udgangspunkt i en medicinsk problemstilling, som analyseres for at kunne udarbejde en problemformulering. Analysen i forbindelse med PBL-tilgangen vil desuden indeholde delelementer i forhold til en MTV tilgang såsom etik, målgruppe og interessentanalyse.

Problemformuleringen vil give grundlag for at arbejde videre med MTV-håndbogens elementer i perspektiv til problemet. Her vil de fire områder, teknologi, patient, organisation og økonomi, blive belyst for at stille mere konkrete spørgsmål. Under hvert område vil der tilhører et indledende metodeafsnit, dette har til formål at beskrive hvilken tilgang der vil blive taget til hvert af de fire forskellige områder. Denne fremgangsmåden betyder at der vil være tale om en problem- og teknologiorienteret MTV, da der søges at finde en løsning på et medicinsk problem gennem en vurdering af, hvorvidt en ny teknologi vil afhjælpe det probleme, som problemanalysen belyser, og om denne vil kunne forbedre nuværende løsningsmetoder. 

Teknologiafsnittet vil indeholde en beskrivelse af egenskaberne for de nuværende teknologier, samt en undersøgelse af den alternative behandlingsmetode, der ligger til grund for MTV'en. Efter den nuværende og de alternative teknologier er beskrevet, vil disse blive sammenholdt, med henblik på at finde fordele og ulemper ved løsningsforslagene.

Patientafsnittet i MTV’en vil bestå i en afgræng af  patientgruppen, med henblik på at gøre problemet mere fokuseret, hvorved målgruppen kan undersøges nærmere  i forhold til teknologien . Der undersøges blandt andet hvorvidt teknologien vil have en betydelig påvirkning på patienternes hverdag, men den kan forbedre deres livskvalitet og om der skal tages højde for eventuelle etiske problemstillinger.

Den organisatoriske analyse vil hovedsageligt behandle ændringer i interaktionen mellem  patienter og sundhedspersonale, samt det organisatoriske aspekt i forhold til samarbejdet mellem forskellige sundhedsinstitutioner.

Som et led i MTV'en vil det økonomiske aspekt blive undersøgt med udgangspunkt i at finde frem til omkostningerne relateret til de teknologiske løsninger, som er undersøgt i teknologianalysen. Her undersøges desuden hvilke besparelser eller ekstraudgifter, der kan forekomme ved implementering af den nye teknologi.

Analysen af de fire MTV-elementer vil dernæst blive anvendt i syntesen, der indeholder en diskussion med udgangspunkt i fordele og ulemper ved både den nuværende og den undersøgte teknologi. Herigennem vil PBL-metoden også komme til udtryk, i og med syntesen leder frem til en konklusion, som vil besvare problemformuleringen.

MTV’en vil primært blive dokumenteret ved brug af videnskabelig litteratur fundet fra forskellige videnskabelige databaser. For at overskueliggøre dette vil der sideløbende med MTV’ens udformning blive udarbejdet en søgeprotokol. I søge protokollen vil der blandt andet være inklusion og eksklusionskriterier  for at kunne fokusere søgningen til det mest relevante litteratur i forhold til de fire områder i MTV’en. Formålet med søge protokollen er dels at få et overblik over de kilder der anvendes og primært for at kunne dokumentere MTV’ens indhold, da det er muligt ved hjælp af søgeprotokollen at se hvor, hvad og hvordan der er søgt litteratur, hvorved det er muligt at genskabe MTV’ens indhold. Søge protokollen kan findes i bilag XX.


%\section{Metode} \label{metode}

%Denne medicinske teknologivurdering (MTV) vil afvige fra opbygningen beskrevet i MTV-håndbogen, som følge af projektet samtidig indeholder elementer fra problembaseret læring (PBL). Som et resultat af blandingen, tages der udgangspunkt i en medicinsk problemstilling, som analyseres for at udarbejde en problemformulering. Analysen i forbindelse med PBL-tilgangen vil desuden indeholde MTV-elementer såsom etik, målgruppe og interessentanalyse.

%Efterfølgende vil problemformuleringen skabe grundlag for at arbejde videre med MTV-håndbogens elementer. Her vil de fire områder, teknologi, patient, organisation og økonomi, blive anvendt til at stille mere konkrete spørgsmål. Fremgangsmåden betyder at der vil være tale om en problem- og teknologiorienteret MTV, da der søges at finde en løsning på et medicinsk problem gennem en vurdering af, hvorvidt en ny teknologi vil afhjælpe de problemer, der er ved den nuværende løsningsmetode.

%Teknologiafsnittet vil indeholde en beskrivelse af egenskaberne for den nuværende teknologi, samt en undersøgelse af den alternative behandlingsmetode, der ligger til grunde for MTV'en. Efter den nuværende og den alternative teknologi er beskrevet, vil disse blive sammenholdt, med henblik på at finde fordele og ulemper ved de to løsningsforslag.

%I forbindelse med patientafsnittet i MTV-modellen afgrænses patientgruppen, med henblik på at gøre problemet konkret, hvorved målgruppen for teknologien kan undersøges nærmere. Der undersøges blandt andet hvorvidt teknologien vil have en betydelig påvirkning på patienternes hverdag og om der skal tages højde for etiske problemstillinger.

%Den organisatoriske analyse vil hovedsageligt behandle ændringer i interaktionen mellem patienter og sundhedspersonale, samt det organisatoriske aspekt i forhold til samarbejdet mellem forskellige sundhedsinstitutioner.

%Som et led i MTV'en vil det økonomiske aspekt blive undersøgt med udgangspunkt i, at finde frem til omkostningerne relateret til de teknologiske løsninger, som er undersøgt i teknologianalysen. Her undersøges desuden hvilke besparelser eller ekstraudgifter, der kan forekomme ved implementering af den nye teknologi.

%Analysen af de fire MTV-elementer vil dernæst blive anvendt i syntesen, der indeholder en diskussion med udgangspunkt i fordele og ulemper ved både den nuværende og den undersøgte teknologi. Herigennem vil PBL-metoden også komme til udtryk, i og med syntesen leder frem til en konklusion, som vil besvare den indledende problemformulering.