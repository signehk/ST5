\chapter{Indledning} \label{sec:indledning}
I Danmark dør 4.500 mennesker årligt i forbindelse med fysisk inaktivitet, svarende til $7-8~\%$ af alle dødsfald \citep{aagaard2014}. Fysisk inaktivitet har konsekvenser for kroppens fysiologiske tilstand og helbred, da det er en risikofaktor for blandt andet psykiske sygdomme, livsstilssygdomme, såsom type-2 diabetes eller visse hjertekarsygdomme samt en for tidlig død for blandt andet patienter med type-2 diabetes og hypertension \citep{motionsraad2007}. 

Statens Institut for Folkesundhed har desuden fundet, at fysisk inaktive personer dør 5-6 år tidligere end aktive personer, og manglende fysisk aktivitet anses som værende en af de mest betydende faktorer i relation til for tidlig død på verdensplan. Ud over dette, resulterer fysisk inaktivitet nationalt årligt i $100.000$ hospitalsindlæggelser, $3,1$ millioner fraværsdage, $2,6$ millioner kontakter til praktiserende læge og $1.200$ førtidspensioner \citep{christensen2012}. Fysisk inaktivitet påvirker blandt andet kroppens kredsløb, muskler, knogler og metabolisme, hvilket på sigt vil resultere i en reduceret arbejdskapacitet for kroppen og et eventuelt funktionstab \citep{motionsraad2007}.

Aktivitet i dagligdagen er nødvendigt i alle aldersgrupper, og Sundhedsstyrelsen anbefaler hertil, at voksne bør være aktive minimum 30 minutter dagligt med moderat intensitet \citep{pedersen2011}. Fysisk aktivitet kan anvendes til at forebygge flere sygdomme, og en struktureret fysisk træning kan yderligere benyttes som en del af en behandling eller til at forebygge en eventuel videreudvikling af flere sygdomme \citep{motionsraad2007}. Af denne grund betragtes fysisk aktivitet som værende relevant under sygdomsbehandling. Herunder betragtes både muligheden for præcis monitorering og potentiel motivation til højere aktivitetsniveau, uden patienterne udsættes for en større arbejdsopgave i forbindelse med den nye teknologi.

For at monitorere om det daglige mål for aktivitetsniveau opfyldes, kan anvendes forskellige aktivitetstrackere, såsom skridttællere og aktivitetsarmbånd. Anvendelsen af disse, til privat brug, er i de seneste år steget, og sådanne trackere kan antageligvis vurderes også at være brugbare i sundhedssektoren til behandling eller forebyggelse af forskellige sygdomme, hvor fysisk aktivitet kan indgå i behandlingsforløbet \citep{statista2016}.

Behandling af patienter foregår, afhængigt af sygdommens karakter, enten i den primære eller sekundære sundhedssektor. I den primære sektor, der er borgernes lokale adgang til sundhedsvæsenet, kan patienter blandt andet komme i kontakt med praktiserende læger i almene praksisser, som kan stille en diagnose og behandle patienter for adskillige sygdomme. Hvis der kræves yderligere komplekse undersøgelser eller behandlinger af en patient, henvises vedkommende til den sekundære sektor, der består af de danske sygehuse \citep{vedsted2014}. 

I dette projekt fokuseres på den almene praksis. I almen praksis er samarbejdet mellem læge og patient tæt, hvorved lægen kan følge den enkelte patients liv, for at kunne gøre behandlingsforløbet bedst muligt. Foruden at give borgerne adgang til lægebesøg er den almene praksis med til at reducere presset på sygehusene i forhold til antal henvendelser \citep{vedsted2014}.
\newpage
Det er herved relevant at undersøge, hvordan alment praktiserende læger monitorerer aktivitetsniveauet hos patienter med sygdomme, der kan behandles eller forebygges med et øget aktivitetsniveau.  


\section{Initierende problem}
På baggrund af ovenstående indledning formuleres et initierende problem, som danner udgangspunkt for problemanalysen. 

\begin{center}
\textit{Hvordan monitoreres patienters aktivitetsniveau i dagligdagen som led i en sygdomsbehandling?}
\end{center}

