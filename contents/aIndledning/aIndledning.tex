\chapter{Indledning}
I Danmark dør 4.500 mennesker årligt som følge af fysisk inaktivitet, hvor fysisk inaktivitet jf. Sundhedsstyrelsen defineres som værende mindre end 2,5 times fysisk aktivitet per uge. \citep{aagaard2014} Fysisk inaktivitet har konsekvenser for kroppens fysiologiske tilstand og helbred, da det er en risikofaktor for psykiske sygdomme, livsstilssygdomme såsom type-2 diabetes eller visse hjertekarsygdomme, samt en for tidlig død for blandt andet patienter med type-2 diabetes og hypertension. \citep{motionsraad2007} 

Fysisk inaktivitet påvirker blandt andet kroppens kredsløb, muskler, knogler og metabolisme, hvilket vil resultere i en reduceret arbejdskapacitet for kroppen og eventuelt funktionstab. På længere sigt kan fysisk inaktivitet øge risikoen for tidlig død, da det er dokumenteret, at regelmæssig fysisk aktivitet nedsætter risikoen for tidlig død. \citep{motionsraad2007}

Sundhedsstyrelsen anbefaler, at voksne bør være aktive minimum 30 minutter dagligt med moderat intensitet, hvilket forstås som 40-59 $\%$ af den maksimale iltoptagelse pågældende eller motion, hvor man bliver lettere forpustet, men hvor det er muligt at føre en samtale.
Aktivitet i dagligdagen er nødvendigt i alle aldersgrupper, og anbefalingerne er specificeret til de enkelte aldersgrupper. Herunder er det understreget, at børn skal være fysisk aktive minimum 60 minutter dagligt, samt at ældre yderligere skal lave udstrækningsøvelser. \citep{pedersen2011}

Fysisk aktivitet kan anvendes til at forebygge flere sygdomme, og en struktureret fysisk træning kan yderligere benyttes som en del af en behandling eller til at forebygge en eventuel videreudvikling af flere sygdomme. \citep{motionsraad2007} Dette kræver, at der fokuseres på fysisk aktivitet under behandling af patienter, hvor dette kan have en positiv effekt.

%\textit{Telemedicin kan benyttes som en del af denne behandling, hvor hjemmemonitorering af fysisk aktivitet er en mulighed. Aktivitetsniveauet kan derved registreres af patienten selv, og dette kan efterfølgende evalueres af egen praktiserende læge eller andet sundhedsfagligt personale. På denne måde kan patient og den primære sundhedssektor spare besøg og transporter ved brug af opfølgning på indberettet data. \citep{medcom2010} - skal dette med? mere problemanalyse?}


\section{Initierende problem}
Hvordan monitoreres/dokumenteres patienters aktivitetsniveau som led i en behandling?  
