\chapter{Indledning} \label{sec:indledning}
I Danmark dør 4.500 mennesker årligt i forbindelse med fysisk inaktivitet, svarende til $7-8~\%$ af alle dødsfald \citep{aagaard2014}. Fysisk inaktivitet har konsekvenser for kroppens fysiologiske tilstand og helbred, da det er en risikofaktor for psykiske sygdomme, livsstilssygdomme, såsom type-2 diabetes eller visse hjertekarsygdomme, samt en for tidlig død for blandt andet patienter med type-2 diabetes og hypertension \citep{motionsraad2007}. 

Statens Institut for Folkesundhed har desuden fundet, at fysisk inaktive personer dør 5-6 år tidligere end aktive personer, og manglende aktivitet anses som værende en af de mest betydende faktorer i relation til for tidlig død på verdensplan. Ud over dette, resulterer fysisk inaktivitet nationalt årligt i $100.000$ hospitalsindlæggelser, $3,1$ millioner fraværsdage, $2,6$ millioner kontakter til praktiserende læge og $1.200$ førtidspensioner \citep{christensen2012}.

Fysisk inaktivitet påvirker blandt andet kroppens kredsløb, muskler, knogler og metabolisme, hvilket vil resultere i en reduceret arbejdskapacitet for kroppen og et eventuelt funktionstab \citep{motionsraad2007}.

Aktivitet i dagligdagen er nødvendigt i alle aldersgrupper, og anbefalingerne er specificeret til de enkelte aldersgrupper.
Sundhedsstyrelsen anbefaler, at voksne bør være aktive minimum 30 minutter dagligt med moderat intensitet \citep{pedersen2011}.

Fysisk aktivitet kan anvendes til at forebygge flere sygdomme, og en struktureret fysisk træning kan yderligere benyttes som en del af en behandling eller til at forebygge en eventuel videreudvikling af flere sygdomme \citep{motionsraad2007}. Dette kræver, at der fokuseres på fysisk aktivitet under behandling af patienter.


Afhængigt af sygdommens stadie vil patienten befinde sig i enten den primære eller sekundær sundhedssektor. I den primære sundhedssektor, som er borgernes lokale adgang til sundhedsvæsenet, kan patienterne komme i kontant med praktiserende læger, som kan opstille en diagnose og behandle patienterne for adskillige sygdomme. Hvis der kræves yderligere komplekse undersøgelser eller behandlinger af patient videre sendes vedkommende til den sekundær sundhedssektor, bestående af syghuset. 

I dette projekt fokuseres på almen praksis, som også er en del af det danske sundhedsvæsenet. I almen praksis er samarbejdet mellem læge og patient tæt, hvor hertil lægen kan følge den enkelte patients sociale liv til blandt andet formål at kunne løse og forstå kroniske sygdomme som patienterne døjer med. 
Foruden at give borgerne en lige adgang til lægebesøg er almen praksis også med til at lemme presset på sygehusene ift. antal henvendelser. \citep{vedsted2014}





\section{Initierende problem}
\textit{Hvordan monitoreres/dokumenteres patienters aktivitetsniveau i dagligdagen som led i en sygdomsbehandling?}  
