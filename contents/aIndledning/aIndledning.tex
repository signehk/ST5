\chapter{Indledning} \label{sec:indledning}
I Danmark dør 4.500 mennesker årligt i forbindelse med fysisk inaktivitet, svarende til $7-8~\%$ af alle dødsfald \citep{aagaard2014}. Fysisk inaktivitet har konsekvenser for kroppens fysiologiske tilstand og helbred, da det er en risikofaktor for psykiske sygdomme, livsstilssygdomme, såsom type-2 diabetes eller visse hjertekarsygdomme, samt en for tidlig død for blandt andet patienter med type-2 diabetes og hypertension \citep{motionsraad2007}. 

Statens Institut for Folkesundhed har desuden fundet, at fysisk inaktive personer dør 5-6 år tidligere end aktive personer, og manglende aktivitet anses som værende en af de mest betydende faktorer i relation til for tidlig død på verdensplan. Ud over dette, resulterer fysisk inaktivitet nationalt årligt i $100.000$ hospitalsindlæggelser, $3,1$ millioner fraværsdage, $2,6$ millioner kontakter til praktiserende læge og $1.200$ førtidspensioner \citep{christensen2012}.

Fysisk inaktivitet påvirker blandt andet kroppens kredsløb, muskler, knogler og metabolisme, hvilket vil resultere i en reduceret arbejdskapacitet for kroppen og et eventuelt funktionstab \citep{motionsraad2007}.

Aktivitet i dagligdagen er nødvendigt i alle aldersgrupper, og anbefalingerne er specificeret til de enkelte aldersgrupper.
Sundhedsstyrelsen anbefaler, at voksne bør være aktive minimum 30 minutter dagligt med moderat intensitet \citep{pedersen2011}.

Fysisk aktivitet kan anvendes til at forebygge flere sygdomme, og en struktureret fysisk træning kan yderligere benyttes som en del af en behandling eller til at forebygge en eventuel videreudvikling af flere sygdomme \citep{motionsraad2007}. Dette kræver, at der fokuseres på fysisk aktivitet under behandling af patienter.

For at måle om det daglige mål for aktivitetsniveau opfyldes, kan blandt andet anvendes forskellige aktivitetstrackere, såsom skridttællere og aktivitetsarmbånd. Anvendelsen af disse, til privat brug, er i de seneste år steget \citep{statista2016}, og sådanne trackere kunne herved tænkes også at være brugbare i sundhedssektoren til behandling eller forebyggelse af forskellige sygdomme.

Behandlingen af patienter foregår, afhængig af sygdommen, enten i den primære eller sekundære sundhedssektor. I den primære sektor, der er borgernes lokale adgang til sundhedsvæsenet, kan patienter blandt andet komme i kontakt med praktiserende læger i almene praksisser, som kan stille en diagnose og behandle patienter for adskillige sygdomme. Hvis der kræves yderligere komplekse undersøgelser eller behandlinger af en patient, henvises vedkommende til den sekundære sektor, der består af de danske sygehuse \citep{vedsted2014}. 

I dette projekt fokuseres på den almene praksis. I almen praksis er samarbejdet mellem læge og patient tæt, hvorved lægen kan følge den enkelte patients liv, for at kunne gøre behandlingsforløbet bedst mulig. Foruden at give borgerne adgang til lægebesøg er den almene praksis med til at reducere presset på sygehusene i forhold til antal henvendelser \citep{vedsted2014}.

Det er herved relevant at undersøge, hvordan praktiserende læger i den almene praksis monitorerer aktivitetsniveauet hos patienter med sygdomme, der kan behandles eller forebygges med et øget aktivitetsniveau.  


\section{Initierende problem}
\textit{Hvordan monitoreres/dokumenteres patienters aktivitetsniveau i dagligdagen som led i en sygdomsbehandling?}  

