\chapter{Søgestrategi}\label{sec:metode_soeg}

MTV-delen af rapporten vil primært blive udarbejdet og dokumenteret ud fra videnskabelig litteratur fundet i forskellige videnskabelige databaser. Søgninger laves med udgangspunkt i problemformuleringen, hvor MTV-spørgsmålene yderligere specificerer den valgte problemformulering til hvert MTV-aspekt. 

For at overskueliggøre litteratursøgningerne vil der sideløbende med MTV’ens udformning lægges fokus på søgningernes kvalitet. Dette gøres blandt andet ved at formulere inklusions- og eksklusionskriterier for at kunne fokusere søgningen til det mest relevante litteratur i forhold til MTV-spørgsmålene. Hertil fokuseres også på, at søgeordene dækker spørgsmålene, så eventuel brugbar litteratur ikke frasorteres grundet for specifikke eksklusionskriterier. 

\section{Evidensniveau} \label{sec:met_evidens}

Yderligere \textit{rettes der fokus imod} kildekritik og herunder kildernes evidensniveau, hvor kilderne rangeres fra 1 til 7, hvor 1 angiver den højeste kvalitet af kilde, og 7 er den laveste \citep{mtvhaandbog}.  

\begin{enumerate}
\item Meta-analyser og systematiske oversigtsartikler
\item Randomiserede kontrollerede undersøgelser
\item Ikke-randomiserede kontrollerede undersøgelser
\item Kohorteundersøgelser
\item Case-kontrol undersøgelser
\item Deskriptive undersøgelser og mindre serier
\item Konsensusrapporter, ikke-systematiske oversigtsartikler, ledere, ekspertudtalelser og lærebøger
\end{enumerate}

\noindent
Under litteratursøgningerne ønskes det at benytte litteratur, der er højt rangeret, selvom projektgruppen erkender, at meget af den anvendte litteratur uundgåeligt vil være såkaldt "grå litteratur". 
