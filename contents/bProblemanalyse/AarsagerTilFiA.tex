\subsection{Årsager til fysisk inaktivitet}
Fysisk inaktivitet er forårsaget af forskellige faktorer, eksempelvis livsstil og den teknologiske udvikling gennem tiden. Manglende tid, motivation og interesse er dog nogle af de overordnede årsager til fysisk inaktivitet \citep{ottesen2005}.  

\subsubsection{Teknologiske faktorer}  
Siden den industrielle revolution er teknologi et område, der er i konstant udvikling, og anvendes blandt andet som skåneredskab for at aflaste den almene arbejder for fysisk hårdt arbejde, samt invaliditet heraf \citep{hallal2012}. 
Ligeledes har udviklingen ledt til en reduktion i mængden af fysisk aktivitet, der er krævet for at komme igennem hverdagen. Dette betyder blandt andet let adgang til mad og drikkevarer, som ikke kræver stor energiomsætning for at skaffe \citep{motionsraad2007,hallal2012}. Transport foregår ofte med bil eller bus, og teknologier som tv, trådløs kommunikation, internet og lignende bidrager til fysisk inaktivitet \citep{hallal2012}.  

\subsubsection{Kropslige faktorer}
På verdensplan ses det, at fysisk inaktivitet stiger i takt med alderen \citep{guthold2008}. 
Årsagen til dette hos danske ældre er, at de ikke føler det nødvendige overskud til fysisk aktivitet efter hverdagens daglige gøremål. 
Nogle overvægtige oplever frygt og manglende motivation ved fysisk aktivitet, da de forbinder det med ubehag og usikkerhed i, hvad deres krop reelt kan holde til. 
Psykiske forhindringer for fysisk aktivitet fremtræder som flovhed for at vise sig frem i et træningscenter, samt at individer ikke føler, de passer ind med omgivelserne under fysisk aktivitet. 
Dertil forekommer ligeledes manglende motivation og/eller interesse \citep{ottesen2005}.

\subsubsection{Økonomiske faktorer}
Fysisk inaktivitet kan også være forårsaget af økonomiske årsager, hvor eksempelvis betaling for medlemskab af et træningscenter vil sætte en begrænsning for nogle personer. Yderligere kan fysisk aktivitet opfattes som værende for tidskrævende eller besværligt at få plads til i hverdagen \citep{ottesen2005}.