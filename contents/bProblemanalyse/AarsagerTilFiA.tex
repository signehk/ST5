\subsection{Årsager til fysisk inaktivitet}
Fysisk inaktivitet er forårsaget af forskellige faktorer, som eksempelvis livsstil og den teknologiske udvikling gennem tiden. Manglende tid, motivation og interesse er dog en af de overordnede årsager til fysisk inaktivitet \citep{ottesen2005}.  

\subsubsection{Teknologiske faktorer}  
Siden den industrielle revolution er teknologi et område, der er i konstant udvikling, og anvendes blandt andet som skåneredskab for at aflaste den almene arbejder for fysisk hårdt arbejde, samt invaliditet heraf \citep{hallal2012}. 
Ligeledes har udviklingen ledt til en reduktion i mængden af fysisk aktivitet krævet for at komme igennem hverdagen \citep{hallal2012, motionsraad2007}. Transport foregår ofte med bil eller bus, og teknologier som tv, trådløs kommunikation, internet og lignende bidrager til fysisk inaktivitet \citep{hallal2012}.  

\subsubsection{Kropslige faktorer}
Alder er blandt disse faktorer, hvor det på verdensplan ses at fysiske inaktivitet stiger i takt med at alderen \citep{guthold2008}. 
Årsagen til dette hos danske ældre er, at de ikke føler det nødvendige overskud, til fysisk aktivitet efter de stadig sværere gøremål i hverdagen. 
Overvægtige oplever frygt og manglende motivation ved fysisk aktivitet, idet de forbinder det med ubehag og usikkerhed i hvad deres krop reelt kan holde til \citep{ottesen2005}. 
Psykiske forhindringer for fysisk aktivitet fremtræder som flovhed for at vise sig frem i et træningscenter, samt at individer ikke føler de passer ind med omgivelserne ved aktivitet. 
Dertil forekommer ligeledes manglende motivation og/eller interresse \citep{ottesen2005}.

\subsubsection{Økonomiske faktorer}
Fysisk inaktivitet kan også være forårsaget af økonomiske årsager, hvor eksempelvis betaling for medlemskab af et træningscenter vil sætte en begrænsning for nogle personer. Yderligere forbindes fysisk aktivitet med noget, der er for tidskrævende eller besværligt at få plads til i hverdagenen \citep{ottesen2005}.