\section{Problemformulering}

%Her vil være en opsummering af problemanalysen, for at tydeliggøre hvorfor problemformuleringen lyder som følgende:

Det er påvist, at mange sygdomsramte personer har gavn af fysisk aktivitet som en behandling eller en metode til at forebygge sygdomsprogression, samt at fysisk inaktivitet kan være en faktor i forbindelse med udviklingen af flere sygdomme \citep{motionsraad2007,pedersen2011}
Af denne grund vælges der at tage udgangspunkt i hypertension, som 20 \% af befolkningen i Danmark lider af, da fysisk inaktivitet øger risikoen for hypertension, og da motion har en blodtrykssænkende effekt \citep{pedersen2011,olsen2015}. 

Det ønskes at undgå eller udskyde non-farmakologiske behandlinger og viderendelser fra almen praksis til hypertensionsklinikker og dermed spare sundhedsvæsenet penge samt forbedre hypertensive patienters livskvalitet...
 
Så vidt muligt bør begrænsningen ske gennem forbedringer i behandlingsmetoderne hos den praktiserende læge, ved at skabe større mulighed for monitorering af hverdagsvaner såsom fysisk aktivitet, så behandlingen sker non-farmakologisk.

anses det som værende relevant at undersøge, om fysisk aktivitet kan benyttes som en større del af behandlingen af hypertension ved bedre monitorering af patienters aktivitetsniveau. 


\textit{Hvilke effekter vil implementeringen af aktivitetsarmbånd til registrering og objektivisering af fysisk aktivitet hos hypertensive patienter have i den almene praksis?}

\textit{Hvilke påvirkninger vil implementeringen af aktivitetsarmbånd i den almene praksis til registrering og objektivisering af fysisk aktivitet have hos hypertensive patienter?}