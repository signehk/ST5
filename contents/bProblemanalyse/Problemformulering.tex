\section{Problemformulering}

%Her vil være en opsummering af problemanalysen, for at tydeliggøre hvorfor problemformuleringen lyder som følgende:

Det er påvist, at mange sygdomsramte personer har gavn af fysisk aktivitet som en behandling eller en metode til at forebygge sygdomsprogression, samt at fysisk inaktivitet kan være en faktor i forbindelse med udviklingen af flere sygdomme \citep{motionsraad2007,pedersen2011}
Af denne grund vælges der at tage udgangspunkt i hypertension, som $20~\%$ den voksne danske befolkning lider af, da fysisk inaktivitet øger risikoen for hypertension, og da motion har en blodtrykssænkende effekt \citep{pedersen2011,olsen2015}. 

Det ønskes at begrænse antallet af farmakologiske behandlinger og viderendelser fra almen praksis til hypertensionsklinikker og dermed spare sundhedsvæsenet penge samt forbedre hypertensive patienters livskvalitet. Så vidt muligt bør denne begrænsning ske gennem forbedringer i behandlingsmetoderne hos den praktiserende læge ved at skabe større mulighed for monitorering af hverdagsvaner såsom fysisk aktivitet, så behandlingen sker non-farmakologisk.

Den nuværende subjektive målemetode er ikke altid i stand til at repræsentere den reelle fysiske aktivitet, da patienter har tendens til at overvurdere mængde eller intensitet af deres ydede fysiske aktivitet \citep{motionsraad2007,pedersen2011,prince2008}. Alternativt kan benyttes objektive målemetoder til et mere konkret og upartisk indblik i patienters aktivitetsmønstre. 

Det vælges herunder at fokusere på Fitbit Flex armbåndet til aktivitetstracking, da denne teknologi fremstår med færrest ulemper jævnfør \autoref{sec:prob_sammenfatning} og \autoref{afgraensning_tek}, hvilket leder frem til den valgte problemformulering: 

%\textit{Hvilke effekter vil implementeringen af aktivitetsarmbånd til registrering og objektivisering af fysisk aktivitet hos hypertensive patienter have i den almene praksis?}

\begin{center}
\textit{Hvilke påvirkninger vil implementeringen af Fitbit Flex i den almene praksis til registrering og objektivisering af fysisk aktivitet have hos hypertensive patienter i sundhedssektoren?}
\end{center}