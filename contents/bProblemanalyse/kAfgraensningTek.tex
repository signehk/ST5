\section{Teknologiafgrænsning} \label{afgraensning_tek}

Igennem en undersøgelse af hvilke funktioner, der vil være relevante i forbindelse med aktivitetstracking, opstilles krav som udgangspunkt for valget af den endelige teknologi. Kravene til funktion vil blive stillet ud fra den primære aktivitetsform hos patientgruppen, således aktivitetsarmbåndet er optimeret til netop denne aktivitetstype.

I Danmark stiger prævalensen af hypertension med alderen. Det ses blandt andet, at der kun er $1~\%$ af de $20-29$ årige, som lider af hypertension, mens omkring $69~\%$ af de $80-89$ årige har sygdommen \citep{olsen2015}. Som følge af den forøgede risiko for hypertension i sammenhæng med alderen, ses der på den primære anbefalede fysiske aktivitet for ældre over $65$ år, hvilket er $30$ minutters aktivitet med moderat intensitet om dagen og mindst $2$ gange $20$ minutters muskelstyrkende eller konditionsforøgende aktivitet om ugen \citep{pedersen2011}.

Hos ældre anses gang over $6$ km/t som konditionsforøgende aktivitet, og gang med $4-5$ km/t som moderat aktivitet. Med udgangspunkt i foregående, samt Sundhedsstyrelsens anbefalinger i \citetalias{pedersen2011} tages udgangspunkt i gangregistrering med mulighed for udvidelse til svømning og cykling \citep{pedersen2011}.

\subsection{Krav til funktionalitet}

Såfremt aktivitetsarmbåndet skal anvendes i hverdagen, er det vigtigt, at det er kompakt og bærbar, samt at det ikke har behov for opladning på daglig basis. Som følge af at den primære aktivitet for patientgruppen er gang, er det vigtigt, at enheden kan måle dette præcist, således målingerne kan anvendes som valide data. 

Da Sundhedsstyrelsen også anbefaler svømning og cykling, hvis patienten har mulighed for dette, vil det være relevant, men ikke påkrævet, at aktivitetsarmbåndet har mulighed for at måle denne type aktivitet. Registrering af disse aktiviteter kræver både vandtæthed og GPS eller mulighed for kommunikation med en ekstern cykelcomputer på patientens cykel.

\subsection{Valg af aktivitetsarmbånd}

For at finde den mest optimale aktivitetsarmbånd til formålet, tages der udgangspunkt i studier, som har undersøgt præcisionen af forskellige aktivitetsarmbånd ved blandt andet antal skridt, energiforbrug og afstand. Ud over dette er brugerfladen også bedømt, hvorfor dette også er relevant at tage med i overvejelserne. Det er valgt at fokusere på Fitbit Flex, da dette indeholder den nødvendige teknologi til tracking, samt at muligheden for reproducering af målinger er høj \citep{kaewkannate2016}. Oven i dette udkom Fitbit Flex $2$ i $2016$, og den nye version giver mulighed for tracking af svømning, hvilket er væsentligt for patiengruppen, men som følge af mangel på studier vedrørende repeterbarheden og præcisionen af den nye model, fokuseres på den gamle model \citep{fitbitflex}.

Yderligere fordele ved Fitbit Flex inkluderer muligheden for at gemme data i op til $30$ dage, muligheden for at sammenligne med andres aktivitet, vandtæthed og kompatibilitet med fitness-apps på smartphones og computer \citep{kaewkannate2016, fitbitflex}. Især de sociale egenskaber ved Fitbit's armbånd, samt muligheden for at tracke aktiviteten med apps, kan give anledning til øget aktivitetsniveau hos patienterne \citep{karapanos2016, rooksby2014}. Fitbit Flex er dog ikke udstyret med GPS, men hvis funktionen er nødvendig for tracking af bestemte aktiviteter, har patienten mulighed for at kombinere GPS-data fra eksempelvis smartphones med Fitbit's data \citep{fitbitflex}.