\section{Følger af fysisk inaktivitet}

\subsection{Fysiske følger}

Der foregår en lang række fysiologiske processer i kroppen, alle disse er i høj grad tilpasset til det miljø vi lever i på jorden. Tyngdekraften udgør en belastning på vores legeme som sammen med de bevægelser vi udfører når vi er fysisk aktive, skaber et stress på kroppen. Hvis kroppen ikke udsættes for stress tilpasser kroppen sig ved at nedgradere de biologiske mekanismer, omvendt forstærkes de når vi stimulerer dem. Blandt disse biologiske mekanismer kan nævnes kredsløbet, stofskiftet, muskelvækst og knoglevækst.\citep{motionsraad2007}

\subsection{Kredsløb}
Kredsløbet er en af de mekanismer som påvirkes relativt hurtigt ved fysisk inaktivitet. Et studie af Convertino fra Brooks Air Force Base som foregik over 4 uger har påvist et fald i aerob kapacitet (VO2max), som angiver den maksimale iltoptaglese i kroppen under fysisk arbejde i forhold til tid, med 5 til 6 \% pr. uge. Personerne som blev testet var både kvinder og mænd i aldersgruppen 18 til 45 år. Et fald i aerob kapacitet kan skyldes en reducering af hjertets slagvolumen både i hvile og under arbejde grundet reducering i kroppens samlede blodvolumen, for at kompencere for dette øges pulsen for at opretholde minutvolumen af blod der pumpes ud i kroppen. Et fald i blodvolumen udgør en kortsigtet reducering af aerob kapacitet.\citep{Convertino1995} Dog ses det over længere tid med inaktivitet reduceret iltekstraktion i det perifere kredsløb, dette ses efter 12 ugers inaktivitet \citep{Coyle1985}.

\subsection{Muskelvæv}
Ved fysisk inaktivitet stimuleres musklerne i mindre grad hvilket fører til tab af muskelmasse grundet hastigheden for proteinnedbrydning i musklerne forløber hurtigere end proteinnydannelse, også kaldet proteinsyntese, musklerne bliver derfor mindre, hvilket betegnes muskelatrofi. Flere studier påpeger at der efter 1 til 2 ugers inaktivitet kan ses en reduktion i muskelmasse og at reduktionen af muskelmasse udelukkende skyldes en reduceret proteinsyntese \citep{Douglas2006}. \citep{Bloomsfield1995} Desuden vil der også opleves et betydeligt tab af muskelkraft hos personer der er inaktive over længere tid {Bloomfield1995}. 

\subsection{Knoglevæv og sener}
Ligesom musklerne skal knogler og sener stimuleres for at kunne opretholde sin styrke. Hvis ikke dette væv stimuleres for eksempel gennem aktivitet som inkluderer en form for stress, ved påvirkning dynamiske stød blandt andet ved hjælp fra tyngdekraften, vil der begynde at ske nedbrydning af knoglevævet. Allerede efter 1 uge vil der kunne blive observeret øget calciumudskillelse i urin og afføring. Dog varer det ofte op mod 1 til 2 måneder før der kan dedekteres forandringer i knoglernes mineralindhold, da knoglevæv omsættes langsomt. \citep{Bloomfield1995}

\subsection{Stofskifte}
Stofskiftet er med i reguleringen af de kemiske processer som sker i kroppen. Hormoner spiller en vigtig rolle inden for stofskiftet, her i blandt hormonet insulin som er vigtigt for glukoseoptagelse i musklerne og for regulering af glukosekoncentrationen i vores blod. En inaktiv livsstil vil føre til nedsat insulinfølsomhed og derfor en nedsat evne til at regulerer glukosekoncentrationen i blodet. Allerede efter en uge med inaktivitet kan der ses en reducering i musklernes insulinfølsomhed. \citep{Mikines1991} Grunden til at dette sker, kan skyldes at der bliver mindre af det glukosetransporterende protein GLUT4 i munkecellerne. Endvidere vil muskelatrofi føre til at der er mindre muskelvæv glukosen kan optages i. \citep{Tabata1999}

Disse fysiske påvirkninger forårsaget af inaktivitet kan være årsagende til mere alvorlige kroniske lidelser hvis der fortsættes en inaktiv livsstil. Fysisk inaktivitet kan fører til insulin resistens, hvilket øger risikoen for type 2 diabetes, fedme, samt andre livsstilssygdomme. \citep{motionsraad2007}
