\subsection{Fysiske følger af fysisk inaktivitet}
Der foregår en lang række fysiologiske processer i kroppen, alle disse er i høj grad tilpasset til det miljø, der er på jorden. 
Tyngdekraften udgør en belastning på kroppen, som sammen med bevægelser, under fysisk aktivitet, skaber et stress på kroppen. 
Hvis kroppen ikke udsættes for dette stress, tilpasses den, ved at nedgradere de biologiske mekanismer og processer. Omvendt forstærkes de når ved stimulation. 
Blandt disse biologiske mekanismer og processer kan nævnes kredsløbet, stofskiftet, muskelvækst og knoglevækst \citep{motionsraad2007}.

\subsubsection{Kredsløb}
Kredsløbet er en af de mekanismer som påvirkes relativt hurtigt ved fysisk inaktivitet. 
Et studie af \citeauthor{Convertino1995}, som foregik over 4 uger, har påvist et fald i aerob kapacitet (VO2max), som angiver den maksimale iltoptagelse i kroppen under fysisk arbejde i forhold til tid, med 5-6 \% pr. uge. 
Personerne som blev testet var både kvinder og mænd i aldersgruppen 18 til 45 år. 
Et fald i aerob kapacitet kan skyldes en reducering af hjertets slagvolumen både i hvile og under arbejde, grundet reducering i kroppens samlede blodvolumen. 
For at kompencere for dette øges pulsen for at opretholde minutvolumen af blod der pumpes ud i kroppen. 
Et fald i blodvolumen udgør en kortsigtet reducering af aerob kapacitet \citep{Convertino1995}. 
% Dog ses det at over længere tid med inaktivitet en reduceret iltekstraktion i det perifere kredsløb, dette kan ses efter cirka 12 ugers inaktivitet \citep{Coyle1985}.
Tidsperioder med inaktivitet varende længere end ca. 12 uger kan der yderligere ses en reduceret iltekstraktion i det perifere kredsløb \citep{Coyle1985}.

\subsubsection{Muskelvæv}
Ved fysisk inaktivitet stimuleres musklerne i mindre grad, hvilket fører til tab af muskelmasse grundet hastigheden for proteinnedbrydning i musklerne forløber hurtigere end proteinnydannelse, også kaldet proteinsyntese. 
Musklerne bliver derfor mindre, hvilket betegnes muskelatrofi. 
Flere studier påpeger, at der efter 1 til 2 ugers inaktivitet, kan ses en reduktion i muskelmasse, og at reduktionen af muskelmasse udelukkende skyldes en reduceret proteinsyntese \citep{Douglas2006, Bloomfield1995}. 
Desuden vil der også opleves et betydeligt tab af muskelkraft hos personer, der er inaktive over længere tid \citep{Bloomfield1995}. 

\subsubsection{Knoglevæv}
Ligesom musklerne, skal knogler og sener stimuleres, for at kunne opretholde deres styrke. 
Hvis ikke vævet stimuleres for eksempel gennem aktivitet, som inkluderer en form for stress, ved påvirkning dynamiske stød blandt andet ved hjælp fra tyngdekraften, vil der begynde at ske nedbrydning af knoglevævet. 
Allerede efter 1 uge har et studie af \citeauthor{Bloomfield1995} kunnet observere øget calciumudskillelse i urin og afføring. 
Dog varer det ofte op mod 1 til 2 måneder før der kan detekteres forandringer i knoglernes mineralindhold, da knoglevæv omsættes langsomt \citep{Bloomfield1995}.

\subsubsection{Stofskifte}
Stofskiftet er med i reguleringen af de kemiske processer, som sker i kroppen. 
Hormoner spiller en vigtig rolle inden for stofskiftet, heriblandt hormonet insulin, som er vigtigt for glukoseoptagelse i musklerne og for regulering af glukosekoncentrationen i blodet. 
En inaktiv livsstil vil føre til nedsat insulinfølsomhed og derfor en nedsat evne til at regulere glukosekoncentrationen i blodet. 
Allerede efter en uge med inaktivitet kan der ses en reducering i musklernes insulinfølsomhed ifølge studiet lavet af \citeauthor{Mikines1991}. 
Grunden til, at dette sker, kan skyldes, at der bliver mindre af det glukosetransporterende protein GLUT4 i munkecellerne. 
Endvidere vil muskelatrofi føre til, at der er mindre muskelvæv hvori glukosen kan optages \citep{Tabata1999}.

\noindent
Alle disse fysiske påvirkninger forårsaget af inaktivitet, kan være årsag til flere alvorlige kroniske lidelser, hvis der fortsættes en inaktiv livsstil. 
Fysisk inaktivitet kan eksempelvis føre til insulin resistens, hvilket øger risikoen for type-2 diabetes. 
Andre sygdomme som kan nævnes er osteoporose (knogleskørhed), hjertekar sygdomme og overvægt \citep{motionsraad2007}.