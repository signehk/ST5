\subsection{Fysiologiske følger af fysisk inaktivitet}\label{sec:FoeglerAfFiA}
Hvis kroppen ikke udsættes for belastning ved fysisk aktivitet, tilpasses den ved at nedgradere nogle biologiske mekanismer og processer. Omvendt forstærkes de ved stimulation. 
Blandt disse biologiske mekanismer og processer kan nævnes kredsløbet, stofskiftet, muskel- og knoglevækst \citep{motionsraad2007}.

De fysiske påvirkninger, forårsaget af inaktivitet, der beskrives i dette afsnit, kan være årsag til flere alvorlige kroniske lidelser, hvis der fortsættes en inaktiv livsstil. 
Fysisk inaktivitet kan eksempelvis føre til osteoporose, hjertekarsygdomme, overvægt samt insulinresistens, der øger risikoen for type-2 diabetes \citep{motionsraad2007}.

\subsubsection{Kredsløb}
Kredsløbet er en af de mekanismer som påvirkes relativt hurtigt ved fysisk inaktivitet. 
Et studie af \citeauthor{Convertino1995}, som foregik over fire uger, har påvist et fald i aerob kapacitet, som angiver den maksimale iltoptagelse i kroppen under fysisk arbejde over tid, med 5-6 \% pr. uge. 
Testpersonerne var både kvinder og mænd i aldersgruppen 18 til 45 år. 
Et fald i aerob kapacitet kan skyldes en reducering af hjertets slagvolumen både i hvile og under arbejde, grundet reducering i kroppens samlede blodvolumen. 
For at kompensere for dette øges pulsen for at opretholde minutvolumen af blod, der pumpes ud i kroppen. 
Et fald i blodvolumen udgør en kortsigtet reducering af aerob kapacitet \citep{Convertino1995}. 
Under tidsperioder med inaktivitet varende længere end ca. 12 uger, kan der yderligere ses en reduceret iltekstraktion i det perifere kredsløb \citep{Coyle1985}.

\subsubsection{Muskelvæv}
Ved fysisk inaktivitet stimuleres musklerne i mindre grad, hvilket fører til tab af muskelmasse grundet, at hastigheden for proteinnedbrydning i musklerne forløber hurtigere end proteinnydannelse, også kaldet proteinsyntese. 
Musklerne bliver derfor mindre, hvilket betegnes muskelatrofi. 
Flere studier påpeger, at der efter én til to ugers inaktivitet, kan ses en reduktion i muskelmasse, og at reduktionen af muskelmasse udelukkende skyldes en reduceret proteinsyntese \citep{Douglas2006, Bloomfield1995}. 
Desuden vil der også opleves et betydeligt tab af muskelkraft hos personer, der er inaktive over længere tid \citep{Bloomfield1995}. 

\subsubsection{Knoglevæv}
Ligesom musklerne, skal knogler og sener stimuleres, for at kunne opretholde deres styrke. 
Hvis ikke vævet stimuleres eksempelvis gennem fysisk aktivitet, som inkluderer en form for stress, vil der begynde at ske en nedbrydning af knoglevævet. 
Allerede efter én uge har et studie af \citeauthor{Bloomfield1995} kunnet observere øget calciumudskillelse i urin og afføring. 
Dog varer det ofte op mod én til to måneder før, der kan detekteres forandringer i knoglernes mineralindhold, da knoglevæv omsættes langsomt \citep{Bloomfield1995}.

\subsubsection{Stofskifte}
Stofskiftet indgår i reguleringen af de kemiske processer i kroppen. 
Hormoner spiller en vigtig rolle inden for stofskiftet, heriblandt hormonet insulin, som er vigtigt for glukoseoptagelse i musklerne og for regulering af glukosekoncentrationen i blodet. 
En inaktiv livsstil vil føre til nedsat insulinfølsomhed og derfor en nedsat evne til at regulere glukosekoncentrationen i blodet. 
Efter en uges inaktivitet kan der ses en reducering i musklernes insulinfølsomhed ifølge  \citeauthor{Mikines1991}. 
Grunden til dette kan skyldes, at der bliver mindre af det glukosetransporterende protein GLUT4 i muskelcellerne. 
Endvidere vil muskelatrofi føre til, at der er mindre muskelvæv, hvori glukosen kan optages \citep{Tabata1999}.