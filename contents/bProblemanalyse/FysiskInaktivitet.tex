\section{Fysisk inaktivitet} \label{sec:fys_inaktivitet}

Definitionen af både fysisk aktivitet og inaktivitet varierer afhængigt af, hvilken sundhedsinstans, der har opstillet definitionen. Center for Disease Control and Prevention (CDC) i USA anbefaler mindst $30$ minutters moderat arbejdsintensitet, såsom rask gang eller havearbejde, $5$ dage om ugen, eller $20$ minutters aktivitet af høj intensitet $3$ dage om ugen \citep{motionsraad2007,christensen2012}.
Samtidig definerer CDC forskellige niveauer af fysisk inaktivitet, hvoraf disse er henholdsvis anbefalet fysisk aktivitet, utilstrækkelig fysisk aktivitet, inaktivitet samt inaktivitet i fritiden. Heraf svarer utilstrækkelig fysisk aktivitet til et aktivitetsniveau ved moderat eller høj intensitet, der ligger under det anbefalede aktivitetsniveau, hvor der dog udføres mere end $10$ minutters fysisk aktivitet ugentligt. Ved niveuaet inaktivitet udføres der mindre end $10$ minutters ugentligt fysisk aktivitet ved moderat eller høj intensitet. Der er desuden ikke rapporteret fysisk aktivitet i den foregående måned i fritiden i niveauet inaktivitet  \citep{motionsraad2007,christensen2012}.

Sundheds- og sygelighedsundersøgelsen af \citeauthor{christensen2012} definerer fysisk inaktivitet ud fra ét enkelt spørgsmål vedrørende den mest passende beskrivelse af patientens fritidsaktiviteter igennem det sidste år. Svarmulighederne til dette spørgsmål er hård træning flere gange om ugen, motionsidræt eller tungt arbejde mindst fire timer om ugen, lettere motion mindst fire timer om ugen samt stillesiddende aktivitet. Besvarer patienten spørgsmålet med "Læser, ser fjernsyn eller har anden stillesiddende beskæftigelse", kategoriseres patienten som værende fysisk inaktiv \citep{motionsraad2007,christensen2012}.

Både Sundhedsstyrelsen og World Health Organization (WHO) definerer fysisk inaktivitet, som værende mindre end $2,5$ timers fysisk aktivitet om ugen. Det er valgt at tage udgangspunkt i Sundhedsstyrelsen og WHO's definition af fysisk inaktivitet, når begrebet omtales senere i projektet. Ud fra denne definition anslår Sundhedsstyrelsen, at $30-40~\%$ af den voksne danske befolkning er fysisk inaktive \citep{motionsraad2007}. 