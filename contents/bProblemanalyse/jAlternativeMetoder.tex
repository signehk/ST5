\section{Alternative metoder til aktivitetsmåling}
%% Objektiv metode - Overordnet

\subsection{Pedometer}
Skridttællerens primær funktion er at vise antal gået skridt indenfor en bestemt afstand.
Skridttælleren kan hertil også måle antal forbrændte kalorier, totale træningstid og afstanden
som brugeren har gået afhængigt af designet. Skridttæller kaldes også pedometer, og findes i
både mekanisk og elektronisk form. [1]

\subsection{Dobbeltmærket van}
Dobbeltmærket vand anvendes til måling af energiforbruget i en periode på 1 - 2 uger. Metoden er baseret på at der først indtages en mængde vand, som indeholder isotoperne brint (2^H) og ilt (18^O), som afviger fra normalt vand ved at have flere neutroner. Disse isotoper bliver optaget i vævsvæsken og bliver fordelt rundt omkring kroppen. Brintisotopen (2^H) udskilles som vand mens iltisotopen (18^O) både kan udskilles som kuldioxid (CO^2) og vand. CO^2 produktionen kan beregnes ved at trække eliminationen af 2^H fra eliminationen af 18^O. CO^2 produktionen bruges her som et udtryk for energiforbruget. [2] [3] [4] 
En begrænsning ved brugen af dobbeltmærket vand metoden er at den kun viser det gennemsnitlige aktivitetsmønster over en periode på 1-2 uger, i stedet for aktiviteten fra dag til dag [3]. Denne metode kan dog kombineres med den subjektive metode, spørgeskema, som belæg. Derudover kræver metoden tilgængelighed af isotoperne og indsprøjtning af disse for at kunne udfører målingen [3].

\subsection{Pulsmåler}
Pulsmålere bruges til at måle hjertefrekvensen. Der findes forskellige metoder til at detektere
puls, for eksempel måling af den elektriske spændingsforskel under hjertets cyklus. Denne
metode anvender typisk et bælte som patienten har rundt om thorax. [2] Nogle pulsmåler indeholder elektroder, og ved kontakt med hudens overflade vil den elektriske spændingsforskel blive målt. Fordelen ved pulsmålere er blandt andet tilstrækkelig hukommelse med høj tidsopløsning, god sammenhæng mellem pulsfrekvens og arbejdsintensitet ved en moderat intensitet eller højere. [2] En anden målemetode kaldet pulsoximeter måler iltmætningen i blodet for heraf at kunne registrere pulsen. Oximeteret består af lysemitterende dioder som udsender lysstråler gennem vævet og opfanges af en fotodetektor, hvorved iltindholdet i blodet kan bestemmes med udgangspunkt i (noget med lysmængde og iltmætning)[5] Selvom alle aldersgrupper kan anvende pulsmåling skal man være opmærksom på medicinering, da dette kan have virkning på hjertets rytme. Eksempelvis kan betablokkere sænke pulsen. [2] [6]
Selvom pulsmålere giver et godt overblik over pulsfrekvensen ved et moderat eller højere intensitet, indebærer den også en begrænsning, ved registrering af pulsen i forbindelse med inaktivitet ved let aktivitet (Vel ved inaktivitet og let aktivitet?). For at pulsen ikke bliver påvirket af følelsesmæssige ændringer på kroppen, såsom forskrækkelse, hvor energiforbruget vil afvige lidt, bruges flex-puls metoden. Denne metode har først en kalibreringsligning som bruges til at bestemme sammenhængen mellem arbejdsintensitet og puls hos den enkelte person. Ud fra kalibreringsligningen findes hvilepulsen, som kan bruges til at finde en flex-puls dvs. gennemsnittet mellem hvilepulsen og pulsen under letteste arbejde. [3]

\subsection{Aktivitetsarmbånd}
Aktivitetsarmbånd, som også er et elektronisk måleinstrument, bruges af forskellige
målgrupper for eksempel elite atleter/udøvere eller til almindelig dagligdags brug til holde styr på den fysiske aktivitet. Aktivitetsarmbånd er primært en kombination af pulsmåling og skridttæller. [7] Afhængigt af designet af aktivitetsarmbåndet kan yderligere funktioner/egenskaber såsom søvn monitorering, antal forbrændte/indtaget kalorier, sende notifikationer til mobilen, bærebare under vand og alarmering måles. [8] [9]  
Aktivitetsarmbånd kan udover skridttælling også måle fysiske parametre som puls, den tilbagelagte distance, søvn og kalorieindtag samt kalorier forbrændt [2] [7]. Aktivitetsarmbåndet kan blive synkroniseret til andre enheder såsom computer og mobil, på denne måde kan data blive overført, analyseret og sammenlignet over en længere periode. [2]
I forhold til aktivitetsmåler har et studie sammenlignet 8 forskellige aktivitetsmålere med et Oxygon Mobile, som er en bærbar udstyr der måler den metaboliske respons ved udførelse af arbejde. Studiet viste en gennemsnitlig procent fejl i et spændet 9% - 24%. [11] [10]

\subsection{Sammenligning/sammenfatning}
Til monitorering af aktivitetsniveautet hos kronikere i praksis kan nævnte objektive målemetoder anvendes. Det afgørende for valget af målemetode er hensigten med monitoreringen. Hvis patientens energiforbrug ønskes at blive målt dagligt er dobbeltmærket vand ikke optimalt. Derimod kan skridttæller anvendes, da denne som tidligere nævnt kan vise antal forbrændte kalorier. Både skridttæller og aktivitetsarmbånd kan patienten monitorere på håndleddet og kan anvendes under træning eller i hverdagen.Teknologierne kan bruges i forbindelse med selvkontrol af aktivitetsniveau, hvilket vil betyde at patienten har mere ansvar for monitorering af aktivitetsniveauet. Fordelen ved et aktivitetsarmbånd er målingen af pulsen og andre eventuelle ønsket fysiologiske parametre og således udvides spændet for hvilke sygdomme der kan arbejdes på.  Da målgruppen er bred skal aktivitetsarmbåndet tilpasses den enkelte person. En udfordring i aktivitetsarmbånd er hvordan man sørger for patienten beholder armbåndet under hele målingsperioden. Da aktivitetsarmbåndet indeholder nødvendige målingsparametre vurderes dette mest hensigtsmæssigt at arbejde videre på i projektet. 


Indledning
En anden måde at dokumentere fysisk aktivitet på er ved anvendelse af objektive målemetoder. Disse metoder er ikke præget af patienternes egen vurdering af den fysiske aktivitet, men måler derimod mængden af aktivitet direkte. Dette kan give mere præcist måling end subjektive målemetoder som er baseret på patients egne erindringer/oplevelse af den fysiske aktivitetsniveau, hvilket ikke altid er i overensstemmelse med definitionen af fysisk aktivitet.   Disse metoder metoderne kombineres med hinanden eller bruges individuelt afhængigt af analysens formål. Derudover kan metoderne også kombineres med de subjektive metoder til at styrke dokumentationen af den udførte fysiske aktivitet. En stigning af brugen af objektive metoder til måling af fysisk aktivitet ses hos nyere undersøgelser. Denne metode er blevet mere udbredt gennem de seneste år, hvor den fysiske aktivitet måles ved anvendelse af for eksempel dobbeltmærket vand, accelerometre/pedometre, pulsmåler og aktivitetsarmbånd. [2] [3] 



%Et studie af Eduardo Ferriolli et al (Physical Activity Monitoring: A Responsive and Meaningful Patient-Centered Outcome for Surgery, Chemotherapy, or Radiotherapy?) målte 

%[1] http://www.explainthatstuff.com/how-pedometers-work.html 

%[2] 

%[3]http://sundhedsstyrelsen.dk/publ/MER/2007/FYSISK_INAKTIVITET-KONSEKVENSER_OG_SAMMENHAENGE2007.PDF (objektive målemetoder)

%[4] Pulse Oximetry Training Manual (se under mappen litteratur)

%Alt muligt andet pis:

%“Background: With the ever-increasing availability of health information technology (HIT) enabling health consumers to measure, store, and manage their health data (e.g., self-tracking devices) ”
%FYSISK_INAKTIVITET-KONSEKVENSER_OG_SAMMENHAENGE2007.PDF