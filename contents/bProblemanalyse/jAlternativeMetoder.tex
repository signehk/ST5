\section{Alternative metoder til aktivitetsmåling}
%% Objektiv metode - Overordnet

En anden måde at dokumentere fysisk aktivitet er ved anvendelse af objektivemålemetoder.
Den objektive metode er ikke præget af patienternes egen vurdering af den fysiske aktivitet, men måler den direkte \cite{pedersen2011, adamo2009}. 
Denne metode er blevet mere udbredt over de seneste år \citep{pedersen2011}, hvor den fysiske aktivitet måles ved anvendelse af for eksempel double labeled water, accelerometere/pedometer eller pulsmåler \cite{pedersen2011, motionsraad2007, adamo}. 


%Udover aktivitetsdagbog kan mere objektive målemetoder såsom pulsmåling, dobbeltmærket vand, accelerometer og skridttæller eller en kombination af flere anvendes til monitorering af aktivitetsniveauet hos individer [1?],\cite{motionsraad2007}. 

%% Objektiv metode - Konkret målemetode

Pulsmåler bruges til at måle hjertefrekvensen. 
Der findes flere forskellige metoder til at detektere pulsen, her i blandt ved at måle den elektriske spændingsforskel som hjertet udsender ved hjælp af elektroder på hudens overflade, denne metode anvender typisk et bælte som brugeren har rundt om thorax \cite{motionsraad2007} [MANGLER]. 

%% Objektiv metode - Konkret målemetode

En anden metode kaldet pulsoximeter, måler iltmætningen i blodet for heraf at kunne registrere pulsen [4?]. 
En pulsmåler indeholder elektroder, og ved kontakt med hudens overflade vil den elektriske spændingsforskel blive målt. 

Selvom pulsmålere giver et godt overblik over pulsfrekvensen ved et moderat eller højere intensitet, indebærer den også en begrænsning, ved registrering af pulsen i forbindelse med inaktivitet ved let aktivitet. 
For at pulsen ikke bliver påvirket af følelsesmæssige ændringer på kroppen, såsom forskrækkelse, hvor energiforbruget vil afvige lidt,  bruges flex-puls metoden. 
Denne metode har først en kalibreringsligning som bruges til at bestemme sammenhængen mellem arbejdsintensitet og puls hos den enkelte person. 
Ud fra kalibreringsligningen findes hvilepulsen, som kan bruges til at finde en flex-puls dvs. gennemsnittet mellem hvilepulsen og pulsen under letteste arbejde. \cite{motionsraad2007}

Dobbeltmærket vand er en metode som måler energiomsætning i kroppen. 

%% Objektiv metode - Konkret målemetode

Skridttællerens primær funktion er at vise antal gået skridt indenfor en bestemt afstand. 
Skridttælleren kan hertil også måle antal forbrændte kalorier, totale træningstid og afstanden som brugeren har gået afhængigt af  designet. Skridttæller kaldes også pedometer, og findes i både mekanisk og elektronisk form. [1?]

Både skridttæller og aktivitetsarmbånd kan patienten monitorere på håndleddet og kan bruges under træning eller i hverdagen. 
Teknologierne kan bruges i forbindelse med selvkontrol af aktivitetsniveau, hvilket vil betyde at patienten har mere ansvar for monitorering af aktivitetsniveauet. 

Aktivitetsarmbånd, som også er et elektronisk måleinstrument, bruges af forskellige målgrupper for eksempel elite atleter/udøvere, som forbereder eller øver til en konkurrence. 
Aktivitetsarmbånd kan udover skridttælling også måle fysiske parametre som puls, søvn og kalorieindtag samt kalorier forbrændt. [2?] 

Aktivitetsarmbåndet kan blive synkroniseret til andre enheder såsom computer og mobil, på denne måde kan data også blive overført. 

%Et studie af Eduardo Ferriolli et al (Physical Activity Monitoring: A Responsive and Meaningful Patient-Centered Outcome for Surgery, Chemotherapy, or Radiotherapy?) målte 

%[1] http://www.explainthatstuff.com/how-pedometers-work.html 

%[2] 

%[3]http://sundhedsstyrelsen.dk/publ/MER/2007/FYSISK_INAKTIVITET-KONSEKVENSER_OG_SAMMENHAENGE2007.PDF (objektive målemetoder)

%[4] Pulse Oximetry Training Manual (se under mappen litteratur)

%Alt muligt andet pis:

%“Background: With the ever-increasing availability of health information technology (HIT) enabling health consumers to measure, store, and manage their health data (e.g., self-tracking devices) ”
%FYSISK_INAKTIVITET-KONSEKVENSER_OG_SAMMENHAENGE2007.PDF