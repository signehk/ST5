\section{Alternative metoder til aktivitetsmåling} \label{sec:alternativemetoder}

For at forbedre bestemmelsen af patienters aktivitetsmønster, er det relevant at undersøge, hvilke nye metoder, der potentielt kan anvendes i almen praksis. Her fokuseres hovedsageligt på fremtidige objektive metoder foreslået af Sundhedsstyrelsen, med henblik på at opnå større præcision i monitorering over længere perioder \citep{motionsraad2007}.

\subsection{Dobbeltmærket vand}

Målemetoden dobbeltmærket vand anvendes til måling af det overordnede energiforbrug i en periode på $1-2$ uger. I starten af måleperioden indtager patienten en afmålt mængde vand med brint- og iltisotoper, som har et højere antal neutroner end normalt vand. Efter indtagelse vil isotoperne blive optaget i vævsvæsken og fordelt i kroppen, hvor brintisotopen udskilles som vand, mens iltisotopen både kan udskilles som kuldioxid og vand. Med udgangspunkt i dette, kan produktionen af kuldioxid beregnes ud fra væskeprøver, ved at trække antallet af eliminerede brintisotoper fra eliminationen af iltisotoper. Herved kan kuldioxidsproduktionen bruges som udtryk for energiforbruget \citep{motionsraad2007,pedersen2011}.

Begrænsningen ved denne metode, er at der kun opnås indblik i det gennemsnitlige aktivitetsmønster over måleperioden, i stedet for aktiviteten for hver enkelt dag. Metoden kan med fordel kombineres med spørgeskemaer, for at opnå større indblik i patientens aktivitetsmønster. 
Ulemper ved metoden er, at der forinden skal tages urin-, spyt- eller blodprøver før dosering af isotoperne, samt op til flere prøver gennem måleperioden og høje økonomiske udgifter til indkøb af isotoper. Dobbeltmærket vand anvendes i dag primært i forskningssammenhæng og ikke hos alment praktiserende læger \citep{motionsraad2007}.

\subsection{Pulsmåler}

Pulsmålere anvendes til at måle hjertefrekvensen. Til dette anvendes eksempelvis et bælte rundt om thorax, der kan måle den elektriske spændingsforskel under hjertets cyklus. Pulsmålere, der måler den elektriske spændingsforskel, kræver, at elektroderne i måleren har kontakt med hudens overflade, og fordelen ved denne målemetode er længerevarende målinger af høj tidsopløsning og god sammenhæng mellem puls og arbejdsintensitet ved moderat til hård aktivitet \citep{motionsraad2007}. 

Alle aldersgrupper kan anvende pulsmålere, men afhængigt af medicin kan pulsen stige eller falde, hvilket lægen skal tage højde for ved måling. Ved måling vil flex-puls fremgangsmåden oftest anvendes, for at undgå pludselige ændringer ved eksempelvis følelsesmæssige påvirkninger. Her kalibreres måleren med udgangspunkt i sammenhængen mellem arbejdsintensitet og puls hos den enkelte person, hvorved flex-pulsen findes som gennemsnit af hvilepulsen og puls ved let arbejde. Når patientens puls måles efterfølgende, vil en puls over flex-pulsen oversættes til energiforbrug gennem en kalibreringsligning, mens en puls under flex-pulsen vil blive oversat til energiforbrug ved hvilestofskiftet \citep{motionsraad2007}.

\subsection{Aktivitetsarmbånd}

Aktivitetsarmbånd består ofte af en kombination af pulsmålere og skridttællere. Afhængigt af mærke og model, vil der være mulighed for flere funktioner, såsom søvnmonitorering, estimat af antal forbrændte kalorier, GPS og anvendelse i sammenhæng med andet elektronisk udstyr, ved eksempelvis synkronisering og analyse af de optagede data ved aktivitet. Synkronisering og analyse kan derefter anvendes til at opnå overblik over aktivitet gennem længere perioder \citep{pedersen2011, rudner2016, chiauzzi2014}.

Ved anvendelse af aktivitetsmålere er der en fejlmargin, som i et studie er fundet til mellem $9~\%$ og $24~\%$. Studiet sammenlignede otte forskellige aktivitetsmålere med et bærbart system, der måler den metaboliske respons ved aktivitet. Det er desuden fundet, at blandt andet præcisionen for skridttælling og tilbagelagt afstand varierer mellem forskellige aktivitetsarmbånd \citep{chiauzzi2014}.

\subsection{Sammenfatning} \label{sec:prob_sammenfatning}

Til monitorering af aktivitetsniveauet hos hypertensive patienter i almen praksis vil det være relevant at anvende en eller flere af ovenstående målemetoder, med henblik på at opnå et mere konkret og objektivt indblik i patienternes aktivitetsmønstre. Fordelen ved metoderne er, at der opstår bias i mindre grad som følge af grundene beskrevet i \autoref{NuMetode}, mens ulemper involverer blandt andet pris og tilvænning til ny elektronik. 

For at opnå højest mulig præcision i løbet af dagen og fra dag til dag, vælges det at frasortere dobbeltmærket vand og skridttællere som følge af, at disse metoder anvendes til måling af gennemsnittet i en længere periode. Pulsmåling kan også anvendes til aktivitetsmonitorering, men vælges ikke som fokusområde i denne MTV, hvilket vælges da medicin kan have en indflydelse på pulsmålerens vurdering af fysisk aktivitet. 

Da aktivitetsarmbånd giver en god mulighed for at opnå indsigt i patientens daglige aktivitetsmønster, eftersom de nemt kan bæres døgnet rundt og giver mulighed for synkronisering med blandt andet computere, hvorved dataoverførsel og -analyse gøres let i hjemmet og ved lægebesøg, vælges disse som fokusområde for MTVen. Samtidigt er det muligt at finde aktivitetsarmbånd, som kan kende forskel på reel aktivitet og nogle få skridt mellem eksempelvis sofa og køkken, hvor skridttællere også vil måle få skridt som værende fysisk aktivitet. Ved anvendelse af aktivitetsarmbånd frem for simple skridttællere, kan der af den grund opnås et mere præcist og detaljeret billede af patientens aktivitetsmønster i løbet af dagen.