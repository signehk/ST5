\section{Nuværende metoder til aktivitetsmåling}

Inden for det danske sygehusvæsen, defineres fysisk aktivitet som værende en aktivitet, der forhøjer energiomsætningen. 
Dette betyder at alt mellem indkøb og gåture, til målrettet fysisk træning, kan defineres som værende fysisk aktivitet.\citep{gupta2013, terkelsen2015}

Sundhedsstyrelsen anbefaler desuden et aktivitetsniveau på mindst 30 minutters motion af moderat intensitet hver dag hele ugen. 
I forbindelse med dette, er moderat densitet blevet defineret som $40-59\%$ af maksimal iltoptagelse, $64-74\%$ af makspuls eller aktivitet, der gør patienten lettere forpustet, uden at forhindre muligheden for samtale. 
For at patienten bliver defineret som værende fysisk inaktiv, kræver det af den grund mindre end $2.5$ timers fysisk aktivitet om ugen.\citep{gupta2013}

I forbindelse med monitorering af aktivitetsniveauet for patienter ved klinikbesøg, kan den fysiske aktivitet bestemmes med udgangspunkt i flere forskellige undersøgelsesmetoder \citep{gupta2013}. 
Måden hvorpå aktiviteten monitoreres, kan opdeles i to kategorier: objektiv og subjektiv \citep{gupta2013, adamo2009}. 

%% Subejtiv metoder - Overordnet

En almindelig subjektiv metode, der anvendes er selvudfyldt dokumentation, der typisk giver et indblik i type af aktivitet, intensitet, hyppighed, samt tidsperiode for ydet aktivitet \citep{adamo2009}. Dertil er der forskellige måder at dokumentere den fysiske aktivitet, som f.eks. en aktivitetslog, aktivitetsdagbog, spørgeskemaer og lignende \citep{adamo2009}. 

%% Subjektive metoder - Konkrete beskrivelser

Spørgeskemaer tager udgangspunkt i faste spørgsmål omhandlende patientens fysiske aktivitet i løbet af dagligdagen \citep{muller2009}. 
Disse omhandler blandt andet transport til og fra arbejde, motionsvaner, tid brugt foran eksempelvis computer eller TV og ønsker om eventuelle ændringer af patientens aktivitetsvaner \citep{gupta2013, vestergaard2012}. 

Alternativt anvendes aktivitetsdagbøger \citep{muller2009} for at opnå en mere fyldestgørende indsigt i patientens aktivitetsmønster \citep{gupta2013}. 
Dagbogen fungerer som en logbog, hvori den primære aktivitet siden sidste notation, nedskrives med bestemte intervaller. 
Denne monitoreringsmetode giver et bedre indblik i patientens fysiske aktivitet gennem dagen, men er også mere tidskrævende at anvende for især patient men også læge.\citep{gupta2013}

%% Subjektive metoder - Overordnet ulemper 

Disse subjektive metode anvendes på grund af dens lave omkostning, lave patientbyrde, og generelle accept, samtidig med at den er velegnet til dokumentation af diversiteten i forhold til hvilken fysisk aktivitet der er ydet \citep{adamo2009}.

Denne type aktivitetsførelse forbindes dog med en fejlrepræsentation i forhold til den reelle fysiske aktivitet. Da det er en subjektiv dokumentationsmetoder, har patienter en tendens til enten at over- eller undervurderer deres egnen fysiske aktivitet \citep{adamo2009}. 
Et studie oplyser at 72 \% af patienter, af alderen 19 eller derunder, overestimerer deres fysiske aktivitet ved selvudfyldelse, i forhold til aktiviteten målt med objektiv/direkte aktivitetsførelse (accelerometer, pedometer, og lignende.) \citep{adamo2009.}

Problematikken er således at de subjektive metoder ikke altid er i stand til at repræsentere den reelle fysiske aktivitet, selvom metoderne anses som værende valide \citep{FYSISK_AKTIVITET, gupta2013}. 

%%%%%%%%%%%%%%%%%%%%%%%%%%%%%%%%%%%%%%%%%%%%%%%

Som et led i behandling af kronikere, såsom overvægtige eller diabetespatienter, udleveres der også skridttællere (accelerometre)\citep{muller2009, jensen2012, snorgaard2010}. 
Accelerometret vil give et mere detaljeret overblik over patientens aktivitetsmønster end spørgeskemaer og dagbøger, grundet muligheden for at monitorere kontinuert gennem længere tid. 
Der opstår dog komplikationer i forbindelse med anvendelsen, som følge af accelerometrets manglende evne til at opfange forskellige aktiviteter. 
Af den grund anvendes det kun til at danne et billede af, hvor meget tid patienten bruger på generel bevægelse. \citep{gupta2013}