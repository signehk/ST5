\section{Nuværende metoder til aktivitetsmåling} \label{NuMetode}

I forbindelse med monitorering af patienters aktivitetsniveau ved besøg hos praktiserende læge, kan mængden af fysisk aktivitet bestemmes med udgangspunkt i flere forskellige undersøgelsesmetoder \citep{motionsraad2007}. 
Måden, hvorpå aktiviteten monitoreres, kan opdeles i to kategorier: subjektiv og objektiv \citep{motionsraad2007, adamo2009}. 
 
%% Subejtiv metoder - Overordnet
\subsection{Subjektive målemetoder}

En almindelig subjektiv metode, der anvendes, er selvudfyldt eller -rapporteret dokumentation, der typisk giver et indblik i typen af aktivitet, intensitet, hyppighed samt tidsperiode for hver enkelt aktivitet \citep{adamo2009}. Dertil er der forskellige måder at dokumentere denne fysiske aktivitet - eksempelvis aktivitetslog, aktivitetsdagbog eller spørgeskema \citep{adamo2009}. 

%% Subjektive metoder - Konkrete beskrivelser

Spørgeskemaer tager udgangspunkt i faste spørgsmål omhandlende patientens fysiske aktivitet i løbet af dagligdagen \citep{muller2009}. 
Disse omhandler blandt andet transport til og fra arbejde, motionsvaner, tid brugt foran eksempelvis computer eller TV samt ønsker om eventuelle ændringer af patientens aktivitetsvaner \citep{motionsraad2007, vestergaard2012}. 

Alternativt anvendes aktivitetsdagbøger for at opnå en mere fyldestgørende indsigt i patientens aktivitetsmønster \citep{motionsraad2007,muller2009}. 
Dagbogen fungerer som en logbog, hvori den primære aktivitet siden sidste notation, nedskrives med bestemte intervaller. 
Denne monitoreringsmetode giver et indblik i patientens fysiske aktivitet gennem dagen, men er også mere tidskrævende at anvende for patienten \citep{motionsraad2007}.

%% Subjektive metoder - Overordnet ulemper 

Subjektive metoder anvendes især grundet deres lave omkostning, ofte lave byrde for patienten, samtidigt med, at de er velegnede til dokumentation af diversiteten i forhold til, hvilken fysisk aktivitet, der er ydet \citep{adamo2009}.

Da det er en subjektiv dokumentationsmetode, har patienter en tendens til enten at over- eller undervurdere mængde eller intensitet af deres ydede fysiske aktivitet \citep{adamo2009}. 
En metaanalyse oplyser blandt andet, at grupper af mænd og kvinder i gennemsnit overestimerer deres fysiske aktivitet med $44~\%$ i forhold til aktiviteten målt med objektiv eller direkte aktivitetsførelse ved brug af en skridttæller eller lignende. \citep{prince2008}.

Denne type aktivitetsregistrering forbindes dog med en fejlrepræsentation i forhold til den reelle fysiske aktivitet. 

Problematikken er således at de subjektive metoder ikke altid er i stand til at repræsentere den reelle fysiske aktivitet, selvom metoderne anses som værende valide \citep{pedersen2011, motionsraad2007}. 

%%%%%%%%%%%%%%%%%%%%%%%%%%%%%%%%%%%%%%%%%%%%%%%
\subsection{Objektive målemetoder}

Som et led i behandling af visse kronikere, såsom overvægtige eller diabetespatienter, kan også udleveres skridttællere, der benytter accelerometre til at registrere fysisk aktivitet som gang eller løb \citep{muller2009, jensen2012, snorgaard2010}. 
Skridttælleren kan give et mere beskrivende overblik af patientens daglige aktivitetsniveau end spørgeskemaer og dagbøger, grundet muligheden for at monitorere kontinuert gennem længere tid uden at være tidskrævende og dermed til gene for patienten. Brugen af en skridttæller giver dog ikke mulighed for at se aktivitetsmønsteret i løbet af dagen, med mindre patienten ofte noterer sin foreløbige aktivitet, da den udelukkende viser antal skridt siden sidste nulstilling.  
Der opstår yderligere komplikationer i forbindelse med anvendelsen, da accelerometre i form af skridttællere ikke er i stand til at måle forskellige former for aktiviteter udover gang og løb. 
Af disse grunde anvendes skridttællere kun til at danne et billede af, hvor meget tid patienten bruger på generel bevægelse  \citep{motionsraad2007}.