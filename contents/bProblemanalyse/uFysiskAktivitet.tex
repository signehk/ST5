\section{Fysisk aktivitet}\label{sec:prob_fysaktiv}

I det danske sundhedsvæsen defineres fysisk aktivitet som værende en aktivitet, der forhøjer energiomsætningen. Dette betyder, at alt fra indkøb og gåture til målrettet fysisk træning, kan defineres som værende fysisk aktivitet \citep{motionsraad2007, terkelsen2015}.

Som nævnt i \autoref{sec:indledning} anbefaler Sundhedsstyrelsen et aktivitetsniveau på minimum 30 minutters motion af moderat intensitet hver dag. I forbindelse med dette, er moderat intensitet defineret som $40-59~\%$ af maksimal iltoptagelse, $64-74~\%$ af makspuls eller som et aktivitetsniveau, der gør patienten lettere forpustet, uden at forhindre muligheden for samtale \citep{motionsraad2007}.
Ud over anbefalingerne til voksne er det understreget, at børn skal være fysisk aktive minimum 60 minutter dagligt \citep{pedersen2011}. 
