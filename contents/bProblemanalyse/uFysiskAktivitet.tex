\section{Fysisk aktivitet}

Inden for det danske sundhedsvæsen, defineres fysisk aktivitet som værende en aktivitet, der forhøjer energiomsætningen. Dette betyder at alt mellem indkøb og gåture, til målrettet fysisk træning, kan defineres som værende fysisk aktivitet\citep{motionsraad2007, terkelsen2015}.

Som nævnt i \ref{sec:indledning} anbefaler Sundhedsstyrelsen et aktivitetsniveau på mindst 30 minutters motion af moderat intensitet hver dag. I forbindelse med dette, er moderat intensitet defineret som $40-59~\%$ af maksimal iltoptagelse, $64-74~\%$ af makspuls eller aktivitet, hvilket gør patienten lettere forpustet, uden at forhindre muligheden for samtale. 

Definitionen af både fysisk aktivitet og inaktivitet varierer, afhængigt af hvilken sundhedsinstans, der har opstillet definitionen. Center for Disease Control and Prevention (CDC) i USA. har valgt at anbefale mindst $30$ minutters moderat arbejdsintensitet, såsom rask gang eller havearbejde, $5$ dage om ugen, eller 20 minutters aktivitet af høj intensitet 3 dage om ugen. Samtidig definerer CDC forskellige niveauer af fysisk inaktivitet, som går fra utilstrækkelig fysisk aktivitet med mere end $10$ minutters aktivitet om ugen. Under dette niveau er inaktivitet, der defineres som mindre end $10$ minutters aktivitet om ugen \citep{motionsraad2007}.

Sundheds- og sygelighedsundersøgelsen definere fysisk inaktivitet ud fra et enkelt spørgsmål, vedrørende den bedste beskrivelse af patientens fritidsaktiviteter. Her indeholder $3$ af $4$ svar forskellige niveauer af aktivitet, og besvarer patienten spørgsmålet med svaret "Læser, ser på fjernsyn eller har anden stillesiddende beskæftigelse.", kategoriseres patienten som fysisk inakiv\citep{motionsraad2007}.

Både Sundhedsstyrelsen og World Health Organization (WHO) definere fysisk inaktivitet som værende mindre end 2,5 timers fysisk aktivitet om ugen. Af den grund vælges det i projektet, at tage udgangspunkt i Sundhedsstyrelsen og WHO's definition af fysisk inaktivitet, når begrebet omtales senere i projektet\citep{motionsraad2007}.