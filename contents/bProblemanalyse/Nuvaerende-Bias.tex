\section{Bias ved nuværende anvendelsesmetoder}
Der findes forskellige måder at dokumentere fysisk aktivitet. En almindelig metode der anvendes er selvudfyldt dokumentation, som f.eks. en aktivitetslog, aktivitetsdagbog, spørgeskemaer og lignende \citep{adamo2009}. 
Metoden anvendes på grund af dens lave omkostninger, lave patientbyrde, og generelle accept, samtidig med at den er velegnet til dokumentation af diversiteten i forhold til hvilken fysisk aktivitet der er ydet \citep{adamo2009}.
Ud fra den selvudfyldte dokumentation dannes der typisk et indblik over type af aktivitet, intensitet, hyppighed, samt tidsperiode for ydet aktivitet \citep{adamo2009}. 

Denne type aktivitetsførelse forbindes dog med en fejlrepræsentation i forhold til den reelle fysiske aktivitet, da de noteret værdier er præget af subjektive holdninger \citep{adamo2009}. Dette påvirker den noteret værdi, i den forstand at patienter har en tendens til enten at over- eller undervurderer den fysiske aktivitet \citep{adamo2009}.
Et studie oplyser at 72 \% af patienter, af alderen 19 eller derunder, overestimerer deres fysiske aktivitet ved selvudfyldelse, i forhold til aktiviteten målt med objektiv/direkte målemetoder (accelerometer, pedometer, og lignende.) \citep{adamo2009}


En aktivitetslog kan også virke motiverende \citep{adamo2009}.

Et studie fremhæver den direkte målemetode som værende den mest bedre egnet til standard sammenligning \citep{adamo2009}.

En sund og aktiv livsstil, kan for børn være en god forhindringsmetode for at inaktivitet ikke leder videre ind i det voksne liv \citep{adamo2009}. 

Det indikeres også at tiltrods for at spørgeskemaet er der lav praktisk, og mere kost effektive, så skal der tages forbehold på data, da de ikke oplyser den reelle data \citep{adamo2009}. 

Dette er ikke tilfældet ved anvendelse af accelerometre, DLW, og lignende, da dette er baseret på objektive målinger \citep{adamo2009}. 

Specielt børn kan have svært ved at udfylde informationerne i spørgeskemaet, da de oftest er drevet af impulshandlinger hvori der ydes kortvarigt men intens fysisk aktivitet \citep{adamo2009}. 