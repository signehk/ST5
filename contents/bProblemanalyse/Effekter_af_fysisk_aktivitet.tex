\subsection{Effekter af fysisk aktivitet}
Fysisk aktivitet kan påvirke kroppens fysiologiske tilstand på mange måder, herunder kan det i forskellige grader forbedre blandt andet immunforsvar, lungefunktion, blodtryk, muskelstyrke- og udholdenhed samt kroppens bevægelighed og vægt. Desuden bemærkes en forbedring af glukosetransportering til muskelcellerne, hvilket medfører, at insulinniveauet er lavere hos folk, der udfører regelmæssig fysisk aktivitet. \citep{andersen2001, martini2015}. Dette betyder, at forskellige sygdomme, der relateres til nogle af de nævnte fysiologiske funktioner, kan påvirkes ved fysisk aktivitet.

Flere studier indikerer, at fysisk aktivitet kan have en forebyggende effekt på forskellige folke- og livsstilssygdomme \citep{warburton2010}. Nogle af disse folke- og livsstilssygdomme er muskel-og skeletlidelser, stress, samt en række kredsløbssygdomme såsom hjertekarsygdomme, hypertension, overvægt og type-2 diabetes. Foruden disse forebygger fysisk aktivitet også nogle kræfttyper, herunder tyktarmskræft og brystkræft. De nævnte lidelser er gældende for alle aldersgrupper, og foruden disse er særlige effekter af fysisk aktivitet gældende for enkelte aldersgrupper. Eksempelvis udskyder eller reducerer den ældre del af befolkningen, der udfører fysisk aktivitet, den aldersrelaterede reduktion i funktionsevne, som forventes med alderen. Risikoen for apopleksi og islæmisk hjertesygdom nedsættes samtidig som følge af fysisk aktivitet hos ældre \citep{pedersen2011,
warburton2010}. 

Fysisk aktivitet kan ligeledes have en psykisk effekt. Det kan blandt andet forebygge psykiske lidelser som depression, angst og demens cite{pedersen2011}. De psykologiske påvirkninger kan skyldes, at endorfinkoncentrationen i blodet øges ved fysisk aktivitet. Endorfiner virker som kroppens egen produktion af morfinlignende stoffer \citep{kessing2016}. Større overskud, mere selvtillid samt bedre social trivsel er også ofte en effekt af fysisk aktivitet \citep{sundhedsstyrelsen2006}. 

Mange af de nævnte sygdomme, både de fysiske og psykiske, kan desuden behandles med fysisk aktivitet. Fysisk aktivitet kan være den primære behandlingsmetode eller det kan være en del af behandlingen, eksempelvis i samspil med medicinsk behandling. Type-2 diabetikere og hypertensive er eksempler på patientgrupper, hvor fysisk aktivitet ofte er en del af behandlingsforløbet, hvor graden af lidelsen har betydning for om fysisk aktivitet og andre livsstilsændringer er den primære behandling eller om behandlingen skal kombineres med medicin. Ved behandling af visse sygdomme eller tilstande skal der tages hensyn til, hvilken form for fysisk aktivitet, der egner sig til forskellige patientgrupper, da det ellers kan have en skadende effekt. Nogle af disse patientgrupper er eksempelvis artrosepatienter, der specielt skal undgå overbelastning af led. Det kan også være gravide, som skal undgå fysisk aktivitet, hvor uventede stød kan forekomme \citep{andersen2001, pedersen2011}.
