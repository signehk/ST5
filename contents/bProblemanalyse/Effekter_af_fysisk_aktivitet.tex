\section{Effekter af fysisk aktivitet}
Fysisk aktivitet kan påvirke kroppen fysiologisk på mange måder, herunder kan det i forskellige grader forbedre blandt andet immunforsvar, lungefunktion, blodtryk, muskelstyrke- og udholdenhed samt kroppens bevægelighed og vægt. Desuden sker en forbedring af glukosetransportering til muskelcellerne, hvilket medfører, at insulinniveauet er lavere hos folk, der udfører fysisk aktivitet. cite{andersen2001, martini2015}. Dette medfører, at forskellige sygdomme, der relateres til nogen af de nævnte fysiologiske funktioner, kan påvirkes ved fysisk aktivitet.

Flere studier indikerer, at fysisk aktivitet kan have en forebyggende effekt på forskellige folke- og livsstilssygdomme cite{warburton2010}. Nogle af disse folke- og livsstilssygdomme er muskel-og skeletlidelser, stress, samt en række kredsløbssygdomme såsom hjertekarsygdomme, hypertension, overvægt og type-2 diabetes. Foruden disse forebygger fysisk aktivitet nogle kræfttyper, herunder tyktarmskræft og brystkræft. De nævnte lidelser er gældende for alle aldersgrupper, og foruden disse er særlige effekter af fysisk aktivitet gældende for enkelte aldersgrupper. Eksempelvis udskyder eller reducerer den ældre befolkning, der udfører fysisk aktivitet, det aldersrelaterede reducering i funktionsevne, som forventes med alderen. Risikoen for apopleksi og islæmisk hjertesygdom nedsættes som følge af fysisk aktivitet hos ældre \cite{pedersen2011,
warburton2010}. 

Fysisk aktivitet kan ligeledes have en psykisk effekt. Det kan blandt andet forebygge psykiske lidelser som depression, angst og demens cite{pedersen2011}. De psykologiske påvirkninger kan skyldes, at endorfinkoncentrationen i blodet ændres ved fysisk aktivitet, og endorfin virker som kroppens egen produktion af morfinlignende stoffer \cite{kessing2016}. Større overskud, mere selvtillid og overskud samt bedre social trivsel kan ligeledes være en effekt af fysisk aktivitet \cite{sundhedsstyrelsen2006}. 

Mange af de nævnte sygdomme, både de fysiske og psykiske, kan desuden behandles med fysisk aktivitet. Den fysiske aktivitet kan være den primære behandlingsmetode eller det kan være en del af behandlingen, eksempelvis i samspil med medicinsk behandling. Type-2 diabetikere og hypertensive er eksempler på patientgrupper, hvor fysisk aktivitet ofte er en del af behandlingsforløbet, hvor graden af lidelsen har betydning for om fysisk aktivitet og andre livsstilsændringer er den primære behandling eller om det kombineres med medicinsk behandling. Ved behandling af sygdomme skal tages hensyn til, hvilken form for fysisk aktivitet, som udføres af forskellige patientgrupper, da det ellers kan have en skadende effekt. Nogle af disse patientgrupper er eksempelvis artrosepatienter, der specielt skal undgå overbelastning af led, samt gravide, som skal undgå fysisk aktivitet, hvor uventede stød kan forekomme \cite{andersen2001, pedersen2011}.
