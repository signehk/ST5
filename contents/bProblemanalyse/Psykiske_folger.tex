\subsection{Psykiske følger af fysisk inaktivitet}
Fysisk inaktivitet er en risikofaktor for visse psykiske lidelser. Eksempelvis er det påvist, at forekomsten af depression er lavere bland fysisk aktive end bland fysisk inaktive \citep{motionsraad2007}. Ud over depression er der nogen evidens for, at andre psykiske sygdomme såsom angst, misburg, skizofreni og spiseforstyrrelser kan have gavn af større eller mindre mængde fysisk aktivitet i relation til sygdomsbehandlingen \citep{kessing2016}. Fysisk inaktivitet kan både have en rolle for sygdomsudviklingen samt den videre progredieren af sygdommen, hvor fysisk inaktivitet kan forværre symptomer og patientens generelle tilstand \citep{motionsraad2007,kessing2016}.

\subsubsection{Depression samt følelsesmæssig trivsel}
I et studie af \textbf{henvisning til Galper kilde}, undersøges sammenhængen mellem fysisk inaktivitet og depression samt følelsesmæssig trivsel. Forsøgspersonerne hertil blev delt op i en grupper af inaktive, utilstrækkeligt aktive, tilstrækkeligt aktive og meget aktive og disse grupper blev så vurderet, om de havde depressive symptomer, og om de trivedes følelsesmæssigt. 
Til dette benyttedes en skala, The General Well-Being Schedule (GWB), som dermed forsøger at kvantificere forsøgspersonernes følelsesmæssige trivsel \citep{galper2006}.

\textbf{Indsæt billeder!} 

Resultater herfra viser, at fysisk inaktive, især kvinder, har en højere tendens til depressive symptomer, end andre, der er mere fysisk aktive. På samme måde fremgik det af studiet, at fysisk inaktive forsøgspersoner  ikke trives \textit{så godt} følelsesmæssigt, som de, der var mere aktive. Der er derved en sammenhæng mellem fysisk inaktivitet og psykiske følger, som eksempelvis depression og forværret følelsesmæssig trivsel \citep{galper2006}. Yderligere er der evidens for, at fysisk inaktivitet forværrer allerede eksisterende depressionstilstande samt dårlig følelsesmæssig trivsel \citep{motionsraad2007}.
