\section{Sygdomsafgrænsning}
Som nævnt i \autoref{sec:effekterafaktivitet}, er fysisk inaktivitet en negativ faktor i forbindelse med flere sygdomme - både fysiske og psykiske. Det er yderligere påvist, at mange sygdomsramte personer har gavn af fysisk aktivitet som en behandling eller en metode til at forebygge sygdomsprogression \cite{motionsraad2007,pedersen2011}. Fysisk aktivitet har effekt ved mange typer sygdomme, som påvirker forskellige aldersgrupper, hvorfor fysisk aktivitet generelt kan siges at være gavnligt, hvilket er årsagen til der eksisterer anbefalinger for alle aldersgrupper om fysisk aktivitet \cite{pedersen2011}. Af denne grund vælges der at tage udgangspunkt i én sygdom og fysisk aktivitets påvirkning på netop denne lidelse som fokusområde i dette projekt.

Hypertension udgør en risikofaktor for følger som apopleksi, myokardieinfarkt, hjerteinsufcciens samt pludselig død, og ifølge nuværende definitioner af hypertension har omkring $20~\%$ af befolkningen denne sygdom \cite{pedersen2011}. Fysisk inaktivitet øger risikoen for hypertension, og motion har en synlig blodtrykssænkende effekt \cite{olsen2015}. Af den grund vælges hypertension som udgangspunktet for projektet og problemanalysen. 