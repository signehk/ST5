\section{Sygdomsafgrænsning}
Som nævnt i \autoref{sec:effekterafaktivitet}, er fysisk inaktivitet forbundet med fysiologiske og psykiske konsekvenser, hvortil det er påvist, at mange sygdomsramte personer har gavn af fysisk aktivitet som en behandling eller en metode til at forebygge sygdomsprogression \cite{motionsraad2007,pedersen2011}. Fysisk aktivitet har effekt ved mange typer sygdomme, som påvirker forskellige aldersgrupper, hvorfor fysisk aktivitet generelt kan siges at være gavnligt \cite{pedersen2011}. Der vælges at tage udgangspunkt i én sygdom og fysisk aktivitets påvirkning på denne lidelse som fokusområde i dette projekt, da dette giver mulighed for at fokusere på en konkret patientgruppe og behandlingen af denne i analysen af de fire MTV-aspekter.

Hypertension udgør en risikofaktor for følger som apopleksi, myokardieinfarkt, hjerteinsufficiens samt pludselig død, og ifølge nuværende definitioner af hypertension har omkring $20~\%$ af befolkningen denne sygdom, hvorfor det kan betegnes som værende en folkesygdom \cite{pedersen2011}. Fysisk inaktivitet øger risikoen for hypertension, og motion har en synlig blodtrykssænkende effekt \cite{olsen2015}. Af den grund vælges hypertension som udgangspunktet for projektet og problemanalysen. 