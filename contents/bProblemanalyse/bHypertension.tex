\section{Hypertension}

Ud af de 20 \% voksne danskere med hypertension, er omkring 30 \% af disse ikke diagnosticeret, idet der ofte ikke er tydelige symptomer på lidelsen. \cite{kronborg2008} 
Der er en række sundhedsmæssige risici forbundet med hypertension. Blandt andet er længerevarende hypertension ofte årsag til kronisk nyresvigt og hjerte-kar-sygdomme. Hypertension medfører et øget pres på kroppens blodkar, hvilket forøger risikoen for udvikling af arteriesklerose, aneurismer, hjerteanfald og apopleksi [2]. Det kan være svært at estimere de nøjagtige tal for dødeligheden som følge af hypertension, idet folk med hypertension ofte dør af følgevirkninger heraf og årsagen til dødsfaldet kan være uklar. Ifølge Statens Institut for Folkesundhed er omkring 4 \% af alle dødsfald i Danmark relateret til hypertension. \cite{juel2006} \textbf{(ikke heeeelt sikker)}
 
På trods af de sundhedsmæssige risici ved hypertension får 2/3 af de diagnosticerede patienter ikke tilstrækkelig behandling, således at de kan komme ned under det anbefalede blodtryk. \cite{paulsen2012}
Blodtryk er karakteriseret ved et systolisk og et diastolisk blodtryk, som henholdsvis er trykket i arterierne, når hjertet trækker sig sammen og mellem to hjerteslag. Blodtryk skrives som ”systole/diastole” og måles i enheden millimeter kviksølv (mmHg). Det anbefales, at blodtrykket ligger under 140/90 mmHg, hvor blodtryk over denne grænse betegnes hypertension. Ligger blodtrykket mellem 120/80 og 139/89 mmHg kaldes dette prehypertension og der bør gøres opmærksom på dette for at undgå hypertension. [2]
I de fleste tilfælde er årsagen til hypertension ukendt, men der er patientgrupper, der har særlig høj risiko for at udvikle hypertension. En lidelse, der ofte forbindes med hypertension, er diabetes. De to lidelser er begge resultatet af metabolisk syndrom, som er forstyrrelser i kroppens metabolisme og er ofte grundet overvægt. \cite{cheung2012} \textbf{(Skal der skrives mere om det?)}
Behandling af hypertension kan ske farmakologisk eller non-farmakologisk. Alle patienter med hypertension bør i non-farmakologisk behandlingen, som består af en række anbefalinger, der bør følges, herunder er eksempelvis motion og kostændringer. Ved farmakologisk behandling tages der højde for graden af hypertension samt hvorvidt der er udviklet følgesygdomme. \cite{lodberg2016}
