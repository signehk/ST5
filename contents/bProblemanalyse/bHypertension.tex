\section{Hypertension}

Af de 20 \% voksne danskere med hypertension, er omkring 30 \% ikke diagnosticeret. Dette skyldes, at der ofte ikke er tydelige symptomer på lidelsen \cite{kronborg2008}. Symptomer, der kan forekomme ved hypertension, er træthed, hovedpine, næseblod, hjertebanken og åndenød ved anstrengelse. Idet hypertension i de fleste tilfælde ikke medfører symptomer, opdages lidelsen derfor ofte ved et tilfælde \cite{olsen2015}.
% olsen2015: https://www.sundhed.dk/borger/patienthaandbogen/hjerte-og-blodkar/sygdomme/hoejt-blodtryk-hypertension/hypertension-symptomer/ 
% Ikke optimal kilde, men andre kilder siger det sammen, både danske og engelske. 
Der er en række sundhedsmæssige risici forbundet med hypertension, idet sygdommen medfører et øget pres på kroppens blodkar, hvilket forøger risikoen for udvikling af arteriesklerose, aneurismer, hjerteanfald og apopleksi. Længerevarende hypertension er af denne grund ofte årsag til kronisk nyresvigt og hjerte-kar-sygdomme \cite{martini2015}. Det kan være svært at estimere de nøjagtige tal for dødeligheden som følge af hypertension, idet patienterne ofte dør af følgevirkninger heraf, og årsagen til dødsfaldet kan være uklar. Ifølge Statens Institut for Folkesundhed er omkring 4 \% af alle dødsfald i Danmark relateret til hypertension \cite{juel2006}.
 
På trods af de sundhedsmæssige risici ved hypertension får 2/3 af de diagnosticerede patienter ikke tilstrækkelig behandling, således at de kan opnå det anbefalede blodtryk \cite{paulsen2012}.
Blodtryk er karakteriseret ved et systolisk og et diastolisk blodtryk, som henholdsvis er trykket i arterierne, når hjertet trækker sig sammen under systole, og trykket mellem to hjerteslag under diastole. Blodtryk skrives som ”systole/diastole” og måles i enheden millimeter kviksølv (mmHg). Det anbefales, at blodtrykket er under 140/90 mmHg, hvor et blodtryk over denne grænse betegnes hypertension. Er blodtrykket mellem 120/80 og 139/89 mmHg kaldes dette prehypertension, og der bør gøres opmærksom på dette for at undgå hypertension \cite{martini2015}.
% Martinibogen skal i kildelisten.

I de fleste tilfælde er årsagen til hypertension ukendt, men der er patientgrupper, der har særlig høj risiko for at udvikle hypertension. En lidelse, der ofte forbindes med hypertension, er diabetes. De to lidelser er begge resultatet af metabolisk syndrom, som er forstyrrelser i kroppens metabolisme og forekommer ofte grundet overvægt \cite{cheung2012}.

Behandling af hypertension kan ske farmakologisk eller non-farmakologisk. Ved farmakologisk behandling tages der højde for graden af hypertension, samt hvorvidt der er udviklet følgesygdomme. Alle patienter med hypertension bør behandles non-farmakologisk, hvilket betyder de får en række anbefalinger fra lægen, der bør følges. Herunder blandt andet ændring af motions- og kostvaner. Hypertensive patienter bør jævnligt ved konsultationer få kontrolleret blodtrykket, hvor lægen eller sygeplejersken desuden kan følge op på patientens vægt, kost og aktivitetsniveau. Normalt vil hypertension kunne behandles i almen praksis, men i tilfælde af behandlingsresistent hypertension, hvor blandt andet motion, kostændringer og formindsket alkoholindtag ikke kan udrede sygdommen, vil patienterne opleve at blive videresendt til en hypertensionsklinik \cite{lodberg2016, bech2015}.