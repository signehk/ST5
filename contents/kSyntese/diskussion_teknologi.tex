\section{Teknologi} \label{sec:dis_teknologi}
Valget af Fitbit Flex er et simpelt alternativt i forhold til andre aktivitetsarmbånd på markedet. 
Grundet armbåndet kun anvender et accelerometer til registrering af aktivitet, er den kun velegnet til gang eller løb, hvortil mange andre aktivitetstyper ikke kan registreres. 
Nødvendigheden for at kunne tracke andre aktivitetstyper, kan dog vurderes som værende af mindre betydning, da målgruppen er hypertensive patienter, hvortil tilfælde af hypertension stiger med alderen. 
Dermed er det primært ældre der vil være målgruppe, hvor gang og løb er anbefalet for at mindske eventuelle skader, samt at dette i forvejen er den mest udøvet aktivitetstype blandt ældre. 

Til trods for at Fitbit Flex armbåndet er i stand til at registrere gang og løb, er den ikke velegnet til formidling af information, da den kun kan kommunikerer via 5 LED’er og vibration. 
Dette gør at patienten skal have adgang til enten smartphone eller pc, for at se konkret information omkring aktivitet eller batteritilstand. 
Der findes andre armbånd fra Fitbit, samt andre producenter, der er kommet med et mere omfattende display, således det er muligt for patienten at få de førnævnte informationer via armbåndet. 
Dertil ville armbåndet blive en mere selvstændig enhed og kun være afhængig af andet elektronik ved synkronisering af data. 
Anvendelsen af én enhed til sammenhold af informationer kan dog påvirke formålet med simpliciteten af armbåndet til gavn for målgruppen. 

På smartphone-enheden bliver data formidlet på engelsk, dette kan yderligere ses som en begrænsning for den kun dansk- eller andet sprogtalende population blandt målgruppen. 
Hertil kan en kontrolgruppe være relevant til undersøgelse af behovet for tilpasning af sproget, eller om nuværende formidlingen er intuitiv. 

Ved jævnlig synkronisering, kan der bevares et detaljeret overblik over registeret aktivitet. 
Hertil vides dog ikke om dette er et behov for at lægen at kan få korrekt indblik i patientens aktivitetsniveau. 
Den detaljeret data kan dog være en fordel ved vejledning omkring aktivitet, såfremt patienten opfylder det anbefalede aktivitetsniveau. 
Pålideligheden af Fitbit Flex er også relevant, da der ideelt ønskes et armbånd der måler den ydet aktivitet 100 \% korrekt, dog er der dette ikke muligt. 
Hertil tillades individuel tilpasning af armbåndet, således en bedre repræsentation af aktivitetsniveau opnås. 
Et af argumenterne for valget af Fitbit Flex var at den præcision ved måling af gang, og at den ellers havde en tendens til at underestimere antal skridt taget. 
Dette er anset som en fordel, da dette er med til at sikre at patienten opfylder aktiviteten i forhold til armbåndet registrerer. 


% Fitbit Flex 2 
