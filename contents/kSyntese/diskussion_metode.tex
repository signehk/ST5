\section{Metode}
Litteraturen har været valgt med udgangspunkt i inklusions- og eksklusionskriterier. Disse kriterier har varieret efter MTV-analysernes vidensbehov og den ønskede litteraturs formål. Det har ved manglende litteratur, været nødvendigt at opstille flere inklusionskriterier, med henblik på at finde anvendelig og relevant litteratur. Dette har blandt andet gjort sig gældende, da der flere gange er benyttet litteratur om aktivitetsarmbånd generelt og ikke udelukkende Fitbit Flex. Kompromiset har dog haft betydning for analysernes antagelser og konklusioner, hvortil bestemte områder forbliver spekulative. Økonomianalysen forbliver eksempelvis spekulativ, da der ikke eksisterer litteratur specifikt om implementering af aktivitetsarmbånd i sundhedssektoren. 

Den inkluderede litteratur er blevet sammenholdt med evidensniveauet, der fremgår af \autoref{sec:met_evidens}, for at vurdere hver enkelt kildes troværdighed. Ranglisten for evidensniveau er ikke blevet anvendt med henblik på at give hver kilde en bestemt rang, men blot for at være opmærksom på kildens evidensniveau. Såfremt det ikke har været muligt at finde kilder af høj evidens, er lavtrangerede kilders resultater verificeret ved gennemlæsning af flere kilder for at styrke validiteten af resultaterne.  Dette gøres, da det antages, at overensstemmelse af multiple kilder tyder på, at resultaterne er valide. 

Nogle af analyserne kunne gøres mere specifikke ved dataindsamling i form af interviews eller spørgeskemaer. Dog er dette ikke set som værende nødvendigt, da det antages at den tilgængelige litteratur er tilstrækkelig. Spørgeskemaer til danske patienter med hypertension kunne eksempelvis udarbejdes med det formål at opnå viden om, hvorvidt et aktivitetsarmbånd, såsom Fitbit Flex, kan være en motivationsfaktor til at øge mængden af fysisk aktivitet eller ej. Yderligere vil det være relevant at finde ud af, om armbåndet vil have en bevidsthedsøgende effekt i forbindelse med fysisk aktivitet som behandling af hypertension. 

For at få indblik i, hvordan Fitbit-applikationen formidler data om patientens aktivitetsniveau, er der foretaget observationer af denne applikation. Dette er udelukkende gjort med udgangspunkt i applikationen, hvorfor det ikke  har været muligt at se, hvordan data synkroniseres fra Fitbit Flex. En mere omstændig observation kunne være foretaget ved inklusion af Fitbit Flex-armbåndet, hvilket blandt andet kunne bidrage til en beskrivelse af armbåndets synkroniseringsprocess, brugervenligheden, samt hvor hurtigt det reelt drænes for strøm ved dagligt brug.  