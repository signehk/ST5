\section{Metode}
Litteraturen har været valgt ud fra inklusions- og eksklusionskriterier. Disse kriterier har varieret efter formål og vidensbehov for de forskellige analyser. Det har ved manglende litteratur, været nødvendigt at opstille flere inklusionskriterier, med henblik på at finde anvendelig og relevant litteratur. Dette har dog haft betydning for antagelser og konklusioner der har kunne tages under de forskellige analyser, hvortil bestemte områder kun forbliver spekulative. 

Den inkluderede litteratur er blevet sammenholdt med det opstillede evidensniveau, for at vurdere kildens troværdighed. Ranglisten er ikke blevet anvendt med henblik på at give hver enkelt kilde en bestemt rang, men blot på at give et indtryk af kildens evidensniveau. Såfremt det ikke har været muligt at finde kilder af høj evidens, er der i stedet blevet fundet flere lavt rangeret kilder med samme udtalelser og konklusioner, for at styrke validiteten af kilderne. 

Nogle af analyserne kunne gøres mere omfattende ved udarbejdelse af interviews eller spørgeskemaer, dog har dette ikke været anset nødvendigt grundet tilgængeligt litteratur har været tilstrækkeligt.

Observationer foretaget af Fitbit Flex har kun været relateret til Fitbit applikationen. Hertil har det været muligt at få indblik i, hvordan data synkroniseres og formidles til brugeren, dog kunne en mere omstændig observation have været foretaget ved indkludering af selve Fitbit Flex armbåndet. Dette ville kunne have bidraget til en beskrivelse af brugervenligheden, samt hvor hurtigt armbåndet reelt drænes for strøm.  
