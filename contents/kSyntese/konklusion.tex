\chapter{Konklusion}

Som konklusion på MTV'en besvares projektets problemformulering:

~
\textit{Hvilke påvirkninger vil implementeringen af Fitbit Flex i den almene praksis til registrering og objektivisering af fysisk aktivitet have hos hypertensive patienter i sundhedssektoren?}
~

\noindent I MTV-analysen er det fundet, at brugen af Fitbit Flex i den primære sektor, vil give de praktiserende læger et mere objektivt grundlag for vurdering af hypertensive patienters aktivitetsniveau. Sammenlignet med den nuværende metode, hvor der primært anvendes spørgeskemaer, vil implementeringen af Fitbit Flex resultere i, at eventuel bias ved vurderingen af aktivitetsmønstret forsvinder som følge af, at der bliver sat præcise tal på patienternes aktivitet.

På baggrund af de undersøgte studier kan det påvises, at anvendelsen af aktivitetsarmbånd resulterer i en mere aktiv hverdag, som følge af den motiverende faktor i forbindelse med en forøget indsigt i eget aktivitetsmønster. Ud fra dette kan det konkluderes, at implementeringen af Fitbit Flex vil have en positiv indvirkning på hypertensive patienters aktivitetsniveau, hvilket kan forbedre patienternes tilstand.

Ved indførelse af Fitbit Flex, vil patientforløbet og antallet af videresendelser fra primær til sekundær sektor kunne påvirkes. Dette sker som følge af, at patienterne opnår en sundere livsstil ved anvendelse af armbåndet, hvorfor færre vil opleve en forværring af den hypertensive tilstand. Derved er det en mulighed at holde dem i den primære sektor, hvor der anvendes livsstilsændringer og blodtryksmedicin uden indlæggelser til behandling af sygdommen. Implementeringen vil påvirke den primære sektor i form af efteruddannelse til de indblandede parter, der skal udlevere armbåndet og instruere patienter i anvendelsen af trackeren.

Trods den motiverende faktor, som giver mulighed for et forhøjet aktivitetsniveau, vil Fitbit Flex ikke være et alternativ til anden behandling. Armbåndet vil i stedet fungere som et hjælpemiddel til både læger og patienter, hvor det giver bedre vurderingsgrundlag og potentiale til sundere livsstil, som vil kunne betyde, at færre patienter viderestilles ved forværring af hypertension.