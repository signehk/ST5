\section{Patient} \label{sec:dis_patient}
På trods af, at \citeauthor{mercer2016} beskriver brug af aktivitetstrackers i en gruppe af kronisk syge, der er over 50 år, kommer denne artikel ikke med en endelig konklusion på anvendeligheden af disse. Dette skyldes, at patienterne i artiklen og denne rapports målgruppe, har blandede forhold til modtagelse og brug af teknologien. Størstedelen har imidlertid haft et godt udbytte og følt sig motiveret af teknologien, mens andre har følt sig meget udfordrede i brugen af en ny teknologi, da de ikke er vant til dette i deres dagligdag. Det er herved vigtigt at overveje, hvordan ikke-teknologivante patienter bedst muligt kan benytte teknologien; eksempelvis ved hjælp fra familie, venner eller plejepersonale. Dette skal undersøges yderligere før en eventuel implementering af Fitbit Flex. 

Det er værd at reflektere over, om Fitbit Flex har en høj motivationsfaktor, eftersom den visuelle feedback i form af armbåndets LED'er er begrænset i forhold til andre aktivitetsarmbånd. 
Teknologivante patienter vil muligvis have et større udbytte af aktivitetsmonitoreringen, hvis armbåndet gjorde brug af GPS, da dette giver mere præcise målinger af tilbagelagt afstand samt medfører, at aktivitetsarmbåndet kan benyttes til monitorering af andre aktivitetsformer end gang og løb. Ved overvejelse af en implementering af aktivitetsarmbånd med GPS, skal de tilhørende etiske aspekter i form af en følelse af overvågning undersøges.
Nogle patienter vil kunne have gavn af flere informationer direkte fra armbåndet, heriblandt antal aktive timer og tilbagelagt afstand, som en yderligere motivation - dette vil kunne lade sig gøre igennem de enheder, som Fitbit Flex synkroniserer med, men informationen kan ikke tilgås fra armbåndet i sig selv. Det skal hertil vurderes, om enkelheden af Fitbit Flex er en fordel for nogle patienter, da dette gør aktivitetsarmbåndet simpelt at interagere med for de, der muligvis ikke er vant til brug af ny teknologi. 

