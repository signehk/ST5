\chapter{Perspektivering}
Rapporten fokuserer på implementeringen af Fitbit Flex som en del af behandlingen af hypertension i almen praksis. Fitbit Flex blev valgt i \autoref{afgraensning_tek}, blandt andet på grund af aktivitetsarmbåndets funktioner og reproducerbarhed. I 2016 er Fitbit Flex 2 udkommet, og det vil være en mulighed at anvende dette i stedet for, grundet yderligere funktioner, såsom at tracke svømning, der kan være relevant for hypertensive patienter. Flere undersøgelser skal foretages for at bekræfte, at blandt andet aktivitetsarmbåndets nøjagtighed og reproducerbarhed ikke er forringet i forhold til den forrige model. Andre aktivitetsarmbånd kan også anvendes, hvis det vælges at prioritere andre egenskaber og funktioner. Nogle aktivitetstrackere har indbygget GPS, så det er muligt for brugeren at tracke andre aktivitetsformer ud over gang og løb. Der er desuden andre aktivitetsarmbånd,  som anses for at være mere brugervenlige, så hvis dette er en afgørende egenskab, fremfor eksempelvis reproducerbarheden, kan disse vælges.

Hvis implementeringen af Fitbit Flex i almen praksis skal tages op til overvejelse, bør yderligere undersøgelser foretages forinden, for at give et bedre evidensbaseret grundlag for beslutningen. Hertil vil en testperiode være relevant at indføre på få klinikker for at undersøge det reelle udbytte.
Yderligere er der ikke fundet studier, der undersøger sammenhængen mellem brugen af Fitbit Flex og den egentlige effekt i forhold til hypertension. For at kunne konkludere noget konkret om effekten, som brugen af Fitbit Flex har, bør der i praksis derfor laves yderligere undersøgelser. Herved vil der også kunne laves en mere fyldestgørende økonomisk analyse, da en teknologisk effektstørrelse har stor betydning for, om en ny teknologi kan svare sig at implementere.

Anvendelsen af Fitbit Flex til monitorering af aktivitetsniveau i den almene praksis kan desuden overvejes at indføres ved andre kroniske sygdomme, foruden hypertension, hvor et øget aktivitetsniveau vil have positiv effekt på sygdomsforløbet.
