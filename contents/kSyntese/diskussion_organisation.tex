\section{Organisation} \label{sec:dis_organisation}

Fitbit Flex vil antageligvis ikke være tilbøjelig til at skabe ændringer i patientens hverdag, eftersom teknologien ikke er nær så krævende og forstyrrende at bære, som andre teknologier, der kan lånes med hjem fra lægen; eksempelvis EKG-apparat med elektroder eller døgnblodtryksmåler, som aktiverer blodtryksmanchetten med et fast interval. Udlån af Fitbit Flex vil derfor kræve, at patienten introduceres til teknologien, hvorefter forløbet vil være tilsvarende udleveringen af andet monitoreringsudstyr i en periode, eksempelvis over en måned.

En anden mulighed for implementering vil være, at patienterne selv anskaffer armbåndet, hvorved aktivitetsmonitoreringen ikke begrænses til perioden, hvor armbåndet er udlånt af sundhedssektoren. Dette vil envidere sikre, at motiveringsfaktoren beskrevet i \autoref{sec:dis_patient} opretholdes længere end perioden, hvor armbåndet er udleveret. Af denne grund vil det være relevant at undersøge muligheden for at skabe en tilskudsordning, således patienterne har mulighed for at købe et armbånd til nedsat pris efter endt monitorering med det udlånte armbånd.

I tilfælde, hvor patienten oplever problemer med teknologien, eller har spørgsmål vedrørende funktionen, bør hverken læger eller sygeplejersker agere support-personale, som følge af blandt andet høje udgifter og et til tider presset arbejdsskema. Her er det muligt at kontakte Fitbit direkte på deres hjemmeside, eller det sted som udbyder udstyret, og som alternativ kan lægesekretærer introduceres til armbåndet, således de kan hjælpe med eventuelle problemer.

Fitbit Flex kommer ikke til at fungere som et alternativ til den nuværende behandling, men som et supplement til antihypertensiva. Som det ses på \autoref{fig:behandlingsvejl} vedrørende behandlingsforløb i relation til forskellige grader af hypertension ses, at alle hypertensive patienter, uanset graden, har gavn af livsstilsændringer. Af denne grund vil armbåndet være anvendeligt som både supplement og potentiel erstatning for andre behandlingsmetoder afhængigt af graden af hypertension. Såfremt det skal erstatte blodtryksmedicin ved mild grad af hypertension, kræver det en undersøgelse af effekten på aktivitetsniveauet og blodtrykket hos patienter, som får udleveret armbåndet.

Armbåndet kan, grundet motivationsfaktoren beskrevet i \autoref{sec:dis_patient} øge aktivitetsniveauet, og derfor potentielt have en positiv indvirkning på blodtrykket hos patienterne, hvorfor det kan reducerende antallet af videresendte patienter fra primær til sekundær sektor. Den motiverende faktor er påvist, når aktivitetsarmbåndet benyttes, hvorfor det er usikkert, om motivationen vil vare ved efter endt monitoreringsperiode. Potentielt vil dette have en positiv effekt på den sekundære sektor ved at lette arbejdsbyrden og spare penge i forbindelse med blandt andet indlæggelser og undersøgelser, hvis den motiverende faktor er vedvarende.

%For at sikre effekten af aktivitetsarmbånd som monitorering og motivatonsfaktor til hypertensive patienter, vil en testperiode kunne være relevant, at indføre på få klinikker til at starte med og derefter udvide til flere, hvis disse viser positive resultater.