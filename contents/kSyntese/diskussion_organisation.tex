\section{Organisation} \label{sec:dis_organisation}

Ved implementering af Fitbit Flex til monitorering af patienternes aktivitetsniveau, kan der drages paralleller med eksempelvis døgnblodtryksmåling, EKG-måling i hjemmet eller søvnmonitorering, hvor patienten ligeledes får udleveret teknologisk udstyr til monitorering i forbindelse med sygdom. Derfor vil udlevering af udstyr til patienten ikke være en ny udfordring for sundhedspersonalet, hvorfor udlevering af Fitbit Flex blot kræver en introduktion til teknologien, hvorefter forløbet vil være tilsvarende udleveringen af andet monitoreringsudstyr. 

Samtidig vil Fitbit Flex være mindre tilbøjeligt til at skabe ændringer i patientens hverdag, eftersom teknologien ikke er nær så krævende og forstyrrende at bære, som eksempelvis EKG-apparatet med elektroder, eller døgnblodtryksmåleren, som aktiverer blodtryksmanchetten med et fast interval. Andre muligheder for implementering vil være, at patienterne selv anskaffer armbåndet, hvorved aktivitetsmonitoreringen ikke begrænses til perioden, hvor armbåndet er udlånt af sundhedssektoren. Dette vil envidere sikre, at motiveringsfaktoren beskrevet i \autoref{sec:dis_patient} opretholdes længere end perioden, hvor armbåndet er udleveret. Af den grund vil det være relevant at skabe en tilbudsordning, således patienterne har mulighed for at købe et armbånd med rabat efter endt monitorering med det udlånte armbånd.

I tilfælde hvor patienten oplever problemer med teknologien, eller har spørgsmål vedrørende funktionen, bør hverken læger eller sygeplejersker agere support-personale, som følge af blandt andet høje udgifter og et til tider presset arbejdsskema. Her er det muligt at kontakte Fitbit på deres engelske hjemmeside, og som alternativ kan lægesekretærer introduceres til armbåndet, således de kan hjælpe med eventuelle problemer.

Fitbit Flex kommer ikke til at fungere som et alternativ til den nuværende behandling, men blot som et supplement og potentiel erstatning for blodtryksmedicin ved let hypertension. Som det ses på \autoref{fig:behandlingsvejl} vedrørende behandlingsforløb i relation til forskellige grader af hypertension, ses det at alle hypertensive patienter, uanset graden, har gavn af livsstilsændringer. Derfor vil armbåndet være anvendeligt i forbindelse med alle grader af hypertension, hvor det vil kunne fungere som både supplement og potentiel erstatning for andre behandlingsmetoder. Såfremt det skal erstatte blodtryksmedicin ved mild grad af hypertension, kræver det en undersøgelse af effekten på aktivitetsniveauet og blodtrykket hos patienter, som får udleveret armbåndet.

Armbåndet kan, grundet motivationsfaktoren beskrevet i \autoref{sec:dis_patient}, potentielt have en positiv indvirkning på blodtrykket hos patienterne, hvorfor det kan have betydning for videresendelse af patienter fra primær til sekundær sektor. Dette vil have en positiv effekt på den sekundære sektor, ved at lette arbejdsbyrden og spare penge i forbindelse med blandt andet indlæggelser og undersøgelser.

