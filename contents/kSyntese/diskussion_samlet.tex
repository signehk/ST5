\section{Samlet diskussion}

For at kunne implementere Fitbit Flex til udredning og behandling af hypertensive patienter skal der gøres en række overvejelser. Disse omhandler blandt andet tidshorisonten for motivation ved brug af armbåndet, implementeringsmetoden og målgruppens accept af den nye teknologi. 

Ved implementeringen skal det fremhæves, at armbåndet ikke fungere som et alternativ til den nuværende behandling, men blot som en motivationsfaktor til livsstilsændringer. Jævnfør \autoref{sec:beh_hypertension}, vil livsstilsændringer gavne patienterne uanset hvilken grad af hypertension, patienten lider af, hvorfor det er relevant at motivere dem til en mere aktiv hverdag, samtidig med at opnå større mulighed for monitorering af livsstilsændringer.

I forbindelse med livsstilsændringerne, er tidshorisonten for den motiverende faktor, som beskrevet i \autoref{sec:dis_patient}, vigtigt at tage med i overvejelserne. Lægen vil grundet denne faktor ikke nødvendigvis få et præcist billede af patientens almindelige fysiske aktivitet, som følge af patienten sandsynligvis dyrker mere motion i monitoreringsperioden. Af den grund vil en undersøgelse af motivationsfaktoren, samt hvorvidt denne aftager efter endt monitoreringsperiode, kunne danne grundlag for en evidensbaseret konklusion vedrørende langtidseffekten af implementeringen.

Ved en undersøgelse af videre aktivitetsmønster efter endt monitoreringsforløb, kan det i samme omgang overvejes at undersøge brugervenligheden og præcisionen for det nye Fitbit Flex 2. Dette kan med fordel gøres således at den nyeste tilgængelige teknologi implementeres, hvis undersøgelserne viser positive resultater ved Fitbit Flex 2.

Med henblik på at opretholde den motiverende faktor i form af den visuelle repræsentation af egen aktivitet, kan det samtidig overvejes at give patienterne muligheden for at købe et Fitbit Flex, eller andet aktivitetsarmbånd, billigere gennem sundhedsvæsenet. Dette vil formentlig kunne vedligeholde motivationen til at fortsætte livsstilsændringerne i form af både mere motion, men potentielt også anvende søvn- og fødevarertracking, for at opnå større indsigt i egen sundhed. Her skal det samtidig nævnes, at armbåndet bør implementeres på lige fod med døgnblodtryksmålere og EKG-målere til brug i hjemmet, hvor sundhedsvæsenet har ansvaret for indkøb og vedligehold, således patienten ikke oplever udgifter i forbindelse med monitorering af aktiviteten, førend de tilbydes at købe et armbånd efter endt monitoreringsperiode.