\section{Samlet diskussion}

For at kunne implementere Fitbit Flex til udredning og behandling af hypertensive patienter skal der gøres en række overvejelser. Disse omhandler blandt andet tidshorisonten for motivation ved brug af armbåndet, implementeringsmetoden og patientgruppens accept af den nye teknologi. 

Ved implementeringen skal det fremhæves, at armbåndet ikke fungerer som et fuldkomment alternativ til den nuværende behandling, men som en motivationsfaktor til livsstilsændringer, der kan udgøre den non-farmakologiske del af behandlingen. Jævnfør \autoref{sec:beh_hypertension}, vil livsstilsændringer gavne patienterne uanset, hvilken grad af hypertension patienten lider af, hvorfor det er relevant at motivere dem til en mere aktiv hverdag, samtidig med at opnå større mulighed for monitorering af livsstilsændringer.

I forbindelse med livsstilsændringerne er den mulige motiverende faktor, som beskrevet i \autoref{sec:dis_patient}, vigtig at tage med i overvejelserne, når implementeringsmetoden skal vælges. Ved hjemlån af armbåndet, vil lægen grundet denne motiverende faktor ikke nødvendigvis kunne få et præcist billede af patientens almindelige fysiske aktivitet, hvis patienten dyrker mere motion i monitoreringsperioden. Af den grund vil en undersøgelse af motivationsfaktoren, samt hvorvidt denne aftager efter endt monitoreringsperiode, kunne danne grundlag for en evidensbaseret konklusion vedrørende langtidseffekten af implementeringen.

Ved en undersøgelse af videre aktivitetsmønster efter endt monitoreringsforløb, kan det i samme omgang overvejes at undersøge brugervenligheden og præcisionen af det nye Fitbit Flex 2. Dette kan med fordel gøres således, at den nyeste tilgængelige teknologi implementeres, hvis undersøgelser viser positive resultater ved brug af Fitbit Flex 2. Vedrørende valget af Fitbit Flex, vil det være en fordel, at armbåndet ikke har et display, som viser brugerens progression. Dette kan gøre at brugeren er mere tilbøjelig til at synkronisere med smartphone eller PC, som følge af nysgerrighed vedrørende egen aktivitet, hvorved brugeren ikke glemmer at uploade data. Havde uret haft indbygget display, kunne brugeren nøjes med at kigge på uret, for at holde sig opdateret vedrørende aktivitetsniveauet, og dermed glemme at synkronisere med Fitbit-kontoen.

Med henblik på at opretholde den motiverende faktor i form af den visuelle repræsentation af egen aktivitet, kan det samtidig overvejes en anden implementeringsmetode; at give patienterne muligheden for at købe et Fitbit Flex-aktivitetsarmbånd med tilskud gennem sundhedsvæsenet. Dette vil formentlig kunne vedligeholde motivationen til at fortsætte livsstilsændringerne i form af øget aktivitetsniveau, men potentielt vil patienter også kunne anvende søvn- og fødevaretracking, for at opnå større indsigt i egen sundhed. Her skal det samtidigt nævnes, at armbåndet, hvis det vurderes omkostningseffektivt, bør implementeres på lige fod med hjemmeblodtryksmålere og EKG-målere til brug i hjemmet, hvor sundhedsvæsenet har ansvaret for indkøb og vedligehold. Således oplever patienten ikke udgifter i forbindelse med monitorering af aktiviteten, førend de overvejer at købe et armbånd efter endt monitoreringsperiode.