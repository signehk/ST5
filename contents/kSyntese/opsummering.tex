\section{Opsummering}
%Skrive en indledning der kort fortæller problemet og resultater (mini opsummering af hele rapporten.)
\subsection{Problemstillingen}
Det er i studier fundet, at $50~\%$ af danskere over $50$ år har hypertension, og at $20~\%$ af den voksne danske befolkning har diagnosen. Til behandling af hypertension anbefales patienter i alle behandlingsgrupper fysisk aktivitet som en del af behandlingen. Sundhedsstyrelsen vurderer, at $30-40~\%$ af befolkningen er fysisk inaktive, og dermed dyrker mindre end $2,5$ timers moderat fysisk aktivitet om ugen. Alment praktiserende læger har en udfordring i form af at monitorere patienter med hypertensions aktivitetsniveau, da patienter på nuværende tidspunkt adspørges om deres subjektive aktivitetsniveau. Det viser sig hertil, at nogle har en tendens til at overestimere sin mængde af fysisk aktivitet. Hertil besluttes at evaluere muligheden for implementering af en objektiv målemetode i form af et Fitbit Flex-aktivitetsarmbånd ved hjælp af en MTV-analyse. 
% Resultater
\subsection{Resultater}
Fitbit Flex består af et aktivitetsarmbånd, der kan registrere skridt ved hjælp af et accelerometer, og som kan visualisere, hvor langt patienten er fra sit daglige mål. Aktivitetsniveauet kan yderligere visualiseres via internettet, hvor den ydede aktivitet samles, så en alment praktiserende læge kan få overblik over patientens gennemsnitlige fysiske aktivitet. Dette kan gøres, hvis Fitbit Flex implementeres som en teknologi, lægen udlåner til patienter, der vurderes til at have gavn af en monitorering af aktivitetsniveau over en periode. Hvis en implementering således har effekt, så færre patienter er fysisk inaktive, og dermed benytter aktivitet som en non-farmakologisk behandling, kan dette resultere i færre henvisninger til den sekundære sektor, udskydelse af risikofaktorer og muligvis færre udgifter til medicinering. På denne måde vil der kunne følge besparelser med implementeringen, som skal opholdes med udgifterne til efteruddannelse og indkøb af armbånd. 

Armbåndet findes i flere studier brugervenligt, hvorfor det vurderes, at patienter uden betydelig teknologikendskab også vil have gavn af armbåndet. Såfremt nogle patienter ønsker at benytte Fitbit Flex' motiverende faktorer,for eksempel sammenkobling med en smartphone eller en computer, der begge giver mulighed for et mere detaljeret overblik over den ydede fysiske aktivitet, skal patienten have mere kendskab til brugen af teknologi, hvorfor alle patienter ikke nødvendigvis vil få det samme ud af brugen af armbåndet.