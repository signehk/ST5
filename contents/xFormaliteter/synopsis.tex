Denne rapport belyser idéen af at anvende et aktivitetsarmbånd til monitorering af hypertensive patienters aktivitetsniveau. 
Hypertension er en folkesygdom, og i Danmark er det vurderet, at 20 \% har sygdommen. Disse patienter anbefales fysisk aktivitet som en del af behandlingen, da dette resulterer i et reduceret blodtryk, hvorpå risikoen for følgesygdomme og medicinering ligeledes kan reduceres eller udskydes. 
Formålet har således været at undersøge et udvalgt aktivitetsarmbånd, Fitbit Flex, og vurdere, om dette hensigtsmæssigt kan anvendes i behandlings- og/eller udredningsforløb af hypertension. 
Den registrerede aktivitet vil give læger og andet relevant sundhedsfagligt personale indblik i patienters aktivitetsniveau, hvorved en korrekt rådgivning kan gives til patienten. 

Vurderingen af Fitbit Flex som registrering- og objektiviseringsmetode er fortaget med udgangspunkt i metoden for en medicinsk teknologi vurdering.  

\vspace{5cm}

\textbf{Hvad konkluderes der?}