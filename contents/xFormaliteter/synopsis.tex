Denne rapport omhandler anvendelsen af aktivitetsarmbånd til monitorering af hypertensive patienters aktivitetsniveau. 
Hypertension er en folkesygdom, og i Danmark er det vurderet, at 20 \% har sygdommen. Disse patienter anbefales fysisk aktivitet som en del af behandlingen, da dette resulterer i et reduceret blodtryk, hvorpå risikoen for følgesygdomme og medicinering ligeledes kan reduceres eller udskydes. 
De nuværende subjektive målemetoder er ofte præget af bias, da patienter fejlestimerer deres eget aktivitetsniveau. Formålet har således været at undersøge et udvalgt aktivitetsarmbånd, Fitbit Flex, og vurdere, om dette hensigtsmæssigt kan anvendes i behandling af hypertension for at opnå mere objektive målinger. 
Den registrerede aktivitet vil give læger og andet relevant sundhedsfagligt personale indblik i patienters aktivitetsniveau, hvorved en korrekt rådgivning og behandling kan gives til patienten. 

Vurderingen af Fitbit Flex som registrering- og objektiviseringsmetode er fortaget med udgangspunkt i metoden for en medicinsk teknologivurdering.  

På baggrund af flere studier er det vist, at aktivitetsarmbånd, ud over at kunne monitorere aktivitetsniveau, har potentiale til at motivere personer til at være mere fysisk aktive, hvilket vil kunne medføre en gavnlig effekt på hypertensive patienters tilstand. En objektiv monitorering vil samtidigt være muligt som følge af, at armbåndet gemmer de målte data, hvilket kan give grundlag for vejledning og anbefalinger om, hvordan patientens behandlingsforløb kan struktureres ved fremtidig behandling.

Implementeringen af aktivitetsarmbånd vil indledningsvist kræve ressourcer i den primære sundhedssektor, men såfremt monitorering med aktivitetsarmbåndet viser sig effektivt for behandlingen af hypertensive patienter, vil dette potentielt kunne medføre en forøget livskvalitet for nogle patienter. 