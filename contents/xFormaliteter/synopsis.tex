Denne rapport belyser ideen af anvende et aktivitetsarmbånd til monitorering af hypertensive patienter's aktivitetsniveau. 
Hypertension er en folkesygdom og i Danmark er det vurderet til at 20 \% har sygdommen. Disse patienter anbefales fysisk aktivitet, da dette resulterer i reduktion i blodtryk, hvorpå risikoen for følgesygdomme og medicinering ligeledes reduceres eller udskydes. 
Formålet har således været at undersøge et udvalgt aktivitetsarmbånd, Fitbit Flex, og vurdere om dette kan anvendes i behandlings- og/eller udredningsforløb af hypertension. 
Den registrerede aktivitet vil give læger og andet relevant sundhedsfagligt personale indblik i patienters aktivitetsniveau, hvoraf en korrekt rådgivning kan gives til patienten. 

Vurderingen af Fitbit Flex som registrering- og objektiviseringsmetode er fortaget med udgangspunkt i håndbogen for medicinsk teknologi vurdering.  

\vspace{5cm}

\textbf{Hvad konkluderes der?}