Denne rapport belyser idéen af at anvende et aktivitetsarmbånd til monitorering af hypertensive patienters aktivitetsniveau. 
Hypertension er en folkesygdom, og i Danmark er det vurderet, at 20 \% har sygdommen. Disse patienter anbefales fysisk aktivitet som en del af behandlingen, da dette resulterer i et reduceret blodtryk, hvorpå risikoen for følgesygdomme og medicinering ligeledes kan reduceres eller udskydes. 
Formålet har således været at undersøge et udvalgt aktivitetsarmbånd, Fitbit Flex, og vurdere, om dette hensigtsmæssigt kan anvendes i behandlings- og/eller udredningsforløb af hypertension. 
Den registrerede aktivitet vil give læger og andet relevant sundhedsfagligt personale indblik i patienters aktivitetsniveau, hvorved en korrekt rådgivning kan gives til patienten. 

Vurderingen af Fitbit Flex som registrering- og objektiviseringsmetode er fortaget med udgangspunkt i metoden for en medicinsk teknologi vurdering.  

På baggrund af flere studier er det vist at aktivitetsarmbånd har potentialet til at motivere personer til at være mere fysisk aktive, hvilket vil kunne medføre en gavnlig effekt på hypertensive patienters tilstand. Yderligere ved at data gemmes ved brug af en applikation tilhørende aktivitetsarmbåndet, muliggøres en objektiv monitorering af bæren af armbåndet, hvilket vil kunne give grundlag og viden for lægen, om vejledning og anbefalinger om hvordan patientens behandlingsforløb kan struktureres ved fremtidig behandling.

Implementeringen af aktivitetsarmbånd vil indledningsvist kræve resourser i den primære sundhedssektor, men såfremt aktivitetsarmbånd viser sig effektivt for behandlingen af hypertensive patienter, vil resultatet kunne afspejle sig i et overordnet lettere behandlingsforløb.