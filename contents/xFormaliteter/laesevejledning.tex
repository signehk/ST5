Skriv forord her.... 


\section*{Læsevejledning}
Denne rapport består af et initierende problem, en problemanalyse, en problemformulering, MTV-spørgsmål og -analyser samt en syntese af disse analyser, der gerne skal besvare problemformuleringen. 

Det initiernede problem og problemanalysen belyser og analyserer projektets problemstillinger og leder frem til en problemformulering igennem en problemafgrænsning. MTV-spørgsmålene og -analyserne beskæftiger sig med de fire MTV-elementer; patient, teknologi, organisation og økonomi. Syntesen dækker over en diskussion af MTV-analyserne, en konklusion på problemformuleringen samt en perspektivering til valgte teknologi i projektet. 


\subsection*{Kildeangivelse}
I denne rapport bliver kilder angivet ved Vancouver-metoden, hvor kilden henvises til som et nummer i kantede parenteser. Information omkring kilden findes i litteraturlisten.


%Teknologiafsnittet vil beskrive den valgte teknologi, og hvilke variationer af teknologien der eksistere i dag. En sammenligning af variationerne vil blive fortaget, med henblik på at fremhæve fordele og ulemper. Yderligere vil teknologien også blive sammenlignet med de nuværende løsningsmuligheder der anvendes i dag, for at se hvordan de adskiller sig fra hinanden.  

%Patientafsnittet i MTV’en undersøges den afgrænsede patientgruppe nærmere i forhold til teknologien. Der undersøges blandt andet om teknologien vil have en betydelig påvirkning på patienternes hverdag, og om den kan forbedre deres livskvalitet. Yderligere undersøges om eventuelle etiske problemstillinger forekommer ved anvendelse af teknologien.

%Den organisatoriske analyse vil hovedsageligt behandle ændringer i interaktionen mellem patienter og sundhedspersonale, samt det organisatoriske aspekt i forhold til samarbejdet mellem forskellige sundhedsinstitutioner: primær og sekundær sundhedssektor.

%Det økonomiske aspekt blive undersøgt, med udgangspunkt i at finde frem til omkostningerne relateret til de teknologiske løsninger, som er undersøgt i teknologianalysen. Dette omhandler eventuelle besparelser eller ekstraudgifter, der kan forekomme ved implementering af den nye teknologi.







%\section{Metode} \label{metode}

%Denne medicinske teknologivurdering (MTV) vil afvige fra opbygningen beskrevet i MTV-håndbogen, som følge af projektet samtidig indeholder elementer fra problembaseret læring (PBL). Som et resultat af blandingen, tages der udgangspunkt i en medicinsk problemstilling, som analyseres for at udarbejde en problemformulering. Analysen i forbindelse med PBL-tilgangen vil desuden indeholde MTV-elementer såsom etik, målgruppe og interessentanalyse.

%Efterfølgende vil problemformuleringen skabe grundlag for at arbejde videre med MTV-håndbogens elementer. Her vil de fire områder, teknologi, patient, organisation og økonomi, blive anvendt til at stille mere konkrete spørgsmål. Fremgangsmåden betyder at der vil være tale om en problem- og teknologiorienteret MTV, da der søges at finde en løsning på et medicinsk problem gennem en vurdering af, hvorvidt en ny teknologi vil afhjælpe de problemer, der er ved den nuværende løsningsmetode.

%Teknologiafsnittet vil indeholde en beskrivelse af egenskaberne for den nuværende teknologi, samt en undersøgelse af den alternative behandlingsmetode, der ligger til grunde for MTV'en. Efter den nuværende og den alternative teknologi er beskrevet, vil disse blive sammenholdt, med henblik på at finde fordele og ulemper ved de to løsningsforslag.

%I forbindelse med patientafsnittet i MTV-modellen afgrænses patientgruppen, med henblik på at gøre problemet konkret, hvorved målgruppen for teknologien kan undersøges nærmere. Der undersøges blandt andet hvorvidt teknologien vil have en betydelig påvirkning på patienternes hverdag og om der skal tages højde for etiske problemstillinger.

%Den organisatoriske analyse vil hovedsageligt behandle ændringer i interaktionen mellem patienter og sundhedspersonale, samt det organisatoriske aspekt i forhold til samarbejdet mellem forskellige sundhedsinstitutioner.

%Som et led i MTV'en vil det økonomiske aspekt blive undersøgt med udgangspunkt i, at finde frem til omkostningerne relateret til de teknologiske løsninger, som er undersøgt i teknologianalysen. Her undersøges desuden hvilke besparelser eller ekstraudgifter, der kan forekomme ved implementering af den nye teknologi.

%Analysen af de fire MTV-elementer vil dernæst blive anvendt i syntesen, der indeholder en diskussion med udgangspunkt i fordele og ulemper ved både den nuværende og den undersøgte teknologi. Herigennem vil PBL-metoden også komme til udtryk, i og med syntesen leder frem til en konklusion, som vil besvare den indledende problemformulering. 


\section*{Ordliste}
\begin{description}[leftmargin=!,labelwidth=\widthof{\bfseries The longeeeeeeest label}]
\item [Aktivitetstracker] Apparat der måler aktivitet, men ikke nødvendigvis i form af et armbånd.
\item [Applikation] Et program, der anvendes under et operativsystem, som bruges til at løse specifikke opgaver
\item [CES-D] Center for Epidemiologic Studies Depression Scale 
\item [Demens] Symptomer på svigtende hjernefunktion.
\item [GPS] Global Positioning System
\item [GWB] The General Well-Being Schedule
\item [Likert skala] Skala til måling af holdninger.
\item [MEMS] Micro Electro-Mechanical System
\item [MTV] Medicinsk Teknologi Vurdering
\item [Osteoporose] Knogleskørhed
\item [Parkinsons sygdom] Kronisk nervesygdom karakteriseret ved langsomme bevægelser, stivhed i musklerne og rysten.
\item [PLO] Praktiserende Lægers Organisation
\item [QALY] Quality Adjusted Life Year
\item [RLTN] Regionernes Lønnings- og TakstNævn
\end{description}
