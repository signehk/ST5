\chapter*{Forord}

Denne rapport er skrevet af gruppe 5404, sundhedsteknologistuderende på 5. semester, Aalborg Universitet, i projektperioden fra den 2. september til den 19. december 2016.

Rapporten er skrevet i relation til en Medicinsk TeknologiVurdering (MTV) med det overordnede tema 'Klinisk teknologi'. 

Projektet omhandler aktivitetsarmbånd som non-farmakologisk behandling af hypertensive patienter i almen praksis med formålet at objektivt monitorere patienternes fysisk aktivitet i hverdagen. Samtidigt anvendes armbåndet til at give alment praktiserende læger et mere objektivt billede af patienternes daglige aktivitetsniveau, sammenlignet med subjektive målemetoder, samt for at øge patientens bevidsthed om deres daglige aktivitetsniveau. 

Aktivitetsarmbåndet bliver analyseret i forhold til de fire aspekter i MTVen; teknologi, patient, organisation og økonomi, for at kunne vurdere, om teknologien vil være et egnet værktøj til behandling af hypertensive patienter.

Der rettes tak til vejledere Ole Hejlesen, Morten Sig Ager Jensen, samt Mads Nibe Stausholm for vejledning og konstruktiv kritik igennem projektperioden. 
