\chapter*{Forord}

Denne rapport er skrevet af gruppe 5404, sundhedsteknologistuderende på 5. semester, Aalborg Universitet i projektperioden fra 2. september til 19. december 2016.

Rapporten er skrevet i relation til en medicinsk teknologivurdering med det overordnede tema 'klinisk teknologi'. 

Projektet omhandler teknologien aktivitetsarmbånd som non farmakologisk behandling af hypertensive patienter i den almene praksis, med det formål at monitorere og motivere patienterne til en mere fysisk aktiv hverdag. Samtidig anvendes armbåndet til at give lægen i den almene praksis et mere objektivt billede af patienternes daglige aktivitetsniveau sammenlignet med subjektive målemetoder, samt for at øge patientems bevidsthed om deres daglige aktivitetsniveau. 

Teknologien aktivitetsarmbånd bliver analyseret i forhold til de fire aspekter, teknologi, patient, organisation og økonomi, for at kunne vurdere om teknologien vil være et egnet værktøj til behandling af hypertensive patienter.

Der rettes tak til vejler Ole Hejlesen, Morten Sig Ager Jensen, samt Mads Nibe Stausholm for god vejledning og konstruktiv kritik igennem projektperioden. \\\\\\\\\\

\begin{comment}
\begin{tabular}{lcl}
   \hspace{-1cm} \rule{7cm}{0.5pt} & \hspace{1cm} & \rule{7cm}{0.5pt} \\
   \hspace{-1cm}  Birgithe Klemann Rasmussen&  & Mads Kristensen
\end{tabular}
\vspace{1.5cm}


\begin{tabular}{lcl}
   \hspace{-1cm} \rule{7cm}{0.5pt} & \hspace{1cm} & \rule{7cm}{0.5pt} \\
    \hspace{-1cm} Signe Hejgaard Kristoffersen & & Simon Brunn
\end{tabular}
\vspace{1.5cm}
      

\begin{tabular}{lcl}
    \hspace{-1cm} \rule{7cm}{0.5pt} & \hspace{1cm} & \rule{7cm}{0.5pt} \\
  \hspace{-1cm} Suado Ali Haji Diriyi & & Toby Steven Waterstone
\end{tabular}
\vspace{1.5cm}
\end{comment}
  
\newpage