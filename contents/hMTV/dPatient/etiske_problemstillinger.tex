\section{Etiske problemstillinger}

Ved implementering af ny teknologi eller nye ideer i sundhedssektoren vil der ofte opstå etiske problemstillinger, som skal adresseres. Der vil derfor i dette afsnit blive forsøgt belyst, hvilke etiske problemstillinger Fitbit Flex, som aktivitetsmonitoreringsenhed, vil kunne skabe i sundhedssektoren.

Rapporten af \citeauthor{patienthome2015} beskriver, at hvis monitoreringen er hyppig, kan det føles som overvågning for nogle patienter. Det kan derfor blive en problemstilling, som skal tages op med patienten, hvorvidt vedkommende ønsker monitoreringen \citep{patienthome2015, SundhedsstyrelsenPatientersRetsstilling2016}.

Studierne af \citeauthor{Mittelstand2011} og \citeauthor{Nordgren2013} har undersøgt en række etiske afspekter ved brugen af enheder, der monitorerer patienten i privatlivet. Studierne understreger visse etiske aspekter vedrørende monitorering af patienten i privatlivet.

Af disse etiske aspekter vil Fitbit Flex primært kunne komme til at ramme områder, som omhandler personligt privatliv og persondata, synlighed, autonomi, pålidelighed, helbred, samt uafhængighed \citep{Mittelstand2011, Nordgren2013}.


\subsection{Privatliv og persondata}

Som tidligere nævnt, vil nogle patienter muligvis komme til at føle den kontinuerlige monitorering som overvågning. Fitbit Flex kan følge patienters aktivitetsniveau i form af skridt og aktive timer, herved vil eventuelle private detaljer ved aktiviteter ikke kunne spores, da Fitbit Flex ikke inkluderer GPS.
Data omhandlende fysisk aktivitet er ikke direkte personfølsomt, da det ikke indeholder CPR-nummer eller kan bruges til at identificere eller skade patienten, hvis brugernavnet til Fitbit-kontoen anonymiseres. På denne måde kan dette data ikke misbruges \citep{Mittelstand2011}.


\subsection{Synlighed}

Synlighed vedrører enhedens fysiske fremtræden og om, hvorvidt nogle patienter vil komme til at føle sig stigmatiseret som syg, hvorved teknologien vil være et symbol for sygdommen, som er synlig for andre \citep{Mittelstand2011}. Fitbit Flex bæres som og ligner et armbåndsur og anses derfor ikke som værende generende for patienten, da patienten på denne måde ikke skiller sig ud. Aktivitetsarmbånd bruges derudover også af dele af befolkningen, der ikke er syge, men som ønsker at monitorere deres aktivitetsniveau. 

\subsection{Autonomi}

Behandlingen, som patienten modtager, skal være forståelig, og patienten skal være indbefattet med, hvad behandlingen kommer til at betyde for vedkommendes liv \citep{Mittelstand2011}. Lægens introduktion af teknologien er derfor vigtig. Patienten vil skulle tilpasse sig en ny livsstil ved implementeringen af Fitbit Flex, da monitoreringen har til formål at sætte konkrete tal på patientens daglige aktivitet.

\subsection{Pålidelighed}

Pålidelighed spiller en rolle for, at den dokumenterede aktivitet, som Fitbit Flex registrerer, kan gengives tilstrækkeligt præcist \citep{Nordgren2013}.
Hvis Fitbit Flex ikke viser patienternes fysiske aktivitetsniveau, eksempelvis da visse former for aktivitet ikke kan registreres, som nævnt i \autoref{noejagtighed}, kan monitoreringen betyde, at lægen underestimerer mængden af fysisk aktivitet. Dette kan betyde, at patienten får en forkert vejledning eller behandling. 
 
\subsection{Helbred}

Monitoreringen har til hensigt at øge patientens aktivitetsniveau og herved gøre patientens helbredstilstand bedre - eller undgå at forværre den. Fitbit Flex har som aktivitetstracker til formål at dokumentere den fysiske aktivitet, patienterne udfører, samt motivere patienterne til at udføre mere aktivitet i hverdagen, ved for eksempel at sætte mål for patienten. Den kontinuerlige måling kan medføre, at patientens liv vil centrere sig omkring sundhed og sygdom. Dette vil kunne betyde, at patienten vil opfatte sig selv som en syg person, i stedet for som en person med en sygdom \citep{Nordgren2013}.

\subsection{Uafhængighed}

Ved udlevering af et Fitbit Flex-armbånd, er det i forventningen om, at patienten skal få en mere aktiv hverdag. I den forbindelse skal det overvejes, om patienten vil føle sig forpligtet eller tvunget til at dyrke mere motion, fremfor at føle sig motiveret og opfordret. Patienten skal ikke føle sig tvunget til at fuldføre denne forventning, da teknologien ikke har til formål at bestemme over patienternes aktiviteter, og hvad de foretager sig. \citep{Nordgren2013}

\subsection{Yderligere etiske overvejelser}
Et andet etisk aspekt ved patientmonitorering er, hvem og hvor mange der skal have ejerskab og ansvar over det data, som bliver indsamlet. Data skal prioriteres, så det kun er det data, man kan bruge til den reelle behandling, som bliver gemt. Der skal derfor tages stilling til, hvad der er relevant data \citep{patienthome2015}. Eksempelvis vil det relevante data fra Fitbit Flex være de data, som fortæller noget om, hvor aktiv patienten er i hverdagen, altså; aktive timer, skridt gået og forbrændte kalorier. Data som søvnovervågning eller andet, som aktivitetarmbåndet kan registere, vil være overflødigt for aktivitetsarmbåndets brug i forhold til behandling af hypertension, da dette ikke giver nogen direkte indikation vedrørende patientens aktivitetsniveau. Data, som sendes til lægen, skal derfor begrænses, så det tilpasses til det, som er nødvendigt for patientens forløb. 


%\textbf{Det kan godt være at nogle af tingene enten skal beskrives bedre, på en anden måde, jeg er også lidt i tvivl om strukturen på dette afsnit... om der er for meget "gentagende" med eller bare for meget med. Nogle af sætningerne skal nok måske højest sandsynligt også lige formuleres lidt bedre :) }











%Privatliv
%Personlig privatliv og person data er ... spørgsmplet er om PHM ses på som personlig data som andet personlig data, graden af følsomhed

%Synlighed, 
%Størrelse og vægt på den benyttede PHM enhed. Er det noget som generer patienten i hverdagen, fordi den er tung og klundtet eller bare en af delene? 

%Medicinering, 
%monitorering skal ikke bare anvendes for monitoreringens skyld.. fordi man kan gør man det, sådan skal det ikke være

%Social issolation, 
%Der er mindre brug for plejepersonale, dette leder til mindre kontakt med patienterne, det kan evt. laves en løsning i at patienterne kan kommunikere sammen gennem monitoreringsenheden... noget noget...

%Autonomi, 
%Selvbestemmelse, monitoreringsenheden kan lave om på patienternes daglige rutiner, ændre på patienternes selvbestemmelse... Patienternes handlinger.

%Princip balance, 


%Informeret samtykke og udvikling, 
%Fuld forståelse af teknologeiens påvirkning for patienterne kan ikke forudsiges direkte da det ikke er blevet gjordt endnu.

%Påvirkning på hjemmehjælpere.
%Hjemmehjælperne får færre opgaver at tage sig af, da teknologien tager over, hvilket også betyder at patienterne oplever mindre personalekontakt, hvilket er nødvendigt for nogen..