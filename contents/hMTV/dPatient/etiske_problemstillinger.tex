\section{Etiske problemstillinger}
%Indhold: I dette afsnit vil vi se på, om der er etiske problemstillinger ved, at lægen kan monitorere patientens aktivitetsniveau i hverdagen, og i så fald disse dilemmaer opstår heraf. Vi vil herunder drage paralleller til teknologier, hvor lignende problemstillinger opstår (fx hjemmeblodtryksmåler). 

Ved implementering af ny teknologi eller nye ideer i sundhedssektoren vil der ofte opstå etiske problemstillinger, som skal adresseres. Der vil derfor i dette afsnit blive forsøgt belyst, hvilke etiske problemstillinger, Fitbit Flex som aktivitetsmonitoreringsenhed, vil kunne skabe i sundhedssektoren.

Rapporten af \citeauthor{patienthome2015} beskriver, at hvis monitoreringen er hyppig, kan det føles som overvågning for nogle patienter. Det kan derfor blive en problemstilling, som skal tages op med patienten, hvorvidt vedkommende ønsker monitoreringen \citep{patienthome2015, SundhedsstyrelsenPatientersRetsstilling2016}.

Studierne af \citeauthor{Mittelstand2011} og \citeauthor{Nordgren2013} har undersøgt en række etiske afspekter ved brugen af enheder, der monitorerer patienten i privatlivet. Studierne understeger visse etiske aspekter vedrørende monitorering af patienten i privatlivet.% Disse aspekter er, personlig privatliv og persondata, synlighed, medicinering, social issolation, aunonomi, pricip balance, informeret samtykke og udvikling, samt påvirkning på hjemmehjælpere. (jeg ved ikke helt om alle skal nævnes eller kun dem som er relevante for os, ordet aspekter bruges desuden ret meget :S) 
Af disse etiske aspekter vil Fitbit Flex primært kunne komme til at ramme områder, som omhandler personligt privatliv og persondata, synlighed, autonomi, pålidelighed, brugervenlighed, helbred, samt uafhængighed \citep{Mittelstand2011, Nordgren2013}.


\subsection{Privatliv og persondata:}

Ved at patienten bliver monitoreret kontinuerligt i løbet af hele dagen, vil nogle brugere kunne komme til at se det som overvågnig som nævnt tidligere. Fitbit Flex kan følge patienters aktivitetsniveau i form af skridt og aktive timer, herved vil alle deres aktiviteter ikke kunne spores ned til mindste detalje da Fitbit Flex ikke inkluderer GPS, dog vil det stadig kunne give nogle spekulationer ved visse patienter muligvis....
Der kan stilles spørgsmålstegn til om data om fysisk aktivitet falder ind under kategorien over følsom persondata, om det kan misbruges på nogen måde. Dog vil det ikke være lige så følsomt som data der vedrører patienten direkte, så som CPR nummer eller andet som kan bruges til at identificere eller skade patienten \citep{Mittelstand2011}.

\noindent \\
\textbf{Synlighed:}
\noindent \\
Synlighed vedrører enhedens fysiske fremtræden, altså om enheden er stor, klodset og tung for patienten at gå med, så den  genere patienten i hverdagen. Fitbit Flex bæres som et andet armbåndsur og ses derfor ikke for generende for patienten, med mindre der er personlige ting som gør at de ikke kan gå med uret, for eksempel allergier eller andet. Yderligere kan det også overvejes at nogle patienter vil komme til at føle sig stigmatiseret, hvorved de kan gå i en tro om den overbevisning om at de er syge, teknologien kan være symbol for dette \citep{Mittelstand2011}. 

\noindent \\
\textbf{Autonomi:}
\noindent \\
Den behandling patienten modtager skal være forståelig og patienten skal være indbefattet med hvad behandlingen kommer til at betyde for vedkommendes liv. Introduktionen af teknologien for patienten er derfor vigtig. Hvis teknologien kan ændre på patientens daglige rutiner, vil patienten skulle tilpasse sig en ny livsstil, hvilket Fitbit Flex nemt kan komme til at gøre, da et af formålene med Fitbit Flex er at motivere patienterne til at være mere aktive. Denne omvendig kan være svær for nogen \citep{Mittelstand2011}. %... Selvbestemmelse...

\noindent \\
\textbf{Pålidelighed:}
\noindent \\
Pålidelighed spiller en rolle for at den dokumenterede aktivitet som Fitbit Flex registrerer kan gengive den aktivitet som patienten har udført præcist. Afhængig af behandlingen kan pålidelighed være vigtigere, for eksempel ved måling af blodglukoseniveau hos diabetikere. Hvis Fitbit Flex ikke viser patienternes fysiske aktivitet som den reelt er, eksempelvis ved at den har svært ved at registre visse former for aktivitet \autoref{xxx} kan behandlingen have en negativ effekt for forløbet... \citep{Nordgren2013}. Hvor godt Fitbit Flex repræcenterer fysisk aktivitet er beskrevet i afsnit \autoref{pålidelighedsafsnittet}.

\noindent \\
\textbf{Brugervenlighed:}
\noindent \\
Brugervenlighed er vigtig for at teknologiens anvendelse, både for patienten og behandleren. Teknologien skal helst være nem at anvende så der ikke opstår misforståelser mellem de parter som teknologien berører, samt for at mindske risikoen for at teknologien fylder for meget i patientens hverdag  \citep{Nordgren2013}. Brugervenlighed for Fitbit Flex beskrives i afsnit \autoref{brugervenlighed}.
 
\noindent \\
\textbf{Helbred:}
\noindent \\
Behandlingen som patienten modtager har til hensigt at gøre patientens helbredstilstand bedre, eller undgå at den bliver værre og ikke forværre den. Fitbit Flex har som aktivitetstracker, til formål at dokumentere den fysiske aktivitet patienterne udfører, samt motivere patienterne til at udføre mere aktivitet i hverdagen, ved at for eksempel at sætte mål for patienten \citep{Nordgren2013}. 

\noindent \\
\textbf{Uafhængighed:}
\noindent \\
Ved at patienterne modtager et Fitbit Flex, er det i den forventning om at de skal få en mere aktiv hverdag. Patienter er ikke tvunget til at fuldføre denne forventning, da teknologi ikke skal bestemme over patienternes aktiviteter og hvad de foretager sig. \citep{Nordgren2013}
\\

Et andet etisk aspekt ved patient monitorering er, hvem og hvor mange der skal have ejerskab og ansvar over den data som der bliver indsamlet, samt hvordan det undgås at drukne i datamængder. Data skal prioriteres så det kun er det data man kan bruge til den reele behandling som bliver "gemt", der skal derfor tages stilling til hvad der er vigtigt \citep{patienthome2015}. Eksempelvis vil data fra Fitbit Flex, som er relevant for lægen, de data som fortæller noget om hvor aktiv patienten er i hverdagen, altså data som aktive timer, skridt gået, samt forbrændte kalorier. Data som søvnovervågning eller andet aktivitetarmbåndet kan registere vil være overflødigt for formålet med aktivitetsarmbåndets brug i forhold til hypertension, da dette ikke giver nogen direkte indikation vedrørende patientens aktivitetsniveau. Data skal derfor begrænses så det tilpasses til det som er nødvendigt for patientens forløb. 

Generelt er problematikken vedrørende de etikske aspekter ved Fitbit Flex lille sammenlignet med andre teknologier som indgår til behandling af patienter. \textbf{Tanken med dette var lidt at lave en slagt opsummering af det der er skrevet, ved ikke om det er nødvendlig, det er det måske ikke}

\textbf{Det kan godt være at nogle af tingene enten skal beskrives bedre, på en anden måde, jeg er også lidt i tvivl om strukturen på dette afsnit... om der er for meget "gentagende" med eller bare for meget med. Nogle af sætningerne skal nok måske højest sandsynligt også lige formuleres lidt bedre :) }











%Privatliv
%Personlig privatliv og person data er ... spørgsmplet er om PHM ses på som personlig data som andet personlig data, graden af følsomhed

%Synlighed, 
%Størrelse og vægt på den benyttede PHM enhed. Er det noget som generer patienten i hverdagen, fordi den er tung og klundtet eller bare en af delene? 

%Medicinering, 
%monitorering skal ikke bare anvendes for monitoreringens skyld.. fordi man kan gør man det, sådan skal det ikke være

%Social issolation, 
%Der er mindre brug for plejepersonale, dette leder til mindre kontakt med patienterne, det kan evt. laves en løsning i at patienterne kan kommunikere sammen gennem monitoreringsenheden... noget noget...

%Autonomi, 
%Selvbestemmelse, monitoreringsenheden kan lave om på patienternes daglige rutiner, ændre på patienternes selvbestemmelse... Patienternes handlinger.

%Princip balance, 


%Informeret samtykke og udvikling, 
%Fuld forståelse af teknologeiens påvirkning for patienterne kan ikke forudsiges direkte da det ikke er blevet gjordt endnu.

%Påvirkning på hjemmehjælpere.
%Hjemmehjælperne får færre opgaver at tage sig af, da teknologien tager over, hvilket også betyder at patienterne oplever mindre personalekontakt, hvilket er nødvendigt for nogen..