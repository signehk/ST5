\section{Etiske problemstillinger}
%Indhold: I dette afsnit vil vi se på, om der er etiske problemstillinger ved, at lægen kan monitorere patientens aktivitetsniveau i hverdagen, og i så fald disse dilemmaer opstår heraf. Vi vil herunder drage paralleller til teknologier, hvor lignende problemstillinger opstår (fx hjemmeblodtryksmåler). 

Ved implementering af ny teknologi eller nye ideer i sundhedssektoren vil der ofte opstå etiske problemstillinger som skal adreseres. Der vil derfor i dette afsnit blive forsøgt belyst hvilke etiske problemstillinger, Fitbit Flex som aktivitets monitoreringsenhed, vil kunne skabe i sundhedsektoren.

Rapporten \author{patienthome2015} beskriver, at hvis monitoreringen bliver hyppig og meget tæt på, kan det for nogle patienter komme til at føles som overvågning. Det kan derfor blive et dilemma som man må tage op med patienten. Om vedkommende ønsker at modtage behandlingen, til trods for at der monitoreres kontinuert. I forhold til Fitbit Flex som monitorerer kontinuerlig aktivitet hos patienten, kan patienten selv tage stilling til om de ønsker behandlingen \citep{patienthome2015}.

Studiet af \author{Mittelstand2011} har undersøgt en række etiske afspekter ved brugen af enheder der monitorerer patienten i privatlivet. Studiet understeger visse afpekter ved de etiske aspekter vedrørende monitorering af patienten i privatlivet. Disse aspekter er, personlig privatliv og persondata, synlighed, medicinering, social issolation, aunonomi, pricip balance, informeret samtykke og udvikling, samt påvirkning på hjemmehjælpere. (jeg ved ikke helt om alle skal nævnes eller kun dem som er relevante for os) 
Af disse etiske aspekter vil Fitbit Flex primært kunne komme til at ramme områder som personlig privatliv og persondata, synlighed, samt autonomi \citep{Mittelstand2011}.

\textbf{Privatliv og persondata}
Ved at patienten bliver monitoreret kontinuerligt i løbet af hele dagen, vil nogle brugere kunne komme til at se det som overvågnig som nævnt tidligere. Fitbit Flex kan følge patienters aktivitetsniveau i form af skridt og aktive timer, herved vil alle deres aktiviteter ikke kunne spores ned til mindste detalje, dog vil det stadig kunne give nogle spekulationer ved visse patienter muligvis....
Der kan stilles spørgsmålstegn til om data om fysisk aktivitet falder ind under kategorien over følsom persondata, om det kan misbruges på nogen måde. Dog vil det ikke være lige så følsomt som data der vedrører patienten direkte, så som CPR nummer eller andet som kan bruges til at identificere eller skade patienten \citep{Mittelstand2011}.

\textbf{Synlighed}
Dette aspekt vedrører enhedens synlighed, altså om enhedn er stor, klodset og tung for patienten at gå med, så den måske genere patienten i hverdagen. Fitbit Flex, bæres som et andet armbåndsur og ses derfor ikke for generende for patienten, med mindre der er personlige ting som gør at de ikke kan gå med uret, for eksempel allergier eller andet.

§\textbf{Autonomi}
Hvis enheden ændre på patientens daglige rutiner, hvilket den nemt kan komme til, vil patienten skulle tilpasse sig en ny livsstil, hvilket Fitbit Flex nemt kan komme til at gøre, da formålet er at motivere patienter til at være mere aktive. Denne omvendig kan være svær for nogen. 

Et andet studie \author{Nordgren2013} addresere ogsp andre aspekter som, enhedens pålidlighed, i dette tilfælde hvor godt Fitbit Flex repræcenterer patienternes fysiske aktivitet, dette er beskrevet i afsnit \autoref{pålidelighedsafsnittet}. Brugervenlighed, om enheden er nem at anvende så der foreksempel opstår misforståelser, brugervenlighed beskrives i afsnit \autoref{brugervenlighed}. Livskvalitets værdier, hvilken påvirkning enheden har på patienterne i form af livskvalitet, både positive og negative. Uafhænginghed, sikkerhed ved monitorering af patienter i privatlivet, samt pris... \citep{Nordgren2013}...

Et andet etisk aspekt ved patient monitorering er, hvem og hvor mange der skal have ejerskab og ansvar over den data som der bliver indsamlet, samt hvordan det undgås at drukne i datamængder. Data skal prioriteres så det kun er det data man kan bruge til den reele behandling som bliver "gemt", der skal derfor tages stilling til hvad der er vigtigt \citep{patienthome2015}. Eksempelvis vil data fra Fitbit Flex, som er relevant for lægen, de data som fortæller noget om hvor aktiv patienten er i hverdagen, altså data som aktive timer, skridt gået, samt forbrændte kalorier. Data som søvnovervågning eller andet aktivitetarmbåndet kan registere vil være overflødigt for formålet med aktivitetsarmbåndets brug i forhold til hypertension, da dette ikke giver nogen direkte indikation vedrørende patientens aktivitetsniveau. Data skal derfor begrænses så det tilpasses til det som er nødvendigt for patientens forløb. 

\textbf{Det kan godt være at nogle af tingene enten skal beskrives bedre, på en anden måde, jeg er også lidt i tvivl om strukturen på dette afsnit... om der er for meget "gentagende" med eller bare for meget. Nogle af sætningerne skal nok også lige formuleres bedre :) }

%Privatliv
%Personlig privatliv og person data er ... spørgsmplet er om PHM ses på som personlig data som andet personlig data, graden af følsomhed

%Synlighed, 
%Størrelse og vægt på den benyttede PHM enhed. Er det noget som generer patienten i hverdagen, fordi den er tung og klundtet eller bare en af delene? 

%Medicinering, 
%monitorering skal ikke bare anvendes for monitoreringens skyld.. fordi man kan gør man det, sådan skal det ikke være

%Social issolation, 
%Der er mindre brug for plejepersonale, dette leder til mindre kontakt med patienterne, det kan evt. laves en løsning i at patienterne kan kommunikere sammen gennem monitoreringsenheden... noget noget...

%Autonomi, 
%Monitoreringsenheden kan lave om på patienternes daglige rutiner, ændre på patienternes selvbestemmelse... Patienternes handlinger.

%Princip balance, 


%Informeret samtykke og udvikling, 
%Fuld forståelse af teknologeiens påvirkning for patienterne kan ikke forudsiges direkte da det ikke er blevet gjordt endnu.

%Påvirkning på hjemmehjælpere.
%Hjemmehjælperne får færre opgaver at tage sig af, da teknologien tager over, hvilket også betyder at patienterne oplever mindre personalekontakt, hvilket er nødvendigt for nogen..