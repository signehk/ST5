\section{Etiske problemstillinger}
%Indhold: I dette afsnit vil vi se på, om der er etiske problemstillinger ved, at lægen kan monitorere patientens aktivitetsniveau i hverdagen, og i så fald disse dilemmaer opstår heraf. Vi vil herunder drage paralleller til teknologier, hvor lignende problemstillinger opstår (fx hjemmeblodtryksmåler). 

Ved implementering af ny teknologi eller nye ideer i sundhedssektoren vil der ofte opstå etiske problemstillinger som skal adreseres. Der vil derfor i dette afsnit blive forsøgt belyst hvilke etiske problemstillinger, Fitbit Flex som aktivitets monitoreringsenhed, vil kunne skabe.

Rapporten \author{patienthome2015} beskriver, at hvis monitoreringen bliver hyppig og meget tæt på, kan det for nogle patienter komme til at føles som overvågning. Det kan derfor blive et dilemma som man må tage op med patienten. Om vedkommende ønsker at modtage behandlingen, til trods for at der monitoreres kontinuert. I forhold til Fitbit Flex som monitorerer kontinuert aktivitet hos patienten, kan patienten selv tage stilling til om de ønsker behandlingen \citep{patienthome2015}.

Et andet etisk aspekt er, hvem og hvor mange der skal have ejerskab og ansvar over den data som der bliver indsamlet, samt hvordan der undgås at drukne i datamængder. Data skal prioriteres så det kun er det data man kan bruge til den reele behandling som bliver "gemt", der skal derfor tages stilling til hvad der er vigtigt \citep{patienthome2015}. Eksempelvis vil data fra Fitbit Flex som er relevant for lægen, de data som fortæller noget om hvor aktiv patienten er i hverdagen, altså data som aktive timer, skridt gået, samt forbrændte kalorier. Data som søvnovervågning eller andet aktivitetarmbåndet kan registere vil være overflødigt for formålet med aktivitetsarmbåndets brug i forhold til hypertension.

Studiet af \author{Mittelstand2011} har undersøgt en række etiske afspekter ved brugen af enheder der monitorerer patienten i privatlivet. Studiet understeger visse afpekter ved de etiske aspekter ved PHM, disse er:

\textbf{Privatliv}
Personlig privatliv og person data er ... spørgsmplet er om PHM ses på som personlig data som andet personlig data, graden af følsomhed

Synlighed, 
Størrelse og vægt på den benyttede PHM enhed. Er det noget som generer patienten i hverdagen, fordi den er tung og klundtet eller bare en af delene? 

Medicinering, 
monitorering skal ikke bare anvendes for monitoreringens skyld.. fordi man kan gør man det, sådan skal det ikke være

Social issolation, 
Der er mindre brug for plejepersonale, dette leder til mindre kontakt med patienterne, det kan evt. laves en løsning i at patienterne kan kommunikere sammen gennem monitoreringsenheden... noget noget...

Autonomi, 
Monitoreringsenheden kan lave om på patienternes daglige rutiner, ændre på patienternes selvbestemmelse... Patienternes handlinger.

Princip balance, 


Informeret samtykke og udvikling, 
Fuld forståelse af teknologeiens påvirkning for patienterne kan ikke forudsiges direkte da det ikke er blevet gjordt endnu.

Påvirkning på hjemmehjælpere.
Hjemmehjælperne får færre opgaver at tage sig af, da teknologien tager over, hvilket også betyder at patienterne oplever mindre personalekontakt, hvilket er nødvendigt for nogen..

\citep{Mittelstand2011}

Et andet studie .... addresere andre aspekter som, pålidlighed, brugervenlighed, pris, livskvalitets værdier, helbred, afhænginghed, sikkerhed, social kontakt ...