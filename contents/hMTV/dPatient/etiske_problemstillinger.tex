\section{Etiske problemstillinger}
Indhold: I dette afsnit vil vi se på, om der er etiske problemstillinger ved, at lægen kan monitorere patientens aktivitetsniveau i hverdagen, og i så fald disse dilemmaer opstår heraf. Vi vil herunder drage paralleller til teknologier, hvor lignende problemstillinger opstår (fx hjemmeblodtryksmåler). 


Hvis monitoreringen bliver hyppig og meget tæt på, kan det for nogen patienter komme til at føles som overvågning.

Et dilemma som man må tage op med patienten. Om vedkommende ønsker at "modtage" denne behandling, til trods for at der monitoreres kontinuert. 

Et andet aspekt er, hvem og hvor mange der skal have ejerskab og ansvar over det data som der bliver indsamlet, samt hvordan man undgår at drukne i datamængder. data skal prioriteres, hvad er vigtigt?.. f.eks. ved blodtrykskontroller.. (gælder både for behandler og borger)

Muligvis inddrage spiromagic, et hjemmemonitorerings spirometer til at opdage tidlige symptomer på KOL, for at indsætte hurtig og effektiv behandlingsforløb for patienter med KOL/optakt til KOL...


[1] http://www.patientathome.dk/media/128206/2015_11_16_patient_home_rapport_digital.pdf



Studiet af .... ... som har undersøgt de etiske afspekter ved brugen af PHM (privacy home monotering)  understeger visse afpekter ved de etiske aspekter ved PHM, disse er:

Privatliv,
personlig privatliv og person data.... spørgsmplet er om PHM ses på som personlig data som andet personlig data, graden af følsomhed

Synlighed, 
Størrelse og vægt på den benyttede PHM enhed. Er det noget som generer patienten i hverdagen, fordi den er tung og klundtet eller bare en af delene? 

Medicinering, 
monitorering skal ikke bare anvendes for monitoreringens skyld.. fordi man kan gør man det, sådan skal det ikke være

Social issolation, 
Der er mindre brug for plejepersonale, dette leder til mindre kontakt med patienterne, det kan evt. laves en løsning i at patienterne kan kommunikere sammen gennem monitoreringsenheden... noget noget...

Autonomi, 
Monitoreringsenheden kan lave om på patienternes daglige rutiner

Princip balance, 


Informeret samtykke og udvikling, 


Påvirkning på hjemmehjælpere.



[2] http://www.academia.edu/2014198/Ethical_Issues_of_Personal_Health_Monitoring_A_Literature_Review 