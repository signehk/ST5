Patienten
Metode
Til analyse af patienten og hvordan teknologien påvirker denne kan model (figur 7.1) anvendes. Her analyseres sociale forhold, kommunikative forhold, økonomiske forhold, individuelle forhold og etiske forhold, samt sammenspillet mellem disse. (Nogle af aspekterne kan vægte mere end andre alt efter hvilken teknologi der er tale om, med en aktivitets tracker er det nok mest, (listet i prioriteret rækkefælge) sociale forhold, herunder hvordan det påvirker patientens arbejds/uddannelses liv, familie, fritid og generel livskvalitet. Individuelle forhold, herunder hvordan patienten oplever teknologien, tilfredshed, motivation, tryghed mm. Kommunikative forhold, herunder hvordan der kommunikeres fra f.eks. sygehus til patient og omvendt, fra teknologi til patient og sygehus. Etiske forhold kan måske indgå i og med at patienternes aktivitet måles og bliver opsamlet som data, disse data bliver set af patienten og lægen, måske deler patienten sine resultater over sociale medier, kan disse data misbruges af andre? GPS, placering kan ses. Økonomiske forhold, altså om teknologien giver udgifter for patienten eller påvirker vedkommendes økonomi.).
Vi kan evt. modificere modellen så den passer til de ting vi vælger at lægge vægt på, i tilfælde af at der er aspekter som ikke er relevante. 


billede her

(Inklusions og eksklusions kriterier skal muligvis opstilles for at kunne fokusere modellen på den målgruppe som vi vælger at fokusere på)


MTV-spørgsmål
-	Er teknologien brugervenlig og motiverer den patienten til at få en mere aktiv hverdag?
-	Hvordan påvirker teknologien patienternes individuelle og sociale forhold i dagligdagen?
-	Hvor stor en andel af patienter oplever en positiv virkning ved anvendelse af teknologien og hvad spiller en rolle for at teknologien giver et succesfuldt forløb?
-	Hvor meget ansvar har patienten ved anvendelsen af teknologien?
-	Hvad er effekten af at anvende teknologien for patienten? Og er disse effekter kortsigtede, langtidssigtede eller både korttidssigtede og langtidssigtede?
-	Er der nogle etiske aspekter ved at monitorere patientens aktivitet, i så fald hvilke og dilemmaer opstår heraf?
-	Skal der være bestemte kriterier opfyldt for at patienten kan få en aktivitets tracker?
(man kan måske godt kombinere nogle af MTV-spørgsmålene så der ikke er så mange hvis det er  ) 
