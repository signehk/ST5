\section{Patientkriterier for tildeling af aktivitetsarmbånd i almen praksis}
%\begin{itemize}
%\item Hvilke kriterier skal patienten opfylde for at få tildelt et aktivitetsarmbånd? (fx hvor slem hypertension, hypertensive der ikke motionerer nok/overvægtige, hypertensive med andre følgesygdomme mm.)
%\end{itemize}

\noindent
Det vil være fordelagtigt at definere nogle kriterier, som patienten vil skulle overholde for at få tildelt et aktivitetsarmbånd til monitorering af fysisk aktivitet til hypertensive patienter. Dette gøres for indskrænke gruppen af patienter for at sikre, at aktivitetsarmbåndene gives til patienter, der vil få mest gavn af denne form for ekstra monitorering, så omkostningseffektiviten holdes så lav som muligt.

Dette kan eksempelvis defineres ud fra graden af fysisk inaktivitet, tendens til overvurdering/estimering af egen fysisk aktivitet og patienter med høj risiko for udvikling af hypertension eller følger til hypertension. 

Disse kriterier kunne være, at patienten er fysisk inaktiv ud fra definitionen i \autoref{sec:fys_inaktivitet}, at egen læge vurderer, at patienten overestimerer mængden af fysisk aktivitet, de dyrker til dagligt, eller at egen læge vurderer, at patienten er \textit{på vej} til at udvikle symptomer på hypertension eller følger til tilstanden, der vil forringe patientens livskvalitet.


\section{Brugervenlighed}
\begin{itemize}
\item Er aktivitetsarmbåndet brugervenligt/let at anvende/lære at anvende?
\item Hvad kræver det af patienten at anvende aktivitetsarmbåndet?
\end{itemize}
Tag evt. udgangspunkt i: 
\begin{itemize}
\item E Karapanos: Wellbeing in the Making: Peoples' Experiences with Wearable Activity Trackers
\end{itemize}

\subsection{Patientens ansvar ved anvendelse af aktivitetsarmbånd}
\begin{itemize}
\item Skal patienten selv sørge for at anskaffe sig uret? Batterier mm. (organisatorisk?)
\item Skal patienten sørge for at lægen får data?
\item Er det patientens ansvar at lære at anvende teknologien? (organisatorisk?)
\end{itemize}

\subsection{Motivation}
\begin{itemize}
\item Motiverer det folk til at anvende det hvis det er brugervenligt?
\item Gider folk at anvende det?
\end{itemize}

\section{Patientens individuelle og sociale forhold i dagligdagen (ved anvendelse af aktivitetsarmbånd)}
\begin{itemize}
\item Påvirker det/ hvordan påvirker det patientens sociale forhold i dagligdagen/ dagligdagen generelt? Har patienterne lyst til at gå med det (fx udseendemæssigt)?
\end{itemize}

\section{Effekter for patienten ved anvendelse af aktivitetsarmbånd}
\subsection{Effekter (og tidshorisonten på effekterne)}
\subsection{Motivation for patienten}
\subsubsection{Motivation til en mere aktiv hverdag}
\subsubsection{Demotiverende}
(fx hvis der ikke sker en ændring)
\subsection{Andel af patienter, der oplever en positiv virkning/effekt}

\section{Etiske aspekter ved anvendelse af aktivitetsarmbånd}
\begin{itemize}
\item Er der etiske problemstillinger ved at lægen så præcist kan se patientens aktivitetsniveau i hverdagen?
\item GPS i uret – patientens lokalitet.
\item Hvilke dilemmaer opstår af de etiske problemstillinger, hvis der er nogen?
\end{itemize}

\section{Delkonklusion}
\begin{itemize}
\item Hvad har betydning for, at teknologien giver patienten et positivt forløb?
\end{itemize}

