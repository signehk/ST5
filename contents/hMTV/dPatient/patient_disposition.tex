\section{Patientkriterier for tildeling af aktivitetsarmbånd i almen praksis}
\begin{itemize}
\item Hvilke kriterier skal patienten opfylde for at få tildelt et aktivitetsarmbånd? (fx hvor slem hypertension, hypertensive der ikke motionerer nok/overvægtige, hypertensive med andre følgesygdomme mm.)
\end{itemize}

\section{Brugervenlighed}
\begin{itemize}
\item Er aktivitetsarmbåndet brugervenligt/let at anvende/lære at anvende?
\item Hvad kræver det af patienten at anvende aktivitetsarmbåndet?
\end{itemize}

\subsection{Patientens ansvar ved anvendelse af aktivitetsarmbånd} (måske egen overskrift?)
\begin{itemize}
\item Skal patienten selv sørge for at anskaffe sig uret? Batterier mm. (organisatorisk?)
\item Skal patienten sørge for at lægen får data?
\item Er det patientens ansvar at lære at anvende teknologien? (organisatorisk?)
\end{itemize}

\subsection{Motivation}
\begin{itemize}
\item Motiverer det folk til at anvende det hvis det er brugervenligt?
\item Gider folk at anvende det?
\end{itemize}

\section{Patientens individuelle og sociale forhold i dagligdagen (ved anvendelse af aktivitetsarmbånd)}
\begin{itemize}
\item Påvirker det/ hvordan påvirker det patientens sociale forhold i dagligdagen/ dagligdagen generelt? Har patienterne lyst til at gå med det (fx udseendemæssigt)?
\end{itemize}

\section{Effekter for patienten ved anvendelse af aktivitetsarmbånd}
\subsection{Effekter (og tidshorisonten på effekterne)}
\subsection{Motivation for patienten}
\subsubsection{Motivation til en mere aktiv hverdag}
\subsubsection{Demotiverende}
(fx hvis der ikke sker en ændring)
\subsection{Andel af patienter, der oplever en positiv virkning/effekt}

\section{Etiske aspekter ved anvendelse af aktivitetsarmbånd}
\begin{itemize}
\item Er der etiske problemstillinger ved at lægen så præcist kan se patientens aktivitetsniveau i hverdagen?
\item GPS i uret – patientens lokalitet.
\item Hvilke dilemmaer opstår af de etiske problemstillinger, hvis der er nogen?
\end{itemize}

\section{Hvad har betydning for, at teknologien giver patienten et positivt forløb?}
(under hvilken overskrift? - mangler overskrift), fx brugervenlighed, patientens ansvar 
