\section{Patientkriterier for tildeling af aktivitetsarmbånd i almen praksis}
%\begin{itemize}
%\item Hvilke kriterier skal patienten opfylde for at få tildelt et aktivitetsarmbånd? (fx hvor slem hypertension, hypertensive der ikke motionerer nok/overvægtige, hypertensive med andre følgesygdomme mm.)
%\end{itemize}

\noindent
Det vil være fordelagtigt at definere nogle kriterier, som patienten vil skulle overholde for at få tildelt et aktivitetsarmbånd til monitorering af fysisk aktivitet til hypertensive patienter. Dette gøres for at indskrænke gruppen af patienter for at sikre, at aktivitetsarmbåndene gives til patienter, der vil få mest gavn af denne form for ekstra monitorering, så omkostningseffektiviteten holdes så lav som muligt.

Dette kan eksempelvis defineres ud fra graden af fysisk inaktivitet, tendens til overvurdering/estimering af egen fysisk aktivitet og patienter med høj risiko for udvikling af hypertension eller følger til hypertension. 

Disse kriterier kunne være, at patienten er fysisk inaktiv ud fra definitionen i \autoref{sec:fys_inaktivitet}, at egen læge vurderer, at patienten overestimerer mængden af fysisk aktivitet, som de dyrker til dagligt, eller at egen læge vurderer, at patienten har høj sandsynlighed for at udvikle symptomer på hypertension eller følger til tilstanden, der vil forringe patientens livskvalitet. Dette vil indskrænke gruppen af patienter, der vil få udleveret et aktivitetsarmbånd til de, der vil få mest gavn af brugen af armbåndet. 


\section{Brugervenlighed}
\begin{itemize}
\item Er aktivitetsarmbåndet brugervenligt/let at anvende/lære at anvende?
\item Hvad kræver det af patienten at anvende aktivitetsarmbåndet?
\end{itemize}
Tag evt. udgangspunkt i: 
\begin{itemize}
\item E Karapanos: Wellbeing in the Making: Peoples' Experiences with Wearable Activity Trackers
\item Acceptance of Commercially Available Wearable Activity Trackers Among Adults Aged Over 50 and With Chronic Illness: A Mixed-Methods Evaluation
\item A Qualitative Analysis of User Experiences With a Self-Tracker for Activity, Sleep, and Diet
\item Personal informatics for everyday life: How users without prior self-tracking experience engage with personal data
\item Kanitthika Kaewkannate og Soochan Kim: A comparison of wearable fitness devices
\end{itemize}

\subsection{Patientens ansvar ved anvendelse af aktivitetsarmbånd}
\begin{itemize}
\item Skal patienten selv sørge for at anskaffe sig uret? Batterier mm. (organisatorisk?)(OHH SÅ VI SNAKKER OGSÅ OM HVOR VIDT DET SKAL VÆRE BRUGERBETALING ELLER ANDET??)
\item Skal patienten sørge for at lægen får data?
\item Er det patientens ansvar at lære at anvende teknologien? (organisatorisk?)
\end{itemize}

\subsection{Motivation}
\begin{itemize}
\item Motiverer det folk (PATIENTER?? ELLER HVEM ER FOLK) til at anvende det hvis det er brugervenligt?
\item Gider folk (WHO IS THIS FOLK o.0) at anvende det?
\end{itemize}

\section{Patientens individuelle og sociale forhold i dagligdagen (ved anvendelse af aktivitetsarmbånd)}
%\begin{itemize}
%\item Påvirker det/ hvordan påvirker det patientens sociale forhold i dagligdagen/ dagligdagen generelt? Har patienterne lyst til at gå med det (fx udseendemæssigt)?
%\end{itemize}
\subsection{Sociale forhold}
En implementering af et aktivitetsarmbånd til brug af patienter, vil også påvirke patienten og dennes sociale forhold. Et aktivitetsarmbånd muliggør sammenligning med andre brugere af aktivitetsarmbånd over internettet, hvis patienten ønsker dette. Dette skaber en form for onlinefællesskab, hvor patienterne kan interagere med andre, der muligvis har lignende mål vedrørende daglig fysisk aktivitet \citep{karapanos2016}. 
Dette giver mulighed for, at patienten kan sammenligne sig med, og konkurrere mod, venner, kollegaer, familie, fremmede eller blot egne tidligere rekorder. På denne måde kan der skabes incitament til motion, hvis der konkurreres mod andre, da det vil virke som en motiverende faktor \citep{rooksby2014}.

I forhold til den valgte patientgruppe, kan alderen af patienten være afgørende, da prævalensen af hypertension stiger med alderen. I aldersgruppen $>50$ år har næsten $50~\%$ af befolkningen hypertension \citep{kronborg2008}. Denne aldersgruppe, især den ældste del af patienterne, vil ikke nødvendigvis kunne benytte sig af de sociale aspekter af aktivitetsarmbåndene, hvis de ikke er bekendte med sociale medier til dagligt. Disse vil udelukkende få gavn af de simple funktioner af et aktivitetsarmbånd, hvorfor det skal tages højde for, at alle patienter ikke vil få det samme udbytte af brugen af teknologien \citep{mercer2016}. Hvis den ældre er i stand til at synkronisere sit armbånd, så familiemedlemmer vil kunne tilgå dennes data via internettet, vil dette muligvis kunne fungere som en motiverende faktor, hvis de er klar over, at familien følger med i deres aktivitetsniveau.

\subsection{Individuelle forhold}
I forhold til den valgte patientgruppe, kan der være nogle individuelle forhold, der afgør, om aktivitetsarmbåndet vil blive brugt af patienterne. Dette kan især være aldersgruppen af patienterne. 

Denne gruppe af patienter kan være tilbageholdende over for ny sundhedsteknologi, som de selv skal betjene, da dette kræver en indsigt i, hvordan denne slags teknologi fungerer. Ikke alle i aldersgruppen, $>50$ år, har meget erfaring med brug af denne type teknologi, hvilket kan gøre nogle patienter tilbageholdende fra at tage teknologien til sig, selvom den er relativt brugervenlig \citep{mercer2016}. 

I et studie af en gruppe af kronisk syge i alderen $>50$ år, ville $73~\%$ af studiets deltagere købe en aktivitetstracker, da de generelt var tilfredse med én eller flere af de afprøvede aktivitetstrackere. I dette studie lagde patienterne blandt andet vægt på, om den var behageligt at gå med, og om den var pæn, så den fungerede som en form for smykke \citep{mercer2016}.

\section{Effekter for patienten ved anvendelse af aktivitetsarmbånd}
\subsection{Effekter (og tidshorisonten på effekterne)}
\subsection{Motivation for patienten}
\subsubsection{Motivation til en mere aktiv hverdag}
\subsubsection{Demotiverende}
(fx hvis der ikke sker en ændring)
\subsection{Andel af patienter, der oplever en positiv virkning/effekt}

\section{Etiske aspekter ved anvendelse af aktivitetsarmbånd}
\begin{itemize}
\item Er der etiske problemstillinger ved at lægen så præcist kan se patientens aktivitetsniveau i hverdagen?
\item GPS i uret – patientens lokalitet.
\item Hvilke dilemmaer opstår af de etiske problemstillinger, hvis der er nogen?
\end{itemize}

\section{Delkonklusion}
\begin{itemize}
\item Hvad har betydning for, at teknologien giver patienten et positivt forløb?
\end{itemize}

