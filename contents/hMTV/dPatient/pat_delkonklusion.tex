\section{Delkonklusion}

Med henblik på at opnå et succesfuldt forløb ved brug af aktivitetsarmbåndene, er det nødvendigt at opstille kriterier for patienten, da eksempelvis behandlingsresistent hypertension eller andre sygdomme kan have indvirkning på forløbet. Her vil det være op til den enkelte praktiserende læge at vurdere, hvorvidt patienten er i stand til at anvende aktivitetsarmbånd i forbindelse med behandlingen af hypertension. Det er i et studie fundet at $73~\%$ af forsøgspersoner over $50$ år efter endt studie, havde interesse i anskaffelse af aktivitetsarmbånd, hvorfor aldersgruppen ikke er et argument for afvisning af aktivitetsarmbånd.

Ved anvendelse af aktivitetsarmbåndet, er det vigtigt at patienterne finder det let anvendeligt i hverdagen, hvorfor både brugertilfredshed og -venlighed har en væsentlig betydning for, hvilke resultater anvendelsen vil give. I de undersøgte studier, er det fundet at brugere syntes Fitbit Flex var let at anvende og havde en underholdende brugerflade, hvilket er fundet som en motiverende faktor i forhold til kontinuert anvendelse af armbåndet. Af andre motiverende faktorer til kontinuert anvendelse, kan nævnes muligheden for at undersøge ændringer i egne aktivitetsvaner, nem synkronisering, samt komfort og design af aktivitetsarmbåndet, for at undgå at skille sig ud ved aktivitetsmonitorering.

I et studie, som introducerede kronisk syge over $50$ år til forskellige aktivitetstrackere, er det desuden fundet at monitoreringen af og målsætning for egen aktivitet, motiverer patienter til højere aktivitetsniveau. Dette vil være gavnligt for sygdomsforløbet, såfremt effekten på patienternes aktivitetsniveau, vil være det samme som vist i studiet. Her er det yderligere fundet, at de sociale egenskaber i Fitbit's brugerflade, kan give grundlag for yderligere motivation til højere aktivitetsniveau, hos patienter med kendskab til andre sociale medier, mens monitoreringen potentielt kan påvirke aktivitetsniveauet i en negativ retning.

Samtidig blev det fundet, at patienter uden tidligere erfaring med smartphones eller tablets, ikke havde yderligere problemer med tilvænning til aktivitetstrackerne, sammenlignet med patienter, som anvendte tidligere nævnte elektronikudstyr i hverdagen. Hvis samme mønster opstår hos hypertensive patienter i almen praksis, antages det at patienter uden stor erfaring med IT også vil være i stand til at anvende Fitbit Flex. 

Ved undersøgelse af de etiske problemstillinger ved anvendelse af Fitbit Flex, er den største problematik, at lægen har mulighed for kontinuert overvågning af patienten, hvilket kræver lægen i samråd med patienten, fastsætter hvilke data lægen har adgang til. Patienten kan potentielt anse den kontinuerte overvågning, som en krænkelse af privatlivet. Som udgangspunkt vil aktivitetsmønstret uden tilhørende GPS-tracking, anses som dataindsamling på niveau med døgnblodtryksmåling eller monitorering af hjerterytme, hvorfor det antages at patienterne er villige til at acceptere overvågning af aktivitetsvaner efter samtale med lægen. 