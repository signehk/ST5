\section{Patientkriterier for tildeling af aktivitetsarmbånd} \label{sec:kriterier}

Det vil være fordelagtigt at definere kriterier, som patienten vil skulle overholde for at få tildelt et Fitbit Flex til monitorering af fysisk aktivitet til hypertensive patienter. Dette gøres for at indskrænke gruppen af patienter for at sikre, at aktivitetsarmbåndene gives til patienter, der vil få mest gavn af denne form for ekstra monitorering, så omkostningseffektiviteten holdes så lav som muligt.

Dette kan eksempelvis defineres ud fra graden af fysisk inaktivitet, tendens til overestimering af egen fysisk aktivitet og patienter med høj risiko for udvikling af hypertension eller følger af hypertension. 

Disse kriterier kan være, at patienten er fysisk inaktiv ud fra definitionen i \autoref{sec:fys_inaktivitet}, at egen læge vurderer, at patienten overestimerer mængden af fysisk aktivitet, som de dyrker til dagligt, eller at egen læge vurderer, at patienten har høj sandsynlighed for at udvikle symptomer på hypertension eller følger til tilstanden, der vil forringe patientens livskvalitet. Dette vil indskrænke gruppen af patienter, der vil få udleveret Fitbit Flex til dem, der vil få mest gavn af brugen af armbåndet. 
De opstillede kriterier skal sammenholdes med lægelig vurdering, da der kan forekomme tilfælde, hvor patienten stadig er egnet Fitbit Flex. 

\subsection*{Kriterier for anvendelse af Fitbit Flex:}
\begin{itemize}
\item Må ikke lide af sygdomme, der har indvirkning på vedvarende anvendelse af armbåndet eller dets virkemåde. 
	\begin{itemize}
	\item Heriblandt neurologiske sygdomme: parkinsons syndrom, demens, alzheimers med flere.
	\end{itemize}

\item Må ikke befinde sig i tilstanden af behandlingsresistent hypertension. 

\item Må ikke være diagnostiseret med sekundær hypertension, jævnfør definitionen i \autoref{sec:dia_hypertension} %eller andre variationer af hypertension hvor en blodtryksreducerende effekt ikke kan opnås med fysisk aktivitet. 

\item Være i stand til at kunne følge de fysiske anbefalinger for at opnå blodtryksreducerende effekt. 
	\begin{itemize}
	\item Betydende faktorer: lammelse, manglende legemesdele med mere, som gør dem ude af stand til at være fysisk aktive. 
	\end{itemize}  

\end{itemize}




%Tilføj noget med at man ikke må være resistent overfor behandling, man skal kunne bevæge sig og det skal være primær hypertension (som også skal defineres!!!!!).