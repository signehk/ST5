\section{Patientens sociale og individuelle forhold i dagligdagen}

... lidt intro her. 

\subsection{Sociale forhold}
En implementering af et aktivitetsarmbånd til brug af patienter, vil også påvirke patienten og dennes sociale forhold. Et aktivitetsarmbånd muliggør sammenligning med andre brugere af aktivitetsarmbånd over internettet, hvis patienten ønsker dette. Dette skaber en form for onlinefællesskab, hvor patienterne kan interagere med andre, der muligvis har lignende mål vedrørende daglig fysisk aktivitet \citep{karapanos2016}. 
Dette giver mulighed for, at patienten kan sammenligne sig med, og konkurrere mod, venner, kollegaer, familie, fremmede eller blot egne tidligere rekorder. På denne måde kan der skabes incitament til motion, hvis der konkurreres mod andre, da det vil virke som en motiverende faktor \citep{rooksby2014}.

I forhold til den valgte patientgruppe, kan alderen af patienten være afgørende, da prævalensen af hypertension stiger med alderen. I aldersgruppen $>50$ år har næsten $50~\%$ af befolkningen hypertension \citep{kronborg2008}. Denne aldersgruppe, især den ældste del af patienterne, vil ikke nødvendigvis kunne benytte sig af de sociale aspekter af aktivitetsarmbåndene, hvis de ikke er bekendte med sociale medier til dagligt. Disse vil udelukkende få gavn af de simple funktioner af et aktivitetsarmbånd, hvorfor det skal tages højde for, at alle patienter ikke vil få det samme udbytte af brugen af teknologien \citep{mercer2016}. Hvis den ældre er i stand til at synkronisere sit armbånd, så familiemedlemmer vil kunne tilgå dennes data via internettet, vil dette muligvis kunne fungere som en motiverende faktor, hvis de er klar over, at familien følger med i deres aktivitetsniveau.

\subsection{Individuelle forhold}
I forhold til den valgte patientgruppe, kan der være nogle individuelle forhold, der afgør, om aktivitetsarmbåndet vil blive brugt af patienterne. Dette kan især være aldersgruppen af patienterne. 

Denne gruppe af patienter kan være tilbageholdende over for ny sundhedsteknologi, som de selv skal betjene, da dette kræver en indsigt i, hvordan denne slags teknologi fungerer. Ikke alle i aldersgruppen, $>50$ år, har meget erfaring med brug af denne type teknologi, hvilket kan gøre nogle patienter tilbageholdende fra at tage teknologien til sig, selvom den er relativt brugervenlig \citep{mercer2016}. 

I et studie af en gruppe af kronisk syge i alderen $>50$ år, ville $73~\%$ af studiets deltagere købe en aktivitetstracker, da de generelt var tilfredse med én eller flere af de afprøvede aktivitetstrackere. I dette studie lagde patienterne blandt andet vægt på, om den var behageligt at gå med, og om den var pæn, så den fungerede som en form for smykke \citep{mercer2016}.