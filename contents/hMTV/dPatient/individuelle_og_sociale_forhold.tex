\section{Patientens sociale og individuelle forhold i dagligdagen}
Idet en patient får tildelt et Fitbit Flex-aktivitetsarmbånd, er der forskellige tilhørende faktorer, der kan have betydning for patientens brug af armbåndet. 
Dette relaterer til, hvor avanceret teknologien er, og hvilke muligheder der er for at formidle den registrerede aktivitet for brugeren selv og omgangskreds. Dertil er det muligt at opdele disse faktorer i sociale og individuelle forhold og kan være hæmmende eller motiverende for patienten.   

\subsection{Sociale forhold}
En implementering af Fitbit Flex kan påvirke patienten og dennes sociale forhold. Fitbit Flex muliggør sammenligning med andre brugere af aktivitetsarmbånd over internettet, hvis patienten ønsker dette. Dette skaber en form for onlinefællesskab, hvor patienterne kan interagere med andre, der kan have lignende mål vedrørende daglig fysisk aktivitet \citep{karapanos2016}. 
Dette giver mulighed for, at patienten kan sammenligne sig med og konkurrere mod venner, kollegaer, familie, fremmede eller blot egne tidligere rekorder. På denne måde kan der skabes incitament til motion, hvis der konkurreres mod andre, da det kan virke som en motiverende faktor \citep{rooksby2014}.

I forhold til den valgte patientgruppe, kan alderen af patienten være afgørende, da prævalensen af hypertension stiger med alderen. I aldersgruppen $>50$ år har næsten $50~\%$ af befolkningen hypertension \citep{kronborg2008}. Denne aldersgruppe, især den ældste del af patienterne, vil ikke nødvendigvis kunne benytte sig af de sociale aspekter af aktivitetsarmbåndene, hvis de ikke er bekendte med sociale medier til dagligt. Disse vil udelukkende få gavn af de simple funktioner af et aktivitetsarmbånd, hvorfor det skal tages højde for, at alle patienter ikke vil få det samme udbytte af brugen af teknologien \citep{mercer2016}. Hvis de ældre patienter er i stand til at synkronisere sit armbånd, så familiemedlemmer vil kunne tilgå dennes data via internettet, vil dette muligvis kunne fungere som en motiverende faktor, hvis de er klar over og har givet samtykke til, at familien følger med i deres aktivitetsniveau.

\subsection{Individuelle forhold}
I forhold til den valgte patientgruppe, kan der være nogle individuelle forhold, der afgør, om aktivitetsarmbåndet vil blive brugt af patienterne. Dette kan især være aldersgruppen af patienterne. 

Den ældre del af patientgruppen kan være tilbageholdende over for ny sundhedsteknologi, som de selv skal betjene, da dette kræver en indsigt i, hvordan denne slags teknologi fungerer. Ikke alle i aldersgruppen, $>50$ år, har erfaring med brug af denne type teknologi, hvilket kan gøre nogle patienter tilbageholdende fra at tage teknologien til sig, selvom den er relativt brugervenlig \citep{mercer2016}. 

I et studie omhandlende en gruppe af kronisk syge i alderen $>50$ år, som havde gået med aktivitetsarmbånd over en periode, ville $73~\%$ af studiets deltagere købe en aktivitetstracker, da de generelt var tilfredse med én eller flere af de afprøvede aktivitetstrackere. I dette studie lagde patienterne blandt andet vægt på, om den var behageligt at gå med, og om den var pæn, så den fungerede som en form for smykke \citep{mercer2016}.