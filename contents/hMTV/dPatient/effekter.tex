\section{Effekter af monitorering af aktivitetsniveau}
%Indhold: I dette afsnit vil vi analysere effekter af en implementering af Fitbit Flex med hensigt på monitorering af aktivitetsniveau. Vi vil herunder komme ind på, hvilke effekter der er mulige at opnå, samt tidshorisonten på effekterne, om teknologien vil være en motiverende faktor for patienten til en mere aktiv hverdag, og om den kan være demotiverende, fx hvis der ikke sker en ændring. Vi vil se, om vi kan estimere, hvor stor en andel af patienter, der oplever en positiv virkning/effekt. 

Hypertensive patienters anvendelse af Fitbit Flex kan have forskellige virkninger på patienter og deres holdning til fysisk aktivitet, og det kan desuden påvirke forholdet mellem patient og læge.

Der er ikke fundet tal, der direkte viser sammenhængen mellem anvendelse af aktivitetsarmbånd og effekten af dette, så det er usikkert, hvor mange patienter, der reelt vil have en positiv effekt af anvendelse af Fitbit Flex. Ud fra forskellige studier kan det estimeres, hvordan patienter med hypertension vil påvirkes ved implementering af aktivitetsarmbånd som en del af behandlingen.

Der er en række studier, blandt andet studiet af \citeauthor{mercer2016}, der indikerer, at brugen af aktivitetsarmbånd kan give motivation til en mere aktiv hverdag. Det nævnes i studiet, at anvendelsen af aktivitetsarmbåndet giver brugeren bevidsthed om egen sundhed og aktivitetsniveau, hvilket i nogle tilfælde kan medføre et øget aktivitetsniveau. 

Testpersonerne, der deltog i studiet, blev, ved anvendelse af en likert skala, spurgt om, hvorvidt de følte, at aktivitetsarmbånd hjalp dem med at blive mere aktive. Til dette bliver der i gennemsnit givet et neutralt svar, således at der hverken var uenighed eller enighed om, at aktivitetsarmbånd medførte øget aktivitetsniveau hos den enkelte \citep{mercer2016}. Det kan derfor antages, at halvdelen af testpersonerne har oplevet, at aktivitetsarmbånd havde en positiv effekt på deres aktivitetsniveau.

Studiet konkluderer, at der er potentiale i at anvende aktivitetsarmbånd til at forbedre kronikeres motionsvaner. Desuden nævnes det, at implementering af aktivitetsarmbånd i sundhedssektoren ville kunne forbedre relationen mellem patient og læge, da det kan hjælpe lægen med at give patienten et bedre indblik i og bedre vejledning om patientens fysiske sundhed og vigtigheden af det \citep{mercer2016}. 

Et studie af \citeauthor{nelson2016} har undersøgt sammenhængen mellem anvendelse af aktivitetsarmbånd og testpersonens følelse af empowerment, hvilket er en følelse af handleevne og kontrol over beslutninger, der påvirker deres helbred \citep{toennesen2005}.

I studiet konkluderes, at forskellige egenskaber, såsom muligheden for at være en del af et fællesskab via en app og den feedback aktivitetsarmbåndet giver, har en positiv indflydelse på brugerens følelse af empowerment. Det faktum, at testpersonerne blev monitoreret havde dog ingen virkning på følelsen af empowerment, hvilket ifølge studiet kan skyldes, at monitorering associeres med negative konsekvenser, såsom invasion af privatlivet.
Studiet finder ligeledes, at jo større en brugers følelse af empowerment er, des mere tilskyndes denne til at opnå sine opstillede mål. Det påpeges dog i studiet, at testpersonernes generelle engagement til at opnå opstillede mål, var lavere end det er set i andre studier. Dette kan skyldes, at testpersonerne i dette studie ikke fik opstillet generelle mål for deres fysiske aktivitet, hvorimod testpersoner i andre studier skulle forsøge at nå mål opstillet af eller i samarbejde med andre. 

%- Nogle folk er motiveret. Nogle føler selv at det hjælper på deres sundhed. Kan få folk til at engagere sig mere i at opnå opstillede mål (måske mål opstillet sammen med læge, så det ikke er helt egne mål).  antage, at det vil have positiv på nogle patienter. Kilder siger desuden at kun patienter der er ”motiveret” til det kommer i non-farmakologisk behandling. 

%\citep{nelson2016}: http://www.sciencedirect.com.zorac.aub.aau.dk/science/article/pii/S0747563216302369 
