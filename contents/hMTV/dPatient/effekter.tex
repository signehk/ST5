\section{Effekter af monitorering af aktivitetsniveau}
Hypertensive patienters anvendelse af Fitbit Flex kan have forskellige virkninger på patienter og deres holdning til fysisk aktivitet, og det kan desuden påvirke forholdet mellem patient og læge.

Der er ikke fundet tal, der direkte viser sammenhængen mellem anvendelse af aktivitetsarmbånd og effekten af dette i forhold til dette projekts problemstilling, så det er usikkert, hvor mange patienter, der reelt vil have en positiv effekt af anvendelse af Fitbit Flex. Ud fra forskellige studier kan det estimeres, hvordan patienter med hypertension vil påvirkes ved implementering af aktivitetsarmbånd som en del af behandlingen.

En række studier, blandt andet  \citeauthor{mercer2016}, indikerer, at brugen af aktivitetsarmbånd kan give motivation til en mere aktiv hverdag. Det nævnes i studiet, at anvendelsen af aktivitetsarmbåndet giver brugeren bevidsthed om egen sundhed og aktivitetsniveau, hvilket i nogle tilfælde kan medføre et øget aktivitetsniveau. 

Testpersonerne, der deltog i studiet, blev, ved anvendelse af en likert skala, spurgt om, hvorvidt de følte, at aktivitetstrackere og skridttællere hjalp dem med at blive mere aktive, samt om aktivitetstrackere gjorde det lettere at være mere aktiv. Til disse spørgsmål fik studiets aktivitetstrackere højere gennemsnitlig score end en skridttæller \citep{mercer2016}. Det kan derfor antages, at brugen af en aktivitetstracker har en højere effekt end brugen af en skridttæller, og at aktivitetsarmbånd havde en positiv effekt på kronikernes aktivitetsniveau.

Studiet konkluderer, at der er potentiale i at anvende aktivitetsarmbånd til at forbedre kronikeres motionsvaner. Desuden nævnes det, at implementering af aktivitetsarmbånd i sundhedssektoren ville kunne forbedre relationen mellem patient og læge, da det kan hjælpe lægen med at give patienten et bedre indblik i og bedre vejledning om patientens fysiske sundhed og vigtigheden af det \citep{mercer2016}. 

Et studie af \citeauthor{nelson2016} har undersøgt sammenhængen mellem anvendelse af aktivitetsarmbånd og testpersonens følelse af empowerment, hvilket er en følelse af handleevne og kontrol over beslutninger, der påvirker deres helbred \citep{toennesen2005}.

I studiet konkluderes, at forskellige egenskaber, såsom muligheden for at være en del af et fællesskab via en applikation og den feedback aktivitetsarmbåndet giver, har en positiv indflydelse på brugerens følelse af empowerment. Det faktum, at testpersonerne blev monitoreret havde dog ingen virkning på følelsen af empowerment, hvilket ifølge studiet kan skyldes, at monitorering associeres med negative konsekvenser, såsom invasion af privatlivet.
Studiet finder ligeledes, at jo større en brugers følelse af empowerment er, des mere tilskyndes denne til at opnå sine opstillede mål. Det påpeges dog i studiet, at testpersonernes generelle engagement til at opnå opstillede mål, var lavere end det er set i andre studier. Dette kan skyldes, at testpersonerne i dette studie ikke fik opstillet generelle mål for deres fysiske aktivitet, hvorimod testpersoner i andre studier skulle forsøge at nå mål opstillet af eller i samarbejde med andre. 
