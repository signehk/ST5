\section{Disposition for organisation}

I dette afsnit...

\section{Patientforløb med hypertension}
(Inddragelse af Leavit modellen i dette afsnit, vi kan måske starte med den og præsentere hvilke aktører der er i systemet og nævne dem overordnet og derefter bygge ud med de efterfølgende afsnit, på en eller anden måde, eller noget i den stil)
\begin{itemize}
\item Samspil mellem primær og sekundær sektor
\begin{itemize}
\item Hvor mange patienter bliver henvist til den sekundære sektor? Hvordan vil antallet som henvises blive påvirket hvis patienterne har anvendt aktivitetsarmbånd (og dermed været mere fysisk aktive)?
\end{itemize}
\item Udredning/diagnostisering af hypertension
\begin{itemize}
\item Hvor og hvordan foregår dette?
\end{itemize}
\item Hvor og hvordan behandles hypertension
\item Aktivitetsarmbånnd i den nuværende organisation
\begin{itemize}
\item Hvordan passer disse ind i den nuværende organisation? 
\item Vil de kunne erstatte allerede eksisterende "teknologier"
\end{itemize}
\end{itemize}

\section{Nye opgaver ved implementering af aktivitetsarmbånd}
\begin{itemize}
\item Efteruddannelse af personale
\begin{itemize}
\item Hvem skal efteruddannes og hvordan foregår dette. Lære hvordan udstyret fungerer, lære at instruere i det, så patienterne kan bruge det selv.
\item Analysering af data fra aktivitetsarmbåndet og hvilket og hvor meget data er der tale om? 
\end{itemize}
\item Indkøb af udstyr
\begin{itemize}
\item Hvilket udstyr er der tale om og hvad skal det kunne? (det specificeres i teknologi afsnittet)
\item Hvor skal udstyret købes ind fra?
\end{itemize}
\item Information og kontakt
\begin{itemize}
\item Hvordan kan lægen få yderligere oplysninger om teknologien hvis dette er nødvendig, f.eks. ved opgradering? 
\item Hvordan kan patienter få information om teknologien, hvis de f.eks. har brug for hjælp til at bruge den og hvem skal de kontakte?
\end{itemize}
\end{itemize}

\section{Delkonklussion}

