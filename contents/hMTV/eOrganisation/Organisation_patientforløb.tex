\section{Patientforløb med hypertension}
Tages der udgangspunkt i Leavitts modificerede organisationsmodel, er der en række variable i organisationen, der vil påvirkes, hvis der sker en organisationsforandring. Disse variable er henholdsvis struktur, teknologi, aktører, opgaver samt omgivelser \citep{mtvhaandbog}. \\
Indføres den nye teknologi, aktivitetsarmbånd, som et led i behandling mod hypertension, kan dette påvirke de resterende organisationsvariable i forskellige grader afhængig af, hvilken virkning teknologien har på behandlingsforløbet. \\
Aktørerne, som er lægerne eller sygeplejerskerne i den almene klinik, vil få til opgave at lære at anvende aktivitetsarmbånd som en del af et behandlingsforløb for hypertensive patienter. Indførelse af aktivitetsarmbånd kræver derfor, at de almen praktiserende læger ønsker at innovere behandlingen og anvende den alternative behandlingsmetode i klinikkerne. Nogle almen praktiserende læger kan muligvis være skeptiske overfor indførelse af en ny teknologi, såsom aktivitetsarmbånd. De læger, der har interesse i at afprøve den nye teknologi, kan forsøge at indføre det i praksis. Har indførelsen en positiv effekt, kan teknologien muligvis udvides til flere almene klinikker. \\
For at gøre det mere attraktivt for de praktiserende læger, kan der indføres et honorar for anvendelse af aktivitetsarmbånd som en del af behandlingen for hypertension. Ved at honorere anvendelse af nye teknologier i almen praksis, kan anvendelsen af udstyret øges, hvilket gør sig gældende ved eksempelvis hjemmeblodtryksmåling \citep{bang2006}. 

\subsection{Samspil mellem primær og sekundær sektor}
Hypertensive patienter kan, hvis der er behov for det, af den almene praktiserende læge blive henvist til forskellige afdelinger afhængig af, hvad lægen vurderer nødvendigt. Regionen, hvor patienten er bosat i, kan have betydning for, hvor patienten henvises til. I den østlige del af Region Midt findes eksempelvis Blodtrykscenteret, hvor Nyremedicinsk, Hjertemedicinsk og Endokronologisk Afdeling i samarbejde behandler hypertension og eventuelle følgesygdomme eller sekundære årsager til hypertension. Patienter kan henvises til Blodtrykscenteret, hvis én af en række indikationer opfyldes. Dette kan eksempelvis være ved behandlingsresistent hypertension, hypertensive patienter med nogle former for hjertekarsygdomme, mistanke om sekundær hypertension, eller hvis nyopdaget hypertension skal verificeres ved hjælp af døgnblodtryksmåling. Ved en henvisning til Blodtrykscenteret bør patienten inden have fået foretaget en udredning af egen læge, hvor informationer om denne bør vedlægges ved henvisningen, så det ikke er nødvendigt at lave undersøgelserne igen \citep{aarhusuniversitetshopsital}. I andre regioner uden et center lignende Blodtrykscenteret vil hypertensive patienter blive henvist til en afdeling, der beskæftiger sig med det konkrete problem, hvilket typisk vil være enten nefrologisk, kardiologisk eller endokronologisk afdeling. Her er eksempelvis Nyremedicinsk Afdeling i Aalborg, som også har specifikke krav til, hvilke hypertensive patienter, der kan modtages, idet der ikke er kapacitet til andre end de sværeste tilfælde. Disse kan eksempelvis være patienter med behandlingsresistent eller sekundær hypertension samt unge patienter, idet der er større risiko for sekundær hypertension i denne aldersgruppe. Desuden modtages patienter med svær akut blodtryksforhøjelse med organpåvirkning. Hvis der foruden nyrepåvirkning også er problemer med lungeødem eller brystsmerter, henvises der til kardiologisk afdeling \citep{buur2011}. Når det vurderes af lægerne på den pågældende afdeling, at patienten er klar til det, videregives patienten igen til den almen praktiserende læge, hvor patienten blandt andet kan gå til kontrol \citep{sundhedsstyrelsen2010, lodberg2016}.
Hvis aktivitetsarmbånd bliver indført som en del af behandlingen mod hypertension kan dette muligvis påvirke denne struktur i organisationen ved at ændre antallet af patienter, der bliver henvist til forskellige afdelinger. Hvis aktivitetsarmbånd har en positiv effekt, og flere patienter får det bedre af at få et højere aktivitetsniveau, vil der sandsynligvis være færre følgevirkninger såsom nyresygdomme og hjerteproblemer, og det vil derfor ikke være nødvendigt at henvise patienterne. 

\subsection{Diagnose og udredning af hypertension}
Hypertension giver sjældent symptomer og opdages derfor ofte ved en tilfældighed ved eksempelvis sundhedstjek hos den almen praktiserende læge. Diagnosen hypertension kan ikke stilles efter blot en enkelt måling foretaget hos lægen, da patienten kan være nervøs og dermed påvirke resultatet. Patienten bør få foretaget enten en døgnblodtryksmåling eller hjemmeblodtryksmåling, hvis målinger i klinikken viser forhøjet blodtryk. Patienten kan desuden sidde i et rum uden tilstedeværelse af sundhedspersonale og få foretaget automatiske blodtryksmålinger \citep{lodberg2016, bech2015}. \\
Viser blodtryksmålinger et forhøjet blodtryk skal patienten igennem en videre udredning. Patientens tidligere sygehistorie vil betragtes, herunder blandt andet forskellige risikofaktorer for hypertension, og om der er familiær disposition til hypertension, diabetes eller nyresygdomme mm. Foruden dette foretages en objektiv undersøgelse af patienten, hvor blandt andet højde, vægt og abdominalomfang måles. Der tages desuden EKG målinger, blodprøver og urinprøver, og hvis der er kliniske tegn på hjertesvigt, foretages røntgen af thorax og ekkokardiografi. Dette kan foregår paraklinisk, hvor patienten bliver henvist til sekundærsektoren \citep{lodberg2016, bech2015}. \\
Er den hypertensive patient under 40 år, har et meget højt blodtryk eller har behandlingsrefraktær hypertension, bør patienten undersøges for sekundær hypertension for at sikre, at det ikke er bagvedliggende sygdomme, der resulterer i hypertension \citep{lodberg2016}. Sekundær hypertension forekommer hos mindre end 5 \% af tilfældene, og den hyppigste årsag er nyresygdomme \citep{lodberg2008}. Patienten kan henvises til nefrologisk afdeling for yderligere undersøgelser, hvis der findes eller er mistanke om en bagvedliggende nyresygdom \citep{lodberg2016, sundhedsstyrelsen2010}. 

\subsection{Behandling af hypertension}
Når den almen praktiserende læge har diagnosticeret og vurderet patienten kan behandlingen påbegyndes. Behandling af hypertension afhænger af, hvilken grad af hypertension patienten har samt, hvorvidt det er sekundær hypertension. Det er herved forskelligt hvor og hvem, der varetager behandlingen. Hvis det opdages, at patienten har sekundær hypertension påbegyndes behandlingen på den pågældende afdeling i sekundærsektoren, der varetager sig opgaver omhandlende patientens sygdom. Har patienten eksempelvis en neurologisk sygdom, er det neuromedicinsk afdeling, der behandler patienten. Her varetages behandlingen indtil patienten er stabil, hvorefter en praktiserende læge igen kan overtage patienten og kontrollere den videre behandling \citep{sundhedssyrelsen2010}. Opstår følgesygdomme af hypertension… Er det behandlingsrefraktær hypertension…
I den almene klinik har lægen mulighed for at behandle hypertensive patienter farmakologisk og non-farmakologisk. Behandlingen vurderes ud fra, om patienten har risiko for kardiovaskulær sygdom, hvor lægen blandt andet undersøger om patienten har hypertensive organskader, diabetes og nyresygdomme \citep{promedicin2016}. \\
Non-farmakologisk behandling består af en række anbefalinger og omlægning af livsstil, herunder rygestop, motion, kostændringer og saltindtag. Alle hypertensive patienter bør som udgangspunkt komme i non-farmakologisk behandling. Dog skal patienter med et blodtryk, der ligger over 180/105 mmHg, starte på en farmakologisk behandling for at sænke blodtrykket inden der udføres jævnligt og/eller intensiv fysisk aktivitet \citep{pedersen2016}.
Mængden af forskellige præparater, patienten bør tage, afhænger af de tidligere nævnte risikofaktorer, som lægen undersøger for, samt hvor højt blodtrykket er. Har patienten eksempelvis en mild grad af hypertension, hvor blodtrykket ligger mellem 140 og 159 systole eller 90 og 99 diastole, kan patienten muligvis nøjes med livsstilsændringer, hvis der ikke er nogle risikofaktorer. Er der risikofaktorer, eksempelvis diabetes bør der senere tillægges blodtryksmedicin \citep{bech2015}. Audit Projekt Odense (AOP), som er en forskningsenhed for almen praksis, har udgivet en rapport omhandlende registreringer af hypertension i 184 almene praksisser. Ifølge denne rapport var der 35 \% af de registrerede hypertensive patienter, der fik for lidt motion, og for lavt aktivitetsniveau var dermed den tredje hyppigste risikofaktor \citep{munch2007}. \\
Ifølge rapporten af AOP er 11 \% af de registrerede hypertensive patienter i non-farmakologisk behandling, samtidig er det kun 1,7 \% af de registrerede patienter, der ikke får nogle former for farmakologisk behandling \citep{munch2007}.
 (Jeg ved ikke hvor jeg er på vej hen… Hjælp. Er alt det her overhovedet relevant at have med?)
Anvendelse af aktivitetsarmbånd til behandling af hypertension vil høre under kategorien non-farmakologisk behandling, hvor lægen eller sygeplejersken vil få endnu en opgave, idet denne først skal oplære patienten i brug af armbåndet og siden følge op på aktivitetsniveauet. Lægen kan ud fra data om aktivitetsniveauet vejlede patienten om motion. Skulle der ske tilbagegang i behandlingen kan lægen desuden tjekke om det muligvis kunne skyldes et lavere aktivitetsniveau end normalt. 

\subsection{Aktivitetsarmbånd i den nuværende organisation}
I rapporten af APO er der en kontaktperson i hvert af de deltagende klinikker, der har svaret på et spørgeskema. De bliver adspurgt om, hvorvidt det ønskes, at personale involveres mere i patientbehandlingen af hypertension, hvortil 67,3 \% svarer ja. Ud af disse svarer omkring 49 \%, at der ønskes øget involvering indenfor vejledning om motion og fysisk aktivitet \citep{munch2007}. 
(Hvad så)
Blodtrykscenteret i Region Midtjylland reklamerer med, at forskning og implementering af nye behandlingsmetoder varetages af dem…  \citep{aarhusuniversitetshospital} (og noget??)
Interessentanalyse?


% lodberg2008: http://www.dahs.dk/fileadmin/2008_en_faelles_klinisk_vejledningX22.pdf 
% bang2006: http://www.dahs.dk/fileadmin/BTmaaling_version-17.pdf 
% aarhusuniversitetshospital: http://www.auh.dk/om-auh/afdelinger/nyremedicinsk-afdeling/til-fagfolk/om-nyremedicinsk-afdeling/blodtrykscentret/ 
% buur2011: https://pri.rn.dk/Assets/14144/Henvisning-Nyremed.pdf 
% sundhedsstyrelsen2010: http://sundhedsstyrelsen.dk/~/media/63BE82801BB441DCAC638C81D81F8963.ashx (er der andre kilder der hedder dette?)
% munck2007: http://www.apo-danmark.dk/files/pub/1832.pdf  (patienternes svar: http://www.apo-danmark.dk/files/pub/1833.pdf ) (Fra 2010: http://www.apo-danmark.dk/files/pub/3697.pdf ) 
% promedicin2016: http://pro.medicin.dk/sygdomme/sygdom/318336 
% pedersen2016: http://pro.medicin.dk/Specielleemner/Emner/318420#a000 
% bech2015: http://www.dahs.dk/fileadmin/user_upload/2013_opdateringer/opdateringer_2014/opdateringer_2015/behandlingsvejledning_2015_juni_final.pdf 
