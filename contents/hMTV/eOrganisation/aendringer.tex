\section{Organisatoriske ændringer}\label{sec:org_aendringer}
Ved implementering af Fitbit Flex vil der forekomme ændringer i patientforløbet og i vidensbehovet for sundhedspersonale og patienter. 

Implementeringen vil påvirke patientforløbet, idet det skal vurderes, hvor i forløbet patienten skal introduceres for aktivitetsarmbåndet. 
Med udgangspunkt i \autoref{fig:patientforloeb}, anses det mest relevant at introducere patienten for og udlåne Fitbit Flex under behandlingsforløbet. Dette er vurderet ud fra, at patienten på daværende tidspunkt i patientforløbet vil blive givet en række anbefalinger, heriblandt fordelene ved fysisk aktivitet. 
Det anses ligeledes, at aktivitetsarmbåndet vil være fordelagtigt i relation til fremtidige kontroller, da dette vil give en indikation om, hvorvidt patientens aktivitetsniveau er stigende eller faldende fra forhenværende kontroller.   

Andre organisatoriske ændringer ses i relation til videnbehovet for at anvende teknologien i en medicinsk sammenhæng. Hertil skal både patient og læge have viden om teknologien. Det essentielle er, at både patient og læge skal være indforstået med, hvordan armbåndet bruges som et middel til dokumentation af patientens aktivitetsniveau. Hertil skal lægen vide, hvordan motion kan have en virkning på blodtrykket, og hvad det vil betyde for patienten, hvis vedkommende ikke er tilstrækkeligt aktiv i forhold til Sundhedsstyrelsens anbefalinger, som nævnt i \autoref{sec:prob_fysaktiv}. Sammenhængen mellem aktivitet og blodtryk er beskrevet i \autoref{sec:effekterafaktivitet}.

Lægen skal yderligere kunne vurdere, hvilke hypertensive patienter, der er egnet til at få et armbånd, hvorfor der i \autoref{sec:kriterier} opstilles forslag til patientkriterier for tildeling af et Fitbit Flex-aktivitetsarmbånd. I forlængelse med dette skal teknologien også tilpasses den enkelte patient, hvilket vil betyde, at lægen yderligere skal have færdigheder i brugertilpasning, som er beskrevet i \autoref{sec:brugertilpasning}. Lægen skal ved udlevering af Fitbit Flex kunne instruere patienten i brugen af armbåndet, samt de tilhørende dele og programmer, således patienten er i stand til at anvende armbåndets forskellige funktioner på egen hånd. \\

\noindent
Instruktionen af armbåndet vil som minimum skulle indeholde viden omkring: 
\begin{itemize}
\item Anskaffelse og installation af applikation på smartphone, computer eller lignende enhed
\item Hvordan armbåndet virker i forhold til, hvordan det registrerer fysisk aktivitet 
\item Brugerfladen af applikationen, samt hvordan patienten selv kan aflæse den registrerede fysiske aktivitet
\item Vigtigheden af jævnlig synkronisering af data , samt hvordan dette gøres
\item Hvordan batteriniveau aflæses, og hvordan batteriet genoplades \\
\end{itemize}


\noindent
Yderligere udleveres en brugermanual med armbåndet, som patienten kan anvende i tilfælde af tvivl eller spørgsmål. Det registrerede data skal efterfølgende også kunne aflæses af lægen, som derfor selv skal have indblik i brugerinterfacets funktioner.

