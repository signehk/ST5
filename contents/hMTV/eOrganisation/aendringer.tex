\section{Organisatoriske ændringer}\label{sec:org_aendringer}

For at anvende FitBit Flex i en medicinsk sammenhæng, skal både patient og læge have viden om teknologien. Det essentielle er at både patient og læge skal være indforstået med, hvordan armbåndet bruges som et middel til dokumentation af patientens aktivitetsmønster.

Hertil skal lægen vide hvordan motion kan have en virkning på blodtrykket, og hvad det vil betyde for patienten hvis vedkommende ikke er tilstrækkeligt aktiv, i forhold til Sundhedsstyrelsens anbefalinger. Sammenhæng mellem aktivitet og blodtryk er beskrevet i \autoref{sec:effekterafaktivitet}.

Lægen skal yderligere kunne vurdere hvilke hypertensive patienter, der er egnet til at få et armbånd, hvorfor der i \autoref{sec:kriterier} opstilles forslag til patient-kriterier for tildeling af et Fitbit Flex. I forlængelse med dette skal teknologien også tilpasses den enkelte patient, hvilket vil betyde at lægen yderligere skal have færdigheder i brugertilpasning, som er beskrevet i \autoref{sec:brugertilpasning}. 

Derudover skal lægen ved udlevering af Fitbit Flex, kunne instruere patienten i brugen af armbåndet, samt de tilhørende dele og programmer, således patienten er i stand til at anvende armbåndets forskellige funktioner på egen hånd. 

Instruktionen af armbåndet ville som minimum indeholde viden omkring: 
\begin{itemize}
\item Forståelse vedrørende anskaffelse og installation af applikation på mobil, computer eller lignende enhed. 
\item Forudsætningerne for at armbåndet virker, i forhold til hvordan registrer fysisk aktivitet. 
\item Brugerfladen af applikationen, samt hvordan patienten selv kan aflæse den registrerede fysiske aktivitet.  
\item Vigtigheden af jævnlig synkronisering af data. 
\item Hvordan batteriniveau aflæses og hvordan det genoplades. 
\end{itemize} 

Yderligere udleveres en brugermanual med armbåndet, som patienten kan anvende i tilfælde af tvivl eller spørgsmål.   

Det registrerede data skal efterfølgende også kunne aflæses af lægen, som derfor selv skal have indblik i brugerinterfacets funktioner.

