\section{Patientforløb med hypertension}
Tages der udgangspunkt i Leavitts modificerede organisationsmodel, er der en række variable i organisationen, der vil påvirkes, hvis der sker en organisationsforandring. Disse variable er henholdsvis struktur, teknologi, aktører, opgaver samt omgivelser \citep{mtvhaandbog}.

Indføres aktivitetsarmbånd, som et led i behandling mod hypertension, kan dette påvirke de resterende organisationsvariable i forskellige grader afhængig af, hvilken virkning teknologien har på behandlingsforløbet.

Aktørerne, som er lægerne eller sygeplejerskerne i den almene klinik, vil få til opgave at lære at anvende aktivitetsarmbånd som en del af et behandlingsforløb for hypertensive patienter. Indførelse af aktivitetsarmbånd kræver derfor, at de almen praktiserende læger ønsker at innovere behandlingen og anvende den alternative behandlingsmetode i klinikkerne. Nogle almen praktiserende læger kan muligvis være skeptiske overfor indførelse af en ny teknologi, såsom aktivitetsarmbånd. De læger, der har interesse i at afprøve den nye teknologi, kan forsøge at indføre det i praksis. Har indførelsen en positiv effekt, kan teknologien muligvis udvides til flere almen praksis klinikker.

For at gøre det mere attraktivt for de praktiserende læger, kan der indføres et honorar for anvendelse af aktivitetsarmbånd som en del af behandlingen for hypertension. Ved at honorere anvendelse af nye teknologier i almen praksis, kan anvendelsen af udstyret øges, hvilket gør sig gældende ved eksempelvis hjemmeblodtryksmåling \citep{bang2006}.

\subsection{Diagnose og udredning af hypertension}

Hypertension giver sjældent symptomer og opdages derfor ofte ved en tilfældighed ved eksempelvis sundhedstjek hos den almen praktiserende læge. Diagnosen hypertension kan ikke stilles efter blot en enkelt måling foretaget hos lægen, da patienten kan være nervøs og dermed påvirke resultatet. Patienten bør få foretaget enten en døgnblodtryksmåling eller hjemmeblodtryksmåling, hvis målinger i klinikken viser forhøjet blodtryk. Patienten kan desuden sidde i et rum uden tilstedeværelse af sundhedspersonale og få foretaget blodtryksmålinger med en automatisk blodtryksmåler \citep{lodberg2016, bech2015}.

Viser blodtryksmålinger et forhøjet blodtryk skal patienten igennem en videre udredning. Patientens tidligere sygehistorie vil betragtes, herunder blandt andet forskellige risikofaktorer for hypertension såsom lavt aktivitetsniveau eller diabetes, og om der er familiær disposition til eksempelvis hypertension, diabetes eller nyresygdomme. Foruden dette foretages en objektiv undersøgelse af patienten, hvor blandt andet højde, vægt og abdominalomfang måles. Der tages desuden EKG målinger, blodprøver og urinprøver, og hvis der er kliniske tegn på hjertesvigt, foretages røntgen af thorax og ekkokardiografi. Dette kan foregå i tilfælde, hvor patienten bliver henvist til sekundærsektoren \citep{lodberg2016, bech2015}.

Er den hypertensive patient under $40$ år, har et meget højt blodtryk eller har behandlingsresistent hypertension, bør patienten undersøges for sekundær hypertension for at sikre, at det ikke er bagvedliggende sygdomme, der resulterer i hypertension \citep{lodberg2016}. Sekundær hypertension forekommer hos mindre end $5~\%$ af tilfældene, og den hyppigste årsag er nyresygdomme \citep{lodberg2008}. Patienten kan eksempelvis henvises til neurologisk afdeling for yderligere undersøgelser, hvis der findes eller er mistanke om en bagvedliggende nyresygdom \citep{lodberg2016, sundhedsstyrelsen2010}. 

\subsubsection{Samspil mellem primær og sekundær sektor}
Under udredningen eller behandling kan hypertensive patienter, hvis nødvendigt, blive henvist til forskellige afdelinger af den almene praktiserende læge. Regionen, hvor patienten er bosat i, kan have betydning for, hvor patienten henvises til. Henvisning kan eksempelvis være til Blodtrykscenteret i Region Midtjylland, hvor Nyremedicinsk, Hjertemedicinsk og Endokronologisk Afdeling i samarbejde behandler hypertension og eventuelle følgesygdomme eller sekundære årsager til hypertension. Patienter kan henvises til Blodtrykscenteret, hvis én af en række indikationer opfyldes. Dette kan eksempelvis være ved behandlingsresistent hypertension, hypertensive patienter med nogle former for hjertekarsygdomme, mistanke om sekundær hypertension, eller hvis nyopdaget hypertension skal verificeres ved hjælp af døgnblodtryksmåling. Ved en henvisning til Blodtrykscenteret bør patienten inden have fået foretaget en udredning af egen læge, hvor informationer om denne bør vedlægges ved henvisningen, så det ikke er nødvendigt at lave undersøgelserne igen \citep{aarhusuniversitetshospital}. 

I andre regioner uden et center lignende Blodtrykscenteret vil hypertensive patienter blive henvist til en afdeling, der beskæftiger sig med det konkrete problem, hvilket typisk vil være enten neurologisk, kardiologisk eller endokronologisk afdeling \citep{buur2011}. Når det vurderes af lægerne på den pågældende afdeling, at patienten har fået den tilstrækkelige behandling på afdelingen, videregives patienten igen til den almen praktiserende læge, hvor patienten blandt andet kan gå til kontrol \citep{sundhedsstyrelsen2010, lodberg2016}.

Hvis aktivitetsarmbånd bliver indført som en del af behandlingen mod hypertension kan dette muligvis påvirke denne struktur i organisationen ved at ændre antallet af patienter, der bliver henvist til forskellige afdelinger. Hvis aktivitetsarmbånd har en positiv effekt, og flere patienter får det bedre af at få et højere aktivitetsniveau, kan der muligvis være færre følgevirkninger såsom nyresygdomme og hjerteproblemer, og det vil derfor ikke være nødvendigt at henvise disse patienter til den sekundære sundhedssektor. 

\subsection{Behandling af hypertension}

Når den almen praktiserende læge har diagnosticeret og vurderet patienten kan behandlingen påbegyndes. Behandling af hypertension afhænger af, hvilken grad af hypertension patienten har samt, hvorvidt det er sekundær hypertension. Det er herved forskelligt hvor og hvem, der varetager behandlingen. Hvis det opdages, at patienten har sekundær hypertension påbegyndes behandlingen på den pågældende afdeling i sekundærsektoren, der varetager sig opgaver omhandlende patientens sygdom. Opstår følgevirkninger, der kræver yderligere behandling, henvises patienten ligeledes til en passende afdeling. Får patienten eksempelvis en neurologisk sygdom, er det neurologisk afdeling, der behandler patienten. Her varetages behandlingen indtil patienten er stabil, hvorefter en praktiserende læge igen kan overtage patienten og kontrollere den videre behandling \citep{sundhedsstyrelsen2010}.

Som skrevet i \autoref{sec:hypertension} har lægen i den almene klinik mulighed for at behandle hypertensive patienter farmakologisk og non-farmakologisk. Behandlingen vurderes ud fra, om patienten har risiko for kardiovaskulær sygdom, hvor lægen blandt andet undersøger om patienten har risikofaktorer såsom hypertensive organskader, diabetes og nyresygdomme \citep{promedicin2016}.

%Non-farmakologisk behandling består af en række anbefalinger vedrørende omlægning af livsstil, herunder rygestop, motion, kostændringer og saltindtag. Alle hypertensive patienter bør som udgangspunkt komme i non-farmakologisk behandling. 
Dog skal patienter med et blodtryk, der ligger over $180$/$105$ mmHg, starte på en farmakologisk behandling for at sænke blodtrykket inden der udføres jævnligt og/eller intensiv fysisk aktivitet \citep{pedersen2016}. %Det her skal i problemanalyse

Mængden af forskellige præparater, patienten skal have, afhænger af de tidligere nævnte risikofaktorer, som lægen undersøger for, samt hvor højt blodtrykket er. Har patienten eksempelvis en mild grad af hypertension, hvor blodtrykket ligger mellem $140$ og $159$ systole eller 90 og 99 diastole, kan patienten muligvis nøjes med livsstilsændringer, hvis der ikke er nogle risikofaktorer. Er der risikofaktorer, eksempelvis diabetes bør der senere tillægges blodtryksmedicin \citep{bech2015}. \citeauthor{munck2007}, som er et projekt ved en forskningsenhed for almen praksis, har udgivet en rapport omhandlende registreringer af hypertension i 184 almene praksisser. Ifølge denne rapport var der $35~\%$ af de registrerede hypertensive patienter, der fik for lidt motion, og for lavt aktivitetsniveau var dermed den tredje hyppigste risikofaktor \citep{munck2007}.

Ifølge \cite{munck2007} er $11~\%$ af de registrerede hypertensive patienter i non-farmakologisk behandling, samtidig er det kun $1,7~\%$ af de registrerede patienter, der ikke får nogle former for farmakologisk behandling \citep{munck2007}.
 %(Jeg ved ikke hvor jeg er på vej hen… Hjælp. Er alt det her overhovedet relevant at have med? DET ER COOLT MED TAL, HVOR MANGE DER EGENTLIG BLIVER BEHANDLER ALTSÅ HVOR MANGE DER ER REGISTRERET. JA JEG SYNTES OGSÅ AT DET VIRKER MEGET FORNUFTIGT)
Anvendelse af aktivitetsarmbånd til behandling af hypertension vil høre under kategorien non-farmakologisk behandling, hvor lægen eller sygeplejersken vil få endnu en opgave, idet denne først skal oplære patienten i brug af armbåndet og siden følge op på aktivitetsniveauet. Lægen kan ud fra data om aktivitetsniveauet vejlede patienten om motion. Skulle der ske tilbagegang i behandlingen kan lægen desuden tjekke om det muligvis kunne skyldes et lavere aktivitetsniveau end normalt, eller omvendt en forbedring og om aktivitetsniveauet er steget. Det vil derfor være muligt at sammenligne et objektivt tal for aktivitetsniveau med patientens prognose. 

\subsection{Aktivitetsarmbånd i den nuværende organisation}
Ifølge Leavitts modificerede organisationsmodel er der foruden de originale elementer i Leavitts organisationsmodel også omverden. Ved omverden forstås en begrænset interessentanalyse \citep{mtvhaandbog}. Mulige interessenter for anvendelse af aktivitetsarmbånd som en del af behandling mod hypertension, er de praktisserende læger.

Ifølge \cite{munck2007} blev der udsendt spørgeskemaer til de almene praksisser, der deltog i undersøgelsen. Spørgeskemaerne blev besvaret af en kontaktperson i hver praksis. Der blev blandt andet adspurgt om, hvorvidt det ønskes, at personale involveres mere i patientbehandlingen af hypertension, hvortil $67,3~\%$ svarede ja. Ud af disse svarede omkring $49~\%$, at de ønsker øget involvering indenfor vejledning om motion og fysisk aktivitet \citep{munck2007}. Disse tal kan tyde på, at en stor anddel af de almene praksisser kan være åbne for nye muligheder eller forbedringer indenfor vejledning om fysisk aktivitet. Ved indførelse af aktivitetsarmbånd som en del af behandlingen af hypertension har personalet i den almene praksis mulighed for at blive mere involveret i patientens fysiske aktivtetsvaner og kan dermed få bedre mulighed for at tilpasse vejledningen til patienten. 

Andre mulige interessenter kan også være læger og andet personale i den sekundære sektor. Blodtrykscenteret i Region Midtjylland reklamerer blandt andet med, at forskning og implementering af nye behandlingsmetoder foregår med udgangspunkt i Blodtrykscenteret \citep{aarhusuniversitetshospital}. Foruden dette kan afdelinger på sygehusene, der varetager mange hypertensive patienter have interesse for at få indført behandlingen i de almene klinikker. Som tidligere nævnt, er der blandt andet kun mulighed for at modtage de sværeste hypertensive patienter på Neurologisk Afdeling i Aalborg, da der ikke er kapacitet til andre. Hvis antallet af svære tilfælde formindskes kan det påvirke organisationen. 
%(Jeg ved ikke hvad jeg nu skal skrive? Eller om det sidste her kan skrives, der er ikke noget rigtigt grundlag for det. Det er bare.. I dunno. SOS) 



% lodberg2008: http://www.dahs.dk/fileadmin/2008_en_faelles_klinisk_vejledningX22.pdf 
% bang2006: http://www.dahs.dk/fileadmin/BTmaaling_version-17.pdf 
% aarhusuniversitetshospital: http://www.auh.dk/om-auh/afdelinger/nyremedicinsk-afdeling/til-fagfolk/om-nyremedicinsk-afdeling/blodtrykscentret/ 
% buur2011: https://pri.rn.dk/Assets/14144/Henvisning-Nyremed.pdf 
% sundhedsstyrelsen2010: http://sundhedsstyrelsen.dk/~/media/63BE82801BB441DCAC638C81D81F8963.ashx (er der andre kilder der hedder dette?)
% munck2007: http://www.apo-danmark.dk/files/pub/1832.pdf  (patienternes svar: http://www.apo-danmark.dk/files/pub/1833.pdf ) (Fra 2010: http://www.apo-danmark.dk/files/pub/3697.pdf ) 
% promedicin2016: http://pro.medicin.dk/sygdomme/sygdom/318336 
% pedersen2016: http://pro.medicin.dk/Specielleemner/Emner/318420#a000 
% bech2015: http://www.dahs.dk/fileadmin/user_upload/2013_opdateringer/opdateringer_2014/opdateringer_2015/behandlingsvejledning_2015_juni_final.pdf 
