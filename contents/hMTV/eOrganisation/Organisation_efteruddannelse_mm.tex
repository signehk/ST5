 
\subsection{Efteruddannelse af personale}
Når en ny teknologi indføres i den almene praksis, skal lægerne eller andet sundhedspersonale i praksissen lære at anvende teknologien, hvis den skal anvendes i et behandlingsforløb. 
Ifølge forskere indenfor læring og teknologiforståelse, som arbejder med forskningsprojektet Technucation, er forståelsen for teknologi vigtig i lægernes erfaring, da det er vigtigt at kunne se og forstå, hvordan teknologier anvendes og udnyttes på bedste vis \citep{aarhusuniversitet2013}. Herfor vil eventuel efteruddannelse kunne ses som relevant for lægerne i almen praksis, hvis et nyt redskab, som aktivitetsarmbånd til objektiv monitorering af patienters fysiske aktivitet, skal implementeres for at sikre, at de har en effekt i deres arbejde og, at de ydelser de medfører for patienterne lever op til forventningerne om at kunne monitorere patienternes fysiske aktivitet objektivt. Bogen Teknologiforståelse - på Skoler og Hosptitaler, der er skrevet i samarbejde med Technucation, beskriver teknologien på forskellige områder. Blandt andet beskrives teknologien som vigtig for at mindske fejl i systemet, da der for eksempel kan undgåes ulæselig håndskrift eller lignende ved anvendelse af teknologi. Ved at implementere aktivitetsarmbånd til monitorering af patienters daglige aktivitetsniveau, vil patienternes daglige fysiske aktivitetsniveau blive registreret på en mere objektiv måde end hvis patienterne selv skal fortælle om deres fysiske aktivitet, hvilket vil være mere subjektivt og derfor mindre troværdigt. Dokumentationen for patienternes fysiske aktivitet vil derfor være mere gyldig for at få et bedre overblik over patienten, hvis der anvendes objektive målemetoder \citep{hasse2012}. 
Efteruddannelsesfonden er etableret, så læger i den almene praksis, som deltager i efteruddannelse, har en konto, hvor lægen kan trække beløb fra til fravær, transport, kursusafgift og undervisningsmateriale. Denne konto er på cirka 13000 kr. Dog skal bestemte kriterier være opfyldt for, at en læge kan få dækket udgifterne til efteruddannelse \citep{vedsted2005}. Et af disse kriterier er eksempelvis, at lægen skal arbejde efter overenskomst om almen praksis mellem Praktiserende Lægers Organisation og Regionernes lønnings- og takstnævn \citep{fondenforalmenpraksis2016}. Efteruddannelse af læger kan passende foregå i samarbejde med de uddannelsesgrupper (tolvmandsforeninger), som ofte indgår i en læges netværk. Der vil i denne forbindelse kunne afholdes foredrag af en foredragsholder relateret til aktivitetsarmbånd og brugen i disse og hvordan de fungerer, for at lægerne får indsigt og uddannelse nødvendig for at kunne anvende dem til diagnosticering eller behandling af patienter \citep{vedsted2005}. 
\\
\subsubsection{Analysering af data fra aktivitetsarmbånd}  
Data opsamlet fra aktivitetsarmbåndet skal analyseres med henblik på at se, inden for den periode patienten har gået med armbåndet, hvor meget patienten har været fysisk aktiv, og om denne fysiske aktivitet opfylder målene for at være aktiv nok.  
Med Fitbit Flex, som er yderligere beskrevet i \autoref{XXXX}, vil den data, der er opsamlet blive overført via en trådløs forbindelse, enten til smartphone eller PC, med bluetooth eller ved brug af den trådløse sync dongle, som følger med produktet. Data, der opsamles, inkluderer blandt andet antal skridt gået, hvor lang en distance dette svarer til, antallet af kalorier forbrændt samt diverse grafer, som kan give patienten et overblik over, hvor meget vedkommende er aktiv. Dataene vil da kunne ses som tal eller grafisk ved hjælp af software programmet \citep{fitbitflex}. Disse data vil kunne tastes ind i de databaser, som lægerne bruger, hvis dette er relevant for behandlingen af patienten eller som diagnosticerende middel.
* Hvad sker der med data - hvordan bliver de brugt?

\subsection{Indkøb af udstyr}	


* Patient betaler selv
* tilskud, "recept"
* Udleveret af lægen
* Udlånt 
 
Bla bla...\\
Fitbit Flex kan købes over producentens egen hjemmeside, her er det muligt at forespørge om en større ordre via deres hjemmeside. Detailhandel kan derfor foregå for eventuelt at kunne spare penge ved at handle ind i et større parti af deres produkt.

\subsection{Information og kontakt}
Problemer eller spørgsmål kan opstå enten fra læge til virksomheden, der producerer aktivitetsarmbåndet eller fra patient til læge. Herfor er det nødvendigt at kunne imødegå dette for at undgå, at der sker fejl eller misforståelser under forløbet, hvor patienten får målt deres fysiske aktivitetsniveau igennem hverdagen. 

\subsubsection{Mellem praktiserende afdeling/læge og producent}
Den praktiserende afdeling kan ved hjælp af producenten, Fitbits hjemmeside kontakte deres kundeservice på e-mail eller telefon, hvis der opstår problemer med produktet, eller hvis der er brug for at få en viden, som ikke kan findes på "på egen hånd". 

\subsubsection{Mellem patient/borger og læge}
Hvis en patient efter at have fået udleveret et aktivitetsarmbånd bliver i tvivl om brugen af denne, vil det være nødvendigt for patienten at kunne kontakte lægen for yderligere information. Ved at det er muligt for patienter at kontakte deres læge telefonisk kan mange misforståelser undgås og eventuel fejlbrug af udstyr. Typisk er det i de sene morgentimer lægerne kan kontaktes telefonisk, dog kan der forekmme ventetider alt efter, hvor mange der prøver at komme i kontakt med den pågældende læge.  (Dette skal der lige findes en kilde på og omskrives lidt :) ) \\

% kilde1 - Teknologiforståelse i sundhedsvæsnet (aarhusuniversitet2013)\\ 
% kilde2 - Almen praksis i Danmark (vedsted2005)
% En bog vi kan bruge "Teknologiforståelse på skoler og hospitaler", måske :) \\
% kilde3 - Teknologiforståelse på skoler og hospitaler (bog) (hasse2012)\\
% kilde4 - Fibit flex manual (fitbitflex)
% fondenforalmenpraksis2016: http://www.laeger.dk/portal/pls/portal/!PORTAL.wwpob_page.show?_docname=11221408.PDF

Andre kilder som er værd at kigge på \\\\

http://www.sst.dk/~/media/408300A6052D4D92BA02D486BD617613.ashx \\
http://technucation.dk/laeringsaktivteter/laeringsaktivitet-ny-teknologi/ \\
http://www.sst.dk/~/media/408300A6052D4D92BA02D486BD617613.ashx \\
http://lvvl.dk/file/241981/vv12.pdf \\

http://videnskab.dk/kultur-samfund/ny-teknologi-kan-skade-patienter-og-skoleelever-mere-end-den-gavner

