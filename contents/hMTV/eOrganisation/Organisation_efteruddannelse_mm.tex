\subsection{Efteruddannelse af personale}
\label{sec:efteruddannelse}
Når en ny teknologi indføres i almen praksis, skal lægerne eller andet sundhedspersonale i praksissen lære at anvende teknologien, hvis den skal anvendes i et behandlingsforløb.
 
Ifølge forskere inden for læring og teknologiforståelse er forståelse for og erfaring med teknologien vigtig, for at se og forstå, hvordan teknologien kan anvendes og udnyttes på bedste vis \citep{aarhusuniversitet2013}. 
Af denne grund vil efteruddannelse være relevant for læger i almen praksis, hvis et nyt redskab til objektiv monitorering af patienters fysiske aktivitet, skal implementeres, for at sikre, at Fitbit Flex har en effekt i lægernes arbejde med behandling af hypertension. På denne måde vil de få et optimalt udbytte af teknologien i forhold til at monitorere patienternes fysiske aktivitet objektivt. Bogen \citetalias{hasse2012} beskriver teknologi i sundhedsvæsenet på forskellige områder. Blandt andet beskrives teknologi som vigtig for at mindske fejl i systemet, da der for eksempel kan undgåes ulæselig håndskrift eller lignende. Ved at implementere Fitbit Flex til monitorering af patienters daglige aktivitetsniveau, vil patienternes daglige fysiske aktivitetsniveau blive registreret på en mere objektiv måde, end hvis patienterne selv skal fortælle om deres fysiske aktivitet, hvilket er beskrevet i \autoref{NuMetode}. Dokumentationen for patienternes fysiske aktivitet vil derfor være mere gyldig for at få et overblik over patientens aktivitetsniveau, hvis der anvendes objektive målemetoder \citep{hasse2012}. 

Efteruddannelsesfonden er etableret, så læger i almen praksis, som deltager i efteruddannelse, har en konto, hvor lægen kan dække udgifter i forhold til fravær fra praksis, transport, kursusafgift og undervisningsmateriale. Denne konto er på cirka $13.000$ kroner årligt. Bestemte kriterier skal være opfyldt for, at en læge kan få dækket udgifterne til efteruddannelse \citep{vedsted2005}. Ét af disse kriterier er eksempelvis, at lægen skal arbejde efter overenskomst om almen praksis mellem Praktiserende lægers organisation (PLO) og Regionernes lønnings- og takstnævn (RLTN) \citep{fondenforalmenpraksis2016}. 

Efteruddannelse af læger kan foregå i samarbejde med de uddannelsesgrupper (tolvmandsforeninger), som ofte indgår i en læges netværk. Der vil i denne forbindelse kunne afholdes oplæg af en foredragsholder relateret til aktivitetsarmbånd og brugen i disse og hvordan de fungerer, for at lægerne får indsigt og uddannelse nødvendig for at kunne anvende dem til diagnosticering eller behandling af patienter \citep{vedsted2005}. 

\subsection{Analysering af data fra aktivitetsarmbånd}  
Data, opsamlet fra Fitbit Flex, skal analyseres med henblik på at se, hvor fysisk aktiv patienten har været, og om denne fysiske aktivitet opfylder målene, som lægen sætter for patienten i forhold til ønsket aktivitetsniveau. 

Med Fitbit Flex vil data, der er opsamlet, blive overført ved at synkronisere enheden via en trådløs forbindelse, enten til smartphone med BLE eller til PC ved brug af den trådløse synkroniserings dongle, som følger med produktet. Denne synkronisering kan patienten foretage i eget hjem, hvorefter lægen vil kunne se patientens data  ved at tilgå patientens Fitbit brugerkonto. Dataene vil da kunne ses som tal eller grafisk ved hjælp af softwareprogrammet som hører til Fitbit \citep{fitbitflex}. 

Disse data vil kunne tastes ind i patientjournalen for den enkelte lægepraksis, hvis dette er relevant for behandlingen af patienten. Dette vil enten betyde, at allerede anvendte programmer i den almene praksis skal tilpasses de nye data, eller at den allerede eksisterende platform fra Fitbit skal anvendes, hvorved der vil skulle implementeres et nyt program i den almene praksis, hvor aktivitetsarmbåndet anvendes. 

\subsection{Indkøb af udstyr}	

Hvis Fitbit Flex implementeres som en del af behandlingen af hypertension, skal det besluttes hvem, der har ansvar for anskaffelsen af teknologien. Sundhedsvæsenet kan eksempelvis stå for indkøb af udstyr for at undgå, at patienten selv skal belastes økonomisk. Aktivitetsarmbåndet vil også kunne udleveres i en periode som en monitorering op til en kontrolbesøg hos egen læge. Dette kan foregå ligesom ved udlevering af apparat til hjemmeblodtryksmåling, hvor patienten modtager klare instruktioner fra læge eller sygeplejerske inden brug og kan låne udstyret med hjem.

Fitbit Flex kan købes over producentens egen hjemmeside, her er det muligt at forespørge om en større ordre via hjemmesiden. Detailhandel kan derfor foregå for eventuelt at kunne spare penge ved at handle ind i et større parti af deres produkt.

En anden mulig implementeringsmetode er, at patienten selv står for anskaffelse af aktivitetsarmbåndet med en rabat, hvis lægen anbefaler Fitbit Flex som et led i behandlingen.  

\subsection{Kontakt og information}

Problemer eller spørgsmål kan opstå fra læge til producenten, Fitbit, eller fra patient til læge. Af denne grund er det nødvendigt at kunne besvare disse spørgsmål for at undgå, at der sker fejl eller misforståelser ved brugen af Fitbit Flex, så både patient og læge kan få mest muligt ud af brugen af aktivitetsarmbåndet. 

\subsubsection{Mellem praktiserende afdeling/læge og producent}

Den praktiserende afdeling kan ved hjælp af producenten, Fitbits hjemmeside kontakte deres kundeservice på e-mail eller telefon, hvis der opstår problemer med produktet, eller hvis der er brug for at få en viden, som ikke kan findes "på egen hånd". 

\subsubsection{Mellem patient/borger og læge}

Hvis en patient efter at have fået udleveret et Fitbit Flex-aktivitetsarmbånd bliver i tvivl om brugen af dette, ud over brugsvejledningens indhold, vil det muligvis være nødvendigt for patienten at kunne kontakte lægen eller andet personale, med viden omkring teknologien, for yderligere information og vejledning. Dette kan gøres telefonisk eller via mail, så misforståelser og eventuel fejlbrug af udstyret kan undgås. 

% kilde1 - Teknologiforståelse i sundhedsvæsnet (aarhusuniversitet2013)\\ 
% kilde2 - Almen praksis i Danmark (vedsted2005)
% En bog vi kan bruge "Teknologiforståelse på skoler og hospitaler", måske :) \\
% kilde3 - Teknologiforståelse på skoler og hospitaler (bog) (hasse2012)\\
% kilde4 - Fibit flex manual (fitbitflex)
% fondenforalmenpraksis2016: http://www.laeger.dk/portal/pls/portal/!PORTAL.wwpob_page.show?_docname=11221408.PDF

%Andre kilder som er værd at kigge på \\\\

%http://www.sst.dk/~/media/408300A6052D4D92BA02D486BD617613.ashx \\
%http://technucation.dk/laeringsaktivteter/laeringsaktivitet-ny-teknologi/ \\
%http://www.sst.dk/~/media/408300A6052D4D92BA02D486BD617613.ashx \\
%http://lvvl.dk/file/241981/vv12.pdf \\

%http://videnskab.dk/kultur-samfund/ny-teknologi-kan-skade-patienter-og-skoleelever-mere-end-den-gavner

