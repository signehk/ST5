
\subsection{Efteruddannelse af personale}
Bla bla..\\
Ifølge forskere indenfor læring og teknologiforståelse, som arbejder med forskningsprojektet Technucation, er forståelsen for teknologi vigtig i lægernes erfaring, da det er vigtigt at kunne se og forstå hvordan teknologier anvendes og udnyttes på bedste vis[kilde1]. Herfor vil eventuel efteruddannelse kunne ses som relevant for lægerne i almen praksis, hvis et redskab som aktivitetsarmbånd til objektiv monitorering af patienters fysiske aktivitet skal implementeres, for at de er sikre i deres arbejde og de ydelser de udfører for borgerne/patienterne. 
Efteruddannelse af læger kan passende foregå i samarbejde med de uddannelsesgrupper (toldmandsforeninger) som ofte indgår i en læges netværk. Der vil i denne forbindelse kunne afholdes foredrag af en foredragsholder relateret til aktivitetsarmbånd og brugen i disse og hvordan de fungere, for at lægerne får indsigt og uddannelse nødvendig for at kunne anvende dem til diagnosticering eller behandling af patienter[kilde2]. 
\\
\subsubsection{Analysering af data fra aktivitetsarmbånd}  
Data opsamlet fra aktivitetsarmbåndet skal analyseres med henblik på at se, inden for den periode patienten har gået med armbåndet, hvor meget patienten har været fysisk aktiv, og om denne fysiske aktivitet opfylder målene for at være aktiv nok.  
Afhængig af enheden der anvendes vil den data der er opsamlet blive overført via en trådløs forbindelse, enten bluetooth eller internet, til fitbits software der er kompatibelt med enheden. Denne software kan bruges både på smartphones og PC... Softwaren er beskrevet i teknologiafnittet bla bla.. noget noget,(referance dertil), disse data inkluderer skridt gået, hvor lang en distance dette svarer til, antallet af kalorier forbrændt, diverse grafer som også giver patienten mulighed for let at få et overblik over hvor meget vedkommende er aktiv... mm.. se reference teknologi afsnit. (MÅSKE HENTE APPEN SELV SÅ VI KAN SE HVAD MAN KAN I DEN?). Dataene vil da kunne ses som tal eller grafisk ved hjælp af software programmet. Disse data vil kunne tastes ind i de databaser som lægerne selv bruger hvis dette er relevant for behandlingen af patienten eller som diagnosticerende middel. [kilde i forhold til aktivitetsarmbåndet]

\subsection{Indkøb af udstyr}	
Bla bla...\\
Fitbit flex kan købes over producentens egen hjemmeside, her er det muligt at forespørge om en større ordre via deres hjemmeside, detailhandel kan derfor foregå for eventuelt at kunne spare penge ved at handle ind i et større parti af deres produkt.

\subsection{Information og kontakt}
Problemer eller spørgsmål kan opstå enten fra læge til virksomheden der producere aktivitetsarmbåndet eller fra borger til læge. Herfor er det nødvendigt at kunne imødegå dette   for at undgå at der sker fejl eller misforståelser under forløbet hvor patienten får målt deres fysiske aktivitets livsstil igennem hverdagen. 

\subsubsection{Mellem praktiserende afdeling/læge og producent}
Den praktiserende afdeling kan ved hjælp af producentens/fitbits hjemmeside kontakte deres kundeservice på e-mail eller telefon hvis der opstår problemer med produkter eller der er brug for at få en viden som ikke kan findes "på egen hånd". 

\subsubsection{Mellem patient/borger og læge}
Hvis en borger er efter at have fået udleveret en aktivitetstracker bliver i tvivl om brugen af denne vil det være nødvendigt for denne at kunne kontakte lægen for yderligere information. Ved at det er muligt for borgere at kontakte deres læge telefonisk kan mange misforståelser undgås og evt. fejlbrug af udstyr. Typisk er det i de sene morgentimer lægerne kan kontaktes telefonisk, dog kan der forekmme ventetider alt efter hvor mange der prøver at komme i kontakt med den pågældende læge.  (Dette skal der lige findes en kilde på og omskrives lidt :) ) \\

kilde1 - Teknologiforståelse i sundhedsvæsnet \\
kilde2 - Almen praksis i Danmark
En bog vi kan bruge "Teknologiforståelse på skoler og hospitaler", måske :) 
Andre kilder som er værd at kigge på \\\\

http://www.sst.dk/~/media/408300A6052D4D92BA02D486BD617613.ashx \\
http://technucation.dk/laeringsaktivteter/laeringsaktivitet-ny-teknologi/ \\
http://www.sst.dk/~/media/408300A6052D4D92BA02D486BD617613.ashx \\
http://lvvl.dk/file/241981/vv12.pdf \\

http://videnskab.dk/kultur-samfund/ny-teknologi-kan-skade-patienter-og-skoleelever-mere-end-den-gavner
