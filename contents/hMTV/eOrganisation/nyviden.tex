\subsection{Organisatoriske ændringer}

For at anvende FitBit flex i en medicinsk sammenhæng skal både patient og læge have viden om denne. Det essentielle er at både patient og læge skal være indforstået i hvordan armbåndet bruges som en dokumentation på den udførte aktivitet af patienten. 


Hertil skal lægen også vide hvordan motion kan have en virkning på blodtrykket, og hvad det vil betyde for patienten hvis han/hun bevæger sig mindre end eller det anbefalede. Sammenhæng mellem aktivitet og blodtryk er beskrevet i afsnit \ref{sec:effekterafaktivitet}.


Lægen skal yderligere kunne vurdere hvilke hypertensive patienter, der er egnet til at få et armbånd, derfor skal lægen vide hvilke patient kriterier, der findes for teknologien. I forlængelse med dette skal teknologien også tilpasses den enkelte patient, hvilket vil betyde at lægen yderligere skal have færdigheder i brugertilpasning, som kan læses om i afsnit \ref{sec:brugertilpasning}. 


Derudover skal lægen ved udlevering af Fitbit Flex, kunne forklare teknologiens forskellige dele samt funktionerne af disse til patienten. Dette kan læses i afsnit \ref{sec:teknologibeskrivelse}. Det registrerede data skal efterfølgende også kunne tolkes af lægen, hvorfor det relevant at lægen ved hvordan teknologiens brugerinterfacet fungerer. For at lægen kan se patients udvikling er det nødvendigt at lægen kan introducere patient til selv synkronisering af data, således patienten kan gemme og dermed opbevarer data over en længere periode. 