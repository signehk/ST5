\section{Organisatoriske ændringer}

For at anvende FitBit Flex i en medicinsk sammenhæng, skal både patient og læge have viden om teknologien. Det essentielle er at både patient og læge skal være indforstået med, hvordan armbåndet bruges som et middel til dokumentation af patientens aktivitetsmønster.

Hertil skal lægen vide hvordan motion kan have en virkning på blodtrykket, og hvad det vil betyde for patienten hvis han/hun ikke er tilstrækkeligt aktiv, i forhold til Sundhedsstyrelsens anbefalinger. Sammenhæng mellem aktivitet og blodtryk er beskrevet i afsnit \ref{sec:effekterafaktivitet}.

Lægen skal yderligere kunne vurdere hvilke hypertensive patienter, der er egnet til at få et armbånd, hvorfor der i \autoref{sec:kriterier} opstilles forslag til patient-kriterier for tildeling af et Fitbit Flex. I forlængelse med dette skal teknologien også tilpasses den enkelte patient, hvilket vil betyde at lægen yderligere skal have færdigheder i brugertilpasning, som kan læses om i afsnit \ref{sec:brugertilpasning}. 

Derudover skal lægen ved udlevering af Fitbit Flex, kunne instruere patienten i brugen af armbåndet, samt de tilhørende dele og programmer, således patienten er i stand til at anvende armbåndets forskellige funktioner på egen hånd. Dette kan læses i afsnit \ref{sec:teknologibeskrivelse}. Det registrerede data skal efterfølgende også kunne tolkes af lægen, og det er af den grund igen relevant at lægen har indblik i brugerinterfacets funktioner. For at lægen kan se patients udvikling er det nødvendigt at lægen kan oplære patienten i synkronisering af data, således patienten kan gemme, opbevare og uploade data over en længere periode.