%%%%%%%%%%%%%%%%%%%%%%%%%%%%%%%%%%%%%%%%%%%%%%%%%%%%%%%%
%\chapter{Økonomi}
%Dette afsnit omhandler det økonomiske aspekt ift. MTV analysen.
\chapter{Økonomi}

\section{Metode}
I økonomianalysen undersøges hvilke omkostninger der er og kan forekomme forbundet ved anvendelse og implementering af aktivitetstracker som dokumenteringsenhed for aktivitet i den almene praksis/medicin.
Ligeledes undersøges omkostninger for nuværende anvendelsesmetoder, samt hvilke økonomiske konsekvenser der forekommer når patienten ikke opretholder anbefalet aktivitetskvote.
Dette er med henblik på at fremhæve sundhedsgevinsterne i forhold til udgifterne.   
Omkostningerne og konsekvenser er opgjort af sundhedsøkonomiske analyser, som cost-effectiveness analyse (CEA), cost-utility analyse (CUA) og cost-benefit analyse (CBA), og oplyses i henholdsvis narturlige enheder (f.eks. vunde leveår), kvalitetsjusterede leveår og kroner øre. 
De estimerede værdier fra de forskellige analyser er baseret på eksisterende litteratur samt basale økonomiske udregninger.
Dette giver anledning til følgende MTV-spørgsmål: 

\subsection{MTV-spørgsmål}
 
\begin{itemize}
\item Hvad er omkostningerne ved nuværende monitoreringsmetode, samt konsekvenserne ved utilstrækkelig aktivitetsydelse? 

\item Hvilke omkostninger er forbundet med brug af Fitbit Flex til patienter med hypertension, og hvad er den økonomiske konsekvens af dette, hvis brug af aktivitetsarmbånd resulterer i et øget antal kvalitetsjusterede leveår?


\end{itemize}
 



