\chapter{Økonomi}

\section{Metode}
I økonomianalysen undersøges, hvilke omkostninger der er og kan forekomme forbundet ved anvendelse og implementering af Fitbit Flex som dokumenteringsenhed for aktivitet i almen praksis.
Ligeledes undersøges omkostninger for den nuværende metode til aktivitetsmåling, samt hvilke økonomiske konsekvenser, der forekommer når patienten ikke opretholder anbefalet aktivitetskvote.
Dette er med henblik på at undersøge sundhedsgevinsterne ved øget fokus på fysisk aktivitet i forhold til udgifterne til Fitbit Flex uden at lave en egentligt cost-utitily analyse.   
%Omkostningerne og konsekvenser er opgjort af sundhedsøkonomiske analyser, som cost-effectiveness analyse (CEA), cost-utility analyse (CUA) og cost-benefit analyse (CBA), og oplyses i henholdsvis narturlige enheder (f.eks. vunde leveår), kvalitetsjusterede leveår og kroner øre. 
De estimerede værdier i dette afsnit er baseret på litteratur og udregninger ud fra denne litteratur.
Dette giver anledning til følgende MTV-spørgsmål: 

\subsection{MTV-spørgsmål}
 
\begin{itemize}
\item Hvad er omkostningerne ved nuværende anvendelsesmetoder, samt konsekvenserne ved utilstrækkelig aktivitetsydelse? 

\item Hvilke omkostninger er forbundet med brug af aktivitetsarmbånd til patienter med hypertension, og hvad er den økonomiske konsekvens af dette, hvis brug af aktivitetsarmbånd resulterer i færre udgifter i sundhedssektoren i forbindelse med hypertension?

\end{itemize}
 



