%%%%%%%%%%%%%%%%%%%%%%%%%%%%%%%%%%%%%%%%%%%%%%%%%%%%%%%%
%\chapter{Økonomi}
%Dette afsnit omhandler det økonomiske aspekt ift. MTV analysen.
\chapter{Økonomi}
\section{Spørgsmål}
%Eksempler
%\begin{itemize}
%\item Hvad vil forskellige modeller for vaccinationsprogram have af nytteeffekten i forhold til omkostninger?

%\item Hvordan ville en eventuel screening påvirke organisering og økonomi? 

%\item Hvad er de ressourcemæssige konsekvenser?
%\end{itemize}

\noindent
Økonomisk relateret MTV-spørgsmål:  
\begin{itemize}
\item Hvilke økonomiske ændringer er der forbundet ved udlevering af aktivitetsmålere til patienter med ???.

\item Hvordan ville et aktivitetsarmbånd påvirke organisering og økonomi?

\item Hvilke omkostninger er forbundet med kvantificering af patientaktivitet, i forhold til nuværende anvendelsesmetoder (Aktivitetslog)?.  

\item Hvad er omkostningerne ved nuværende anvendelsesmetoder, samt konsekvenserne ved utilstrækkelig aktivitetsydelse? 
\end{itemize}

\section{Metode}
% Hvad koster den patientgruppe for samfundet? Hvad koster det, hvis de ikke passer deres sygdom ordentlig (i forhold til motion), og dermed får følgevirkninger (fx hospitalsophold, medicin)? (Cost/Benefit analyse??)
% Cost-effectiveness analysen (konsekvenserne måles i naturlige enheder)
%Cost-utility analysen (konsekvenserne måles i kvalitetsjusterede leveår (QALYs))
% Cost-benefit analysen (konsekvenserne opgøres i kroner og øre)
% Hvordan er omkostningerne sammenlignet med alternativerne?
% I forlængelse af, hvad aktivitetstrackeren skal kunne: Er de billige tilstrækkelige, eller er det nødvendigt at købe de dyre?
% Brugerbetaling eller egenbetaling??

I økonomianalysen undersøges hvilke omkostninger der er forbundet med anvendelse af aktivitetsmåler som dokumenteringsenhed for aktivitet i den almene praksis/medicin.
Ligeledes undersøges omkostninger for nuværende anvendelsesmetoder, samt hvilke økonomiske konsekvenser der forekommer når patienten ikke opretholder anbefalet aktivitetskvote.
Dette er med henblik på at fremhæve sundhedsgevinsterne i forhold til udgifterne.   
Omkostningerne og konsekvenser er opgjort af sundhedsøkonomiske analyser, som cost-effectiveness analyse (CEA), cost-utility analyse (CUA) og cost-benefit analyse (CBA), og oplyses i henholdsvis narturlige enheder (f.eks. vunde leveår), kvalitetsjusterede leveår og korner øre. 
De estimeret værdier fra de forskellige analyser er baseret på eksisterne litteratur fundet ved anvendelse af søgeprotokol, samt basale økonomiske udregninger.    

 



