\section{Delkonklusion}

Indledningsvist vil implementeringen af Fitbit Flex i den almene praksis give en udgift i forhold til de anvendte metoder relateret til aktivitetsregistrering. Udsigterne for eventuel besparelse på hypertensionsområdet vil være udtrykt ved en reduktion i brugen af medicin, samt udskydelse af følgesygdomme, såfremt der opnås en positiv effekt ved implementeringen af aktivitetsarmbåndet. Omkostningerne ved en implementering vil give sig til udtryk i form af direkte omkostninger til indkøb, efteruddannelse, samt vedligeholdelse af Fitbit Flex. Anvendelse af Fitbit Flex antages at være tilsvarende udgifter som ved anvendelse af hjemmeblodtryksmonitorering. 

Såfremt Fitbit Flex har en positiv virkning på aktivitetsniveauet hos hypertensive patienter, vil der potentielt forekomme besparelser i den sekundære sundhedssektor i form af færre ambulante besøg og indlæggelser på hospitalet.

Ligeledes vil kunne forekomme besparelser i den sekundære sundhedssektor, samt i forbindelse med medicinudgifterne ved en reduktion i DDD. Hertil kan en testperiode med Fitibit Flex i den almene praksis være værd at overveje for at sikre, om teknologien har den tiltænkte effekt på hypertensive patienter.  