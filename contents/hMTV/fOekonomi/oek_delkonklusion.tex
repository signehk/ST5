\section{Delkonklusion}
%Indhold: Dette afsnit vil indeholde en delkonklusion af denne del af MTV'en og dette kapitel, som forhåbentligt vil lede frem til en endelig konklusion i syntesen. 

%En implementering af Fitbit Flex i den primære sundhedssektor vil "til at starte med" udgøre en omkosning. 

Indledningsvist vil implementeringen af Fitbit Flex i den almene praksis give en udgift i forhold til de anvendte metoder relateret til aktivitetsregistrering. Udsigterne for eventuel besparelse på hypertensionområdet vil være udtrykt ved en reduktion i brugen af medicin, samt udskydelse af følgesygdomme, såfremt der opnås en positiv effekt ved implementeringen af armbåndet. Omkostningerne ved en implementering vil give sig til udtryk i form af direkte omkostninger til indkøb, efteruddannelse, samt vedligeholdelse af Fitbit Flex. Ved anvendelse af Fitbit Flex antages at være tilsvarende udgifter som ved anvendelse af hjemmeblodtryksmonitorering. 

Såfrem Fitbit Flex har en positiv virkning på aktivitetsniveauet hos hypertensive patienter, vil der potentielt forekomme besparelser i den sekundære sundhedssektor i form af færre ambulante besøg og indlæggelser på hospitalet.

Ud fra økonomianalysen antages der at være potentiale til besparelser i sundhedsektoren. Disse besparelser vil forekomme i den sekundære sundhedssektor, samt i forbindelse med medicinudgifterne ved en reduktion i DDD. Hertil kan en testperiode med Fitibit Flex i den almene praksis være værd at overveje, for at få indblik i om teknologien har den tiltænkte effekt på hypertensive patienter.  