\subsection{Omkostninger i sundhedssektoren}
Ved en implementering af Fitbit Flex vil dette medføre ændringer i strukturen af både den primære og sekundære sundhedssektor, der er i forbindelse med behandling af hypertension, som nævnt i \autoref{sec:org_aendringer}, hvor nogle mulige organisatoriske ændringer er beskrevet. Med de organisatoriske ændringer, vil der ligeledes følge ændringer i udgifter til sundhedssektoren. 
Forventningen ved en implementering af Fitbit Flex er, at den vil medføre færre kostelige henvendelser til det offentlige. 
I den primære sektor, vil dette kunne ske, hvis antallet af konsultationer falder. Dette kan eksempelvis ske, hvis nogle fysiske kontroller erstattes med telefoniske konsultationer, da honoraret for en telefonisk konsultation er 110 kroner lavere end en fysisk konsultation. 
 
Ligeledes vil der muligvis kunne være økonomiske besparelser, hvis en øget mængde fysisk aktivitet kan reducere mængden af henvendelser til den sekundære sundhedssektor. Et ambulant besøg er 1.283 kroner dyrere end en konsultation, eller 1.394 kroner dyrere end en telefonisk konsultation ved egen læge. Hvis antallet af hospitalsindlæggelser endvidere kan reduceres, vil udgifterne dertil kunne falde med 12.597 kroner per dag. Regionerne har også udgifter i form af tilskud til medicin, som beskrevet i \autoref{sec:nuv_primaer}. Ved en implementering af Fitbit Flex, vil nogle patienter muligvis kunne udskyde, eller måske helt undgå, brugen af antihypertensiva, hvilket derigennem vil kunne spare regionerne penge på sigt. På denne måde, vil der kunne følge besparelser med en implementering af Fitbit Flex, der muligvis vil kunne udligne udgifterne til implementeringen af aktivitetsarmbåndet, hvis Fitbit Flex som en del af behandlingen til hypertension har en betydelig effekt. 
