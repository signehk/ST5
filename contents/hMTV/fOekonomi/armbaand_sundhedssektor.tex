\subsection{Omkostninger i sundhedssektoren}
% har kombineret primær og sekundær sundhedssektor - gav ikke mening at adskille.. 

%Indhold: I dette afsnit vil vi gerne analysere og belyse mulige udgifter/besparelser ift. et muligt antal (reducerede) henvendelser til almen praksis ved benyttelse af ny teknologi, hvis dette er muligt at estimere. Herunder vil vi se på, om man vil kunne "nøjes" med telefonsamtaler eller emails for nogle af kontrollerne i stedet for egentlige konsultationer. 

%Indhold: I dette afsnit vil vi gerne analysere og belyse udgifter/besparelser ift. et muligt antal (reducerede) hospitalsindlæggelser ved benyttelse af ny teknologi, hvis dette er muligt at estimere.

Ved en implementering af Fitbit Flex, vil dette medføre ændringer i strukturen af den primære og sekundære sundhedssektor, som nævnt i \autoref{sec:org_aendringer}, hvor nogle mulige organisatoriske ændringer er beskrevet. Med de organisatoriske ændringer, vil der ligeledes følge ændringer i udgifter til sundhedssektoren. 
Håbet ved en implementering af Fitbit Flex er, at den vil medføre færre, økonomisk tunge, henvendelser til det offentlige. 

I den primære sektor, vil dette kunne ske, hvis antallet af blodtrykskontroller faldt. Dette kunne eksempelvis ske, hvis nogle fysiske kontroller erstattedes med telefoniske konsultationer, da honoraret for en telefonisk konsultation er 110 kroner lavere end en fysisk konsultation. Ud over denne økonomiske gevinst, skal der tages højde for, at der vil kunne være et øget antal henvendelser til egen læge i forbindelse med vurdering af patientens egnethed til brug af Fitbit Flex, udlevering og instruktioner i brugen af aktivitetsarmbåndet. 
 
Ligeledes vil der muligvis kunne være økonomiske besparelser, hvis en øget mængde fysisk aktivitet vil kunne reducere mængden af henvendelser til den sekundære sundhedssektor. Dette er muligt, da et ambulant besøg er 1.283 kroner dyrere end en konsultation, eller 1.394 kroner dyrere end en telefonisk konsultation ved egen læge. Hvis antallet af hospitalsindlæggelser endvidere kan mindskes, vil udgifterne dertil falde med 12.597 kroner per dag, det ikke er nødvendigt ikke at indlægge en patient. 

Regionerne

På denne måde, vil der kunne følge besparelser med en implementering af Fitbit Flex, der muligvis vil kunne udligne udgifterne til implementeringen af aktivitetsarmbåndet, som nævnt i \autoref{sec:armbaand_omkost}. 

Økonomi-afsnittet bliver nok primært spekulativt, da der er mange faktorer, som kan være voldsomt svære at estimere, men jeg synes, der er nogle fine betragtninger med. Tænk specielt over, at det godt kan være, at man har øgede omkostninger forbundet med armbåndet i primærsektoren, men at det man sparer i sekundærsektoren og i medicinudgifter opvejer dette på samfundsniveau. Det kan jo også være, at konklusionen simpelthen er, at det er for dyrt i forhold til gevinsten. En hård økonomisk konklusion kan nok være svær at nå til.