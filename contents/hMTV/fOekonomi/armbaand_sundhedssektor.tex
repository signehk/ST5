\subsection{Omkostninger i sundhedssektoren}
% har kombineret primær og sekundær sundhedssektor - gav ikke mening at adskille.. 

%Indhold: I dette afsnit vil vi gerne analysere og belyse mulige udgifter/besparelser ift. et muligt antal (reducerede) henvendelser til almen praksis ved benyttelse af ny teknologi, hvis dette er muligt at estimere. Herunder vil vi se på, om man vil kunne "nøjes" med telefonsamtaler eller emails for nogle af kontrollerne i stedet for egentlige konsultationer. 

%Indhold: I dette afsnit vil vi gerne analysere og belyse udgifter/besparelser ift. et muligt antal (reducerede) hospitalsindlæggelser ved benyttelse af ny teknologi, hvis dette er muligt at estimere.

Ved en implementering af Fitbit Flex, vil dette medføre ændringer i strukturen af den primære og sekundære sundhedssektor, som nævnt i \autoref{sec:org_aendringer}, hvor nogle mulige organisatoriske ændringer blev beskrevet. Med de organisatoriske ændringer, vil der ligeledes følge ændringer i udgifter til sundhedssektoren. 

Meningen med implementering af Fitbit Flex er, at det vil medføre færre, økonomisk tunge, henvendelser til det offentlige. I den primære sektor, vil dette kunne ske, hvis antallet af blodtrykskontroller faldt, eksempelvis hvis nogle fysiske kontroller erstattedes med telefoniske konsultationer, da honoraret for en telefonisk konsultation er 110 kroner lavere end en fysisk konsultation. Ligeledes vil der også kunne være økonomiske besparelser, ved ændringer i den sekundære sektor, da ambulante besøg er eksempelvis 1.394 kroner dyrere end en telefonisk konsultation. 

Hvis antallet af indlæggelser endvidere kan mindskes, vil udgifterne dertil falde med 12.597 kroner per dag. På denne måde, vil der kunne følge besparelser med en implementering af Fitbit Flex, der muligvis vil kunne udligne udgifterne til implementeringen af aktivitetsarmbåndet, som nævnt i \autoref{sec:armbaand_omkost}. 