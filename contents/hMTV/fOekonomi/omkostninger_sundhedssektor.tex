\section{Omkostninger i sundhedssektoren}

\subsection{Primær sektor}
Implementering af Fitbit Flex til patienter, som lider af hypertension, vil medføre og/eller ændre de økonomiske udgifter i både den primære og sekundære sundhedssektor. I den primære sundhedssektor, som består af de praktiserende læger, er det relevant at se på udgifter/omkostninger i forbindelse med anvendte målemetoder samt nuværende antal henvendelser til almen praksis. Derudover skal teknologien købes ind, så patienterne kan anvende den, hertil skal det overvejes, om teknologien skal være betalt af det offentlige, eller om den skal være brugerbetalt.

Den subjektive målemetode, der på nuværende tidspunkt benyttes af $27,7~\%$ af praktiserende læger, foregår ved spørgeskema under en konsultation, medfører udgifter i den primære sundhedssektor \citep{munck2007}. Afhængigt af antal konsultationer, som den enkelte kronikere har behov for, kan et spørgeskema følge med hver konsultation, og omkostningerne til denne målemetode vil derved stige. Udarbejdelse og udskrifter af et spørgeskema vil have relativt lave omkostninger.
Audit Projekt Odense udarbejdede i 2007 en rapport om hypertension i almen praksis. Her blev 159 praksis blandt andet spurgt følgende spørgsmål "Sætter I jeres hypertensionspatienter til kontrol med fast tidsinterval? Hvis ja, angiv det typiske interval" hvortil 92,5 \% af disse gjorde dette. Yderligere svarede 56,7 \%, at det typisk blev gjort med et interval på 3 måneder \citep{munck2007}. 

Hvis det, jævnfør \citeauthor{kronborg2008}, antages at $1/5$ af den voksne danske befolkning har hypertension, vil dette svare til omkring 900.000 danskere \citep{folketal2016}.

Statistik årbogen 2016 udarbejdet af Danmarks Statistik viser at den samlede medicinudgift i den primære sundhedssektor i Danmark i 2014 lå på 11,6 mia kroner. Årbogen har i denne sammenhæng påpeget at blodtrykssænkende medicin og hjertemedicin er nogle af de mest anvendte former for medicin i Danmark \citep{dst2016}. 

\subsection{Sekundær sektor}
Tal fra landspatientregistret viser, at der i 2012 var 310 ambulante besøg i forbindelse med blodtryksforhøjelse af ukendt årsag i den private sektor. Dette tal steg i 2013 til 639 ambulante besøg. I 2014 steg tallet yderligere til 1186 i 2014, hvilket vil betyde en stigning på cirka 85,6 \% \citep{sundhedsdatastyrelsen2016} . Med denne stigning vil udgifterne for antal henvendelser i almen praksis dermed også stige proportionelt. 
