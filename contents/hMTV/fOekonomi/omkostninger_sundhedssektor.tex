\section{Omkostninger i sundhedssektoren}
Det er relevant at se på omkostningerne i sundhedssektorens primære og sekundære sektor ved brug af den nuværende målemetode. 

\subsection{Primær sektor}
Den subjektive målemetode, der på nuværende tidspunkt benyttes af $27,7~\%$ af praktiserende læger, foregår ved spørgeskema under en konsultation, medfører udgifter i den primære sundhedssektor \citep{munck2007}. Afhængigt af antal konsultationer, som den enkelte kronikere har behov for, kan et spørgeskema følge med hver konsultation, og omkostningerne til denne målemetode vil derved stige. Udarbejdelse og udskrifter af et spørgeskema vil have relativt lave omkostninger.
\citeauthor{munck2007} udarbejdede i 2007 en rapport om hypertension i almen praksis. Her blev 159 kontaktpersoner i almen praksis spurgt: "Sætter I jeres hypertensionspatienter til kontrol med fast tidsinterval? Hvis ja, angiv det typiske interval". Her svarede 92,5 \%, at de sætter deres patienter til kontrol med et fast tidsinterval, og i gennemsnit er dette interval udregnet til 3,9 måneder \citep{munck2007}. 

Hvis det, jævnfør \citeauthor{kronborg2008}, antages at $1/5$ af den voksne danske befolkning har hypertension, vil dette svare til omkring 900.000 danskere \citep{folketal2016}. Hvis 900.000 danskere skal til lægekonsultation á 137,93 kr hver 3,9. måned, vil dette svare til en årsomkostning for sundhedssektoren på omkring 380 millioner kr. 

Den samlede medicinudgift i den primære sundhedssektor i Danmark i 2014 lå på 11,6 mia kroner, og Danmarks Statistik påpeger i denne sammenhæng, at blodtrykssænkende medicin og hjertemedicin er nogle af de mest anvendte former for medicin i Danmark \citep{dst2016}. 

\subsection{Sekundær sektor}
Tal fra landspatientregistret viser, at der i 2012 var 310 ambulante besøg i forbindelse med blodtryksforhøjelse af ukendt årsag i den private sektor. Dette tal steg i 2013 til 639 ambulante besøg. I 2014 steg tallet yderligere til 1186 i 2014, hvilket vil betyde en stigning på cirka 85,6 \% \citep{sundhedsdatastyrelsen2016} . Med denne stigning vil udgifterne for antal henvendelser i almen praksis dermed også stige proportionelt. 
