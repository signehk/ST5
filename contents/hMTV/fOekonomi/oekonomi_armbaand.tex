
\section{Omkostninger ved implementering af Fitbit Flex}

%Indledning 
\textbf{Denne tekst skulle forestille en indledende tekst til de afsnit som er efterfølgende, men det skal lige tilpasses lidt} 

Implementering af Fitbit Flex til patienter, som lider af hypertension, vil medføre og/eller ændre de økonomiske udgifter i både den primære og sekundære sundhedssektor. I den primære sundhedssektor, som består af de praktiserende læger, er det relevant at se på udgifter/omkostninger i forbindelse med anvendte målemetoder samt nuværende antal henvendelser til almen praksis. Derudover skal teknologien købes ind, så patienterne kan anvende den, hertil skal det overvejes, om teknologien skal være betalt af det offentlige, eller om den skal være brugerbetalt. 


\subsection{Direkte omkostninger}
De direkte omkostninger betegner reelle udgifter forbundet ved indkøb af Fitbit Flex til implementering og brug i den almene praksis.  
Ved implementering relateres de udgifter forbundet til efteruddannelse af personale. Dette berøres ligeledes i \autoref{sec:efteruddannelse}, hvor efteruddannelsesfonden årligt stiller $13.000$ kr. til rådighed \citep{vedsted2005}.  
'Brug' relaterer til de udgifter der kan forekomme ved udlån og introduktion af armbåndet. 

%Indkøb
Indkøbsprisen for et Fitbit Flex armbåndet er $749\ kr$ hvis det købes direkte fra producentens hjemmeside. Dette er dækker dog over de omkostninger der vil være tilknytte til privat køb. 
Ved køb gennem sundhedssektoren kan betragtes dette som et virksomhedskøb, hvortil der ikke pålægges moms afgifter, der udgør $20~\%$ af den opgivede pris. 
Eventuelt kan der indgås købsforhandlinger, hvortil aftale om yderligere besparelse kan aftales ved køb af større mængder. 

 \textbf{Her vil der komme en mere dybdegående beskrivelse af omkostninerne vdr. implementeringerne}
%Implementering og brug 
%Implementering

Til at bedømme omkostningerne forbundet ved brug af Fitbit Flex, tages der udgangspunkt i honorartabellen fra PLO \citep{honorartabel2016}. Tabellen er et opslag over hvilke honorar der kan gives praktiserende læger ved forskellige ydelser.
Da der ikke forekommer nogle direkte omkostninger i forhold til udlån og introduktion af aktivitets armbånd, sammenlignes der i stedet med udgifterne ved hjemmeblodtryksmåling. Udlevering og introduktion af hjemmeblodtryksmåler udstyr vil typisk forekomme ved en konsultation, hvorfor der også vil blive lagt honorar til dette. Honoraret for en konsultation ligger på $137,83$ kr. som der også er skrevet i afsnit \autoref{sec:nuv_primaer}. Udlån og instruktion af hjemmeblodtryksmåler ligger på $141,68$ kr. hvilket vurderes at være sammenlignligt med et aktivitetsarmbånd, da det cirka omfatter de samme procedurer (bare lidt anderledes), dette kan eventulet forekomme flere gange efter behov hos patienterne. Dette er et samlet honorar på $279,51$ kr.

\subsection{Indirekte omkostninger}

\textbf{Tekst omhandldlende evtuelle besparelse som led i en implementering af aktivitetsarmbånd, så som medicin og undgåning af behandlingsresistent hypertension}













\begin{comment}
Hvad koster et Fitbit Flex? 
Hvilke besparelser tilbydes der så sundhedsektoren? 

Hvad koster det så at introducere patienterne til teknologien? 
	Hvad dækker den her introduktion minimum over, for at kunne anvende armbåndet? (Anvendelse af app og hvordan den skal oplades.)
	
Hvad koster det hvis de har spørgsmål vedr. teknologien? 


%Diskussion orienteret 
Langsigtet omkostninger - hvis behandlingen hjælper/ikke hjælper
- Besparelser vedr. medicinering 
- Besparelser vedr. ambulant forløb 
- Forebyggelse af behandlingsresistent hypertension = $$$$



EVT: Dags-takster i sekundær sektor (Ambulant).



En model i almen praktsis for implementeringen af aktivitetsarmbånd?

Honorartabel = \citep{honorartabel2016}
\end{comment}