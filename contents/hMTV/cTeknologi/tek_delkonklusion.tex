\section{Delkonklusion}

Tracking og repræsentering af hypertensive patienters aktivitetsniveau i almen praksis er muligt ved brug af Fitbit Flex og de forskellige elementer af brugerfladen i form af den dertilhørende applikation og Fitbits egen hjemmeside. Ved synkronisering mellem aktivitetsarmbånd og smartphone eller computer, er  det muligt for patienten at følge sit aktivitetsniveau, hvorved data samtidig videregives online til analyse for lægen. Fitbit Flex skal, ved introduktion til patienten, tilpasses brugerens vægt, højde, alder, køn og skridtlængde for at estimere aktiviteten optimalt.

Igennem Fitbits hjemmeside har lægen mulighed for at monitorere antal skridt, tilbagelagt afstand og aktive timer med forskellige aktivitetsniveauer for de seneste $30$ dage. Disse data er tilgængelige for lægen ved at følge patientens konto via egen brugerkonto, frem for at logge ind på hver bruger, for at se den enkelte patients aktivitet. Herved skabes et overblik over patienternes aktivitetsniveau. 

I \autoref{noejagtighed} er det fundet, at Fitbit Flex har en præcision mellem $96,4$ og $99,6~\%$ afhængigt af gangtypen, hvorfor Fitbit Flex antages som værende nøjagtig nok til at repræsentere  patientens aktivitetsniveau. Samtidigt blev det fundet, at målinger med armbåndet har en høj repeterbarhed, hvilket betyder, at pålideligheden i forbindelse med observation af ændrede aktivitetsmønstre hos patienter med hypertension er stor. Dette  betyder, at lægen med kan observere, hvis der sker en stigning eller et fald i patientens aktivitetsniveau, hvilket står i modsætning til den nuværende subjektive målemetode. 
Ulempen ved Fitbit Flex er, at patienterne skal informere lægen om anden aktivitet ud over gang og løb som følge af, at armbåndet ikke er i stand til at måle andre aktivitetstyper.

Overordnet vurderes det, på baggrund af patientgruppens primære aktivitetstype; løb og gang, og Fitbit Flex's egenskaber, at armbåndet vil være et brugbart redskab til objektiv monitorering af patienternes fysiske aktivitet, såfremt patienterne er i stand til at anvende teknologien.  Eftersom Fitbit har lanceret Fitbit Flex 2, der giver mulighed for tracking af svømning, vil det være relevant at overveje implementering af den nye model, såfremt fremtidige studier viser samme høje præcision og repeterbarhed, som fundet ved første udgave af Fitbit Flex.