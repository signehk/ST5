\section{Delkonklusion}

Tracking af patienters aktivitetsniveau med Fitbit Flex i almen praksis, er muligt med de forskellige elementer af brugerfladen i form af både applikationer og Fitbit's egen hjemmeside. For patienten er det muligt at følge egen aktivitet, ved synkronisering mellem armbånd og mobiltelefon eller computer, hvorved data samtidig videregives til analyse på nettet. Fitbit skal ved første anvendelse tilpasses brugerens vægt, højde, alder, køn og skridtlængde, for at estimere aktiviteten optimalt.

Lægen har gennem Fitbit's hjemmeside mulighed for at se antal skridt, tilbagelagt afstand og aktive timer med forskellige aktivitetsniveauer for de seneste $30$ dage. Dette kan ses ved at tilføje patientens konto som "ven", frem for at lægen logger ind på hver bruger, for at se den enkelte patiens aktivitet. 

I \ref{noejagtighed} er det fundet at Fitbit Flex har en præcision mellem $96,4~\%$ og $99,6~\%$ afhængigt af gangtypen, hvorfor armbåndets præcision antages at være høj nok til en estimering af patienternes aktivitetsniveau. Samtidig blev det fundet at målinger med armbåndet har en høj repeterbarhed, hvilket betyder at pålideligheden i forbindelse med observation af ændrede aktivitetsmønstre hos patienterne er stor. Ulempen ved Fitbit Flex, er at patienterne skal informere lægen om anden aktivitet end gang og løb, som følge af armbåndet ikke er i stand til at måle andre aktivitetstyper.

Overordnet vurderes det på baggrund af patientgruppens primære aktivitetstype og Fitbit Flex's egenskaber, at armbåndet vil være et godt reskab til objektiv monitorering af patienternes fysiske aktivitet, såfremt patienterne er i stand til at anvende teknologien. Eftersom Fitbit har lanceret Fitbit Flex 2, der giver mulighed for tracking af svømning, vil det være relevant at overveje implementering af den nye model, såfremt fremtidige studier viser samme høje præcision og repeterbarhed, som fundet ved Fitbit Flex.