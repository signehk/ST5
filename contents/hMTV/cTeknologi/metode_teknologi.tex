\chapter{Teknologi}
\section{Metode}
Dette kapitel har fokus på det teknologiske element, hvor teknologien vil blive karakteriseret, analyseret og vurderet. Dette gøres i henhold til \autoref{sec:metode}. Der er opstillet en række MTV-spørgsmål, som vil redegøre for og vurdere, om aktivitetsarmbånd fra et teknologisk synspunkt kan anvendes til at måle aktivitetsniveau hos patienter med ’sygdom’. Herudover vil det blive undersøgt, hvilke følgevirkninger anvendelse af aktivitetsarmbånd har på patientens sygdom. (Eller noget??) Dette giver anledning til følgende MTV-spørgsmål: 
\subsection{MTV-spørgsmål}
\begin{itemize}
\item Hvordan måles patienters aktivitet på nuværende tidspunkt? (I problemanalysen?)
\item Hvordan fungerer en aktivitets tracker/armbånd(??), og hvordan kan denne anvendes i medicinsk sammenhæng, således at en almen praktiserende læge får dokumenteret patientens aktivitetsniveau?
\item (Er der på nuværende tidspunkt læger, der anvender aktivitetsarmbånd, hvordan får de data fra aktivitetsarmbåndet?)
\item Hvor stor nøjagtighed/pålidelighed er det nødvendigt, at aktivitets armbånd har, hvis de skal anvendes til det ønskede formål, og hvor nøjagtige/pålidelige er de eksisterende aktivitetsarmbånd? (eller omvendt – Hvor nøjagtige er aktivitetsarmbånd, og er pålideligheden stor nok til at kunne anvendes i sundhedssektoren?)
\item Hvilken effekt (eller konsekvenser, følgevirkninger?) har anvendelsen af aktivitetsarmbånd til dokumentering af aktivitetsniveau på patientens sygdom?
\item (Hvilke problematikker kan opstå ved brug af aktivitetsarmbånd?) fneh...

\end{itemize}








