\chapter{Teknologi}
Dette kapitel har fokus på det teknologiske element, hvor teknologien vil blive karakteriseret, analyseret og vurderet.
\section{Metode}
Teknologien er opstillet ud fra en række MTV-spørgsmål, som vil beskrive teknologien og redegøre for og vurdere, hvilke teknologiske krav, aktivitetsarmbåndene skal opfylde for at kunne benyttes til at måle aktivitetsniveau hos hypertensive patienter. 
Dette vil gøres på baggrund af en tilpasset litteratur søgning ved brug af søgeprotokol, hvor nøgleord som 'XXX' anvendes. 
Yderligere fortages beskrivelse af teknologien ud fra egene observationer, ved anvendelse software relateret til teknologien.   
Herudover vil det blive undersøgt, hvilke effekter anvendelse af aktivitetsarmbånd har på patientens sygdom. 
Dette giver anledning til følgende MTV-spørgsmål: 
\subsection{MTV-spørgsmål}
\begin{itemize}
%\item Hvordan måles patienters aktivitet på nuværende tidspunkt? (I problemanalysen?)
\item Hvordan anvendes og registrerer Fitbit Flex fysisk aktivitet, og hvordan kan den anvendes i medicinsk sammenhæng, således at en almen praktiserende læge får dokumenteret patientens aktivitetsniveau?
%\item (Er der på nuværende tidspunkt læger, der anvender aktivitetsarmbånd, hvordan får de data fra aktivitetsarmbåndet?)
%\item Hvor stor nøjagtighed/pålidelighed er det nødvendigt, at aktivitets armbånd har, hvis de skal anvendes til det ønskede formål, og hvor nøjagtige/pålidelige er de eksisterende aktivitetsarmbånd? (eller omvendt – Hvor nøjagtige er aktivitetsarmbånd, og er pålideligheden stor nok til at kunne anvendes i sundhedssektoren?)
\item Repræsenterer Fitbit Flex den fysiske aktivitet tilstrækkeligt, til at data kan anvendes af praktiserende læger som beslutningsgrundlag?
\item Hvilken effekt har anvendelsen af Fitbit Flex til dokumentation af aktivitetsniveau på patientens sygdom?
%\item (Hvilke problematikker kan opstå ved brug af aktivitetsarmbånd?) fneh...
\end{itemize}








