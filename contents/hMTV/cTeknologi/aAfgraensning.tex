\subsection{Teknologiafgrænsning}

Igennem en undersøgelse af hvilke funktioner, der vil være relevante i forbindelse med aktivitetstracking, opstilles krav som udgangspunkt for valget af den endelige teknologi. Kravene til funktion vil blive stillet ud fra den primære aktivitetsform hos patientgruppen, således trackeren er optimeret til netop denne aktivitetstype.

I Danmark stiger prævalensen af hypertension med alderen. Det ses blandt andet at der kun er $1~\%$ af de $20-29$ årige, som lider af hypertension, mens omkring $69~\%$ af de $80-89$ årige har sygdommen \citep{olsen2015}. Som følge af den forøgede risiko for hypertension i sammenhæng med alderen, vil den primære anbefalede fysiske aktivitet for ældre over 65 år, være 30 minutters moderat intensitet om dagen og mindst 2 gange 20 minutters muskelstyrkende eller konditionsforøgende aktivitet om ugen \citep{pedersen2011}.

Hos ældre anses gang over $6$ km/t som konditionsforøgende aktivitet, og gang med $4-5$ km/t som moderat aktivitet. Med udgangspunkt i foregående, samt Sundhedsstyrelsens anbefalinger i 'Fysisk Aktivitet - Håndbog om forebyggelse og behandling at tage udgangspunkt i gangregistrering, med mulighed for udvidelse til svømning og cykling \citep{pedersen2011}.

\subsubsection{Krav til funktionalitet}

Såfremt tracking-enheden skal anvendes i hverdagen, er det vigtigt at den er kompakt og bærbar, samt at den ikke har behov for opladning på daglig basis. Som følge af at den primære aktivitet for patientgruppen er gang, er det vigtigt at enheden kan måle dette præcist, således målingerne kan anvendes som valide data. 

Da Sundhedsstyrelsen også anbefaler svømning og cykling hvis patienten har mulighed for dette, vil det være relevant men ikke påkrævet, at trackingenheden har mulighed for at måle denne type aktivitet. Registrering af disse aktiviteter kræver både vandtæthed og GPS eller mulighed for kommunikation med en ekstern cykelcomputer på patientens cykel.

\subsubsection{Valg af aktivitetstracker}

For at finde den mest optimale aktivitetstracker til formålet, tages der udgangspunkt i studier, som har undersøgt præcisionen af forskellige aktivitetsarmbånd ved blandt andet antal skridt, energiforbrug og afstand. Ud over dette er brugerfladen også bedømt, hvorfor dette også er relevant at tage med i overvejelserne.

I både \citep{evenson2015} og \citep{kaewkannate2016} er det fundet, at FitBits skridttæller har en høj præcision, som ifølge \citep{kaewkannate2016} ligger omkring  $95-99~\%$ afhængigt af om patienten bevæger sig på trapper, løbebånd eller fladt underlag. Her er præcisionen højest ved gang på fladt underlag, mens Fitbit uret scorer lavest ved gang på trapper eller løbebånd \citep{kaewkannate2016}.  (mÅSKE LIGGE UD MED AT GIVE EN LILLE BESKRIVELSE AF HVAD STUDIET(ERNE) HAR UNDERSØGT, HVOR MANGE AKTIVITETSTRACKERER ER DER TALE OM OG MÅSKE HVILKE FOR AT FÅ EN LIDT "BLØDERE AFGRÆNSNINNG TIL FITBIT URENE)

Fitbit-modellernes målinger i forhold til antal skridt, afstand og energiforbrug varierer ikke fra hinanden, ved brug af flere målere af samme model, hvorfor der ikke vil være stor forskel på målte data ved brug af forskellige ure \citep{evenson2015}. Dette er blandt andet relevant ved dataindsamling til studier vedrørende effekten af armbåndet, da de optagede data dermed kan sammenlignes med større validitet. Samtidig undgåes problemer ved kalibrering, hvis patienten skal have byttet uret, som følge af fejlfunktion.

Yderligere fordele ved Fitbit (FLEX? :) )inkludere muligheden for at gemme data i op til $30$ dage, muligheden for at sammenligne med andres aktivitet, vandtæthed og kompitabilitet med fitness-apps på telefon og computer \citep{kaewkannate2016, fitbitflex}. Især de sociale egenskaber ved Fitbit's armbånd, samt muligheden for at tracke aktiviteten med apps, kan give anledning til øget aktivitetsniveau hos patienterne \citep{karapanos2016, rooksby2014}. Fitbit Flex er dog ikke udstyret med GPS, men hvis funktionen er nødvendig for tracking af bestemte aktiviteter, har patienten mulighed for at kombinere GPS-data fra eksempelvis smartphones med Fitbit's data \citep{fitbitflex}. (DER MANGLE MÅSKE LIDT EN LILLE ... DERFOR VÆLGES MODELLEN FLEX FRA FITBIT, DA DENNE... I DETTE AFSNIT :))