\section{Fordele og begrænsninger}\label{sec:tek_fordelebegr}
Sammenlignes anvendelse af aktivitetsarmbåndet Fitbit Flex med nuværende anvendte metoder til monitorering af aktivitetsniveau i almen praksis, er der forskellige fordele og begrænsninger ved denne alternative metode. Som tidligere nævnt i \autoref{sec:alternativemetoder} giver anvendelsen af aktivitetsarmbånd den alment praktiserende læge en mere nøjagtig vurdering af en patients aktivitetsniveau sammenlignet med subjektive besvarelser såsom spørgeskemaer.

Informationer om aktivitetsniveau opsamles automatisk, og det er dermed ikke nødvendigt for patienten selv at holde styr på, hvor meget aktivitet, der udføres. Lægens opfattelse af, hvor meget fysisk aktivitet patienten udfører, afhænger derved heller ikke af patientens hukommelse eller evne til at formidle. Det giver lægen en mere objektiv oversigt over aktiviteten udført over tid, og dette kan hjælpe med at se en eventuel udvikling eller tilbagegang i aktivitetsniveauet hos patienten. 

Det er desuden ikke nødvendigvis ekstra tidskrævende at anvende aktivitetsarmbåndet sammenlignet med de nuværende metoder i \autoref{NuMetode}, såfremt patienten har modtaget tilstrækkelig instruktion vedrørende anvendelse af Fitbit Flex, hvis aktivitetsarmbåndet udleveres i forbindelse med kontrol ved alment praktiserende læge. Problemer kan opstå, hvis patienten oplever besvær ved anvendelsen af aktivitetsarmbåndet, og ikke bruger det korrekt eller vælger helt at undgå at bruge det.

Fitbit Flex er generelt ikke velegnet til tracking af andet end gang og løb. Anden aktivitet end dette kan medføre, at lægen får forkert indblik i patientens aktivitetsniveau, hvis patienten ikke har meddelt lægen, at patienten foruden den registrerede aktivitet også dyrker anden sport. Sammenlignet med de subjektive metoder, der anvendes i klinikkerne, kan alle former for fysisk aktivitet medtages på én gang ved besvarelse af eksempelvis spørgeskema eller under samtale om patientens aktivitetsniveau.
En ulempe ved anvendelse af Fitbit Flex sammenlignet med andre aktivitetsarmbånd er, at der ikke er indbygget GPS i dette. For at opnå højere præcision til måling af en aktivitet, kan en ekstern GPS, eksempelvis i en smartphone, anvendes. Det er dog ikke en nødvendighed for lægen at vide placeringen, da lægen som udgangspunkt kun er interesseret i at kende aktivitetsniveauet.

En begrænsning i forhold til aktivitetsarmbånd sammenlignet med nuværende metoder kan være, at det er en mere teknologisk metode, der kan kræve adgang til internettet gennem en smartphone eller en computer. Størstedelen af de hypertensive patienter er en del af den ældre befolkningsgruppe. I 2014 var der ifølge Danmarks Statistik 41 \% af de 75-89 årige, der aldrig har brugt internettet \citep{dst2014}. Hvis lægen skal have mulighed for at tilgå patientens data vedrørende aktivitetsniveauet, når patienten ikke er fysisk til stede, eksempelvis ved telefonkonsultationer, er det nødvendigt, at patienten har adgang til internettet for at synkronisere data fra aktivitetsarmbåndet med vedkommendes brugerkonto.

Som nævnt i \autoref{sec:teknologibeskrivelse} skal Fitbit Flex synkroniseres med en smartphone eller computer for at kunne visualisere den registrerede aktivitet. Dette kræver, at patienten er i besiddelse af en af disse, hvis patienten selv ønsker at følge med i aktiviteten. Har patienten ikke mulighed for at komme i besiddelse af smartphone eller computer, kan detaljeret data lagres i aktivitetsarmbåndet i op til syv dage. Dette er ikke problematisk, hvis lægen og patienten synkroniserer armbåndet inden for disse syv dage, såfremt det ønskes at se detaljeret data. Alternativt kan Fitbit Flex gemme mindre detaljeret data i op til 30 dage. 
