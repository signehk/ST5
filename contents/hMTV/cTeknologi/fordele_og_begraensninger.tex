\subsection{Fordele og begrænsninger ved teknologien}
Sammenlignes anvendelse af aktivitetsarmbåndet Fitbit Flex med nuværende anvendte metoder til objektivisering af aktivitetsniveau i almen praksis, er der forskellige fordele og begrænsninger ved denne alternative metode. Som tidligere nævnt giver anvendelsen af aktivitetsarmbånd den almen praktiserende læge en mere nøjagtig vurdering af en patients aktivitetsniveau sammenlignet med subjektive besvarelser såsom spørgeskemaer.

Informationer om aktivitetsniveauet opsamles automatisk, og det er dermed ikke nødvendigt for patienten selv at holde styr på, hvor meget aktivitet, der udføres. Lægens opfattelse af, hvor meget fysisk aktivitet patienten udfører afhænger derved heller ikke af patientens hukommelse eller evne til at formidle. Det giver lægen en mere præcis oversigt over aktiviteten udført over tid, og dette kan hjælpe med at se en eventuel udvikling eller tilbagegang i aktivitetsniveauet hos patienten. 

Det er desuden ikke nødvendigvis ekstra tidskrævende at anvende aktivitetsarmbåndet, hvis patienten har modtaget tilstrækkelig information om, hvordan udstyret anvendes korrekt. Problemer kan opstå, hvis patienten oplever besvær ved anvendelsen af aktivitetsarmbåndet, og ikke bruger det korrekt eller vælger helt at undgå at bruge det.

Fitbit Flex har som tidligere nævnt ikke mulighed for at registrere svømning. Udfører en patient fysisk aktivitet i form af svømning, kan dette ikke måles med aktivitetsarmbåndet. Dette kan medføre at lægen får en forkert objektivisering af patientens aktivitetsniveau, hvis patienten ikke har meddelt lægen, at patienten foruden den registrerede aktivitet også svømmer. Sammenlignet med de subjektive metoder, der anvendes i klinikkerne, kan alle former for fysisk aktivitet medtages på én gang ved besvarelse af eksempelvis spørgeskema eller under samtale om patientens aktivitetsniveau.
En ulempe ved anvendelse af Fitbit Flex sammenlignet med andre aktivitetsarmbånd er, at der ikke er indbygget GPS i denne. Ønsker patienten at vide placering, skal en ekstern GPS, eksempelvis på en smartphone, anvendes. Det er dog ikke en nødvendighed for lægen at vide placeringen, da lægen som udgangspunkt kun er interesseret i at kende aktivitetsniveauet. Cykler patienten, kan GPS være en metode til at holde styr på afstanden, der er tilbagelagt. Samme problemstilling som ved svømning kan her opstå, hvis aktivitetsarmbåndet ikke registrerer aktiviteten under cykling.
 
En begrænsning i forhold til aktivitetsarmbånd sammenlignet med nuværende metoder er, at det er en mere teknologisk metode, der kan kræve adgang til smartphone eller PC og muligvis internet. Størstedelen af de hypertensive patienter er en del af den ældre befolkningsgruppe. I 2014 var der ifølge Danmarks Statistik 41 \% af de 75-89 årige, der aldrig har brugt internettet \citep{dst2014}. Hvis lægen skal have mulighed får at tilgå patientens data vedrørende aktivitetsniveauet uden, at patienten er fysisk til stede, eksempelvis ved telefonkonsultationer, er det nødvendigt, at patienten har adgang til internettet for at synkronisere data fra aktivitetsarmbåndet med dennes brugerkonto.

Som nævnt i \autoref{sec:teknologibeskrivelse} skal Fitbit Flex synkroniseres med en smartphone eller PC for at kunne se den registrerede aktivitet. Dette kræver, at patienten er i besiddelse af en af disse, hvis patienten selv ønsker at følge med i aktiviteten. Har patienten ikke mulighed for at komme i besiddelse af smartphone eller PC, kan detaljeret data lagres i aktivitetsarmbåndet i op til syv dage. Dette kan dog imødegås ved, at lægen tjekker data fra aktivitetsarmbåndet indenfor de syv dage, hvis det ønskes at se detaljeret data. Ellers kan Fitbit Flex gemme mindre detaljeret data i op til 30 dage. Fitbit Flex har desuden et batteri, der genoplades ved at koble den til en PC. 
