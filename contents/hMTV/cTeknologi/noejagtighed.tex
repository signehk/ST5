\subsection{Nøjagtighed af aktivitetsmåling}

Præcisionen af Fitbit Flex armbåndet vil have indvirkning på brugbarheden af de målte data, og problematikken vedrørende præcisionen beskrives i følgende afsnit. For at armbåndet kan anvendes i praksis, er det vigtigt at armbåndet ikke har tendens til overestimering af patientens aktivitetsniveau. Dette krav er stillet, som følge af et forhøjet aktivitetsniveau gavner patienten, hvorfor det ikke vil være hensigtsmæssigt at implementere et armbånd, der giver patienten et indtryk af at have opnået den ønskede aktivitetstid, før patienten reelt set har opnået de daglige mål. Her vil det være mere gavnligt for patientens tilstand, hvis armbåndet underestimere aktiviteten, således patienten kommer til at bevæge sig mere end det ønskede mål.

Både \citeauthor{evenson2015} og \citeauthor{kaewkannate2016} har undersøgt præcisionen af forskellige aktivitetsarmbånd. I \citetalias{evenson2015} undersøges allerede eksisterende studier, for derved at opnå evidens for validitet og pålidelighed ved anvendelse af eksempelvis forskellige Fitbit og Jawbone modeller. Det andet studie, \citetalias{kaewkannate2016}, har undersøgt Fitbit Flex, Withing Pulse, Misfit Shine og Jawbone Up$24$. I studiet er både brugertilfredshed, repeterbarhed og præcision undersøgt, hvor aktivitetsarmbåndene testes ved normal gang, gang på trapper og gang på løbebånd. Samtidig testes muligheden for at gentage forsøget med samme resultat også, for at undersøge repeterbarheden \citep{evenson2015, kaewkannate2016}.

I studierne er det fundet, at Fitbit Flex's skridttæller har en høj præcision, som ifølge \citeauthor{kaewkannate2016} ligger mellem $96,4~\%$ til $99,6~\%$, afhængigt af om patienten bevæger sig på trapper, løbebånd eller fladt underlag. Her er præcisionen højest ved gang på fladt underlag, mens Fitbit uret scorer lavest ved gang på trapper eller løbebånd. I undersøgelsen blev de undersøgte aktivitetsarmbånds præcision fundet til omkring $95-99~\%$ afhængigt af gangtypen \citep{kaewkannate2016}.

Fitbit Flex's målinger i forhold til antal skridt, afstand og energiforbrug varierer ikke markant fra hinanden ved samtidig brug af flere Fitbit Flex armbånd, hvorfor der ikke vil være stor ændring på målte data ved eventuel udskiftning af armbånd. Her er repeterbarheden for armbånd båret på højre og venstre håndled $0,90$ for skridt og $0,95$ for kilokalorier \citep{evenson2015}. For Fitbit Flex er repeterbarheden i det andet studie fundet mellem $0,72$ og $0,81$ afhængigt af gangtypen, mens den laveste og højeste repeterbarhed er $0,55$ og $0.89$ for det samlede studie. Her er repeterbarheden fundet ved at se på den samlede afstand forsøgspersonen er gået og den målte afstand for aktivitetsarmbåndene. \citep{kaewkannate2016}. Dette er blandt andet relevant ved dataindsamling til studier vedrørende effekten af armbåndet, da de optagede data dermed kan sammenlignes med større validitet. Samtidig undgåes problemer ved kalibrering, hvis patienten skal have byttet aktivitetsarmbåndet, som følge af fejlfunktion.

Betydningen af over- og underestimering af aktivitetsniveauet, gør at Fitbit Flex anses som værende anvendeligt til aktivitetsestimering hos patienterne i almen praksis, eftersom præcisionen ved almindelig gang er tæt på $100~\%$, mens den underestimerer andre gangtyper \citep{kaewkannate2016}. Derved opnås ekstra aktivitet hos patienten, før han/hun bliver gjort opmærksom på at de daglige mål er nået, hvilket anses som et positivt resultat af fejlestimering ved anvendelse af aktivitetsarmbåndet.