\subsection{Nøjagtighed af aktivitetsmåling}

Præcisionen af Fitbit Flex armbåndet vil have indvirkning på brugbarheden af de målte data, og problematikken i forhold til armbåndets begrænsninger er beskrevet i \ref{afgraensning_tek}. Det blev fundet at armbåndets præcision ved almindelig gang er $99,6~\%$, mens det jævnfør "A comparison of warable fitness devices" af \citep{kaewkannate2016}, underestimerer aktiviteten ved gang på trapper eller løbebånd \citep{kaewkannate2016}.

For at armbåndet kan anvendes i praksis, er det vigtigt at armbåndet ikke har tendens til overestimering af patientens aktivitetsniveau. Dette krav er stillet, som følge af et forhøjet aktivitetsniveau gavner patienten, hvorfor det ikke vil være hensigtsmæssigt at implementere et armbånd, der giver patienten et indtryk af at have opnået den ønskede aktivitetstid, før patienten reelt set har opnået de daglige mål. Her vil det være mere gavnligt for patientens tilstand, hvis armbåndet underestimere aktiviteten, således patienten kommer til at bevæge sig mere end det ønskede mål.

Betydningen af over- og underestimering af aktivitetsniveauet, gør at Fitbit Flex anses som værende anvendeligt til aktivitetsestimering hos patienterne i almen praksis, eftersom præcisionen ved almindelig gang er tæt på $100~\%$, mens den underestimerer andre gangtyper. Derved opnås ekstra aktivitet hos patienten, før han/hun bliver gjort opmærksom på at de daglige mål er nået, hvilket anses som et positivt resultat af fejlestimering ved anvendelse af aktivitetsarmbåndet.